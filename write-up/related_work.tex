\chapter{Related Work}\label{chap:related}

% \begin{guidance}
% 
%     This chapter covers relevant (and typically, recent) research
%     which you build upon (or improve upon). There are two complementary
%     goals for this chapter:
%     \begin{enumerate}
%       \item to show that you know and understand the state of the art; and
%       \item to put your work in context
%     \end{enumerate}
% 
%     Ideally you can tackle both together by providing a critique of
%     related work, and describing what is insufficient (and how you do
%     better!)
% 
%     The related work chapter should usually come either near the front or
%     near the back of the dissertation. The advantage of the former is that
%     you get to build the argument for why your work is important before
%     presenting your solution(s) in later chapters; the advantage of the
%     latter is that don't have to forward reference to your solution too
%     much. The correct choice will depend on what you're writing up, and
%     your own personal preference.
% 
% \end{guidance}

\prechapter{%
    Now that I have described my contributions, I will explain how it relates
    to existing work, leaving brief discussions on future work to the next
    chapter. I strongly believe the field of matrix expression compilation would
    benefit greatly from a comprehensive literature review but unfortunately
    that is beyond the scope of this chapter.
}%

\section{Matrix Expression Compilation}\label{sec:mat_exp_comp}

Most of the projects below try to be fully-automated black-boxes which model
computing a matrix expression as some sort of graph with informal, ad-hoc rules
about what can and should be copied or modified in-place. Allocations,
temporaries and common sub-expressions are details invisible to the programmer,
left to the compiler.

The matrix expression `compiler' as implemented in LT4LA is intended to be a
mere proof-of-concept of how linear types can and arguably should be used
\emph{ergonomically}. I have taken the approach of attempting to help
programmers precisely and explicitly capture, using types, the practices
prevalent in code they already write.

I believe the advantages of my approach are two-fold (1) \emph{predictable}
performance (as defined in Chapter~\ref{chap:eval}) and (2) more accurate
modelling of how low-level kernels handle their resources. My confidence in the
latter claim comes from finding errors in Fortran code output by SymPy to
compute a Kalman filter (as described in Section~\ref{sec:qual_benefits});
errors that would not have passed type-checking had it been translated via
LT4LA as an intermediate representation.

\subsection{SymPy}

SymPy is a symbolic computer algebra system for Python; its matrix expression
compiler~\cite{rocklin_thesis} uses a term-rewrite system, with rules supplied
by a BLAS expert (which must be strongly-normalising, that is, never cause a
loop) but need not be confluent (there can be more than one solution per
expression). Rules include expressions to match on, the expression it can
produce, information about the expressions (such as whether the matrix is
symmetric or full-rank) and information about which variable is updated
in-place.

\subsection{Clak and Cl1ck}

Clak and Cl1ck~\cite{fabregat_thesis} were developed independently around the
same time as SymPy's matrix expression compilation. Clak attempts to produce
\emph{multiple} algorithms for a single matrix expression, by considering a
wider matrix expression grammar and more matrix properties and inference rules.
These algorithms assume basic building blocks such as products and
factorisations. Cl1ck attempts to take on the challenge of writing BLAS/LAPACK
like libraries too, by generating lower-level loop-based blocked routines for
the aforementioned basic building blocks, in the spirit of the
FLAME~\cite{gunnels_flame} project.

\subsection{Linnea and Taco}

Linnea~\cite{linnea} and Taco~\cite{taco} are two newer contenders to Clak and
Cl1ck respectively. Linnea continues the work of Clak to producing real
executable code for \emph{existing} libraries and kernels, as well as
incorporating work on a \emph{generalised} matrix chain algorithm~\cite{gmc}.
Taco (\emph{Tensor Algebra COmpiler}) focuses on emitting efficient routines
for expressions in tensor index notation, with many optimisations for
\emph{sparse} tensors.

\section{Metaprogramming}

Most of the compilers in the aforementioned projects usually built, analysed,
compiled, ran (and in some cases, dynamically loaded) expressions (including
functions) at runtime, similar to how regular expressions are handled in most
languages -- in particular, even when the regular expression is known at
compile time.

In LT4LA, I took the approach of having a concrete syntax and expression
language which was then translated and made available as a \emph{typed
expression} to other modules \emph{at compile time} via the build system. Apart
from convenience in programming and testing, there was nothing inherent in the
approach that prevented me from using OCaml's PPX syntax-extensions so that I
could write normal OCaml expressions from within OCaml and have them checked
for linearity before compilation.

Having a statically compiled language and a build system as so affords the
advantage of eliminating the runtime overheads mentioned at the start of this
section. However, there is some useful information (such as matrix dimensions
for the matrix chain algorithm) which is sometimes known only at runtime (but
once known, usually fixed). In these cases, using \emph{multi-stage
programming} would be a better approach to implementing a matrix expression
compiler.

\subsection{MetaOCaml and Scala LMS}

MetaOCaml~\cite{metaocaml} and Scala with Lightweight Modular
Staging~\cite{scala_lms} are systems which support multi-stage programming. A
typical example of this is the generation of Fast Fourier-Transform kernels,
specialised to a desired array length. Combining this with recent work on
generalising and automatically deriving \emph{partially static} representations
of data~\cite{yallop}, it may be possible to apply such techniques to
\emph{tensor algebra} expressions.

\subsection{Expression Templates in C++ Libraries}

Expression templates are a commonly used compile-time metaprogramming
technique, used by Eigen, uBLAS and Armadillo to name a few. If known at
compile-time, matrix dimensions can also be passed in as template arguments to
ensure operations match (otherwise checking at runtime). In Eigen, such
features are combined with heuristics to enable \emph{lazy evaluation} and
automatically determine whether a sub-expression is evaluated into a temporary
variable or not.

They perform rudimentary pattern-matching and in some cases, loop-fusion, to
avoid evaluating expressions in a purely binary manner (invoking the bane of a
C++ programmer: temporaries and unnecessary copies) when possible (either by
translating to a library kernel call or, as an example, inlining a $v := a + b
+ c$ vector expression into one loop).

As is the case with LT4LA, this approach shares \emph{some} elimination of
runtime overheads, but not all, thanks to the heuristics surrounding lazy
evaluation and evaluating sub-expressions into temporaries. This comes at the
cost of a user being able to easily inspect the generated C++ code, losing
explicitness.

\section{Types}

Apart from lazy evaluation, the following projects show how instead of a (E)DSL
library approach, we could have type-level resource management provided far
more conveniently and naturally at the \emph{language} level. The difference is
that a library can be designed, shipped and used now whereas language features
take time and can have unintended interactions with other language features. My
hope is that once people are convinced of the utility type-level resource
management by using a library, the impetus for integrating such features into
the language follows.

\subsection{Lazy Evaluation}

A particularly strong advantage LT4LA has over other libraries that use lazy
evaluation is, funnily enough, linear types; more precisely, it is the static
and perfect information they guarantee about aliasing: with Owl, every
graph-node has only one node in or out; with Eigen, every ``assignment'' has a
\texttt{noalias} annotation, without the danger of getting it wrong and
invoking undefined behaviour.

This simplifies the rules and exceptions a programmer reasoning about memory
usage needs to remember. Of course, now the programmer has to figure out how
they are using their temporaries, but because matrices are linearly-typed,
redundant copies/missed frees can be pointed out by the compiler, guiding them
towards a satisfying solution.

\subsection{Futhark}

Futhark~\cite{futhark} is a second-order (meaning it supports functions such as
map and fold/reduce) array combinator (meaning array operations can be fused
into streams to reduce temporaries) language designed for efficient parallel
compilation. It supports ML-style modules, loops, limited parametric
polymorphism, size types and uniqueness types (an idea closely related to
linear types). Its combinators are more expressive than typical linear-algebra
library kernels so encourage shorter, more declarative linear-algebra code.

\subsection{Substructural Features in Rust}

Rust~\cite{rust} is a (relatively) new systems programming language aiming to
bring the last two decades of programming language research to the masses in a
usable and friendly manner. Its \emph{borrow-checker} is the feature most
relevant to this thesis because it statically attempts to prevent many
resource-related bugs. Although there are a few linear-algebra libraries for
Rust under development, careful use of its macro system and borrow-checker
could make it the safest and easiest to use language for linear-algebra
projects to come. Its struggle is more likely to be against the inertia of the
large amounts of C++/Fortran code already out there rather than its usability
or benefits. Given this inertia, I believe there is value in taking the DSL
library approach of LT4LA which will work with existing systems.

\subsection{Linear Types in Haskell}

Linear types have been incorporated into a branch of the Glasgow Haskell
Compiler~\cite{retrofitting} in an attempt to provide safe, functional
streaming and IO (after people saw the potential from libraries providing
linearity features). Practical benefits include zero-copy buffers and
eliminating garbage collection in certain situations by allowing the user to
safely manage memory. The fact that it can and has been done gives me hope that
other languages will also see the value and adopt some form of
resource-management in their type systems.

\subsection{Linear and Dependent Types in Idris}\label{subsec:lin_dep_types}

Dimension mismatches are seen as an irritating but small inconvenience when
writing linear-algebra code. However, another error I found in the Fortran code
output by SymPy to compute a Kalman filter was a dimension/transposition error.
Although we would not need full dependent types to solve dimension mismatches
(symbolic size types would be sufficient), managing properties about matrices
could be done at the type-level in a dependently typed setting.

We could then express the usual properties and results of operations at the
type-level, ensuring, for example, that certain functions are called only when
the matrix is symmetric and can be written to. Idris (a Haskell inspired
language with dependent types) has had experimental support for uniqueness
types since its early days and now a linear types extension~\cite{idris_linear}
is also being worked on based on new research around integrating linear and
dependent types~\cite{atkey}.
