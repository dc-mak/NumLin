\section{Introduction}

Programmers writing numerical software often find themselves caught on
the horns of a dilemma. The foundational, low-level linear algebra
libraries such as BLAS and LAPACK offer programmers very precise
control over the memory lifetime and usage of vector and matrix
values. However, this power comes paired with the responsibility to
manually manage the memory associated with each array object, and in
addition to bringing in the familiar difficulties of reasoning about
lifetimes, aliasing and sharing that plague low-level systems
programming; this also moves the APIs away from the linear-algebraic,
mathematical style of thinking that numerical programmers want to use.

As a result, programmers often turn to higher-level languages such as
Matlab, R and NumPy, which offer very high-level array abstractions
that can be viewed as ordinary mathematical values. This makes
programming safer, as well as making prototyping and verification much
easier, since it lets programmers write programs which bear a closer
resemblance to the formulas that the mathematicians and statisticians
designing these algorithms prefer to work with, and ensures that
program bugs will reflect incorrectly-computed values rather than heap
corruption.

The intention is that these languages can use libraries BLAS and
LAPACK, without having to expose programmers to explicit memory
management. However, this benefit comes at a price: because user
programs do not worry about aliasing, the language implementations
cannot in general exploit the underlying features of the low-level
libraries that let them explicitly manage and reuse memory. As a
result, programs written in high-level statistical languages can
be much less memory-efficient than programs that make full use
of the powers the low-level APIs offer. 

So in practice, programmers face a tradeoff: they can eschew safety
and exploit the full power of the underlying linear algebra libraries,
or they can obtain safety at the price of unneeded copies and worse
memory efficiency. In this work, we show that this tradeoff is not a
fundamental one.

\lang\ is a functional programming language whose type system is
designed to enforce the safe usage of the APIs of low-level linear algebra
libraries (such as BLAS/LAPACK).  It does so by combining linear types,
fractional permissions, runtime errors and recursion into a small, easily
understandable, yet expressive set of core constructs.

\lang\ allows a novice to understand and work with complicated linear
algebra library APIs, as well as point out subtle aliasing bugs and
reduce memory usage in existing programs. In fact, we were able to use
\lang\ to find linearity and aliasing bugs in a linear algebra
algorithm that was \emph{generated} by another program
\emph{specifically designed to translate matrix expressions into an
  efficient sequence of calls to linear algebra routines}. We were
also able to reduce the number of temporaries used by the same
algorithm, using \lang's type system to guide us.

\lang's implementation supports several syntactic conveniences as well as a
\emph{usable} integration with real OCaml libraries.

\subsection{Contributions}

In this paper
\begin{itemize}
    \item we describe \lang, a linearly typed language for linear algebra programs
    \item we illustrate that \lang's design and features are well-suited to its
        intended domain with progressively sophisticated examples
    \item we prove \lang's soundness, using a step-indexed logical relation
    \item we describe a very simple, unification based type-inference algorithm
        for polymorphic fractional permissions (similar to ones used for
        parametric polymorphism), demonstrating an alternative approach to
        dataflow analysis \cite{bierhoff}
    \item we describe an implementation that is both compatible with and usable
        from existing code
    \item we show an example of how using \lang\ helped highlight linearity
        and aliasing bugs, and reduce the memory usage of a \emph{generated}
        linear algebra program
    \item we show that using \lang, we can achieve parity with C for linear
        algebra routines, whilst having much better static guarantees about the
        linearity and aliasing behaviour of our programs.
\end{itemize}

