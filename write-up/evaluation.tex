\chapter{TO DO: Evaluation}

\begin{guidance}
    For any practical projects, you should almost certainly have
    some kind of evaluation, and it's often useful to separate
    this out into its own chapter.
\end{guidance}

\prechapter{%
    In this chapter, I will argue the central premise of this thesis: linear
    types are a practical and usable tool to help working programmers
    write code that is both more legible and less resource-hungry than with
    existing linear-algebra frameworks. My project illustrates how to do so in
    a way that can be implemented as a usable \emph{library} for existing
    languages and frameworks that leverages the already impressive amount of
    work gone into optimising them so far.
}%

\section{Set-up}

\section{Results}

\section{Summary}

% Kalman Filter
% Least Squares Linear Regression
% p73 onwards https://people.cs.uchicago.edu/~mrocklin/storage/dissertation.pdf

% \section{Against Overly-Safe Copying}
% Numpy, Julia, Owl

% \section{Against Poor Legibility}
% C++, Fortran
