\documentclass[a4paper,UKenglish]{lipics-v2019}
%This is a template for producing LIPIcs articles. 
%See lipics-manual.pdf for further information.
%for A4 paper format use option "a4paper", for US-letter use option "letterpaper"
%for british hyphenation rules use option "UKenglish", for american hyphenation rules use option "USenglish"
%for section-numbered lemmas etc., use "numberwithinsect"
%for enabling cleveref support, use "cleveref"
%for enabling cleveref support, use "autoref"

\bibliographystyle{plainurl}

\title{NumLin: Linear Types for Linear Algebra}

\author{Dhruv C.~Makwana}{Unaffiliated \url{dhruvmakwana.com} }{dcm41@cam.ac.uk}{[orcid]}{}

\author{Neelakantan R.~Krishnaswami}{Department of Computer Science and Technology, University of Cambridge, United Kingdom}{nk480@cl.cam.ac.uk}{[orcid]}{}

\authorrunning{D.\,C. Makwana and N.\,R. Krishnaswami}

\Copyright{Dhruv C. Makwana and Neelakantan R. Krishnaswami}% LIPIcs license is "CC-BY";  http://creativecommons.org/licenses/by/3.0/

\ccsdesc[300]{Theory of computation~Program specifications}

\keywords{numerical, linear, algebra, types, permissions, OCaml}

\supplement{\url{www.github.com/dc-mak/lt4la}}

%\nolinenumbers %uncomment to disable line numbering

%\hideLIPIcs  %uncomment to remove references to LIPIcs series (logo, DOI, ...), e.g. when preparing a pre-final version to be uploaded to arXiv or another public repository

%Editor-only macros:: begin (do not touch as author)%
\EventEditors{John Q. Open and Joan R. Access}
\EventNoEds{2}
\EventLongTitle{42nd Conference on Very Important Topics (CVIT 2016)}
\EventShortTitle{CVIT 2016}
\EventAcronym{CVIT}
\EventYear{2016}
\EventDate{December 24--27, 2016}
\EventLocation{Little Whinging, United Kingdom}
\EventLogo{}
\SeriesVolume{42}
\ArticleNo{23}

% Source code highlighting
\usepackage[outputdir=../build]{minted}
  % Magic incantation to stop minted from putting red boxes around shit
  \usepackage{etoolbox}
  \makeatletter
  \AtBeginEnvironment{minted}{\dontdofcolorbox}
  \def\dontdofcolorbox{\renewcommand\fcolorbox[4][]{##4}}
  \makeatother
\RecustomVerbatimEnvironment{Verbatim}{BVerbatim}{}
% Convenient inline syntax highlighting
\newmintinline[highl]{ocaml}{breaklines}

\usepackage{amsmath,amssymb}

% Ott Rules
\usepackage{supertabular}%
% \usepackage{geometry}%
\usepackage{ifthen}%
\usepackage{alltt}%hack%

% generated by Ott 0.25 from: semantics.ott
\newcommand{\ottdrule}[4][]{{\displaystyle\frac{\begin{array}{l}#2\end{array}}{#3}\quad\ottdrulename{#4}}}
\newcommand{\ottusedrule}[1]{\[#1\]}
\newcommand{\ottpremise}[1]{ #1 \\}
\newenvironment{ottdefnblock}[3][]{ \framebox{\mbox{#2}} \quad #3 \\[0pt]}{}
\newenvironment{ottfundefnblock}[3][]{ \framebox{\mbox{#2}} \quad #3 \\[0pt]\begin{displaymath}\begin{array}{l}}{\end{array}\end{displaymath}}
\newcommand{\ottfunclause}[2]{ #1 \equiv #2 \\}
\newcommand{\ottnt}[1]{\mathit{#1}}
\newcommand{\ottmv}[1]{\mathit{#1}}
\newcommand{\ottkw}[1]{\mathbf{#1}}
\newcommand{\ottsym}[1]{#1}
\newcommand{\ottcom}[1]{\text{#1}}
\newcommand{\ottdrulename}[1]{\textsc{#1}}
\newcommand{\ottcomplu}[5]{\overline{#1}^{\,#2\in #3 #4 #5}}
\newcommand{\ottcompu}[3]{\overline{#1}^{\,#2<#3}}
\newcommand{\ottcomp}[2]{\overline{#1}^{\,#2}}
\newcommand{\ottgrammartabular}[1]{\begin{supertabular}{llcllllll}#1\end{supertabular}}
\newcommand{\ottmetavartabular}[1]{\begin{supertabular}{ll}#1\end{supertabular}}
\newcommand{\ottrulehead}[3]{$#1$ & & $#2$ & & & \multicolumn{2}{l}{#3}}
\newcommand{\ottprodline}[6]{& & $#1$ & $#2$ & $#3 #4$ & $#5$ & $#6$}
\newcommand{\ottfirstprodline}[6]{\ottprodline{#1}{#2}{#3}{#4}{#5}{#6}}
\newcommand{\ottlongprodline}[2]{& & $#1$ & \multicolumn{4}{l}{$#2$}}
\newcommand{\ottfirstlongprodline}[2]{\ottlongprodline{#1}{#2}}
\newcommand{\ottbindspecprodline}[6]{\ottprodline{#1}{#2}{#3}{#4}{#5}{#6}}
\newcommand{\ottprodnewline}{\\}
\newcommand{\ottinterrule}{\\[5.0mm]}
\newcommand{\ottafterlastrule}{\\}
\newcommand{\ottmetavars}{
\ottmetavartabular{
 $ \mathit{fraccap} ,\, \mathit{fc} $ & \ottcom{fractional capability variable} \\
 $ \mathit{expression} ,\, \mathit{x} ,\, \mathit{g} ,\, \mathit{a} ,\, \mathit{b} $ & \ottcom{expression variable} \\
 $ \mathit{integer} ,\, \mathit{k} $ & \ottcom{integer variable} \\
 $ \mathit{float64} ,\, \mathit{flt} $ & \ottcom{64-bit floating-point variable} \\
}}

\newcommand{\ottterminals}{
\ottrulehead{\ottnt{terminals}}{::=}{}\ottprodnewline
\ottfirstprodline{|}{ \lambda }{}{}{}{}\ottprodnewline
\ottprodline{|}{ \otimes }{}{}{}{}\ottprodnewline
\ottprodline{|}{ \multimap }{}{}{}{}\ottprodnewline
\ottprodline{|}{ \vdash }{}{}{}{}\ottprodnewline
\ottprodline{|}{ \in }{}{}{}{}\ottprodnewline
\ottprodline{|}{ \forall }{}{}{}{}\ottprodnewline
\ottprodline{|}{ \textsf{Cap} }{}{}{}{}\ottprodnewline
\ottprodline{|}{ \textsf{Type} }{}{}{}{}\ottprodnewline
\ottprodline{|}{ \: ! }{}{}{}{}}

\newcommand{\ottf}{
\ottrulehead{\ottnt{f}}{::=}{\ottcom{fractional capability}}\ottprodnewline
\ottfirstprodline{|}{\mathit{fc}}{}{}{}{\ottcom{variable}}\ottprodnewline
\ottprodline{|}{\ottkw{Zero}}{}{}{}{\ottcom{zero}}\ottprodnewline
\ottprodline{|}{\ottkw{Succ} \, \ottnt{f}}{}{}{}{\ottcom{successor}}}

\newcommand{\ottti}{
\ottrulehead{\ottnt{ti}}{::=}{\ottcom{non-linear type}}\ottprodnewline
\ottfirstprodline{|}{\ottkw{int}}{}{}{}{\ottcom{integer}}\ottprodnewline
\ottprodline{|}{\ottnt{f_{{\mathrm{64}}}}}{}{}{}{\ottcom{64-bit floats (doubles)}}\ottprodnewline
\ottprodline{|}{\ottkw{bool}}{}{}{}{\ottcom{booleans}}\ottprodnewline
\ottprodline{|}{\ottnt{t}  \multimap  \ottnt{t'}}{}{}{}{\ottcom{arrow (multiple-use)}}}

\newcommand{\ottt}{
\ottrulehead{\ottnt{t}}{::=}{\ottcom{linear type}}\ottprodnewline
\ottfirstprodline{|}{\ottsym{1}}{}{}{}{\ottcom{unit}}\ottprodnewline
\ottprodline{|}{\: !  \ottsym{(}  \ottnt{ti}  \ottsym{)}}{}{}{}{\ottcom{multiple-use type}}\ottprodnewline
\ottprodline{|}{\ottnt{t}  \otimes  \ottnt{t'}}{}{}{}{\ottcom{pair}}\ottprodnewline
\ottprodline{|}{\ottnt{t}  \multimap  \ottnt{t'}}{}{}{}{\ottcom{arrow (single-use)}}\ottprodnewline
\ottprodline{|}{\forall \, \mathit{fc}  \ottsym{.}  \ottnt{t}}{}{\textsf{bind}\; \mathit{fc}\; \textsf{in}\; \ottnt{t}}{}{\ottcom{frac. cap. abstraction}}\ottprodnewline
\ottprodline{|}{\ottkw{Arr} \, \ottsym{[}  \ottnt{f}  \ottsym{]}}{}{}{}{\ottcom{array}}\ottprodnewline
\ottprodline{|}{\ottnt{t}  \ottsym{\{}  \ottnt{f}  \ottsym{/}  \mathit{fc}  \ottsym{\}}} {\textsf{M}}{}{}{\ottcom{substitution}}}

\newcommand{\ottp}{
\ottrulehead{\ottnt{p}}{::=}{\ottcom{primitive}}\ottprodnewline
\ottfirstprodline{|}{\ottkw{set}}{}{}{}{\ottcom{array index assignment}}\ottprodnewline
\ottprodline{|}{\ottkw{get}}{}{}{}{\ottcom{array indexing}}\ottprodnewline
\ottprodline{|}{\ottkw{add}}{}{}{}{\ottcom{64-bit float addition}}\ottprodnewline
\ottprodline{|}{\ottkw{sub}}{}{}{}{\ottcom{64-bit float subtraction}}\ottprodnewline
\ottprodline{|}{\ottkw{mul}}{}{}{}{\ottcom{64-bit float multiplication}}\ottprodnewline
\ottprodline{|}{\ottkw{div}}{}{}{}{\ottcom{64-bit float division}}\ottprodnewline
\ottprodline{|}{\ottkw{eq}}{}{}{}{\ottcom{64-bit float equality}}\ottprodnewline
\ottprodline{|}{\ottkw{lt}}{}{}{}{\ottcom{64-bit float less-than}}\ottprodnewline
\ottprodline{|}{\ottkw{iadd}}{}{}{}{\ottcom{integer addition}}\ottprodnewline
\ottprodline{|}{\ottkw{isub}}{}{}{}{\ottcom{integer subtraction}}\ottprodnewline
\ottprodline{|}{\ottkw{imul}}{}{}{}{\ottcom{integer multiplication}}\ottprodnewline
\ottprodline{|}{\ottkw{idiv}}{}{}{}{\ottcom{integer division}}\ottprodnewline
\ottprodline{|}{\ottkw{ieq}}{}{}{}{\ottcom{integer equality}}\ottprodnewline
\ottprodline{|}{\ottkw{ilt}}{}{}{}{\ottcom{integer comparsion (less-than)}}\ottprodnewline
\ottprodline{|}{\ottkw{and}}{}{}{}{\ottcom{boolean conjuction}}\ottprodnewline
\ottprodline{|}{\ottkw{or}}{}{}{}{\ottcom{boolean disjunction}}\ottprodnewline
\ottprodline{|}{\ottkw{not}}{}{}{}{\ottcom{boolean negation}}\ottprodnewline
\ottprodline{|}{ \textbf{split\_perm} }{}{}{}{\ottcom{share array}}\ottprodnewline
\ottprodline{|}{ \textbf{merge\_perm} }{}{}{}{\ottcom{unshare array}}\ottprodnewline
\ottprodline{|}{\ottkw{free}}{}{}{}{\ottcom{free arrary}}\ottprodnewline
\ottprodline{|}{\ottkw{copy}}{}{}{}{\ottcom{copy array}}\ottprodnewline
\ottprodline{|}{\ottkw{swap}}{}{}{}{\ottcom{swap array}}\ottprodnewline
\ottprodline{|}{\ottkw{asum}}{}{}{}{\ottcom{$\sum_i | x_i |$}}\ottprodnewline
\ottprodline{|}{\ottkw{axpy}}{}{}{}{\ottcom{$x := \alpha x + y$}}\ottprodnewline
\ottprodline{|}{\ottkw{dot}}{}{}{}{\ottcom{$x \cdot y$}}\ottprodnewline
\ottprodline{|}{ \textbf{nrm2} }{}{}{}{\ottcom{$\| x \| ^ 2$}}\ottprodnewline
\ottprodline{|}{\ottkw{rot}}{}{}{}{\ottcom{plane rotation}}\ottprodnewline
\ottprodline{|}{\ottkw{rotg}}{}{}{}{\ottcom{Givens rotation}}\ottprodnewline
\ottprodline{|}{\ottkw{rotm}}{}{}{}{\ottcom{modified givens rotation}}\ottprodnewline
\ottprodline{|}{\ottkw{rotmg}}{}{}{}{\ottcom{generate modified Givens rotation}}\ottprodnewline
\ottprodline{|}{\ottkw{scal}}{}{}{}{\ottcom{$x := \alpha x$}}\ottprodnewline
\ottprodline{|}{\ottkw{amax}}{}{}{}{\ottcom{index of maximum absolute value}}}

\newcommand{\otte}{
\ottrulehead{\ottnt{e}}{::=}{\ottcom{expression}}\ottprodnewline
\ottfirstprodline{|}{\mathit{x}}{}{}{}{\ottcom{variable}}\ottprodnewline
\ottprodline{|}{\mathit{k}}{}{}{}{\ottcom{integer}}\ottprodnewline
\ottprodline{|}{\mathit{flt}}{}{}{}{\ottcom{64-bit floating-point}}\ottprodnewline
\ottprodline{|}{\ottsym{(}  \ottsym{)}}{}{}{}{\ottcom{unit introduction}}\ottprodnewline
\ottprodline{|}{\ottkw{let} \, \ottsym{(}  \ottsym{)}  \ottsym{=}  \ottnt{e}  \ottsym{;}  \ottnt{e'}}{}{}{}{\ottcom{unit elimination}}\ottprodnewline
\ottprodline{|}{\ottkw{true}}{}{}{}{\ottcom{true (boolean introduction)}}\ottprodnewline
\ottprodline{|}{\ottkw{false}}{}{}{}{\ottcom{false (boolean introduction)}}\ottprodnewline
\ottprodline{|}{\ottkw{if} \, \ottnt{e} \, \ottkw{then} \, \ottnt{e_{{\mathrm{1}}}} \, \ottkw{else} \, \ottnt{e_{{\mathrm{2}}}}}{}{}{}{\ottcom{if (boolean elimination)}}\ottprodnewline
\ottprodline{|}{\ottsym{(}  \ottnt{e}  \ottsym{,}  \ottnt{e'}  \ottsym{)}}{}{}{}{\ottcom{pair introduction}}\ottprodnewline
\ottprodline{|}{\ottkw{let} \, \ottsym{(}  \mathit{a}  \ottsym{,}  \mathit{b}  \ottsym{)}  \ottsym{=}  \ottnt{e}  \ottsym{;}  \ottnt{e'}}{}{\textsf{bind}\; \mathit{a} \cup  \mathit{b}\; \textsf{in}\; \ottnt{e'}}{}{\ottcom{pair elimination}}\ottprodnewline
\ottprodline{|}{\lambda  \mathit{x}  \ottsym{:}  \ottnt{t}  \ottsym{.}  \ottnt{e}}{}{\textsf{bind}\; \mathit{x}\; \textsf{in}\; \ottnt{e}}{}{\ottcom{abstraction}}\ottprodnewline
\ottprodline{|}{\ottkw{fix} \, \mathit{g}  \ottsym{:}  \: !  \ottsym{(}  \ottnt{t}  \multimap  \ottnt{t'}  \ottsym{)}  \ottsym{=}  \ottnt{e}}{}{\textsf{bind}\; \mathit{g}\; \textsf{in}\; \ottnt{e}}{}{\ottcom{fixpoint operator}}\ottprodnewline
\ottprodline{|}{\ottnt{e} \, \ottnt{e'}}{}{}{}{\ottcom{application}}\ottprodnewline
\ottprodline{|}{\ottkw{Array} \, \ottnt{e}}{}{}{}{\ottcom{array introduction}}\ottprodnewline
\ottprodline{|}{\ottkw{let} \, \mathit{x}  \ottsym{=}  \ottnt{e}  \ottsym{;}  \ottnt{e'}}{}{\textsf{bind}\; \mathit{x}\; \textsf{in}\; \ottnt{e'}}{}{\ottcom{array elimination}}\ottprodnewline
\ottprodline{|}{\ottnt{p}}{}{}{}{\ottcom{Level 1 BLAS routine primitives}}\ottprodnewline
\ottprodline{|}{\forall \, \mathit{fc}  \ottsym{.}  \ottnt{e}}{}{}{}{\ottcom{frac. cap. abstraction}}\ottprodnewline
\ottprodline{|}{\ottnt{e}  \ottsym{[}  \ottnt{f}  \ottsym{]}}{}{}{}{\ottcom{frac. cap. specialisation}}}

\newcommand{\ottT}{
\ottrulehead{\Theta}{::=}{\ottcom{fractional capability environment}}\ottprodnewline
\ottfirstprodline{|}{ \cdot }{}{}{}{}\ottprodnewline
\ottprodline{|}{\Theta  \ottsym{,}  \mathit{fc}}{}{}{}{}}

\newcommand{\ottG}{
\ottrulehead{\Gamma}{::=}{\ottcom{linear types environment}}\ottprodnewline
\ottfirstprodline{|}{ \cdot }{}{}{}{}\ottprodnewline
\ottprodline{|}{\Gamma  \ottsym{,}  \mathit{x}  \ottsym{:}  \ottnt{t}}{}{}{}{}\ottprodnewline
\ottprodline{|}{\Gamma  \ottsym{,}  \Gamma'}{}{}{}{}}

\newcommand{\ottD}{
\ottrulehead{\Delta}{::=}{\ottcom{linear types environment}}\ottprodnewline
\ottfirstprodline{|}{ \cdot }{}{}{}{}\ottprodnewline
\ottprodline{|}{\Delta  \ottsym{,}  \mathit{x}  \ottsym{:}  \ottnt{ti}}{}{}{}{}}

\newcommand{\ottformula}{
\ottrulehead{\ottnt{formula}}{::=}{}\ottprodnewline
\ottfirstprodline{|}{\ottnt{judgement}}{}{}{}{}\ottprodnewline
\ottprodline{|}{\mathit{x}  \ottsym{:}  \ottnt{ti} \, \in \, \Delta}{}{}{}{}\ottprodnewline
\ottprodline{|}{\mathit{x}  \ottsym{:}  \ottnt{t} \, \in \, \Gamma}{}{}{}{}\ottprodnewline
\ottprodline{|}{\mathit{fc} \, \in \, \Theta}{}{}{}{}}

\newcommand{\ottWellXXFormed}{
\ottrulehead{\ottnt{Well\_Formed}}{::=}{}\ottprodnewline
\ottfirstprodline{|}{\Theta  \vdash  \ottnt{f} \, \textsf{Cap}}{}{}{}{\ottcom{Valid fractional capabilities}}\ottprodnewline
\ottprodline{|}{\Theta  \vdash  \ottnt{t} \, \textsf{Type}}{}{}{}{\ottcom{Valid types}}}

\newcommand{\ottTypes}{
\ottrulehead{\ottnt{Types}}{::=}{}\ottprodnewline
\ottfirstprodline{|}{\Theta  \ottsym{;}  \Delta  \ottsym{;}  \Gamma  \vdash  \ottnt{e}  \ottsym{:}  \ottnt{t}}{}{}{}{\ottcom{Tying rules for expressions (no primitives yet)}}}

\newcommand{\ottjudgement}{
\ottrulehead{\ottnt{judgement}}{::=}{}\ottprodnewline
\ottfirstprodline{|}{\ottnt{Well\_Formed}}{}{}{}{}\ottprodnewline
\ottprodline{|}{\ottnt{Types}}{}{}{}{}}

\newcommand{\ottuserXXsyntax}{
\ottrulehead{\ottnt{user\_syntax}}{::=}{}\ottprodnewline
\ottfirstprodline{|}{\mathit{fraccap}}{}{}{}{}\ottprodnewline
\ottprodline{|}{\mathit{expression}}{}{}{}{}\ottprodnewline
\ottprodline{|}{\mathit{integer}}{}{}{}{}\ottprodnewline
\ottprodline{|}{\mathit{float64}}{}{}{}{}\ottprodnewline
\ottprodline{|}{\ottnt{terminals}}{}{}{}{}\ottprodnewline
\ottprodline{|}{\ottnt{f}}{}{}{}{}\ottprodnewline
\ottprodline{|}{\ottnt{ti}}{}{}{}{}\ottprodnewline
\ottprodline{|}{\ottnt{t}}{}{}{}{}\ottprodnewline
\ottprodline{|}{\ottnt{p}}{}{}{}{}\ottprodnewline
\ottprodline{|}{\ottnt{e}}{}{}{}{}\ottprodnewline
\ottprodline{|}{\Theta}{}{}{}{}\ottprodnewline
\ottprodline{|}{\Gamma}{}{}{}{}\ottprodnewline
\ottprodline{|}{\Delta}{}{}{}{}\ottprodnewline
\ottprodline{|}{\ottnt{formula}}{}{}{}{}}

\newcommand{\ottgrammar}{\ottgrammartabular{
\ottterminals\ottinterrule
\ottf\ottinterrule
\ottti\ottinterrule
\ottt\ottinterrule
\ottp\ottinterrule
\otte\ottinterrule
\ottT\ottinterrule
\ottG\ottinterrule
\ottD\ottinterrule
\ottformula\ottinterrule
\ottWellXXFormed\ottinterrule
\ottTypes\ottinterrule
\ottjudgement\ottinterrule
\ottuserXXsyntax\ottafterlastrule
}}

% defnss
% defns Well_Formed
%% defn Cap_
\newcommand{\ottdruleWFXXCapXXVar}[1]{\ottdrule[#1]{%
\ottpremise{\mathit{fc} \, \in \, \Theta}%
}{
\Theta  \vdash  \mathit{fc} \, \textsf{Cap}}{%
{\ottdrulename{WF\_Cap\_Var}}{}%
}}


\newcommand{\ottdruleWFXXCapXXZero}[1]{\ottdrule[#1]{%
}{
\Theta  \vdash  \ottkw{Zero} \, \textsf{Cap}}{%
{\ottdrulename{WF\_Cap\_Zero}}{}%
}}


\newcommand{\ottdruleWFXXCapXXSucc}[1]{\ottdrule[#1]{%
\ottpremise{\Theta  \vdash  \ottnt{f} \, \textsf{Cap}}%
}{
\Theta  \vdash  \ottkw{Succ} \, \ottnt{f} \, \textsf{Cap}}{%
{\ottdrulename{WF\_Cap\_Succ}}{}%
}}

\newcommand{\ottdefnWFXXCapXX}[1]{\begin{ottdefnblock}[#1]{$\Theta  \vdash  \ottnt{f} \, \textsf{Cap}$}{\ottcom{Valid fractional capabilities}}
\ottusedrule{\ottdruleWFXXCapXXVar{}}
\ottusedrule{\ottdruleWFXXCapXXZero{}}
\ottusedrule{\ottdruleWFXXCapXXSucc{}}
\end{ottdefnblock}}

%% defn Type_
\newcommand{\ottdruleWFXXTypeXXUnit}[1]{\ottdrule[#1]{%
}{
\Theta  \vdash  \ottsym{1} \, \textsf{Type}}{%
{\ottdrulename{WF\_Type\_Unit}}{}%
}}


\newcommand{\ottdruleWFXXTypeXXInt}[1]{\ottdrule[#1]{%
}{
\Theta  \vdash  \: !  \ottsym{(}  \ottkw{int}  \ottsym{)} \, \textsf{Type}}{%
{\ottdrulename{WF\_Type\_Int}}{}%
}}


\newcommand{\ottdruleWFXXTypeXXFloatSixFour}[1]{\ottdrule[#1]{%
}{
\Theta  \vdash  \: !  \ottsym{(}  \ottnt{f_{{\mathrm{64}}}}  \ottsym{)} \, \textsf{Type}}{%
{\ottdrulename{WF\_Type\_Float64}}{}%
}}


\newcommand{\ottdruleWFXXTypeXXBool}[1]{\ottdrule[#1]{%
}{
\Theta  \vdash  \: !  \ottsym{(}  \ottkw{bool}  \ottsym{)} \, \textsf{Type}}{%
{\ottdrulename{WF\_Type\_Bool}}{}%
}}


\newcommand{\ottdruleWFXXTypeXXPair}[1]{\ottdrule[#1]{%
\ottpremise{\Theta  \vdash  \ottnt{t} \, \textsf{Type}}%
\ottpremise{\Theta  \vdash  \ottnt{t'} \, \textsf{Type}}%
}{
\Theta  \vdash  \ottnt{t}  \otimes  \ottnt{t'} \, \textsf{Type}}{%
{\ottdrulename{WF\_Type\_Pair}}{}%
}}


\newcommand{\ottdruleWFXXTypeXXLollipop}[1]{\ottdrule[#1]{%
\ottpremise{\Theta  \vdash  \ottnt{t} \, \textsf{Type}}%
\ottpremise{\Theta  \vdash  \ottnt{t'} \, \textsf{Type}}%
}{
\Theta  \vdash  \ottnt{t}  \multimap  \ottnt{t'} \, \textsf{Type}}{%
{\ottdrulename{WF\_Type\_Lollipop}}{}%
}}


\newcommand{\ottdruleWFXXTypeXXFixt}[1]{\ottdrule[#1]{%
\ottpremise{\Theta  \vdash  \ottnt{t}  \multimap  \ottnt{t'} \, \textsf{Type}}%
}{
\Theta  \vdash  \: !  \ottsym{(}  \ottnt{t}  \multimap  \ottnt{t'}  \ottsym{)} \, \textsf{Type}}{%
{\ottdrulename{WF\_Type\_Fixt}}{}%
}}


\newcommand{\ottdruleWFXXTypeXXArray}[1]{\ottdrule[#1]{%
\ottpremise{\Theta  \vdash  \ottnt{f} \, \textsf{Cap}}%
}{
\Theta  \vdash  \ottkw{Arr} \, \ottsym{[}  \ottnt{f}  \ottsym{]} \, \textsf{Type}}{%
{\ottdrulename{WF\_Type\_Array}}{}%
}}


\newcommand{\ottdruleWFXXTypeXXForAll}[1]{\ottdrule[#1]{%
\ottpremise{\Theta  \ottsym{,}  \mathit{fc}  \vdash  \ottnt{t} \, \textsf{Type}}%
}{
\Theta  \vdash  \forall \, \mathit{fc}  \ottsym{.}  \ottnt{t} \, \textsf{Type}}{%
{\ottdrulename{WF\_Type\_ForAll}}{}%
}}

\newcommand{\ottdefnWFXXTypeXX}[1]{\begin{ottdefnblock}[#1]{$\Theta  \vdash  \ottnt{t} \, \textsf{Type}$}{\ottcom{Valid types}}
\ottusedrule{\ottdruleWFXXTypeXXUnit{}}
\ottusedrule{\ottdruleWFXXTypeXXInt{}}
\ottusedrule{\ottdruleWFXXTypeXXFloatSixFour{}}
\ottusedrule{\ottdruleWFXXTypeXXBool{}}
\ottusedrule{\ottdruleWFXXTypeXXPair{}}
\ottusedrule{\ottdruleWFXXTypeXXLollipop{}}
\ottusedrule{\ottdruleWFXXTypeXXFixt{}}
\ottusedrule{\ottdruleWFXXTypeXXArray{}}
\ottusedrule{\ottdruleWFXXTypeXXForAll{}}
\end{ottdefnblock}}


\newcommand{\ottdefnsWellXXFormed}{
\ottdefnWFXXCapXX{}\ottdefnWFXXTypeXX{}}

% defns Types
%% defn Type
\newcommand{\ottdruleTyXXVar}[1]{\ottdrule[#1]{%
}{
\Theta  \ottsym{;}  \Delta  \ottsym{;}   \cdot   \ottsym{,}  \mathit{x}  \ottsym{:}  \ottnt{t}  \vdash  \mathit{x}  \ottsym{:}  \ottnt{t}}{%
{\ottdrulename{Ty\_Var}}{}%
}}


\newcommand{\ottdruleTyXXVarXXBang}[1]{\ottdrule[#1]{%
\ottpremise{\mathit{x}  \ottsym{:}  \ottnt{ti} \, \in \, \Delta}%
}{
\Theta  \ottsym{;}  \Delta  \ottsym{;}  \Gamma  \vdash  \mathit{x}  \ottsym{:}  \: !  \ottsym{(}  \ottnt{ti}  \ottsym{)}}{%
{\ottdrulename{Ty\_Var\_Bang}}{}%
}}


\newcommand{\ottdruleTyXXIntXXIntro}[1]{\ottdrule[#1]{%
}{
\Theta  \ottsym{;}  \Delta  \ottsym{;}  \Gamma  \vdash  \mathit{k}  \ottsym{:}  \: !  \ottsym{(}  \ottkw{int}  \ottsym{)}}{%
{\ottdrulename{Ty\_Int\_Intro}}{}%
}}


\newcommand{\ottdruleTyXXFloatSixFourXXIntro}[1]{\ottdrule[#1]{%
}{
\Theta  \ottsym{;}  \Delta  \ottsym{;}  \Gamma  \vdash  \mathit{flt}  \ottsym{:}  \: !  \ottsym{(}  \ottnt{f_{{\mathrm{64}}}}  \ottsym{)}}{%
{\ottdrulename{Ty\_Float64\_Intro}}{}%
}}


\newcommand{\ottdruleTyXXUnitXXIntro}[1]{\ottdrule[#1]{%
}{
\Theta  \ottsym{;}  \Delta  \ottsym{;}   \cdot   \vdash  \ottsym{(}  \ottsym{)}  \ottsym{:}  \ottsym{1}}{%
{\ottdrulename{Ty\_Unit\_Intro}}{}%
}}


\newcommand{\ottdruleTyXXUnitXXElim}[1]{\ottdrule[#1]{%
\ottpremise{\Theta  \ottsym{;}  \Delta  \ottsym{;}  \Gamma  \vdash  \ottnt{e}  \ottsym{:}  \ottsym{1}}%
\ottpremise{\Theta  \ottsym{;}  \Delta  \ottsym{;}  \Gamma'  \vdash  \ottnt{e'}  \ottsym{:}  \ottnt{t}}%
}{
\Theta  \ottsym{;}  \Delta  \ottsym{;}  \Gamma  \ottsym{,}  \Gamma'  \vdash  \ottkw{let} \, \ottsym{(}  \ottsym{)}  \ottsym{=}  \ottnt{e}  \ottsym{;}  \ottnt{e'}  \ottsym{:}  \ottnt{t}}{%
{\ottdrulename{Ty\_Unit\_Elim}}{}%
}}


\newcommand{\ottdruleTyXXBoolXXTrue}[1]{\ottdrule[#1]{%
}{
\Theta  \ottsym{;}  \Delta  \ottsym{;}  \Gamma  \vdash  \ottkw{true}  \ottsym{:}  \: !  \ottsym{(}  \ottkw{bool}  \ottsym{)}}{%
{\ottdrulename{Ty\_Bool\_True}}{}%
}}


\newcommand{\ottdruleTyXXBoolXXFalse}[1]{\ottdrule[#1]{%
}{
\Theta  \ottsym{;}  \Delta  \ottsym{;}  \Gamma  \vdash  \ottkw{false}  \ottsym{:}  \: !  \ottsym{(}  \ottkw{bool}  \ottsym{)}}{%
{\ottdrulename{Ty\_Bool\_False}}{}%
}}


\newcommand{\ottdruleTyXXBoolXXElim}[1]{\ottdrule[#1]{%
\ottpremise{\Theta  \ottsym{;}  \Delta  \ottsym{;}  \Gamma  \vdash  \ottnt{e}  \ottsym{:}  \: !  \ottsym{(}  \ottkw{bool}  \ottsym{)}}%
\ottpremise{\Theta  \ottsym{;}  \Delta  \ottsym{;}  \Gamma_{{\mathrm{1}}}  \vdash  \ottnt{e_{{\mathrm{1}}}}  \ottsym{:}  \ottnt{t}}%
\ottpremise{\Theta  \ottsym{;}  \Delta  \ottsym{;}  \Gamma_{{\mathrm{2}}}  \vdash  \ottnt{e_{{\mathrm{2}}}}  \ottsym{:}  \ottnt{t}}%
}{
\Theta  \ottsym{;}  \Delta  \ottsym{;}  \Gamma  \ottsym{,}  \Gamma_{{\mathrm{1}}}  \ottsym{,}  \Gamma_{{\mathrm{2}}}  \vdash  \ottkw{if} \, \ottnt{e} \, \ottkw{then} \, \ottnt{e_{{\mathrm{1}}}} \, \ottkw{else} \, \ottnt{e_{{\mathrm{2}}}}  \ottsym{:}  \ottnt{t}}{%
{\ottdrulename{Ty\_Bool\_Elim}}{}%
}}


\newcommand{\ottdruleTyXXPairXXIntro}[1]{\ottdrule[#1]{%
\ottpremise{\Theta  \ottsym{;}  \Delta  \ottsym{;}  \Gamma  \vdash  \ottnt{e}  \ottsym{:}  \ottnt{t}}%
\ottpremise{\Theta  \ottsym{;}  \Delta  \ottsym{;}  \Gamma'  \vdash  \ottnt{e'}  \ottsym{:}  \ottnt{t'}}%
}{
\Theta  \ottsym{;}  \Delta  \ottsym{;}  \Gamma  \ottsym{,}  \Gamma'  \vdash  \ottsym{(}  \ottnt{e}  \ottsym{,}  \ottnt{e'}  \ottsym{)}  \ottsym{:}  \ottnt{t}  \otimes  \ottnt{t'}}{%
{\ottdrulename{Ty\_Pair\_Intro}}{}%
}}


\newcommand{\ottdruleTyXXPairXXElim}[1]{\ottdrule[#1]{%
\ottpremise{\Theta  \ottsym{;}  \Delta  \ottsym{;}  \Gamma  \vdash  \ottnt{e_{{\mathrm{12}}}}  \ottsym{:}  \ottnt{t_{{\mathrm{1}}}}  \otimes  \ottnt{t_{{\mathrm{2}}}}}%
\ottpremise{\Theta  \ottsym{;}  \Delta  \ottsym{;}  \Gamma'  \ottsym{,}  \mathit{a}  \ottsym{:}  \ottnt{t_{{\mathrm{1}}}}  \ottsym{,}  \mathit{b}  \ottsym{:}  \ottnt{t_{{\mathrm{2}}}}  \vdash  \ottnt{e}  \ottsym{:}  \ottnt{t}}%
}{
\Theta  \ottsym{;}  \Delta  \ottsym{;}  \Gamma  \ottsym{,}  \Gamma'  \vdash  \ottkw{let} \, \ottsym{(}  \mathit{a}  \ottsym{,}  \mathit{b}  \ottsym{)}  \ottsym{=}  \ottnt{e_{{\mathrm{12}}}}  \ottsym{;}  \ottnt{e}  \ottsym{:}  \ottnt{t}}{%
{\ottdrulename{Ty\_Pair\_Elim}}{}%
}}


\newcommand{\ottdruleTyXXLambdaXXBang}[1]{\ottdrule[#1]{%
\ottpremise{\Theta  \vdash  \: !  \ottsym{(}  \ottnt{ti}  \ottsym{)} \, \textsf{Type}}%
\ottpremise{\Theta  \ottsym{;}  \Delta  \ottsym{,}  \mathit{x}  \ottsym{:}  \ottnt{ti}  \ottsym{;}  \Gamma  \vdash  \ottnt{e}  \ottsym{:}  \ottnt{t}}%
}{
\Theta  \ottsym{;}  \Delta  \ottsym{;}  \Gamma  \vdash  \lambda  \mathit{x}  \ottsym{:}  \: !  \ottsym{(}  \ottnt{ti}  \ottsym{)}  \ottsym{.}  \ottnt{e}  \ottsym{:}  \: !  \ottsym{(}  \ottnt{ti}  \ottsym{)}  \multimap  \ottnt{t}}{%
{\ottdrulename{Ty\_Lambda\_Bang}}{}%
}}


\newcommand{\ottdruleTyXXLambda}[1]{\ottdrule[#1]{%
\ottpremise{\Theta  \vdash  \ottnt{t'} \, \textsf{Type}}%
\ottpremise{\Theta  \ottsym{;}  \Delta  \ottsym{;}  \Gamma  \ottsym{,}  \mathit{x}  \ottsym{:}  \ottnt{t'}  \vdash  \ottnt{e}  \ottsym{:}  \ottnt{t}}%
}{
\Theta  \ottsym{;}  \Delta  \ottsym{;}  \Gamma  \vdash  \lambda  \mathit{x}  \ottsym{:}  \ottnt{t'}  \ottsym{.}  \ottnt{e}  \ottsym{:}  \ottnt{t'}  \multimap  \ottnt{t}}{%
{\ottdrulename{Ty\_Lambda}}{}%
}}


\newcommand{\ottdruleTyXXApp}[1]{\ottdrule[#1]{%
\ottpremise{\Theta  \ottsym{;}  \Delta  \ottsym{;}  \Gamma  \vdash  \ottnt{e}  \ottsym{:}  \ottnt{t'}  \multimap  \ottnt{t}}%
\ottpremise{\Theta  \ottsym{;}  \Delta  \ottsym{;}  \Gamma'  \vdash  \ottnt{e'}  \ottsym{:}  \ottnt{t'}}%
}{
\Theta  \ottsym{;}  \Delta  \ottsym{;}  \Gamma  \ottsym{,}  \Gamma'  \vdash  \ottnt{e} \, \ottnt{e'}  \ottsym{:}  \ottnt{t}}{%
{\ottdrulename{Ty\_App}}{}%
}}


\newcommand{\ottdruleTyXXArrayXXIntro}[1]{\ottdrule[#1]{%
\ottpremise{\Theta  \ottsym{;}  \Delta  \ottsym{;}  \Gamma  \vdash  \ottnt{e}  \ottsym{:}  \: !  \ottsym{(}  \ottkw{int}  \ottsym{)}}%
}{
\Theta  \ottsym{;}  \Delta  \ottsym{;}  \Gamma  \vdash  \ottkw{Array} \, \ottnt{e}  \ottsym{:}  \ottkw{Arr} \, \ottsym{[}  \ottkw{Zero}  \ottsym{]}}{%
{\ottdrulename{Ty\_Array\_Intro}}{}%
}}


\newcommand{\ottdruleTyXXArrayXXElim}[1]{\ottdrule[#1]{%
\ottpremise{\Theta  \ottsym{;}  \Delta  \ottsym{;}  \Gamma  \vdash  \ottnt{e}  \ottsym{:}  \ottkw{Arr} \, \ottsym{[}  \ottnt{f}  \ottsym{]}}%
\ottpremise{\Theta  \ottsym{;}  \Delta  \ottsym{;}  \Gamma'  \ottsym{,}  \mathit{x}  \ottsym{:}  \ottkw{Arr} \, \ottsym{[}  \ottnt{f}  \ottsym{]}  \vdash  \ottnt{e'}  \ottsym{:}  \ottnt{t'}}%
}{
\Theta  \ottsym{;}  \Delta  \ottsym{;}  \Gamma  \ottsym{,}  \Gamma'  \vdash  \ottkw{let} \, \mathit{x}  \ottsym{=}  \ottnt{e}  \ottsym{;}  \ottnt{e'}  \ottsym{:}  \ottnt{t'}}{%
{\ottdrulename{Ty\_Array\_Elim}}{}%
}}


\newcommand{\ottdruleTyXXForAllXXfracXXcap}[1]{\ottdrule[#1]{%
\ottpremise{\Theta  \ottsym{,}  \mathit{fc}  \ottsym{;}  \Delta  \ottsym{;}  \Gamma  \vdash  \ottnt{e}  \ottsym{:}  \ottnt{t}}%
}{
\Theta  \ottsym{;}  \Delta  \ottsym{;}  \Gamma  \vdash  \forall \, \mathit{fc}  \ottsym{.}  \ottnt{e}  \ottsym{:}  \forall \, \mathit{fc}  \ottsym{.}  \ottnt{t}}{%
{\ottdrulename{Ty\_ForAll\_frac\_cap}}{}%
}}


\newcommand{\ottdruleTyXXSpecXXfracXXcap}[1]{\ottdrule[#1]{%
\ottpremise{\Theta  \vdash  \ottnt{f} \, \textsf{Cap}}%
\ottpremise{\Theta  \ottsym{;}  \Delta  \ottsym{;}  \Gamma  \vdash  \ottnt{e}  \ottsym{:}  \forall \, \mathit{fc}  \ottsym{.}  \ottnt{t}}%
}{
\Theta  \ottsym{;}  \Delta  \ottsym{;}  \Gamma  \vdash  \ottnt{e}  \ottsym{[}  \ottnt{f}  \ottsym{]}  \ottsym{:}  \ottnt{t}  \ottsym{\{}  \ottnt{f}  \ottsym{/}  \mathit{fc}  \ottsym{\}}}{%
{\ottdrulename{Ty\_Spec\_frac\_cap}}{}%
}}


\newcommand{\ottdruleTyXXFix}[1]{\ottdrule[#1]{%
\ottpremise{\Theta  \vdash  \ottnt{t}  \multimap  \ottnt{t'} \, \textsf{Type}}%
\ottpremise{\Theta  \ottsym{;}  \Delta  \ottsym{,}  \mathit{g}  \ottsym{:}  \ottnt{t_{{\mathrm{1}}}}  \multimap  \ottnt{t_{{\mathrm{2}}}}  \ottsym{;}   \cdot   \vdash  \ottnt{e}  \ottsym{:}  \ottnt{t_{{\mathrm{1}}}}  \multimap  \ottnt{t_{{\mathrm{2}}}}}%
}{
\Theta  \ottsym{;}  \Delta  \ottsym{;}  \Gamma  \vdash  \ottkw{fix} \, \mathit{g}  \ottsym{:}  \: !  \ottsym{(}  \ottnt{t}  \multimap  \ottnt{t'}  \ottsym{)}  \ottsym{=}  \ottnt{e}  \ottsym{:}  \: !  \ottsym{(}  \ottnt{t}  \multimap  \ottnt{t'}  \ottsym{)}}{%
{\ottdrulename{Ty\_Fix}}{}%
}}

\newcommand{\ottdefnTyXXType}[1]{\begin{ottdefnblock}[#1]{$\Theta  \ottsym{;}  \Delta  \ottsym{;}  \Gamma  \vdash  \ottnt{e}  \ottsym{:}  \ottnt{t}$}{\ottcom{Tying rules for expressions (no primitives yet)}}
\ottusedrule{\ottdruleTyXXVar{}}
\ottusedrule{\ottdruleTyXXVarXXBang{}}
\ottusedrule{\ottdruleTyXXIntXXIntro{}}
\ottusedrule{\ottdruleTyXXFloatSixFourXXIntro{}}
\ottusedrule{\ottdruleTyXXUnitXXIntro{}}
\ottusedrule{\ottdruleTyXXUnitXXElim{}}
\ottusedrule{\ottdruleTyXXBoolXXTrue{}}
\ottusedrule{\ottdruleTyXXBoolXXFalse{}}
\ottusedrule{\ottdruleTyXXBoolXXElim{}}
\ottusedrule{\ottdruleTyXXPairXXIntro{}}
\ottusedrule{\ottdruleTyXXPairXXElim{}}
\ottusedrule{\ottdruleTyXXLambdaXXBang{}}
\ottusedrule{\ottdruleTyXXLambda{}}
\ottusedrule{\ottdruleTyXXApp{}}
\ottusedrule{\ottdruleTyXXArrayXXIntro{}}
\ottusedrule{\ottdruleTyXXArrayXXElim{}}
\ottusedrule{\ottdruleTyXXForAllXXfracXXcap{}}
\ottusedrule{\ottdruleTyXXSpecXXfracXXcap{}}
\ottusedrule{\ottdruleTyXXFix{}}
\end{ottdefnblock}}


\newcommand{\ottdefnsTypes}{
\ottdefnTyXXType{}}

\newcommand{\ottdefnss}{
\ottdefnsWellXXFormed
\ottdefnsTypes
}

\newcommand{\ottall}{\ottmetavars\\[0pt]
\ottgrammar\\[5.0mm]
\ottdefnss}

%
\usepackage{ottlayout}%
\renewcommand{\ottpremise}[1]{\premiseSTY{#1}}%
\renewcommand{\ottusedrule}[1]{\usedruleSTY{#1}}%
\renewcommand{\ottdrule}[4][]{\druleSTY[#1]{#2}{#3}{#4}}%
\renewenvironment{ottdefnblock}[3][]{\defnblockSTY[#1]{#2}{#3}}{\enddefnblockSTY}%

% Proof macros
\newcommand{\den}[3]{ \mathcal{#1}_{#2} [\![ #3 ]\!] }%
\newcommand{\V}[2]{ \den{V}{#1}{#2} }%
\newcommand{\C}[2]{ \den{C}{#1}{#2} }%
 
\newcommand{\Unit}{\ottkw{unit}}%
\newcommand{\Bang}{\ottkw{!}}
\newcommand{\Bool}{\ottkw{bool}}%
\newcommand{\Int}{\ottkw{int}}%
\newcommand{\Elt}{\ottkw{elt}}%
\newcommand{\Mat}{\ottkw{mat}}%
\newcommand{\Zf}{\ottkw{Z}}%
\newcommand{\Sf}{\ottkw{S}}%
\newcommand{\Many}{\ottkw{Many}}%
\newcommand{\dom}{\mathrm{dom}}%
\newcommand{\empH}{\emptyset}%

\usepackage{pf2}
\beforePfSpace{15pt, 10pt, 10pt, 10pt, 5pt, 2pt}
\afterPfSpace{15pt, 10pt, 10pt, 10pt, 5pt, 2pt}
\interStepSpace{15pt, 10pt, 10pt, 10pt, 5pt, 2pt}
\pflongindent%

% Multi-line table cells
% tex.stackexchange.com/questions/2441/how-to-add-a-forced-line-break-inside-a-table-cell#19678
\newcommand{\specialcell}[3][c]{%
  \begin{array}[#1]{@{}#2@{}}#3\end{array}}

\newcommand{\alsocell}[3][c]{%
  \begin{tabular}[#1]{@{}#2@{}}#3\end{tabular}}

% Figures
\usepackage{graphicx}
\usepackage[dvipsnames]{xcolor}
\usepackage{lscape}
\usepackage{pgfplots}

% PL Stuff
\usepackage[nounderscore]{syntax}
\renewcommand{\syntleft}{\normalfont\itshape}
\renewcommand{\syntright}{}
\renewcommand{\ulitleft}{\normalfont\bf}%\syn@ttspace\frenchspacing}
\renewcommand{\ulitright}{}
\renewcommand{\litleft}{\bgroup\ulitleft}
\renewcommand{\litright}{\ulitright\egroup}

% NumLin
\newcommand{\lang}{\textsc{NumLin}}

\begin{document}


\begin{abstract}
    We present \lang, a functional programming language designed to express
    the APIs of low-level linear algebra libraries (such as BLAS/LAPACK) safely
    and explicitly, through a brief description of its key features and several
    illustrative examples. We show that \lang's type system is sound and that
    its implementation improves upon na{\"i}ve implementations of linear
    algebra programs, almost towards C-levels of performance. Lastly, we contrast
    it to other recent developments in linear types and show that using linear
    types and fractional permissions to express the APIs of low-level linear
    algebra libraries is a simple and effective idea.
\end{abstract}

\section{Introduction}

\lang\ is a functional programming language designed to express the APIs of
low-level linear algebra libraries (such as BLAS/LAPACK) safely and explicitly.
It does so by combining linear types, fractional permissions, runtime errors
and recursion into a small, easily understandable, yet expressive set of core
constructs. In addition to this, \lang's implementation supports several
syntactic conveniences as well as a usable integration with real-world OCaml
code.

\subsection{Contributions}
\begin{itemize}
    \item We have designed the \lang\ programming language
    \item We illustrate that the design is sensible with many matrix-y examples
    \item We give a soundness proof for \lang, using a step-indexed logical relation
    \item Incredibly simple type inference algorihtm for polymorphic fractional permissions
        \begin{itemize}
            \item Compare to Bierhof et al's \emph{Fraction Polymorphic Permission Inference},
                which uses a fancy dataflow analysis
            \item We use exactly the same unification algorithm type polymorphism does
        \end{itemize}
    \item We have an implementation - compatible with and usable from existing code!
\end{itemize}



\section{Language Overview and Examples}

\subsection{Short language description}

This should be a high-level description of the language. We should
aim to spend about a page on this section, highlighting the key
properties of \LANG{}:
\begin{itemize}
\item Linearity, obviously
\item Fractional permissions: sharing and unsharing
\item Dynamic errors: matrix dimensions and unsharing (matrix identity checks)
  \begin{itemize}
  \item In the related work, we should point to L3 statically tracking
    pointer identities, but we want to keep the number of type indices
    under control
  \end{itemize}
\end{itemize}

\subsection{Examples}

The more examples, the better. In fact, it's actually impossible to
have too many examples -- it's okay (indeed, desirable) to spend 5-6
pages on examples.

\begin{itemize}

    \item Simple: factorial, shows recursion, !-annotations etc.

    \item Less simple: summing over an array, indexing and safety

    \item Medium: one-dimensioal convolution, permissions

    \item Harder: linear regression, pattern-matching and apparent non-linearity

    \item Harder: L1 norm-minimisation, some frees, re-using memory

    \item Big finish: Kalman filter

\end{itemize}


\section{Formal System}\label{sec:formal_system}

\subsection{Core Type Theory}

The full typing rules are in Appendix \ref{subsec:static_sem}, but the key
ideas are as follow.

\begin{itemize}

    \item A typing judgement consists of $ \Theta; \Delta; \Gamma \vdash e : t$.

    \item $\Theta$ is the environment that tracks which fractional permission variables
        in scope. Fractional permissions (the \textsf{Perm} judgement) and types (the
        \textsf{Type} judgement) are \emph{well-formed} if all of their free fractional
        variables are in $\Theta$.

    \item $\Delta$ is the environment storing non-linearly or \emph{inuitionistically}
        typed variables.

    \item $\Gamma$ is the environment storing linearly typed variables. 

\end{itemize}

Note that rules for typing $()$, booleans, integers and elements are typed
with respect to an \emph{empty} linear environment: this means no linear
values are needed to produce a value of those types.

\[
    \ottdruleTyXXUnitXXIntro{}
\]

Conversely, whenever two or more subexpressions need to be typed, they must
consume a disjoint set of linear values (pairs, let-expressions).  In the case
of if-expressions, both branches must consume the same set of linear values
(disjoint to the ones used to evaluate the condition).

\[
    \ottdruleTyXXBoolXXElim{}
\]

The \highl{Many} introduction and elimination rules are very important.
Producing !-type values may only be done if the expression inside is a
syntactic value which is not a location. This allows all safely duplicable
resources, including functions which capture non-linear resources from their
environments, but prevents producing aliases of (pointers to) arrays and
matrices. This is exactly the same as value-restriction from the world of
parametric polymorphism.

\[
    \ottdruleTyXXBangXXIntro{}
\]

Consuming a !-type value \emph{moves it} from the linear environment $\Gamma$
and \emph{into} the intuitionistic environment $\Delta$. This is exactly why
$\mathbf{let}\ !x = e_1\ \mathbf{in}\ e_2$ desugars to $\mathbf{let\ Many}\ x =
e_1\ \mathbf{in\ } \mathbf{let\ Many}\ x = \mathbf{Many}\ (\mathbf{Many}\ x)\
\mathbf{in}\ e_2$.

\[
    \ottdruleTyXXBangXXElim{}
\]

Rules \textsc{Ty\_Gen} and \textsc{Ty\_Spc} are for fractional permission
generalisation and specialisation respectively. They allow the definition and
use of functions that are polymorphic in the fractional permission index of
their results and one or more of their arguments.

\[
    \ottdruleTyXXGen{} \qquad\qquad \ottdruleTyXXSpc{}
\]

Rule \textsc{Ty\_Fix} shows how recursive functions are typed. Even though
recursive functions are fully annotated, type checking them is interesting for
two reasons: to type check the body of the fixpoint, the type of the recusive
function is in the \emph{intuitionistic} environment $\Delta$ (without this,
you would not be able to write a base case) whilst the argument and its type
are the \emph{only things in the linear environment} $\Gamma$. The latter means
that recursive functions can be type checked in an empty environment (thus be
wrapped in \highl{Many} and used zero or multiple times).

\[
    \ottdruleTyXXFix{}
\]

Lastly, types of almost all \lang\ primitives, as embedded in OCaml's type
system, are shown in Appendix \ref{subsec:primitives}, with some similar ones
(like those for binary arithmetic operators) omitted for brevity. The
main difference between the OCaml type of a primitive like \highl{gemm} and its
\lang\ counterpart is the inclusion of explicit `$\forall$'s.  So,
\highl{float bang -> ('a mat * bool bang) -> ('b mat * bool bang) -> float
bang -> z mat -> ('a mat * 'b mat) * z mat}
will correspond to \\
$!\ottkw{elt} \multimap \forall\, x.\ x \ \ottkw{mat} \ \otimes \ !\ottkw{bool}
\multimap \forall\, y.\  y \ \ottkw{mat} \ \otimes \ !\ottkw{bool} \multimap
\ !\ottkw{elt} \multimap z\ \ottkw{mat} \multimap ( x \ \ottkw{mat} \ \otimes y
\ \ottkw{mat} ) \ \otimes z\ \ottkw{mat}$

\subsection{Dynamic Semantics}\label{subsec:semantics}

The full, small-step transition relation is in Appendix \ref{subsec:dyn_sem},
but the key ideas are as follow.

Heaps ($\sigma$) are multisets containing triples of an abstract location $l$,
a fractional permission $f$ and sized matrices $m_{n,k}$. The notation $l
\mapsto_f m_{k_1, k_2}$ should be read as ``location $l$ represents $f$
ownership over matrix $m$ (of size $k_1 \times k_2$)''.  Each heap-and-expression
either steps to another heap-and-expression or a runtime error $\mathbf{err}$.
In the full grammar definition we see a definition of values and contexts in
the language.

We draw the reader's attention to the definitions relating to fractional
permissions. Specifically, unlike a lambda, the body of a $\ottkw{fun}\, f\!c
\rightarrow \_$ must be a syntactic value. The context $\ottkw{fun}\, f\!c
\rightarrow [-]$ means expressions can be reduced inside a fractional
permission generalisation. This is to emphasize that fractions are merely
\emph{compile-time constructs} and do not affect runtime behaviour. Correct
usage of fractions is enforced by the type system, so programs do not get
stuck. Fractional permissions are specialised using substitution over both the
heap and an expression (\textsc{Op\_Frac\_Perm}).
\[
    \ottdruleOpXXFracXXPerm{}
\]

Like with the static semantics, the interesting rules in the dynamic semantics
are those relating to primitives. Creating a matrix ($\ottkw{matrix}\ k_1\
k_2$) successfully (\textsc{Op\_Matrix}) requires non-negative dimensions and
returns a (fresh) location of a matrix of those dimensions, extending the heap
to reflect that $l$ represents a complete ownership over the new matrix.
\[
    \ottdruleOpXXMatrix{}
\]

Dually, \textsc{Op\_Free}, requires a location represent complete ownership
before removing it and the matrix it points to from the heap.
\[
    \ottdruleOpXXFree{}
\]

Choosing a multiset representation as opposed to a set allows for two
convenient invariants: multiplicity of a triple $l \mapsto_f m_{k_1, k_2}$ in
the heap corresponds to the number of aliases of $l$ in the expression with
type $f\ \ottkw{mat}$ and the sum of all the fractions for $l$ will always be
$1$ (for a closed, well-typed expression). With this in mind, the rules
\textsc{Op\_Share} and \textsc{Op\_Unshare\_Eq} are fairly natural.
\[
    \ottdruleOpXXShare{} \\
\]
\[
    \ottdruleOpXXUnshareXXEq{}
\]

Combining all of these features, we see that \textsc{Op\_Gemm\_Match} requires
that the location being updated ($l_3$) has complete ownership of over matrix
$m_3$ and can thus change what value it stores to $m_1 m_2 + m_3$. In
particular, this places no restriction on $l_2$ and $l_3$: they could be
$\ottkw{share}$d aliases of the same matrix. Transition rules for other
primitives (omitted) follow the same structure: $\mapsto_1$ for any locations
that are written to and $\mapsto_{f\!c}$ for anything else.
\[
    \ottdruleOpXXGemmXXMatch{}
\]

\subsection{Logical Relation}

First, we define an interpretation of heaps with fractional permissions in the
style of Bornat et. al~\cite{bornat} (interpreting the multiset as a partial
map from locations to the sum of all its associated fractions and a matrix) as
well as the n-fold iteration of $\rightarrow$.

\[
    \den{H}{}{\sigma} = \bigstar_{(l,f,m) \in \sigma} [ l \mapsto_f m ]
\]
where
\[
    (\varsigma_1 \star \varsigma_2)(l) \equiv
    \begin{cases}
        \varsigma_1(l) & \textrm{if } l \in \dom(\varsigma_1) \wedge l \notin \dom(\varsigma_2) \\
        \varsigma_2(l) & \textrm{if } l \in \dom(\varsigma_2) \wedge l \notin \dom(\varsigma_1) \\
        (f_1 + f_2, m) & \textrm{if } (f_1, m) = \varsigma_1(l) \wedge (f_2, m) = \varsigma_2(l) \wedge f_1 + f_2 \leq 1 \\
        \textrm{undefined} & \textrm{otherwise}
    \end{cases}
\]

We then define a step-indexed logical relation in the style of Morrisett et.
al~\cite{morrisett}. $(\varsigma, v) \in \V{k}{t}$ means it takes a heap with
exactly $\varsigma$ resources to produce a value $v$ of type $t$ in at most $k$
steps. So, something like a $\ottkw{unit}$ or a $!t$ need no resources, whereas
a $f\, \ottkw{mat}$ needs exactly $f$ ownership of a some matrix and a pair
needs a $\star$ combination of the heaps required for each component.
\begin{align*}
  \V{k}{ \Unit } &= \{ (\empH, \ast) \} \\
  \V{k}{ f \, \Mat } &= \{ (\{ l \mapsto_{2^{-f}} \_ \} , l) \} \\
  \V{k}{ \Bang t } &= \{ (\empH, \Many\, v) \mid (\empH, v) \in \V{k}{t} \} \\
  \V{k}{ t_1 \otimes t_2 } &= \{ (\varsigma_1 \star \varsigma_2, ( v_1, v_2 )) \mid (\varsigma_1, v_1) \in \V{k}{t_1} \wedge (\varsigma_2, v_2) \in \V{k}{t_2} \}
\end{align*}

The definition of $\V{k}{\forall f\!c.\ t}$ says a value and heap
must be the same regardless of what fraction is substituted into both; the
$k-1$ is to take into account fraction specialisation takes ones step
(\textsc{Op\_Spc}).
\[
    \V{k}{ \forall f\!c.\  t } = \{ (\varsigma, \ottkw{fun}\, f\!c \rightarrow \, v) \mid \forall f.\ (\varsigma [ f\!c / f ], v [ f\!c / f ]) \in \V{k-1}{ t [ f\!c / f ] } \}
\]

To understand the definition of $\V{k}{t' \multimap t}$, we must first look at
$\C{k}{t}$, the computational interpretation of types. Intuitively, it is a
combination of a frame rule on heaps (no interference), type-preservation and
termination (in $j < k$ steps) to either an error or a heap-and-expression,
with the further condition that if the expression is a syntactic value then it
is also one semantically.
\begin{align*}
    \C{k}{ t } &= \{ (\varsigma_s, e_s) \mid \forall \, j < k, \sigma_r.\ \varsigma_s \star \varsigma_r \textrm{ defined } \Rightarrow \langle \sigma_s + \sigma_r, e_s \rangle \rightarrow^j \ottkw{err}\ \vee \exists \sigma_f, e_f.\\
               & \qquad \qquad \langle \sigma_s + \sigma_r, e_s \rangle \rightarrow^j \langle \sigma_f + \sigma_r, e_f \rangle \wedge ( e_f \textrm { is a value } \Rightarrow ( \varsigma_f \star \varsigma_r, e_f ) \in \V{k-j}{t} ) \}
\end{align*}

In this light, $\V{k}{t' \multimap t}$ simply says
that $v$ is a function and that evaluating the application of it to any
argument (of the correct type, requiring its own set of resources, bounded by
$k$ steps) satisfies all the aforementioned properties.
\begin{align*}
    \V{k}{ t' \multimap t } &= \{ (\varsigma_v, v ) \mid ( v \equiv \ottkw{fun}\, x : t' \rightarrow e \vee v \equiv \ottkw{fix} (g, x : t' , e : t) ) \, \wedge\\
                            & \qquad \qquad \forall j \leq k, (\varsigma_{v'}, v') \in \V{j}{ t' }.\ \varsigma_v \star \varsigma_v' \textrm{ defined } \Rightarrow (\varsigma_v \star \varsigma_v', v\, v') \in \C{j}{t} \}
\end{align*}

The interpretation of typing environments $\Delta$ and $\Gamma$ are with
respect to an arbitrary substitution of fractional permissions $\theta$. Note
that only the interpretation of $\Gamma$ involves a (potentially) non-empty heap.
\begin{align*}
    \den{I}{k}{ \Delta, x : t } \theta &= \{ \delta[x \mapsto v_x ] \mid \delta \in \den{I}{k}{\Delta}\theta \wedge (\empH, v_x) \in \V{k}{\theta(t)} \} \\
    \den{L}{k}{ \Gamma, x : t } \theta &= \{ (\varsigma \star \varsigma_x, \gamma[x \mapsto v_x ]) \mid (\varsigma, \gamma) \in \den{L}{k}{\Gamma}\theta \wedge (\varsigma_x, v_x) \in \V{k}{\theta(t)} \}
\end{align*}

And so the final semantic interpretation of a typing judgement simply
quantifies over all possible fractional permission substitutions $\theta$,
linear value substitutions $\gamma$, intuitionistic value substitutions
$\delta$ and heaps $\sigma$.  Note that, $\varsigma \equiv \den{H}{}{\theta(\sigma)}$.
\begin{align*}
\den{}{k}{ \Theta; \Delta ; \Gamma \vdash e : t } &= \forall \theta, \delta, \gamma, \sigma.\ \Theta = \dom(\theta) \wedge (\varsigma, \gamma) \in \den{L}{k}{ \Gamma }\theta \wedge \delta \in \den{I}{k}{ \Delta }\theta \Rightarrow \\
                                                     & \qquad \qquad (\varsigma, \theta(\delta(\gamma(e)))) \in \C{k}{ \theta(t) }
\end{align*}

\subsection{Soundness Theorem}

\begin{theorem}
(The Fundamental Lemma of Logical Relations)
\[
    \forall \Theta, \Delta, \Gamma, e, t.\ \Theta; \Delta ; \Gamma \vdash e : t \Rightarrow
    \forall k.\ \den{}{k}{ \Theta; \Delta ; \Gamma \vdash e : t }
\]
\end{theorem}

To prove the above theorem, we need several lemmas; the interesting ones are:
the moral equivalent of the frame rule (\ref{frame}), monotonicity for the
step-index (\ref{subsetKJ}), splitting up environments corresponds to splitting
up heaps (\ref{restriction}) and heap-and-expressions take the same steps of
evaluation under any substitution of their free fractional permissions
(\ref{fracPermSub}).

The proof proceeds by induction on the typing judgement.  The case for
\textsc{Ty\_Fix} is the reason we quantify over the step-index $k$ in the
\emph{conclusion} of the soundness theorem. It allows us to then induct over
the step-index and assume exactly the thing we need to prove at a smaller index.

The case for \textsc{Ty\_Gen} follows a similar pattern, but has the extra
complication of reducing an expression with an arbitrary fractional permission
variable in it, and then instantiating it at the last momemnt to conclude,
which is where \ref{fracPermSub} (heap-and-expressions take the same steps of
evaluation under any substitution of their free fractional permissions) is
used.

The rest of the cases are either very simple base cases (variables, unit,
boolean, integer or element literals) or follow very similar patterns; for
these, only \textsc{Ty\_Let} is presented in full and other similar cases
simply highlight exactly what would be different.  The general idea is to split
up the linear substitution and heap along the same split of $\Gamma/\Gamma'$,
then (by induction) use $\C{k}{-}$ and one `half' of the  linear substitution
and heap to conclude the `first' sub-expression either takes $j< k$ steps to
$\ottkw{err}$ or another heap-and-expression.

In the first case, you use \textsc{Op\_Context\_Err} to conclude the whole
let-expression does the same. Similarly we use \textsc{Op\_Context} $j$ times
in the second case. However, a small book-keeping wrinkle needs to be taken
care of in the case that the heap-and-expression turns into a value in $i \leq
j$ steps: \textsc{Op\_Context} is not functorial for the n-fold iteration of
$\rightarrow$.  Basically, the following is not quite true:
\[
\ottdrule{%
    \ottpremise{\langle  \sigma  \ottsym{,}  \ottnt{e}  \rangle  \rightarrow^j  \langle  \sigma'  \ottsym{,}  \ottnt{e'}  \rangle}%
    }{
    \langle  \sigma  \ottsym{,}  \ottnt{C}  \ottsym{[}  \ottnt{e}  \ottsym{]}  \rangle  \rightarrow^j  \langle  \sigma'  \ottsym{,}  \ottnt{C}  \ottsym{[}  \ottnt{e'}  \ottsym{]}  \rangle}{%
    {\ottdrulename{Op\_Context}}{}%
}
\]
because after the $i$ steps, we need to invoke \textsc{Op\_Let\_Var} to proceed
evalution for any remaining $j-i$ steps. After that, it suffices to use the
induction hypothesis on the second sub-expression to finish the proof.  To do
so, we need to construct a valid linear substitution and heap (i.e., one in
$\den{L}{k}{\Gamma', x : t}\theta$). We take the other `half' of the linear
substitution and heap (from the inital split at the start) and extend it with
$[x \mapsto v]$, (where $x$ is the variable bound in the let-expression and $v$
is the value we assume the first sub-expression evaluated to in $i$ steps).



\chapter{Implementation}

\begin{guidance}
This chapter may be called something else\ldots but in general
the idea is that you have one (or a few) ``meat'' chapters which
describe the work you did in technical detail.
\end{guidance}

\prechapter{%
    I implemented this project in OCaml, however I strongly believe the ideas
    described in this chapter can be applied easily to other languages and also
    are modular enough to extend the OCaml implementation to output to different
    back-end languages. I will show how a small core language with a few features
    can be the target of heavy-desugaring of typical linear-algebra programs. This
    core language can then be elaborated and checked for linearity before
    performing some simple and predictable optimisations and emitting (in this
    particular implementation) OCaml code that is not obviously safe, correct
    and performant.
}%

\section{Core Language}

\section{Elaboration and Inference}

\section{Compiling and Metaprogramming}


\section{Discussion and Related Work}\label{sec:discussion_related_work}

\subsection{Finding Bugs in SymPy's Output}\label{subsec:finding_bugs}

Prior to this project, we had little experience with linear algebra libraries
or the problem of matrix expression compilation. As such, we based our initial
\lang\ implementation of a Kalman filter using BLAS and LAPACK, on a popular
GitHub gist of a Fortran implementation, one that was \emph{automatically
generated} from SymPy's matrix expression compiler~\cite{rocklin_thesis}.

Once we translated the implementation from Fortran to \lang, we attempted to
compile it and found that (to our surprise) it did not type-check. This was
because the original implementation contained incorrect aliasing, unused
variables and unnecessary temporaries, and did not adhere to Fortran's
read/write permissions (with respect to \texttt{intent} annotations
\texttt{in}, \texttt{out} and \texttt{inout}) all of which were now highlighted
by \lang's type system.

The original implementation used 6 temporaries, one of which was immediately
spotted as never being used due to linearity. It also contained two variables
which were marked as \texttt{intent(in)} but would have been written over by
calls to `gemm', spotted by the fractional capabilities feature. Furthermore,
it used a matrix \emph{twice} in a call to `symm', once with a read permission
but once with a \emph{write} permission.  Fortran assumes that any parameter
being written to is not aliased and so this call was not only incorrect, but
illegal according to the standard, both aspects of which were captured by
linearity and fractional capabilities.

Lastly, it contained another unnecessary temporary, however one that was not
obvious without linear types. To spot it, we first performed live-range
splitting (checked by linearity) by hoisting calls to \highl{freeM} and then
annotated the freed matrices with their dimensions.  After doing so and
spotting two disjoint live-ranges of the same size, we replaced a call to
\highl{freeM} followed by allocating call to \highl{copy} with one, in-place
call to \highl{copyM_to}. We believe the ability to boldly refactor code which
manages memory is good evidence of the usefulness of linearity as a tool for
programming.

\subsection{Related Work}

Using linear types for BLAS routines is a particularly good domain fit (given
the implicit restrictions on aliasing arguments), and as a result the idea of
using substructural types to express array computations is not a particularly
new one~\cite{scholz,henriksen,bernardy2016}.  However, many of these designs
have been focused on building languages to \emph{implement} the kernel linear
algebra functions, and as a result, they tend to add additional limitations on
the language design. Both Futhark~\cite{henriksen} and Single Assignment
C~\cite{scholz} omit higher-order functions to facilitate compilation to GPUs.
The work of \cite{bernardy2016} forbids term-level recursion, in order to
ensure that all higher-order computations can be statically normalized away and
thereby maximize opportunities for array fusion.

In contrast, our approach is to begin with the assumption that we can take
existing efficient BLAS-like libraries, and then enforce their correct
\emph{usage} using a linear type discipline with fractional permissions. 

This approach is similar to the one taken in linear algebra libraries for Rust
-- these libraries typically take advantage of the distinction that Rust's type
system offers between mutable views/references to arrays.  The work of
\cite{weiss} and \cite{rustbelt} suggest that Rust's borrow-checker \emph{can
be expressed in simpler terms} using fractional permissions, though to our
knowledge the programmer-visible lifetime analysis in Rust has never been
formalized.

Working explicitly with fractional permissions has two main benefits. First,
our type system demonstrates that type systems for fractional permissions can
be dramatically simpler than existing state-of-the-art approaches, including
both industrial languages like Rust, as well as academic (such as those
developed by \cite{bierhoff}).  Bierhoff \emph{et al}'s type system, much like
Rust's, builds a complex dataflow analysis into the typing rules to infer when
variables can be shared or not. This allows for more natural-looking user
programs, but can create the impression that using fractional permissions
requires a heavy theoretical and engineering effort going well beyond that
needed for supporting basic linear types.

Instead, our approach, of requiring sharing to be made explicit, lets us
demonstrate that the existing unification machinery already in place for
ordinary ML-style type inference can be reused to support fractions. Basically,
we can view sharing a value as dividing a fraction by two, and after taking
logarithms all fractions are Peano numbers, whose equality can be established
with ordinary unification.

This fact is important because there are major upcoming implementations of
linear types such as Linear Haskell~\cite{bernardy2017linear}, which do not
have built-in support for fractional permissions. Instead, Linear Haskell takes
a slightly different definition of linearity, one based on \emph{arrows} as
opposed to \emph{kinds}: for $f : a \multimap b$, if $f u$ is used exactly once
\emph{then} $u$ is used exactly once. Whilst this has the advantage of being
backwards-compatible, it also means that the type system has no built-in
support for the concurrent reader, exclusive writer pattern that fractional
permissions enable.

However, since our type system shows demonstrates unification is ``all one
needs'' for fractions, it should be possible to \emph{encode} \lang's approach
to fractional permissions in Linear Haskell by adding a GADT-style natural
number index to array types tracking the fraction, which should enable
supporting high-performance BLAS bindings in Linear Haskell. Actually
implementing this is something we leave for future work, as there remains one
issue which we do not see a good encoding for. Namely, only having support for
linear functions makes it a bit inconvenient to manipulate linear values
directly -- programs end up taking on a CPS-like structure. This seems to
remain an advantage of a direct implementation of linear types over the Linear
Haskell style.


\subsection{Simplicity and Further Work}

We are pleasantly surprised at how simple the overall design and implementation
of \lang\ is, given its expressive power and usability.  So simple in fact,
that fractions, a convenient theoretical abstraction until this point, could be
implemented by restricting division and multiplication to be by 2 only
\cite{boyland2003}, thus turning any required arithmetic into unification.

Indeed, the focus on getting a working prototype early on (so that we could
test it with real BLAS/LAPACK routines as soon as possible) meant that we only
added features to the type system when it was clear that they were absolutely
necessary: these features were !-types and value-restriction for the
\highl{Many} constructor. 

Going forwards, one may wish to eliminate even more runtime errors from \lang,
by extending its type system. For example, we could have used existential types
to statically track pointer identities~\cite{morrisett}, or parametric
polymorphism.

We could also attempt to catch mismatched dimensions at compile time as well.
While this could be done with generative phantom types~\cite{abe2015simple},
using dependent types may offer more flexibility in \emph{partitioning}
regions~\cite{space_monads} or statically enforcing dimensions related
constraints of the arguments at compile-time.  ATS~\cite{cui2005ats} is an
example of a language which combines linear types with a sophisticated proof
layer. But although it provides BLAS bindings, it does not aim to provide
aliasing restrictions as demonstrated in this paper.

Taking this idea one step even further, since matrix dimensions are typically
fixed at runtime, we could \emph{stage} \lang\ programs and compile matrix
expressions using more sophisticated algorithms~\cite{barthels}. However, it is
worth noting that without care, such algorithms~\cite{rocklin_thesis}, usually
based on graph-based, ad-hoc dataflow analysis, can produce erroneous output
which would not get past a linear type system with fractions.

We also think that this concept (and the general design of its implementation)
need not be limited to linear algebra: we could conceivably `backport' this
idea to other contexts that need linearity (concurrency, single-use
continuations, zero-copy buffer, streaming I/O) or combine it with dependent
types to achieve even more expressive power to split up a single block of
memory into multiple regions in an arbitrary manner~\cite{space_monads}.


\clearpage
\bibliography{ourbib}

\clearpage
\appendix
\section{\lang\ Specification}

\ottstyledefaults{premiselayout=justify}%
\subsection{Static Semantics}\label{subsec:static_sem}
\ottdefnsTypes%

\subsection{Dynamic Semantics}\label{subsec:dyn_sem}
\ottdefnsOpXXSem%

\clearpage
\section{Interpretation}

\subsection{Definitions}

% Changed to multiset because normal disjoint unions and subsets of cartesian
% products for the heap wouldn't capture _multiplicity_:different variables
% in the environment could have identical permissions/types.

Operationally, $\emph{Heap} \sqsubseteq \emph{Loc} \times \emph{Permission}
\times \emph{Matrix} $ (a multiset), denoted with a $\sigma$.\\
Define its \emph{interpretation} to be $\emph{Loc} \rightharpoonup
\emph{Permission} \times \emph{Matrix}$ with $\star:
\emph{Heap} \times \emph{Heap} \rightharpoonup \emph{Heap}$ as follows:
\[
    (\varsigma_1 \star \varsigma_2)(l) \equiv
    \begin{cases}
        \varsigma_1(l) & \textrm{if } l \in \dom(\varsigma_1) \wedge l \notin \dom(\varsigma_2) \\
        \varsigma_2(l) & \textrm{if } l \in \dom(\varsigma_2) \wedge l \notin \dom(\varsigma_1) \\
        (f_1 + f_2, m) & \textrm{if } (f_1, m) = \varsigma_1(l) \wedge (f_2, m) = \varsigma_2(l) \wedge f_1 + f_2 \leq 1 \\
        \textrm{undefined} & \textrm{otherwise}
    \end{cases}
\]
Commutativity and associativity of $\star$ follows from that of $+$.\\
$\varsigma_1 \star \varsigma_2$ is \emph{defined} if it is for all $l \in
\dom(\varsigma_1) \cup \dom(\varsigma_2)$.\\
%Note that $\forall \varsigma.\ \exists \textrm{ cardinally minimal}\, \sigma.\ \varsigma = \den{H}{}{\sigma}$.\\
%{\pf~Binary representation of $\varsigma(l)$ for each $l \in \emph{Loc}$.}\\
\textbf{Implicitly denote} $\varsigma \equiv \den{H}{}{\sigma} \equiv
\bigstar_{(l,f,m) \in \sigma} [ l \mapsto_f m ]$.\\
\\
The $n-$fold iteration for the $StepsTo$ (functional) relation, is also a (functional) relation:
\begin{align*}
    \forall n.\ \ottkw{err} &\rightarrow^n \ottkw{err} &
    \langle \sigma , v \rangle &\rightarrow^n \langle \sigma , v \rangle &
    \langle \sigma , e \rangle &\rightarrow^0 \langle \sigma , e \rangle &
    \langle \sigma , e \rangle &\rightarrow^{n+1} ((\langle \sigma , e \rangle \rightarrow) \rightarrow^n)
\end{align*}
Hence, all bounded iterations end in either an $\ottkw{err}$, a heap-and-expression or a
heap-and-value.

\subsection{Interpretation}

\begin{align*}
  \V{k}{ \Unit } &= \{ (\empH, \ast) \} \\
\\
    \V{k}{ \Bool } &= \{ (\empH, true), (\empH, false) \} \\
\\
    \V{k}{ \Int } &= \{ (\empH, n) \mid 2^{-63} \leq n \leq 2^{63} -1 \} \\
\\
    \V{k}{ \Elt } &= \{ (\empH, f) \mid f \textrm{ a IEEE Float64 } \} \\
\\
    \V{k}{ f \, \Mat } &= \{ (\{ l \mapsto_{2^{-f}} \_ \} , l) \} \\
\\
    \V{k}{ \Bang t } &= \{ (\empH, \Many\, v) \mid (\empH, v) \in \V{k}{t} \} \\
\\
    \V{k}{ \forall fc.\  t } &= \{ (\varsigma, \ottkw{fun}\, fc \rightarrow \, v) \mid \forall f.\ (\varsigma, (\ottkw{fun}\, fc \rightarrow \, v)\, [ f ]) \in \V{k}{ t [ fc / f ] } \} \\
\\
    \V{k}{ t_1 \otimes t_2 } &= \{ (\varsigma_1 \star \varsigma_2, ( v_1, v_2 )) \mid (\varsigma_1, v_1) \in \V{k}{t_1} \wedge (\varsigma_2, v_2) \in \V{k}{t_2} \} \\
\\
% j <= k because beta-reduction is guaranteed to take one step AND our types aren't recursive.
    \V{k}{ t \multimap t' } &= \{ (\varsigma_{v'}, v' ) \mid ( v' \equiv \ottkw{fun}\, x : t \rightarrow e \vee v' \equiv \ottkw{fix} (g, x : t, e : t') ) \, \wedge\\
                            & \qquad \qquad \forall j \leq k, (\varsigma_v, v) \in \V{j}{ t }.\ \varsigma_{v'} \star \varsigma_v \textrm{ defined } \Rightarrow (\varsigma_{v'} \star \varsigma_v, v'\, v) \in \C{j}{ t' } \} \\
\\
% j < k to match L3 Fluet/Ahmed paper
    \C{k}{ t } &= \{ (\varsigma_s, e_s) \mid \forall \, j < k, \sigma_r.\ \varsigma_s \star \varsigma_r \textrm{ defined } \Rightarrow \langle \sigma_s + \sigma_r, e_s \rangle \rightarrow^j \ottkw{err}\ \vee \exists \sigma_f, e_f.\\
               & \qquad \qquad \langle \sigma_s + \sigma_r, e_s \rangle \rightarrow^j \langle \sigma_f + \sigma_r, e_f \rangle \wedge ( e_f \textrm { is a value } \Rightarrow ( \varsigma_f \star \varsigma_r, e_f ) \in \V{k-j}{t} ) \} \\
\\
    \den{I}{k}{ \cdot } \theta &= \{ [] \} \\
\\
    \den{I}{k}{ \Delta, x : t } \theta &= \{ \delta[x \mapsto v_x ] \mid \delta \in \den{I}{k}{\Delta}\theta \wedge (\empH, v_x) \in \V{k}{\theta(t)} \} \\
\\
    \den{L}{k}{ \cdot } \theta &= \{ (\empH, []) \} \\
\\
    \den{L}{k}{ \Gamma, x : t } \theta &= \{ (\varsigma \star \varsigma_x, \gamma[x \mapsto v_x ]) \mid (\varsigma, \gamma) \in \den{L}{k}{\Gamma}\theta \wedge (\varsigma_x, v_x) \in \V{k}{\theta(t)} \} \\
\\
    \varsigma \equiv \den{H}{}{\sigma} &\equiv \bigstar_{(l,f,m) \in \sigma} [ l \mapsto_f m ]\\
\\
\den{}{k}{ \Theta; \Delta ; \Gamma \vdash e : t } &= \forall \theta, \delta, \gamma, \sigma.\ \dom(\Theta) = \dom(\theta) \wedge (\varsigma, \gamma) \in \den{L}{k}{ \Gamma }\theta \wedge \delta \in \den{I}{k}{ \Delta }\theta \Rightarrow \\
                                                 & \qquad \qquad (\varsigma, \gamma(\delta(e))) \in \C{k}{ \theta(t) }
\end{align*}


\section{Lemmas}

\subsection{$
    \forall \sigma_s, \sigma_r, e.\ \varsigma_s \star \varsigma_r \textrm{ defined }
    \Rightarrow \forall n.\ \langle \sigma_s, e \rangle \rightarrow^n =
    \langle \sigma_s + \sigma_r , e \rangle \rightarrow^n
$}\label{frame}

\begin{proof}

    \suffices{By induction on $n$, consider only the cases $\langle \sigma_s, e
        \rangle \rightarrow \langle \sigma_f, e_f \rangle$ where $\sigma_s \neq
        \sigma_f$.\\}

    \pfsketch{~Only \textsc{Op\_\{Free, Matrix, Share, Unshare\_Eq,
        Gemm\_Match\}} change the heap: the rest are either parametric in the
        heap or step to an $\ottkw{err}$.\\}

    \prove{$\langle \sigma_s + \sigma_r, e \rangle \rightarrow
        \langle \sigma_f + \sigma_r, e_f \rangle$.\\}

    \step{}{\case{\textsc{Op\_Free}, $\sigma_s \equiv \sigma' + \{ l
        \mapsto_1 m \}$, $\sigma_f = \sigma'$.}{\pf~Instantiate
        \textsc{Op\_Free} with $(\sigma' + \sigma_r) + \{ l \mapsto_1 m
        \}$,\\valid because $l \notin \dom(\varsigma_r)$ by $\varsigma' \star
        [l \mapsto_1 m] \star \varsigma_r$ defined (assumption).}}

    \step{}{\case{\textsc{Op\_Matrix}}{
        \pf~Rule has no requirements on $\sigma_s$ so will also work with
        $\sigma_s + \sigma_r$.
    }}

    \step{}{\case{\textsc{Op\_Share}, $\sigma_s \equiv \sigma' + \{ l \mapsto_f
        m \}$, $\sigma_f = \sigma' + \{ l \mapsto_{\frac{1}{2} \cdot f} m \} +
        \{ l \mapsto_{\frac{1}{2} \cdot f} m \}$.}{
        \pf~Union-ing $\sigma_r$ does not remove $l \mapsto_f m$, so that can
        be split out of $ \sigma_s + \sigma_r$ as before.
    }}

    \step{}{\case{\textsc{Op\_Unshare\_Eq}, $\sigma_s \equiv \sigma' + \{ l
        \mapsto_{\frac{1}{2} \cdot f} m \} + \{ l \mapsto_{\frac{1}{2} \cdot f}
        m \}$, $\sigma_f = \sigma' + \{ l \mapsto_f m \}$.}{}}

    \begin{proof}

        \step{}{Union-ing $\sigma_r$ does not remove $l \mapsto_{\frac{1}{2}
            \cdot f} m$, so that can still be split out of $ \sigma_s + \sigma_r$.}

        \step{}{There may also be other valid splits introduced by $\sigma_r$.}

        \step{}{However, by assumption of $\varsigma_s \star \varsigma_r$
            defined, any splitting of $\sigma_s + \sigma_r$ will
            satisfy $f \leq 1$.}

    \end{proof}

    \step{}{\case{\textsc{Op\_Gemm\_Match}}{}}

    \begin{proof}

        \step{}{By assumption of $\varsigma_s \star \varsigma_r$ defined,
            either $l_1$ (or $l_2$, or both) are not in $\sigma_r$, or they are
            and the matrix values they point to are the same.}

        \step{}{The permissions (of $l_1$ and/or $l_2$) may differ, but
            \textsc{Op\_Gemm\_Match} universally quantifies over them and
            leaves them unchanged, so they are irrelevant.}

        \step{}{Only the pointed to matrix value at $l_3$ changes.}

        \step{}{\suffices{$l_3 \notin \pi_1 [\sigma_r]$.}}

        \step{}{By assumption of $\varsigma_s \star \varsigma_r$ defined, $l_3
            \notin \dom(\varsigma_r)$.}

        \step{}{Hence $l_3 \notin \pi_1 [\sigma_r]$.}

    \end{proof}

\end{proof}

\subsection{$\forall k, t.\ \V{k}{t} \subseteq \C{k}{t}$}\label{subsetVC}
Follows from definition of $\C{k}{t}$, $\rightarrow^j$ ($\forall n.\ \langle
\sigma , v \rangle \rightarrow^n \langle \sigma , v \rangle$) for arbitrary
$j \leq k$ and \ref{frame}.

\subsection{$\forall \theta, \delta, \gamma, v.\ \theta(\delta(\gamma(v)))\textrm{ is a value.}$}\label{valueSub}

$\theta$ is irrelevant because it only maps fractional permission variables to
fractional permissions. By construction, $\delta$ and $\gamma$ only map
variables to values, and values are closed under substitution.

\subsection{$
    \forall k, \sigma, \sigma', e, e', t.  \ (\varsigma', e') \in \C{k}{t} \wedge
    \langle \sigma, e \rangle \rightarrow \langle \sigma', e' \rangle
    \Rightarrow (\varsigma, e) \in \C{k+1}{t}
$}\label{stepInC}

In the lemma, and for the rest of its proof, $\varsigma = \den{H}{}{\sigma}$.

\begin{proof}

    \assume{arbitrary $j < k + 1$, and $\sigma_r$ such that $\varsigma
    \star \varsigma_r$ defined.\\}

    \step{}{\case{$j=0$. Clearly $\sigma_f = \sigma_s + \sigma_r$ and $e' = e$.}{
        Remains to show that if $e$ is a value then $(\varsigma_s \star
        \varsigma_r, e) \in \V{k}{t}$.\\
        This is true vacuously, because by assumption, $e$ is not a value.}}

    \step{}{\case{$j \geq 1$. We have $\langle \sigma, e \rangle
        \rightarrow^j \, = \langle \sigma', e' \rangle \rightarrow^{j-1}$.}{
        Instantiate $(\varsigma', e') \in \C{k}{t}$, with $j-1 < k$ and
        $\sigma_r$ to conclude the required conditions.}}

\end{proof}

\subsection{$j \leq k \Rightarrow \den{\_\ }{k}{\cdot} \subseteq \den{\_\ }{j}{\cdot}$}\label{subsetKJ}

For the rest of this proof, $\varsigma = \den{H}{}{\sigma}$.\\
Lemma~\ref{stepInC} is the inductive step for this lemma for the $\C{}{}$ case.\\
Need to prove for $\V{}{}$, by induction on $t$ and then index.

\begin{proof}

    \suffices{Consider only $t \multimap t'$ case, rest use $k$ directly on structure of type.}

    \assume{Arbitrary $j \leq k$ and $(\varsigma_{v'}, v') \in \V{k}{t
        \multimap t'}$.}

    \prove{$(\varsigma_{v'}, v') \in \V{j}{t \multimap t'}$.\\}

    \step{}{$v'$ is of the correct syntactic form (lambda or fixpoint) by
        assumption.}

    \step{}{\assume{arbitrary $j' \leq j$ and $(\varsigma_v, v) \in \V{j'}{t}$
        such that $\varsigma_{v'} \star \varsigma_v$ is defined.}}

    \step{}{\suffices{to show $(\varsigma_{v'} \star \varsigma_v, v' v) \in
        \C{j'}{t'}$.}}

    \step{}{This is true by instantiating $(\varsigma_{v'}, v') \in \V{k}{t
        \multimap t'}$ with $j' \leq k$ and $(\varsigma_v, v) \in \V{j'}{t}$.}

\end{proof}

\subsection{$\forall \Delta, \Gamma, t, k, \theta, \delta, \gamma.\ %
    \delta \in \den{I}{k}{\Delta}\theta \wedge \gamma \in \pi_2[\den{L}{k}{\Gamma}\theta]
    \Rightarrow \dom(\Delta) = \dom(\delta)$ and $\dom(\Gamma) = \dom(\gamma)$}\label{samedom}

\pf~By induction on $\Delta$ and $\Gamma$.

\subsection{$\forall k, \Gamma, \Gamma', \theta, \sigma_+, \gamma_+.\ %
    (\varsigma_+, \gamma) \in \den{L}{k}{ \Gamma, \Gamma' }\theta
    \wedge \Gamma, \Gamma' \textrm{ disjoint } \Rightarrow
    \exists \sigma, \gamma, \sigma' , \gamma' .\ \sigma_+ = \sigma + \sigma'
    \wedge \gamma, \gamma' \textrm{ disjoint }
    \wedge \gamma_+ = \gamma \cup \gamma'
    \wedge (\varsigma, \gamma) \in \den{L}{k}{\Gamma}
    \wedge (\varsigma', \gamma') \in \den{L}{k}{\Gamma'} $}\label{restriction}

\pf~By induction on $\Gamma'$.

\subsection{$\forall e, \sigma, e', \sigma', \theta.\
    \langle \sigma, e \rangle \rightarrow \langle \sigma',  e' \rangle
    \Rightarrow \langle \theta(\sigma), \theta(e) \rangle \rightarrow
    \langle \theta(\sigma') , \theta(e') \rangle$}\label{fracPermSub}

\pf~By induction on $\rightarrow$.
\begin{proof}

    \step{}{\assume{Arbitrary $e, \sigma, e', \sigma', \theta$ such that 
        $\langle \sigma, e \rangle \rightarrow \langle \sigma', e' \rangle$.}}

    \step{}{\suffices{To consider only the following rules which mention
        fractional permission variables.}
        \textsc{Op\_Frac\_Perm}, \textsc{Op\_Share}, \textsc{Op\_Unshare\_(N)Eq}
        and \textsc{Op\_Gemm\_(Mis)Match}.}

    \step{}{\case{\textsc{Op\_Frac\_Perm.}}{Because substitution avoids capture,
        \\ $\langle \theta(\sigma), \theta((\ottkw{fun}\: '\!f\!c \rightarrow v) \, [f])
         \rangle \rightarrow \langle \theta(\sigma' \, [f / f\!c]),
        \theta(v \, [f / f\!c]) \rangle$.}}

    \step{}{The rest of the cases are parametric in their use of fractional
        permission variables and so will take the same step ater any substitution.}

    \step{}{\textsc{Corollary:}
        If $\langle \sigma \, [f_1 / f\!c], e \, [f_1 / f\!c] \rangle \rightarrow^n
        \langle \sigma_2, e'_2 \rangle$ and
        $\langle \sigma \, [f_2 / f\!c], e \, [f_2 / f\!c] \rangle \rightarrow^n
        \langle \sigma_2, e'_2 \rangle$, then
        $\exists \, \sigma, e'.\ \sigma_1 = \sigma \, [f_1 / f\!c] \wedge
        \sigma_2 = \sigma \, [f_2 / f\!c] \wedge
        e'_1 = e' \, [f_1 / f\!c] \wedge e'_2 = e' \, [f_2 / f\!c]$.}

\end{proof}


\section{Soundness Proof}

\begin{proof}

    \pfsketch{\ %
        Use the contrapositive both ways. This turns the negated existential into
        witnesses we can work with.\\
    }

    \pflet{%
        $\phi(X) = $\\
        \emph{Note: $\forall X.\ \phi(X)\subseteq X$, $\not\subset\ \equiv\
        \not\subseteq \, \vee \, =$ and $\not\subseteq\ \Rightarrow\ \neq$}\\
        $a$, $b$, $c$ be elements of the Martelli's semiring \\
        $L^+ = a \cup \phi$ \\
        $L = \phi(L^+) = a \otimes (b \oplus c)$ \\
        $M^+ = $ \\
        $M = \phi(M^+) = (a \otimes b) \oplus (a \otimes c)$ \\
    }

    \prove{Distributivity holds, i.e.\ $L=M$.}
    \suffices{%
        Since $\oplus$ and $\otimes$ are commutative (definitions of $\oplus$
        and $\otimes$ are symmetric in their arguments because $\exists x.\
        \exists y.\ P(x,y) \Leftrightarrow \exists y.\ \exists x.\ P(x,y)$ and
        $\cup$ is commutative) it suffices to show only left-distributivity.\\
    }

    \pf{~We show $L \subseteq M$ and $M \subseteq L$.\\}

    \step{}{%
        \case{%
            $L \subseteq M$.
        }{%
            We show $m \notin M \Rightarrow m \notin L$ for arbitrary $m$.\\
            \pf{~We do this by cases on $m \in M^+$}.
            \suffices{\ %
                Because $L \subseteq L^+$, to show $m \notin L$ it suffices to
                show either $m \notin L^+$ or $\exists y \in L^+.\ y \subset m$.
            }
        }
    }

    \begin{proof}

        \step{}{%
            \case{%
                $m \in M^+$.
            }{%
                This means that $ \exists x \in a \otimes b,\ y \in a \otimes
                c.\ x \cup y = m $ and because $m \notin M$, we have $\exists
                x' \in a \otimes b,\ y \in a \otimes c.\ x' \cup y' = m'
                \subset m = x \cup y$. Assume, without loss of generality,
                they are the smallest such $x'$ and $y'$. Because $\phi(X)
                \subseteq X$ for any $X$, we proceed by cases: either $x' \in
                a$ or $y' \in a$ or both $x' \in b$ and $y' \in c$.
            }
        }

        \step{mplusRule}{%
            \case{%
                $m \notin M^+$.
            }{%
                This means $\forall\ x \in \phi(a \cup b),\ y \in \phi(a \cup
                c).\ m \neq x \cup y$.
            }
        }

        \step{}{%
            \textbf{Thus, if $m \notin M$, then $m \notin L$.}~Q.E.D.
        }

    \end{proof}

    \step{}{%
        \case{%
            $M \subseteq L$.
        }{%
            We show $l \notin L \Rightarrow l \notin M$ for arbitrary $l$.
            \suffices{\ %
                Because $M \subseteq M^+$, to show $l \notin M$ it suffices to
                show either $l \notin M^+$ or $\exists y \in M^+.\ y \subset l$.
            }
        }
    }

    \begin{proof}

        \step{lplusRule}{%
            \case{%
                $l \notin L^+$.
            }{%
                This means $l \notin a$ and $l \notin b \oplus c = \phi$.\\
                We conclude from the latter, that $\forall x \in b,\ y \in c.\ 
                x \cup y \neq l$.\\
                We reason by cases on \emph{why} $l \notin a$, to show that
                $\exists y \in M^+.\ y \subset l$ or $l \notin M^+$.
            }
        }

        \step{}{%
            \case{%
                $l \in L^+$.
            }{%
                Under the assumption $l \notin L$, we need only consider two
                cases: the rest produce the contradiction $l \in L$.
            }
        }

    \end{proof}


    \step{}{%
        \textbf{Thus, $L = M$}~Q.E.D.
    }

\end{proof}


\subsection{Well-formed types}
\ottdefnsWellXXFormed%

\section{\lang\ Grammar}\label{sec:grammar_def}
\ottgrammar%

\section{Desugaring \lang}\label{sec:lang_desugar}

\begin{center}
\[\def\arraystretch{1.3}
    \begin{array}{rcl}
    x[e] &
    \Rightarrow &
    \mathbf{get}\ \_\ x\ (e) \;\qquad \textrm{(similarly for matrices)}
\\
    x[e_1] := e_2 &
    \Rightarrow &
    \mathbf{set}\ x\ (e_1)\ (e_2) \quad \textrm{(similarly for matrices)}
\\
\\
    pat & ::= & ()\ \mid\ x\ \mid\ !x\ \mid\ \mathbf{Many\ } pat\ \mid\ (pat, pat)
\\
    \mathbf{let}\ !x = e_1\ \mathbf{in}\ e_2 &
    \Rightarrow &
    \specialcell[t]{l}{\mathbf{let\ Many}\ x = e_1\ \mathbf{in\ } \\
    \mathbf{let\ Many}\ x = \mathbf{Many}\ (\mathbf{Many}\ x)\ \mathbf{in}\ e_2}
\\
    \mathbf{let\ Many} \langle pat_x \rangle\ = e_1\ \mathbf{in}\ e_2 &
    \Rightarrow &
    \specialcell[t]{l}{%
        \mathbf{let\ Many}\ x = x\ \mathbf{in\ } \\
        \mathbf{let\ } \langle pat_x \rangle\ = x\ \mathbf{in\ } e_2}
\\
    \mathbf{let}\ (\langle pat_a \rangle, \langle pat_b \rangle)\  = e_1\ \mathbf{in}\ e_2 &
    \Rightarrow &
    \specialcell[t]{l}{%
        \mathbf{let\ } (a,b)\ = a\_b\ \mathbf{in\ }
        \ \mathbf{let\ } \langle pat_a \rangle\ = a\ \mathbf{in\ } \\
        \mathbf{let\ } \langle pat_b \rangle\ = b\ \mathbf{in\ } e_2}
\\
    \mathbf{fun}\ (\langle pat_x \rangle : t) \rightarrow e &
    \Rightarrow &
    \mathbf{fun}\ (x : t) \rightarrow \mathbf{let\ } \langle pat_x \rangle = x\ \mathbf{in\ } e
\\
\\
    arg & ::= & \langle pat \rangle : t\ \mid\ {'x} \textrm{ (fractional permission variable)}
\\
    \mathbf{fun}\ \langle arg_{1 .. n} \rangle \rightarrow e &
    \Rightarrow &
    \mathbf{fun}\ \langle arg_1 \rangle \rightarrow ..
    \ \mathbf{fun}\ \langle arg_n \rangle \rightarrow e
\\
    \mathbf{let}\ f\ {\langle arg_{1 .. n} \rangle} = e_1\ \mathbf{in}\ e_2 &
    \Rightarrow &
    \mathbf{let}\ f = \mathbf{fun}\ {\langle arg_{1 .. n} \rangle} \rightarrow e_1\
    \mathbf{in}\ e_2
\\
    \mathbf{let}\ !f\ {\langle arg_{1 .. n} \rangle} = e_1\ \mathbf{in}\ e_2 &
    \Rightarrow &
    \mathbf{let\ Many}\ f = \mathbf{Many}\ (\mathbf{fun}\ {\langle arg_{1 .. n} \rangle}
    \rightarrow e_1)\ \mathbf{in}\ e_2
\\
    \mathrm{fixpoint} & \equiv & \mathbf{fix}\ (f, x : t, \mathbf{fun}
    \ {\langle arg_{1 .. n} \rangle} \rightarrow e_1 : {t'} )
\\
    \mathbf{let\ rec}\ f\ (x : t)\ {\langle arg_{1 .. n} \rangle} : {t'} = e_1\ \mathbf{in}\ e_2 &
    \Rightarrow &
    \mathbf{let}\ f = \mathrm{fixpoint}\ \mathbf{in}\ e_2
\\
    \mathbf{let\ rec}\ !f\ (x : t)\ {\langle arg_{1 .. n} \rangle} : {t'} = e_1\ \mathbf{in}\ e_2 &
    \Rightarrow &
    \mathbf{let\ Many}\ f = \mathbf{Many}\ \mathrm{fixpoint}\ \mathbf{in}\ e_2
    \end{array}
\]
\end{center}

\clearpage
\section{Primitives}\label{subsec:primitives}

\vspace*{\fill}
\begin{center}
\begin{minted}[fontsize=\small]{ocaml}
module Arr = Owl.Dense.Ndarray.D
type z = Z
type 'a s = Succ
type 'a arr = A of Arr.arr [@@unboxed]
type 'a mat = M of Arr.arr [@@unboxed]
type 'a bang = Many of 'a [@@unboxed]
module Prim :
sig
  val extract : 'a bang -> 'a
  (** Boolean *)
  val not_ : bool bang -> bool bang
  (** Arithmetic, many omitted for brevity *)
  val addI : int bang -> int bang -> int bang
  val eqI : int bang -> int bang -> bool bang
  (** Arrays *)
  val set : z arr -> int bang -> float bang -> z arr
  val get : 'a arr -> int bang -> 'a arr * float bang
  val share : 'a arr -> 'a s arr * 'a s arr
  val unshare : 'a s arr -> 'a s arr -> 'a arr
  val free : z arr -> unit
  (** Owl *)
  val array : int bang -> z arr
  val copy : 'a arr -> 'a arr * z arr
  val sin : z arr -> z arr
  val hypot : z arr -> 'a arr -> 'a arr * z arr
  (** Level 1 BLAS *)
  val asum : 'a arr -> 'a arr * float bang
  val axpy : float bang -> 'a arr -> z arr -> 'a arr * z arr
  val dot : 'a arr -> 'b arr -> ('a arr * 'b arr) * float bang
  val scal : float bang -> z arr -> z arr
  val amax : 'a arr -> 'a arr * int bang
  (* Matrix, some omitted for brevity *)
  val matrix : int bang -> int bang -> z mat
  val eye : int bang -> z mat
  val copy_mat : 'a mat -> 'a mat * z mat
  val copy_mat_to : 'a mat -> z mat -> 'a mat * z mat
  val size_mat : 'a mat -> 'a mat * (int bang * int bang)
  val transpose : 'a mat -> 'a mat * z mat
  (* Level 3 BLAS/LAPACK *)
  val gemm : float bang -> ('a mat * bool bang) -> ('b mat * bool bang) ->
             float bang -> z mat -> ('a mat * 'b mat) * z mat
  val symm : bool bang -> float bang -> 'a mat -> 'b mat -> float bang ->
             z mat -> ('a mat * 'b mat) * z mat
  val gesv : z mat -> z mat -> z mat * z mat
  val posv : z mat -> z mat -> z mat * z mat
  val potrs : 'a mat -> z mat -> 'a mat * z mat
  val syrk : bool bang -> float bang -> 'a mat -> float bang -> z mat ->
           'a mat * z mat
end
\end{minted}
\end{center}
\vfill

\clearpage
\section{Kalman Filters from \lang\ and C}

\begin{figure}[h]
\begin{center}
    \begin{minted}[fontsize=\footnotesize]{ocaml}
let kalman sigma h mu r_1 data_1 =
  let h, _p_k_n_p_ = Prim.size_mat h in
  let k, n = _p_k_n_p_ in
  let sigma_hT = Prim.matrix n k in
  let (sigma, h), sigma_hT =
    Prim.gemm (Many 1.) (sigma, Many false) (h, Many true) (Many 0.) sigma_hT in
  let (h, sigma_hT), r_2 =
    Prim.gemm (Many 1.) (h, Many false) (sigma_hT, Many false) (Many 1.) r_1 in
  let k_by_k, x = Prim.posv_flip r_2 sigma_hT in
  let (h, mu), data_2 =
    Prim.gemm (Many 1.) (h, Many false) (mu, Many false) (Many (-1.)) data_1 in
  let (x, data_2), new_mu =
    Prim.gemm (Many 1.) (x, Many false) (data_2, Many false) (Many 1.) mu in
  let x_h = Prim.matrix n n in
  let (x, h), x_h =
    Prim.gemm (Many 1.) (x, Many false) (h, Many false) (Many 0.) x_h in
  let () = Prim.free_mat x in
  let sigma, sigma2 = Prim.copy_mat sigma in
  let (sigma, x_h), new_sigma =
    Prim.symm (Many true) (Many (-1.)) sigma x_h (Many 1.) sigma2 in
  let () = Prim.free_mat x_h in
  ((sigma, h), (new_sigma, (new_mu, (k_by_k, data_2))))
    \end{minted}
    \caption{OCaml code for a Kalman filter, generated (at \emph{compile time})
        from the code in Figure~\ref{fig:lang_kalman}, passed through
        \texttt{ocamlformat} for presentation.}\label{fig:ocaml_kalman}

\vspace{\baselineskip}

    \begin{minted}[fontsize=\footnotesize]{c}
static void kalman( const int n,                   const int k,                
                    const double *sigma, /* n,n */ const double *h,   /* k,n */
                    const double *mu,    /* n,1 */ double *r,         /* k,k */
                    double *data,        /* k,1 */ double **ret_sigma /* n,n */ ) {
    double* n_by_k = (double *) malloc(n * k * sizeof(double));
    cblas_dgemm(RowMajor, NoTrans, Trans, n, k, n, 1., sigma, n, h, n, 0., n_by_k, k);
    cblas_dgemm(RowMajor, NoTrans, NoTrans, k, k, n, 1., h, n, n_by_k, k, 1., r, k);
    LAPACKE_dposv(LAPACK_COL_MAJOR, 'U', k, n, r, k, n_by_k, k);
    cblas_dgemm(RowMajor, NoTrans, NoTrans, k, 1, n, 1., h, n, mu, 1, -1., data, 1);
    cblas_dgemm(RowMajor, NoTrans, NoTrans, n, 1, k, 1., n_by_k, k, data, 1, 1., mu, 1);
    double* n_by_n = (double *) malloc(n * n * sizeof(double));
    cblas_dgemm(RowMajor, NoTrans, NoTrans, n, n, k, 1., n_by_k, k, h, n, 0., n_by_n, n);
    free(n_by_k);
    double* n_by_n2 = (double *) malloc(n * n * sizeof(double));
    cblas_dcopy(n*n, sigma, 1, n_by_n2, 1);
    cblas_dsymm(RowMajor, Right, Upper, n, n, -1., sigma, n, n_by_n, n, 1., n_by_n2, n);
    free(n_by_n);
    *ret_sigma = n_by_n2; }
    \end{minted}
    \caption{\textsc{Cblas/Lapacke} implementation of a Kalman filter. I used C instead
        of Fortran because it is what Owl uses under the hood and OCaml FFI
        support for C is better and easier to use than that for Fortran. A distinct
        `measure\_kalman' function that sandwiches a call to this function with
        \texttt{getrusage} is omitted for brevity.}\label{fig:cblas_kalman}

\end{center}
\end{figure}



\end{document}
