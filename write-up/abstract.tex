\newpage
{\normalfont\Huge\sffamily\bfseries TO DO: Abstract}
\vspace{24pt}

In this thesis, I will argue that linear types are an appropriate,
\emph{type-based formalism} for expressing aliasing, read/write permissions,
memory allocation, re-use and deallocation, first, in the context of the APIs
of linear algebra libraries, consequently extending these principles to the
problem of \emph{efficient} matrix-expression compilation.

I will show that framing the problem using linear types can help \emph{reduce
bugs} by making precise and explicit the informal, ad-hoc practices typically
employed by human experts and linear algebra \emph{compilers} and automate
checking them.

As evidence for this argument, I will show programs written with this safety,
precision and explicitness (1) can be just as pleasant and convenient for a
programmer as less efficient, but higher-level linear algebra libraries and (2)
perform just as \emph{efficiently and predictably} as lower-level, less
readable and more error-prone interfaces linear algebra libraries. I will
further show that support for such programs can be implemented \emph{as
libraries} on top of existing programming languages and libraries.

\textbf{Summarise evaluation.}

\textbf{Summary future work.}

\newpage
\vspace*{\fill}
