\section{Discussion and Related Work}\label{sec:discussion_related_work}

\subsection{Finding Bugs in SymPy's Output}\label{subsec:finding_bugs}

Prior to this project, we had little experience with linear algebra libraries
or the problem of matrix expression compilation. As such, we based our initial
\lang\ implementation of a Kalman filter using BLAS and LAPACK, on a popular
GitHub gist of a Fortran implementation, one that was \emph{automatically
generated} from SymPy's matrix expression compiler~\cite{rocklin_thesis}.

Once we translated the implementation from Fortran to \lang, we attempted to
compile it and found that (to our surprise) it did not type-check. This was
because the original implementation contained incorrect aliasing, unused
variables and unnecessary temporaries, and did not adhere to Fortran's
read/write permissions (with respect to \texttt{intent} annotations
\texttt{in}, \texttt{out} and \texttt{inout}) all of which were now highlighted
by \lang's type system.

The original implementation used 6 temporaries, one of which was immediately
spotted as never being used due to linearity. It also contained two variables
which were marked as \texttt{intent(in)} but would have been written over by
calls to `gemm', spotted by the fractional-capabilities feature. Furthermore,
it used a matrix \emph{twice} in a call to `symm', once with a read permission
but once with a \emph{write} permission.  Fortran assumes that any parameter
being written to is not aliased and so this call was not only incorrect, but
illegal according to the standard, both aspects of which were captured by
linearity and fractional-capabilities.

Lastly, it contained another unnecessary temporary, however one that was not
obvious without linear types. To spot it, we first performed live-range
splitting (checked by linearity) by hoisting calls to \highl{freeM} and then
annotated the freed matrices with their dimensions.  After doing so and
spotting two disjoint live-ranges of the same size, we replaced a call to
\highl{freeM} followed by allocating call to \highl{copy} with one, in-place
call to \highl{copyM_to}. We believe the ability to boldly refactor code which
manages memory is good evidence of the usefulness of linearity as a tool for
programming.

\subsection{Related Work}

The main point we want to make is that using linear types for BLAS
is an ``obvious'' idea, but is surprisingly under-explored. 

\begin{itemize}
\item Rust
\item ATS
\item Single-assignment C 
\item Linear Haskell
\item Bernardy and Sveningsson
\item L3
\item Boyland fractional permissions
\item Disadvantage of graph-based approaches to matrix expression generation
\item multi-stage: once dimensions known, they are fixed?
\end{itemize}

\subsection{Simplicity and Further Work}

We are pleasantly surprised at how simple the overall design and implementation
of \lang\ is, given its expressive power and usability. Indeed, the focus on
getting a working prototype early on (so that we could test it with real
BLAS/LAPACK routines as soon as possible) meant that we only added features to
the type system when it was clear that they were absolutely necessary: these
features were !-types and value-restriction for the \highl{Many} constructor.
Although we considered parametric polymorphism, size-types and tracking pointer
identities at the type-level\cite{morrisett} (all of which would certainly be
useful and are potential avenues for further work) it turns out these features
are not necessary.

We believe we have stumbled across a particularly nice problem-solution fit:
the idea of linear types for linear algebra -- one that used to just be
commonly believed folklore -- actually has some merit and is worth exploring.
We also think that this concept (and the general design of its implementation)
need not be limited to linear algebra: we could conceivably `backport' this
idea to other contexts that need linearity (concurrency, single-use
continuations, zero-copy buffer, streaming I/O) or combine it with dependent
types to achieve even more expressive power to split up a single block of
memory into multiple regions in an arbitrary manner\cite{space_monads}.
