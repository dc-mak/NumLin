\section{Kalman Filters from \lang\ and C}

\begin{figure}[tb]
    \centering
    \begin{minted}[fontsize=\footnotesize]{ocaml}
let kalman sigma h mu r_1 data_1 =
  let h, _p_k_n_p_ = Prim.size_mat h in
  let k, n = _p_k_n_p_ in
  let sigma_h = Prim.matrix k n in
  let (sigma, h), sigma_h =
    Prim.symm (Many true) (Many 1.) sigma h (Many 0.) sigma_h
  in
  let (sigma_h, h), r_2 =
    Prim.gemm (Many 1.) (sigma_h, Many false) (h, Many true) (Many 1.) r_1
  in
  let (h, mu), data_2 =
    Prim.gemm (Many 1.) (h, Many false) (mu, Many false) (Many (-1.)) data_1
  in
  let h, new_h = Prim.copy_mat_to h sigma_h in
  let r_2, new_r = Prim.copy_mat r_2 in
  let chol_r, sol_h = Prim.posv new_r new_h in
  let chol_r, sol_data = Prim.potrs chol_r data_2 in
  let () = Prim.free_mat chol_r in
  let h_sol_h = Prim.matrix n n in
  let (h, sol_h), h_sol_h =
    Prim.gemm (Many 1.) (h, Many true) (sol_h, Many false) (Many 0.) h_sol_h
  in
  let () = Prim.free_mat sol_h in
  let h_sol_data = Prim.matrix n (Many 1) in
  let (h, sol_data), h_sol_data =
    Prim.gemm (Many 1.) (h, Many true) (sol_data, Many false) (Many 0.) h_sol_data
  in
  let mu, mu_copy = Prim.copy_mat mu in
  let (sigma, h_sol_data), new_mu =
    Prim.symm (Many false) (Many 1.) sigma h_sol_data (Many 1.) mu_copy
  in
  let () = Prim.free_mat h_sol_data in
  let h_sol_h_sigma = Prim.matrix n n in
  let (sigma, h_sol_h), h_sol_h_sigma =
    Prim.symm (Many true) (Many 1.) sigma h_sol_h (Many 0.) h_sol_h_sigma
  in
  let sigma, sigma_copy = Prim.copy_mat_to sigma h_sol_h in
  let (sigma, h_sol_h_sigma), new_sigma =
    Prim.symm (Many false) (Many (-1.)) sigma h_sol_h_sigma (Many 1.) sigma_copy
  in
  let () = Prim.free_mat h_sol_h_sigma in
  ((sigma, (h, (mu, (r_2, sol_data)))), (new_mu, new_sigma)) )
in
kalman
    \end{minted}
    \caption{OCaml code for a Kalman filter, generated (at \emph{compile time})
        from the code in Figure~\ref{fig:lang_kalman}, passed through
        \texttt{ocamlformat} for presentation.}\label{fig:ocaml_kalman}

\end{figure}

\begin{landscape}
\begin{figure}[p]
    \centering
    \begin{minted}[fontsize=\footnotesize]{c}
static void kalman( const int n,               const int k,                const double *sigma, /* n,n */
                    const double *h, /* k,n */ const double *mu, /* n,1 */ double *r,           /* k,k */
                    double *data,    /* k,1 */ double **ret_mu,  /* k,1 */ double **ret_sigma   /* n,n */ ) {
        double* k_by_n = (double *) malloc(k * n * sizeof(double));
/*16*/  cblas_dsymm(CblasRowMajor, CblasRight, CblasUpper, k, n, 1., sigma, n, h, n, 0., k_by_n, n);
/*17*/  cblas_dgemm(CblasRowMajor, CblasNoTrans, CblasTrans, k, k, n, 1., k_by_n, n, h, n, 1., r, k);
/*18*/  cblas_dgemm(CblasRowMajor, CblasNoTrans, CblasNoTrans, k, 1, n, 1., h, n, mu, 1, -1., data, 1);
/*19*/  cblas_dcopy(k * n, h, 1, k_by_n, 1);
        double* k_by_k = (double *) malloc(k * k * sizeof(double));
/*20*/  cblas_dcopy(k * k, r, 1, k_by_k, 1);
/*21*/  LAPACKE_dposv(LAPACK_ROW_MAJOR, 'U', k, n, k_by_k, k, k_by_n, n);
/*23*/  LAPACKE_dpotrs(LAPACK_ROW_MAJOR, 'U', k, 1, k_by_k, k, data, 1);
        free(k_by_k);
        double* n_by_n = (double *) malloc(n * n * sizeof(double));
/*24*/  cblas_dgemm(CblasRowMajor, CblasTrans, CblasNoTrans, n, n, k, 1., h, n, k_by_n, n, 0., n_by_n, n);
        free(k_by_n);
        double* n_by_1 = (double *) malloc(n * sizeof(double));
/*25*/  cblas_dgemm(CblasRowMajor, CblasTrans, CblasNoTrans, n, 1, k, 1., h, n, data, 1, 0., n_by_1, 1);
        double* new_mu = (double *) malloc(n * sizeof(double));
/*26*/  cblas_dcopy(n, mu, 1, new_mu, 1);
/*27*/  cblas_dsymm(CblasRowMajor, CblasLeft, CblasUpper, n, 1, 1., sigma, n, n_by_1, 1, 1., new_mu, 1);
        free(n_by_1);
        double* n_by_n2 = (double *) malloc(n * n * sizeof(double));
/*28*/  cblas_dsymm(CblasRowMajor, CblasRight, CblasUpper, n, n, 1., sigma, n, n_by_n, n, 0., n_by_n2, n);
/*29*/  cblas_dcopy(n*n, sigma, 1, n_by_n, 1);
/*30*/  cblas_dsymm(CblasRowMajor, CblasLeft, CblasUpper, n, n, -1., sigma, n, n_by_n2, n, 1., n_by_n, n);
        free(n_by_n2);
        *ret_sigma = n_by_n;
        *ret_mu = new_mu; }
    \end{minted}
    \caption{CBLAS/LAPACKE implementation of a Kalman filter. I used C instead
        of Fortran because it is what Owl uses under the hood and OCaml FFI
        support for C is better and easier to use than that for Fortran. A distinct
        `measure\_kalman' function that sandwiches a call to this function with
        \texttt{getrusage} is omitted for brevity.}\label{fig:cblas_kalman}

\end{figure}
\end{landscape}

\section{Specification}

\ottstyledefaults{premiselayout=justify}%
\subsection{Static Semantics}\label{subsec:static_sem}
\ottdefnsTypes%

\subsection{Dynamic Semantics}\label{subsec:dyn_sem}
\ottdefnsOpXXSem%

\section{Interpretation}

\subsection{Definitions}

% Changed to multiset because normal disjoint unions and subsets of cartesian
% products for the heap wouldn't capture _multiplicity_:different variables
% in the environment could have identical permissions/types.

Operationally, $\emph{Heap} \sqsubseteq \emph{Loc} \times \emph{Permission}
\times \emph{Matrix} $ (a multiset), denoted with a $\sigma$.\\
Define its \emph{interpretation} to be $\emph{Loc} \rightharpoonup
\emph{Permission} \times \emph{Matrix}$ with $\star:
\emph{Heap} \times \emph{Heap} \rightharpoonup \emph{Heap}$ as follows:
\[
    (\varsigma_1 \star \varsigma_2)(l) \equiv
    \begin{cases}
        \varsigma_1(l) & \textrm{if } l \in \dom(\varsigma_1) \wedge l \notin \dom(\varsigma_2) \\
        \varsigma_2(l) & \textrm{if } l \in \dom(\varsigma_2) \wedge l \notin \dom(\varsigma_1) \\
        (f_1 + f_2, m) & \textrm{if } (f_1, m) = \varsigma_1(l) \wedge (f_2, m) = \varsigma_2(l) \wedge f_1 + f_2 \leq 1 \\
        \textrm{undefined} & \textrm{otherwise}
    \end{cases}
\]
Commutativity and associativity of $\star$ follows from that of $+$.\\
$\varsigma_1 \star \varsigma_2$ is \emph{defined} if it is for all $l \in
\dom(\varsigma_1) \cup \dom(\varsigma_2)$.\\
\textbf{Implicitly denote} $\varsigma \equiv \den{H}{}{\sigma} \equiv
\bigstar_{(l,f,m) \in \sigma} [ l \mapsto_f m ]$.\\
\\
The $n-$fold iteration for the $StepsTo$ relation, is also a relation:
\begin{align*}
    \forall n.\ \ottkw{err} &\rightarrow^n \ottkw{err}\\
    \forall n.\ \langle \sigma , v \rangle &\rightarrow^n \langle \sigma , v \rangle\\
    \langle \sigma , e \rangle &\rightarrow^0 \langle \sigma , e \rangle\\
    \langle \sigma , e \rangle &\rightarrow^{n+1} \langle \sigma'', e'' \rangle
    \equiv \exists \sigma', e'.\ \langle \sigma , e \rangle \rightarrow \langle
    \sigma' , e' \rangle \wedge \langle \sigma' , e' \rangle \rightarrow^n
    \langle \sigma'', e'' \rangle
\end{align*}
Hence, all bounded iterations end in either an $\ottkw{err}$, a heap-and-expression or a
heap-and-value.

\subsection{Interpretation}

\begin{align*}
  \V{k}{ \Unit } &= \{ (\empH, \ast) \} \\
\\
    \V{k}{ \Bool } &= \{ (\empH, true), (\empH, false) \} \\
\\
    \V{k}{ \Int } &= \{ (\empH, n) \mid 2^{-63} \leq n \leq 2^{63} -1 \} \\
\\
    \V{k}{ \Elt } &= \{ (\empH, f) \mid f \textrm{ a IEEE Float64 } \} \\
\\
    \V{k}{ f \, \Mat } &= \{ (\{ l \mapsto_{2^{-f}} \_ \} , l) \} \\
\\
    \V{k}{ \Bang t } &= \{ (\empH, \Many\, v) \mid (\empH, v) \in \V{k}{t} \} \\
\\
    \V{k}{ \forall fc.\  t } &= \{ (\varsigma, \ottkw{fun}\, fc \rightarrow \, v) \mid \forall f.\ (\varsigma, (\ottkw{fun}\, fc \rightarrow \, v)\, [ f ]) \in \V{k}{ t [ fc / f ] } \} \\
\\
    \V{k}{ t_1 \otimes t_2 } &= \{ (\varsigma_1 \star \varsigma_2, ( v_1, v_2 )) \mid (\varsigma_1, v_1) \in \V{k}{t_1} \wedge (\varsigma_2, v_2) \in \V{k}{t_2} \} \\
\\
    \V{k}{ t \multimap t' } &= \{ (\varsigma, v' ) \mid ( v' = \ottkw{fun}\, x : t \rightarrow e \vee v' = \ottkw{fix} (g, x : t, e : t') ) \, \wedge\\
                            & \qquad \qquad \forall j < k, (\varsigma_v, v) \in \V{j}{ t }.\ \varsigma \star \varsigma_v \textrm{ defined } \Rightarrow (\varsigma \star \varsigma_v, v'\, v) \in \C{j}{ t' } \} \\
\\
    \C{k}{ t } &= \{ (\varsigma_s, e) \mid \forall \, j \leq k, \sigma_r.\ \varsigma_s \star \varsigma_r \textrm{ defined } \Rightarrow \langle \sigma_s + \sigma_r, e \rangle \rightarrow^j \ottkw{err}\ \vee \exists \sigma_f, e'.\\
               & \qquad \qquad \langle \sigma_s + \sigma_r, e \rangle \rightarrow^j \langle \sigma_f + \sigma_r, e' \rangle \wedge ( e' \textrm { is a value } \Rightarrow ( \varsigma_f \star \varsigma_r, e' ) \in \V{k-j}{t} ) \} \\
\\
    \den{I}{k}{ \cdot } \theta &= \{ [] \} \\
\\
    \den{I}{k}{ \Delta, x : t } \theta &= \{ \delta[x \mapsto v_x ] \mid \delta \in \den{I}{k}{\Delta}\theta \wedge (\empH, v_x) \in \V{k}{\theta(t)} \} \\
\\
    \den{L}{k}{ \cdot } \theta &= \{ (\empH, []) \} \\
\\
    \den{L}{k}{ \Gamma, x : t } \theta &= \{ (\varsigma \star \varsigma_x, \gamma[x \mapsto v_x ]) \mid (\varsigma, \gamma) \in \den{L}{k}{\Gamma}\theta \wedge (\varsigma_x, v_x) \in \V{k}{\theta(t)} \} \\
\\
    \varsigma \equiv \den{H}{}{\sigma} &\equiv \bigstar_{(l,f,m) \in \sigma} [ l \mapsto_f m ]\\
\\
\den{}{k}{ \Theta; \Delta ; \Gamma \vdash e : t } &= \forall \theta, \delta, \gamma, \sigma.\ \dom(\Theta) = \dom(\theta) \wedge (\varsigma, \gamma) \in \den{L}{k}{ \Gamma }\theta \wedge \delta \in \den{I}{k}{ \Delta }\theta \Rightarrow \\
                                                 & \qquad \qquad (\varsigma, \gamma(\delta(e))) \in \C{k}{ \theta(t) }
\end{align*}


\section{Lemmas}

\subsection{$
    \forall \sigma_s, \sigma_r, e.\ \varsigma_s \star \varsigma_r \textrm{ defined }
    \Rightarrow \forall n.\ \langle \sigma_s, e \rangle \rightarrow^n =
    \langle \sigma_s + \sigma_r , e \rangle \rightarrow^n
$}\label{frame}

\begin{proof}

    \suffices{By induction on $n$, consider only the cases $\langle \sigma_s, e
        \rangle \rightarrow \langle \sigma_f, e_f \rangle$ where $\sigma_s \neq
        \sigma_f$.\\}

    \pfsketch{~Only \textsc{Op\_\{Free, Matrix, Share, Unshare\_Eq,
        Gemm\_Match\}} change the heap: the rest are either parametric in the
        heap or step to an $\ottkw{err}$.\\}

    \prove{$\langle \sigma_s + \sigma_r, e \rangle \rightarrow
        \langle \sigma_f + \sigma_r, e_f \rangle$.\\}

    \step{}{\case{\textsc{Op\_Free}, $\sigma_s \equiv \sigma' + \{ l
        \mapsto_1 m \}$, $\sigma_f = \sigma'$.}{\pf~Instantiate
        \textsc{Op\_Free} with $(\sigma' + \sigma_r) + \{ l \mapsto_1 m
        \}$,\\valid because $l \notin \dom(\varsigma_r)$ by $\varsigma' \star
        [l \mapsto_1 m] \star \varsigma_r$ defined (assumption).}}

    \step{}{\case{\textsc{Op\_Matrix}}{
        \pf~Rule has no requirements on $\sigma_s$ so will also work with
        $\sigma_s + \sigma_r$.
    }}

    \step{}{\case{\textsc{Op\_Share}, $\sigma_s \equiv \sigma' + \{ l \mapsto_f
        m \}$, $\sigma_f = \sigma' + \{ l \mapsto_{\frac{1}{2} \cdot f} m \} +
        \{ l \mapsto_{\frac{1}{2} \cdot f} m \}$.}{
        \pf~Union-ing $\sigma_r$ does not remove $l \mapsto_f m$, so that can
        be split out of $ \sigma_s + \sigma_r$ as before.
    }}

    \step{}{\case{\textsc{Op\_Unshare\_Eq}, $\sigma_s \equiv \sigma' + \{ l
        \mapsto_{\frac{1}{2} \cdot f} m \} + \{ l \mapsto_{\frac{1}{2} \cdot f}
        m \}$, $\sigma_f = \sigma' + \{ l \mapsto_f m \}$.}{}}

    \begin{proof}

        \step{}{Union-ing $\sigma_r$ does not remove $l \mapsto_{\frac{1}{2}
            \cdot f} m$, so that can still be split out of $ \sigma_s + \sigma_r$.}

        \step{}{There may also be other valid splits introduced by $\sigma_r$.}

        \step{}{However, by assumption of $\varsigma_s \star \varsigma_r$
            defined, any splitting of $\sigma_s + \sigma_r$ will
            satisfy $f \leq 1$.}

    \end{proof}

    \step{}{\case{\textsc{Op\_Gemm\_Match}}{}}

    \begin{proof}

        \step{}{By assumption of $\varsigma_s \star \varsigma_r$ defined,
            either $l_1$ (or $l_2$, or both) are not in $\sigma_r$, or they are
            and the matrix values they point to are the same.}

        \step{}{The permissions (of $l_1$ and/or $l_2$) may differ, but
            \textsc{Op\_Gemm\_Match} universally quantifies over them and
            leaves them unchanged, so they are irrelevant.}

        \step{}{Only the pointed to matrix value at $l_3$ changes.}

        \step{}{\suffices{$l_3 \notin \pi_1 [ \sigma_r ]$.}}

        \step{}{By assumption of $\varsigma_s \star \varsigma_r$ defined, $l_3
            \notin \dom(\varsigma_r)$.}

        \step{}{Hence $l_3 \notin \pi_1 [ \sigma_r ]$.}

    \end{proof}

\end{proof}

\subsection{$\forall k, t.\ \V{k}{t} \subseteq \C{k}{t}$}\label{subsetVC}
Follows from definition of $\C{k}{t}$, $\rightarrow^j$ ($\forall n.\ \langle
\sigma , v \rangle \rightarrow^n \langle \sigma , v \rangle$) for arbitrary
$j \leq k$ and \ref{frame}.

\subsection{$\forall \theta, \delta, \gamma, v.\ \theta(\delta(\gamma(v)))\textrm{ is a value.}$}\label{valueSub}

$\theta$ is irrelevant because it only maps fractional permission variables to
fractional permissions. By construction, $\delta$ and $\gamma$ only map
variables to values, and values are closed under substitution.

\subsection{$
    \forall k, \sigma, \sigma', e, e', t.  \ (\varsigma', e') \in \C{k}{t} \wedge
    \langle \sigma, e \rangle \rightarrow \langle \sigma', e' \rangle
    \Rightarrow (\varsigma, e) \in \C{k+1}{t}
$}\label{stepInC}

In the lemma, and for the rest of its proof, $\varsigma = \den{H}{}{\sigma}$.

\begin{proof}

    \assume{arbitrary $j < k + 1$, and $\sigma_r$ such that $\varsigma
    \star \varsigma_r$ defined.\\}

    \step{}{\case{$j=0$. Clearly $\sigma_f = \sigma_s + \sigma_r$ and $e' = e$.}{
        Remains to show that if $e$ is a value then $(\varsigma_s \star
        \varsigma_r, e) \in \V{k}{t}$.\\
        This is true vacuously, because by assumption, $e$ is not a value.}}

    \step{}{\case{$j \geq 1$. We have $\langle \sigma, e \rangle
        \rightarrow^j\, = \langle \sigma', e' \rangle \rightarrow^{j-1}$.}{
        Instantiate $(\varsigma', e') \in \C{k}{t}$, with $j-1 < k$ and
        $\sigma_r$ to conclude the required conditions.}}

\end{proof}

\subsection{$j \leq k \Rightarrow \den{\_\ }{k}{\cdot} \subseteq \den{\_\ }{j}{\cdot}$}\label{subsetKJ}

For the rest of this proof, $\varsigma = \den{H}{}{\sigma}$.\\
Lemma~\ref{stepInC} is the inductive step for this lemma for the $\C{}{}$ case.\\
Need to prove for $\V{}{}$, by induction on $t$ and then index.

\begin{proof}

    \suffices{Consider only $t \multimap t'$ case, rest use $k$ directly on structure of type.}

    \assume{Arbitrary $j \leq k$ and $(\varsigma_{v'}, v') \in \V{k}{t
        \multimap t'}$.}

    \prove{$(\varsigma_{v'}, v') \in \V{j}{t \multimap t'}$.\\}

    \step{}{$v'$ is of the correct syntactic form (lambda or fixpoint) by
        assumption.}

    \step{}{\assume{arbitrary $j' \leq j$ and $(\varsigma_v, v) \in \V{j'}{t}$
        such that $\varsigma_{v'} \star \varsigma_v$ is defined.}}

    \step{}{\suffices{to show $(\varsigma_{v'} \star \varsigma_v, v' v) \in
        \C{j'}{t'}$.}}

    \step{}{This is true by instantiating $(\varsigma_{v'}, v') \in \V{k}{t
        \multimap t'}$ with $j' \leq k$ and $(\varsigma_v, v) \in \V{j'}{t}$.}

\end{proof}

\subsection{$\forall \Delta, \Gamma, t, k, \theta, \delta, \gamma.\ %
    \delta \in \den{I}{k}{\Delta}\theta \wedge \gamma \in \pi_2[\den{L}{k}{\Gamma}\theta]
    \Rightarrow \dom(\Delta) = \dom(\delta)$ and $\dom(\Gamma) = \dom(\gamma)$}\label{samedom}

\pf~By induction on $\Delta$ and $\Gamma$.

\subsection{$\forall k, \Gamma, \Gamma', \theta, \sigma_+, \gamma_+.\ %
    (\varsigma_+, \gamma) \in \den{L}{k}{ \Gamma, \Gamma' }\theta
    \wedge \Gamma, \Gamma' \textrm{ disjoint } \Rightarrow
    \exists \sigma, \gamma, \sigma' , \gamma' .\ \sigma_+ = \sigma + \sigma'
    \wedge \gamma, \gamma' \textrm{ disjoint }
    \wedge \gamma_+ = \gamma \cup \gamma'
    \wedge (\varsigma, \gamma) \in \den{L}{k}{\Gamma}
    \wedge (\varsigma', \gamma') \in \den{L}{k}{\Gamma'} $}\label{restriction}

\pf~By induction on $\Gamma'$.

\subsection{$\forall e, \sigma, e', \sigma', \theta.\
    \langle \sigma, e \rangle \rightarrow \langle \sigma',  e' \rangle
    \Rightarrow \langle \theta(\sigma), \theta(e) \rangle \rightarrow
    \langle \theta(\sigma') , \theta(e') \rangle$}\label{fracPermSub}

\pf~By induction on $\rightarrow$.
\begin{proof}

    \step{}{\assume{Arbitrary $e, \sigma, e', \sigma', \theta$ such that 
        $\langle \sigma, e \rangle \rightarrow \langle \sigma', e' \rangle$.}}

    \step{}{\suffices{To consider only the following rules which mention
        fractional permission variables.}
        \textsc{Op\_Frac\_Perm}, \textsc{Op\_Share}, \textsc{Op\_Unshare\_(N)Eq}
        and \textsc{Op\_Gemm\_(Mis)Match}.}

    \step{}{\case{\textsc{Op\_Frac\_Perm.}}{Because substitution avoids capture,
        \\ $\langle \theta(\sigma), \theta((\ottkw{fun} f\!c \rightarrow v) \, [f])
         \rangle \rightarrow \langle \theta(\sigma'\, [f\!c/f]),
        \theta(v\, [f\!c/f]) \rangle$.}}

    \step{}{The rest of the cases are parametric in their use of fractional
        permission variables and so will take the same step ater any substitution.}

    \step{}{\textsc{Corollary:}
        If $\langle \sigma \, [f\!c/f_1], e\, [f\!c/f_1] \rangle \rightarrow^n
        \langle \sigma_2, e'_2 \rangle$ and
        $\langle \sigma\, [f\!c/f_2], e \, [f\!c/f_2] \rangle \rightarrow^n
        \langle \sigma_2, e'_2 \rangle$, then
        $\exists\,\sigma, e'.\ \sigma_1 = \sigma\,[f\!c/f_1] \wedge
        \sigma_2 = \sigma\,[f\!c/f_2] \wedge
        e'_1 = e'\,[f\!c/f_1] \wedge e'_2 = e'\,[f\!c/f_2]$.}

\end{proof}


\section{Soundness Proof}

\[
    \forall \Theta, \Delta, \Gamma, e, t.\ \Theta; \Delta ; \Gamma \vdash e : t \Rightarrow
    \den{}{}{ \Theta; \Delta ; \Gamma \vdash e : t }
\]

\begin{proof}

    \pfsketch{ Induction over the typing judgements.\\ }

    \assume{%
        \begin{pfenum}
            \item Arbitrary $\Theta, \Delta, \Gamma, e, t$ such that $\Theta; \Delta ; \Gamma \vdash e : t$.
            \item Arbitrary $\theta, k, \delta, \gamma, \sigma$ such that:
                \begin{pfenum}
                    \item $\textrm{dom}(\Theta) = \textrm{dom}(\theta)$
                    \item $(\sigma, \gamma) \in \den{L}{k}{\Gamma}\theta$\label{inL}
                    \item $\delta \in \den{I}{k}{\Delta}\theta$.\label{inI}
                \end{pfenum}
            \item W.l.o.g., all variables are distinct/$\dom(\Delta)$ and
                $\dom(\Gamma)$ are disjoint.\label{disjoint}
            \item And so that over expressions $\gamma \circ \delta = \delta \circ \gamma$.
            \item By construction, $\dom(\Delta) = \dom(\delta)$ and $\dom(\Gamma) = \dom(\gamma)$.\label{samedom}
            \item ??? $\V{k}{\theta(t)} \subseteq \C{k}{\theta(t)}$.\label{subsetVC}
            \item ??? ``Stronger heap''/frame rule: $\langle \sigma, e \rangle
                \rightarrow^* = \langle \sigma \star \sigma_r , e \rangle \rightarrow^*$.\label{frame}
            \item ??? $\delta(\gamma(v))$ is a value.\label{valueSub}
            \item ??? $j \leq k \Rightarrow \den{\_\ }{k}{\cdot} \subseteq \den{\_\ }{j}{\cdot}$\label{subset}
            \item ??? $(\sigma', e') \in \C{k-1}{\cdot} \wedge \langle \sigma,
                e \rangle \rightarrow \langle \sigma', e' \rangle \Rightarrow
                (\sigma, e) \in \C{k}{\cdot}\label{stepInC}$\\
        \end{pfenum}
    }

    \prove{$(\sigma, \gamma(\delta(e))) \in \C{k}{\theta(t')}$.}
    \assume{Arbitrary $j \leq k$ and $\sigma_{r}$.}
    \suffices{Show whole expression either reduces to $\ottkw{err}$ or takes $j$ steps.\\}

    \step{}{%
        \case{%
            \textsc{Ty\_Let}.
        }{%
            \prove{$(\sigma, \gamma(\delta(\ottkw{let}\, x\, =\, e\, \ottkw{in}\, e'))) \in \C{k}{\theta(t')}$.}

            \suffices{$(\sigma, \ottkw{let}\, x\, =\, \gamma(\delta(e))\,
                \ottkw{in}\, \gamma(\delta(e'))) \in \C{k}{\theta(t')}$.}

        }
    }

    \begin{proof}
        \step{}{By induction,
            \begin{pfenum}
                \item $\den{}{}{\Theta; \Delta; \Gamma \vdash e : t}$\label{IH1}
                \item $\den{}{}{\Theta; \Delta; \Gamma', x : t \vdash e' : t'}$\label{IH2}.
            \end{pfenum}}\pflabel{IH}

        \step{}{By \ref{inL} and induction on $\Gamma'$, we know there exist
                $\sigma_{e'},\; (\sigma_e, \gamma_e) \in \den{L}{k}{\Gamma}$, \\
                such that $\sigma = \sigma_e \star \sigma_{e'}$.}\pflabel{split}

        \step{}{So, using them, $\theta, k, \delta$, and \ref{disjoint} we have
            $(\sigma_e, \gamma_e(e)) \in \C{k}{\theta(t)}$.}

        \step{}{By \ref{disjoint}, $(\sigma_e, \gamma(\delta(e))) \in \C{k}{\theta(t)}$.}

        \step{}{By definition of $\C{k}{\cdot}$ and \pfref{split}, we
            instantiate with $j$ and $\sigma_r = \sigma_{e'}$ to
            conclude that $\langle \sigma , \gamma(\delta(e)) \rangle$
            either reduces to $\ottkw{err}$ or another heap and expression.}

        \step{}{\case{$\ottkw{err}$}{??? By \textsc{Op\_Context\_Err}
            and \ref{frame} with $\sigma_{r}$, the whole expression
            reduces to $\ottkw{err}$ in $j \leq k$ steps. Since $j \leq k$ and
            $\sigma_{r}$ are arbitrary, $(\sigma, \gamma(\delta(\ottkw{let}\,
            x\, =\, e\, \ottkw{in}\, e'))) \in \C{k}{\theta(t')}$.}}

        \step{}{\case{$j$ steps to another heap and expression.}By \textsc{Op\_Context},
            the whole expression does the same.}

        \step{}{If it is not a value, we are done. ??? If it is $({\sigma_e}_f, v)
            \in \V{k-j}{\theta(t)}$ by \ref{valueSub}. \suffices{$({\sigma_e}_f \star
            \sigma_{e'}, \ottkw{let}\, x\, =\, v\, \ottkw{in}\, \gamma(\delta(e'))) \in
            \C{k-j}{\theta(t')}$.} \suffices{??? $({\sigma_e}_f \star \sigma_{e'},
            \gamma(\delta(e')) [x/v]) \in \C{k-j-1}{\theta(t')}$ by \ref{stepInC}.}}

        \step{}{\define{$\gamma_{e'} (y) = v$ if $ y=x $ and $\gamma(y)$ if $y \in \dom(\Gamma')$.}
            ??? Thus, by \ref{subset}, $(\sigma_{e'}, \gamma_{e'}) \in \den{L}{k}{\Gamma',
            x : t}\theta \subseteq \den{L}{k-j-1}{\Gamma', x : t}\theta$.}

        \step{}{Instantiate \ref{IH2} of step \pfref{IH} with $\theta, k-j-1, \delta,
            \gamma_{e'}, \sigma_{e'}$ to conclude \\ $(\sigma_{e'}, \gamma_{e'}(\delta(e')))
            \in \C{k-j-1}{\theta(t')}$.}

        \step{}{By \ref{disjoint}, we have $\gamma(\delta(e')) [x/v] =
            \gamma_{e'}(\delta(e'))$ and by \ref{frame} we conclude \\ $({\sigma_e}_f \star
            \sigma_{e'}, \gamma(\delta(e')) [x/v]) \in \C{k-j-1}{\theta(t')}$}

    \end{proof}

    \step{}{%
        \case{%
            \textsc{Ty\_Pair\_Elim}.
        }{%
            \prove{$(\sigma, \gamma(\delta(\ottkw{let}\, (a, b)\, =\, e\, \ottkw{in}\, e'))) \in \C{k}{\theta(t')}$.}
            {\pf\ Similar to \textsc{Ty\_Let} but with \textsc{Op\_Let\_Pair}}
        }
    }

    \begin{proof}

        \step{}{When $({\sigma_e}_f, v) \in \V{k-j}{\theta(t_1) \otimes \theta(t_2)}$, we have
            $v = (v_1, v_2)$.}

        \step{}{\suffices{??? $(\sigma_{e'}, \gamma(\delta(e'))) \in
            \C{k-j-1}{\theta(t')}$ by \ref{stepInC}.}}

        \step{}{\define{$\gamma_{e'}$ to be the restriction of $\gamma$ to $\dom(\Gamma')$.}
            ??? Thus, by \ref{subset}, $(\sigma_{e'}, \gamma_{e'}[a \mapsto v_1, b \mapsto v_2 ])
            \in \den{L}{k}{\Gamma', a : t_1, b : t_2 }\theta \\ \subseteq \den{L}{k-j-1}{\Gamma',
            a : t_1, b : t_2}\theta$.}

        \step{}{Instantiate $\den{}{}{\Theta; \Delta; \Gamma', a : t_1, b : t_2 \vdash e' : t'}$
            with $\theta, k-j-1, \delta, \gamma_{e'} [a \mapsto v_1, b \mapsto v_2 ] , \sigma_{e'}$.}

        \step{}{??? By \ref{disjoint} $(\sigma_{e'}, \gamma(\delta(e'))) \in
            \C{k-j-1}{\theta(t')}$.}

    \end{proof}

    \step{}{%
        \case{%
            \textsc{Ty\_Bang\_Elim}.
        }{%
            \prove{$(\sigma, \gamma(\delta(\ottkw{let}\, \ottkw{Many}\, x\, =\, e\, \ottkw{in}\, e')))
                \in \C{k}{\theta(t)}$.}
            {\pfsketch\ Similar to \textsc{Ty\_Let}, but with the following key differences.}
        }
    }

    \begin{proof}

        \step{}{When $({\sigma_e}_f, v) \in \V{k-j}{\theta(!t)}$, since $\V{k-j}{\theta(!t)}
            = \V{k-j}{!\theta(t)}$,\\ we have ${\sigma_e}_f = \empH$ and $v = \ottkw{Many}\, v'$
            for some $(\empH, v') \in \V{k-j}{\theta(t)}$.}

        \step{}{\suffices{$(\sigma_{e'}, \ottkw{let}\, \ottkw{Many}\, x\, =\, \ottkw{Many}\,
            v'\, \ottkw{in}\, \gamma(\delta(e'))) \in \C{k-j}{\theta(t)}$.}}

        \step{}{\suffices{$(\sigma_{e'}, \gamma(\delta(e')) [ x / v ]) \in \C{k-j-1}{\theta(t)}$.}}

        \step{}{\define{$\gamma_{e'}$ as the restriction of $\gamma$ to $\dom(\Gamma')$.}}

        \step{}{Instantiate $\den{}{}{\Theta; \Delta, x : t, \Gamma' \vdash e' : t'}$ with
            $\theta, k-j-1, \delta_{e'} = \delta[x \mapsto v'], \gamma_{e'}, \sigma_{e'}$ to
            conclude \\ $(\sigma_{e'}, \gamma_{e'}(\delta_{e'}(e'))) \in \C{k-j-1}{\theta(t)}$.}

        \step{}{??? By \ref{disjoint}, $(\sigma_{e'}, \gamma(\delta(e')) [ x / v ]) \in
            \C{k-j-1}{\theta(t)}$.}

    \end{proof}

    \step{}{%
        \case{%
            \textsc{Ty\_Unit\_Elim}.
        }{%
            \prove{$(\sigma, \gamma(\delta(\ottkw{let}\, ()\, =\, e\, \ottkw{in}\, e'))) \in \C{k}{\theta(t)}$.}
            {\pf\ Similar to \textsc{Ty\_Let} but with \textsc{Op\_Let\_Unit}.}

        }
    }

    \begin{proof}

        \step{}{When $({\sigma_e}_f, v) \in \V{k-j}{\ottkw{unit}}$, we have ${\sigma_e}_f =
            \empH$ and $v = ()$.}

        \step{}{\suffices{??? $(\sigma_{e'}, \gamma(\delta(e'))) \in
            \C{k-j-1}{\theta(t')}$ by \ref{stepInC}.}}

        \step{}{\define{$\gamma_{e'}$ to be the restriction of $\gamma$ to $\dom(\Gamma')$.}
            ??? Thus, by \ref{subset}, $(\sigma_{e'}, \gamma_{e'}) \in \den{L}{k}{\Gamma'}\theta
            \subseteq \den{L}{k-j-1}{\Gamma'}\theta$.}

        \step{}{Instantiate $\den{}{}{\Theta; \Delta; \Gamma' \vdash e' : t'}$ with $\theta,
            k-j-1, \delta, \gamma_{e'}, \sigma_{e'}$.}

        \step{}{??? By \ref{disjoint} $(\sigma_{e'}, \gamma(\delta(e'))) \in
            \C{k-j-1}{\theta(t')}$.}

    \end{proof}

    \step{}{%
        \case{%
            \textsc{Ty\_Bool\_Elim}.
        }{%
            \prove{$(\sigma, \gamma(\delta(\ottkw{if}\, e\, \ottkw{then}\,
                e_1\, \ottkw{else}\, e_2))) \in \C{k}{\theta(t)}$.}
                {\pf\ Similar to \textsc{Ty\_Unit\_Elim} but with \textsc{Op\_If\_\{True,False\}}\\
                and ${\sigma_e}_f = \empH$ and $v = \ottkw{Many}\, \ottkw{true}$ or
                $v = \ottkw{Many}\, \ottkw{false}$.}
        }
    }

    \step{}{%
        \case{%
            \textsc{Ty\_Bang\_Intro}.
        }{%
            \prove{$(\sigma, \gamma(\delta(\ottkw{Many}\, e))) \in \C{k}{\theta(!t)}$.}
            \suffices{$(\sigma, \ottkw{Many}\, \gamma(\delta(e))) \in \C{k}{!\theta(t)}$.}
        }

    }

    \begin{proof}

        \step{}{By assumption of \textsc{Ty\_Bang\_Intro}, $e = v$ for some
            value $v \neq l$, $\Gamma = \empH$ and so \\ $\den{}{}{\Theta;
            \Delta; \cdot \vdash v : t}$ by induction.}

        \step{}{\suffices{$(\empH, \ottkw{Many}\, \delta(v)) \in
            \C{k}{!\theta(t)}$ by \ref{disjoint} and \ref{inL}.}}

        \step{}{Instantiate $\den{}{}{\Theta; \Delta; \cdot \vdash v : t}$
            with $\theta, k, \delta, \gamma = [], \sigma = \empH$ to obtain
            $(\empH, \delta(v)) \in \C{k}{\theta(t)}$.}

        \step{}{Instantiate $(\empH, \delta(v)) \in \C{k}{\theta(t)}$ with $j=0$, and
            $\sigma_r = \empH$, to conclude $(\empH, v) \in \V{k}{\theta(t)}$.}

        \step{}{??? By definition of $\V{k}{!\theta(t)}$, \ref{valueSub} and \ref{subsetVC} we
            have $(\empH, \ottkw{Many}\, \delta(v)) \in \C{k}{!\theta(t)}$.}

    \end{proof}

    \step{}{%
        \case{%
            \textsc{Ty\_Pair\_Intro}.
        }{%
            \prove{$(\sigma, \gamma(\delta(\, ( e, e')\, ))) \in \C{k}{\theta(t \otimes t')}$.}
        }
    }

    % Intro dual of Unit_Elim
    \begin{proof}

        \step{}{By induction,
            \begin{pfenum}
                \item $\den{}{}{\Theta; \Delta; \Gamma \vdash e : t}$
                \item $\den{}{}{\Theta; \Delta; \Gamma' \vdash e' : t'}$.
            \end{pfenum}}

        \step{}{First component reduces to error in $j_1$ steps, so the whole thing does too.}

        \step{}{First component reduces to value in $j_1$ steps, so $\ldots$.}

        \step{}{Second component reduces to error in $j_2$ steps, so the whole thing does too.}

        \step{}{Second component also reduces to value in $j_2$ steps, so $\ldots$.}

        \step{}{Should be possible by IH instantiating with $k - j_2, k - j_1$ (swap, the subtract
            remaining) respectively.}

    \end{proof}

    \step{}{%
        \case{%
            \textsc{Ty\_Lambda}.
        }{%
            \prove{$(\sigma, \gamma(\delta(\ottkw{fun}\, x : t' \rightarrow e))) \in \C{k}{\theta(t' \multimap t)}$.}
        }
    }

    \begin{proof}
    \end{proof}

    \step{}{%
        \case{%
            \textsc{Ty\_App}.
        }{%
            \prove{$(\sigma, \gamma(\delta(\,e \,e'\,))) \in \C{k}{\theta(t)}$.}
        }
    }

    % Intro dual of Let
    \begin{proof}
    \end{proof}

    \step{}{%
        \case{%
            \textsc{Ty\_Gen}.
        }{%
            \prove{$(\sigma, \gamma(\delta(\ottkw{fun}\, fc \rightarrow e))) \in \C{k}{\theta(\forall \, fc.\, t)}$.}
        }
    }

    \begin{proof}
    \end{proof}

    \step{}{%
        \case{%
            \textsc{Ty\_Spc}.
        }{%
            \prove{$(\sigma, \gamma(\delta(e\, [f]))) \in \C{k}{\theta(t\, [fc/f])}$.}
        }
    }

    \begin{proof}
    \end{proof}

    \step{}{%
        \case{%
            \textsc{Ty\_Fix}.
        }{%
            \prove{$(\sigma, \gamma(\delta(\ottkw{fix} (g, x : t, e : t')))) \in
                \C{k}{\theta(!(t \multimap t'))}$. This means $\sigma = \empH$.}
                \suffices{??? to show $ \ldots \in \V{k}{!(\theta(t) \multimap
                    \theta(t'))}$, by \ref{subsetVC}.}
        }
    }

    \begin{proof}

        \step{}{\assume{Arbitrary $j < k$ and $(\sigma, v) \in \V{j}{\theta(t)}$.}}

        \step{}{\suffices{$(\sigma, \letManyG g\, v) \in \C{j}{\theta(t')}$.}}

        \step{}{\pflet{$e_1 = e [ g / \ottkw{fun}\, x : t \rightarrow \letManyG g\, x ]$.}}

        \step{}{\suffices{??? by \ref{stepInC}, $(\sigma, (\ottkw{fun}\, x : t \rightarrow e_1)\, v)
            \in \C{j-1}{\theta(t')}$.}}

        \step{}{\suffices{??? by \ref{stepInC}, $(\sigma, e_1 [ x / v ]) \in \C{j-2}{\theta(t')}$.}}

        \step{}{By induction, we have $\den{}{}{\Theta; \Delta, g : t \multimap t';
            x : t \vdash e : t'}$.}

        \step{}{Instantiate this with $\theta, j-2, \delta[g \mapsto \ottkw{fun}\, x : t
            \rightarrow e_1 ], \gamma = [ x \mapsto v ], \sigma = \empH$.
            \prove{$(\empH, \ottkw{fun}\, x : t \rightarrow e_1) \in \V{j-2}{\theta(t)
            \multimap \theta(t')}$.}}

            \begin{proof}

                \step{}{\suffices{??? by \ref{stepInC}, $(\sigma', e_1 [ x / v']) \in
                    \C{j-2}{\theta(t')}$ for arbitrary $(\sigma', v') \in \V{j-2}{\theta(t)}$.}}

                \step{}{We can again use the induction hypothesis $\den{}{}{\Theta;
                        \Delta, g : t \multimap t'; x : t \vdash e : t'}$.}

                \step{}{But since it's true for $\C{0}{\cdot}$ (base case), it's true
                    by induction ???}

            \end{proof}

            \step{}{Lastly, we show $\delta(\gamma(e)) = e_1 [ x / v ]$, which follows
                by their definitions, \\ to conclude $(\sigma, e_1 [ x / v ]) \in
                \C{j-2}{\theta(t')}$.}

    \end{proof}

    \step{}{%
        \case{%
            \textsc{Ty\_Var\_Lin}.
        }{%
            \prove{$(\sigma, \gamma(\delta(x))) \in \C{k}{\theta(t)}$.}
        }
    }

    \begin{proof}
        \step{}{$\Gamma = \{ x : t\}$ by assumption of \textsc{Ty\_Var\_Lin}.}
        \step{}{\suffices{$(\sigma, \gamma(x)) \in \C{k}{\theta(t)}$ by \ref{disjoint}.}}
        \step{}{By \ref{inL}, there exist $(\sigma_x, v_x) \in \V{k}{\theta(t)}$, such
            that $\sigma=\sigma_x$ and $\gamma = [x \mapsto v_x]$.}
        \step{}{??? Hence, $(\sigma_x, v_x) \in \C{k}{\theta(t)}$, by \ref{subsetVC}.}
    \end{proof}

    \step{}{%
        \case{%
            \textsc{Ty\_Var}.
        }{%
            \prove{$(\sigma, \gamma(\delta(x))) \in \C{k}{\theta(t)}$.}
        }
    }

    \begin{proof}
        \step{}{$x : t \in \Delta$ and $\Gamma = \empH$ by assumption of \textsc{Ty\_Var}.}
        \step{}{\suffices{$(\empH, \delta(x)) \in \C{k}{\theta(t)}$ by \ref{disjoint} and \ref{inL}.}}
        \step{}{By \ref{inI}, there exists $v_x$ such that $(\empH, v_x) \in \V{k}{\theta(t)}$.}
        \step{}{??? Hence, $(\empH, v_x) \in \C{k}{\theta(t)}$, by \ref{subsetVC}.}
    \end{proof}

    \step{}{%
        \case{%
            \textsc{Ty\_Unit\_Intro}.
        }{%
            \prove{$(\sigma, \gamma(\delta(\, ()\,))) \in \C{k}{\theta(\Unit)}$.\\}
        }
    }

    \step{}{%
        \case{%
            \textsc{Ty\_Bool\_True}, \textsc{Ty\_Bool\_False}, \textsc{Ty\_Int\_Intro}, \textsc{Ty\_Elt\_Intro}.
        }{%
            Similar to \textsc{Ty\_Unit\_Intro}.
        }
    }

\end{proof}


\section{Well-formed types}
\ottdefnsWellXXFormed%

\section{Grammar Definition}
\ottgrammar%

\section{Desugaring \lang}

\begin{figure}
\begin{center}
\[\def\arraystretch{1.3}
    \begin{array}{rcl}
    x[e] &
    \Rightarrow &
    \mathbf{get}\ \_\ x\ (e) \;\qquad \textrm{(similarly for matrices)}
\\
    x[e_1] := e_2 &
    \Rightarrow &
    \mathbf{set}\ x\ (e_1)\ (e_2) \quad \textrm{(similarly for matrices)}
\\
\\
    pat & ::= & ()\ \mid\ x\ \mid\ !x\ \mid\ \mathbf{Many\ } pat\ \mid\ (pat, pat)
\\
    \mathbf{let}\ !x = e_1\ \mathbf{in}\ e_2 &
    \Rightarrow &
    \specialcell[t]{l}{\mathbf{let\ Many}\ x = e_1\ \mathbf{in\ } \\
    \mathbf{let\ Many}\ x = \mathbf{Many}\ (\mathbf{Many}\ x)\ \mathbf{in}\ e_2}
\\
    \mathbf{let\ Many} \langle pat_x \rangle\ = e_1\ \mathbf{in}\ e_2 &
    \Rightarrow &
    \specialcell[t]{l}{%
        \mathbf{let\ Many}\ x = x\ \mathbf{in\ } \\
        \mathbf{let\ } \langle pat_x \rangle\ = x\ \mathbf{in\ } e_2}
\\
    \mathbf{let}\ (\langle pat_a \rangle, \langle pat_b \rangle)\  = e_1\ \mathbf{in}\ e_2 &
    \Rightarrow &
    \specialcell[t]{l}{%
        \mathbf{let\ } (a,b)\ = a\_b\ \mathbf{in\ }
        \ \mathbf{let\ } \langle pat_a \rangle\ = a\ \mathbf{in\ } \\
        \mathbf{let\ } \langle pat_b \rangle\ = b\ \mathbf{in\ } e_2}
\\
    \mathbf{fun}\ (\langle pat_x \rangle : t) \rightarrow e &
    \Rightarrow &
    \mathbf{fun}\ (x : t) \rightarrow \mathbf{let\ } \langle pat_x \rangle = x\ \mathbf{in\ } e
\\
\\
    arg & ::= & \langle pat \rangle : t\ \mid\ {'x} \textrm{ (fractional permission variable)}
\\
    \mathbf{fun}\ \langle arg_{1 .. n} \rangle \rightarrow e &
    \Rightarrow &
    \mathbf{fun}\ \langle arg_1 \rangle \rightarrow ..
    \ \mathbf{fun}\ \langle arg_n \rangle \rightarrow e
\\
    \mathbf{let}\ f\ {\langle arg_{1 .. n} \rangle} = e_1\ \mathbf{in}\ e_2 &
    \Rightarrow &
    \mathbf{let}\ f = \mathbf{fun}\ {\langle arg_{1 .. n} \rangle} \rightarrow e_1\
    \mathbf{in}\ e_2
\\
    \mathbf{let}\ !f\ {\langle arg_{1 .. n} \rangle} = e_1\ \mathbf{in}\ e_2 &
    \Rightarrow &
    \mathbf{let\ Many}\ f = \mathbf{Many}\ (\mathbf{fun}\ {\langle arg_{1 .. n} \rangle}
    \rightarrow e_1)\ \mathbf{in}\ e_2
\\
    \mathrm{fixpoint} & \equiv & \mathbf{fix}\ (f, x : t, \mathbf{fun}
    \ {\langle arg_{1 .. n} \rangle} \rightarrow e_1 : {t'} )
\\
    \mathbf{let\ rec}\ f\ (x : t)\ {\langle arg_{1 .. n} \rangle} : {t'} = e_1\ \mathbf{in}\ e_2 &
    \Rightarrow &
    \mathbf{let}\ f = \mathrm{fixpoint}\ \mathbf{in}\ e_2
\\
    \mathbf{let\ rec}\ !f\ (x : t)\ {\langle arg_{1 .. n} \rangle} : {t'} = e_1\ \mathbf{in}\ e_2 &
    \Rightarrow &
    \mathbf{let\ Many}\ f = \mathbf{Many}\ \mathrm{fixpoint}\ \mathbf{in}\ e_2
    \end{array}
\]
\end{center}
\caption{Desugaring from \lang\ concrete syntax to core constructs.}\label{fig:lang_desugar}
\end{figure}

