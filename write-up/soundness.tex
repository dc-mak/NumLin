\section{Proofs}

\subsection{Lemmas}

\subsubsection{$
    \forall \sigma_s, \sigma_r, e.\ \varsigma_s \star \varsigma_r \textrm{ defined }
    \Rightarrow \forall n.\ \langle \sigma_s, e \rangle \rightarrow^n =
    \langle \sigma_s + \sigma_r , e \rangle \rightarrow^n
$}\label{frame}

\begin{proof}

    \suffices{By induction on $n$, consider only the cases $\langle \sigma_s, e
        \rangle \rightarrow \langle \sigma_f, e_f \rangle$ where $\sigma_s \neq
        \sigma_f$.\\}

    \pfsketch{~Only \textsc{Op\_\{Free, Matrix, Share, Unshare\_Eq,
        Gemm\_Match\}} change the heap: the rest are either parametric in the
        heap or step to an $\ottkw{err}$.\\}

    \prove{$\langle \sigma_s + \sigma_r, e \rangle \rightarrow
        \langle \sigma_f + \sigma_r, e_f \rangle$.\\}

    \step{}{\case{\textsc{Op\_Free}, $\sigma_s \equiv \sigma' + \{ l
        \mapsto_1 m \}$, $\sigma_f = \sigma'$.}{\pf~Instantiate
        \textsc{Op\_Free} with $(\sigma' + \sigma_r) + \{ l \mapsto_1 m
        \}$,\\valid because $l \notin \dom(\varsigma_r)$ by $\varsigma' \star
        [l \mapsto_1 m] \star \varsigma_r$ defined (assumption).}}

    \step{}{\case{\textsc{Op\_Matrix}}{
        \pf~Rule has no requirements on $\sigma_s$ so will also work with
        $\sigma_s + \sigma_r$.
    }}

    \step{}{\case{\textsc{Op\_Share}, $\sigma_s \equiv \sigma' + \{ l \mapsto_f
        m \}$, $\sigma_f = \sigma' + \{ l \mapsto_{\frac{1}{2} \cdot f} m \} +
        \{ l \mapsto_{\frac{1}{2} \cdot f} m \}$.}{
        \pf~Union-ing $\sigma_r$ does not remove $l \mapsto_f m$, so that can
        be split out of $ \sigma_s + \sigma_r$ as before.
    }}

    \step{}{\case{\textsc{Op\_Unshare\_Eq}, $\sigma_s \equiv \sigma' + \{ l
        \mapsto_{\frac{1}{2} \cdot f} m \} + \{ l \mapsto_{\frac{1}{2} \cdot f}
        m \}$, $\sigma_f = \sigma' + \{ l \mapsto_f m \}$.}{}}

    \begin{proof}

        \step{}{Union-ing $\sigma_r$ does not remove $l \mapsto_{\frac{1}{2}
            \cdot f} m$, so that can still be split out of $ \sigma_s + \sigma_r$.}

        \step{}{There may also be other valid splits introduced by $\sigma_r$.}

        \step{}{However, by assumption of $\varsigma_s \star \varsigma_r$
            defined, any splitting of $\sigma_s + \sigma_r$ will
            satisfy $f \leq 1$.}

    \end{proof}

    \step{}{\case{\textsc{Op\_Gemm\_Match}}{}}

    \begin{proof}

        \step{}{By assumption of $\varsigma_s \star \varsigma_r$ defined,
            either $l_1$ (or $l_2$, or both) are not in $\sigma_r$, or they are
            and the matrix values they point to are the same.}

        \step{}{The permissions (of $l_1$ and/or $l_2$) may differ, but
            \textsc{Op\_Gemm\_Match} universally quantifies over them and
            leaves them unchanged, so they are irrelevant.}

        \step{}{Only the pointed to matrix value at $l_3$ changes.}

        \step{}{\suffices{$l_3 \notin \pi_1 [ \sigma_r ]$.}}

        \step{}{By assumption of $\varsigma_s \star \varsigma_r$ defined, $l_3
            \notin \dom(\varsigma_r)$.}

        \step{}{Hence $l_3 \notin \pi_1 [ \sigma_r ]$.}

    \end{proof}

\end{proof}

\subsubsection{$\forall k, t.\ \V{k}{t} \subseteq \C{k}{t}$}\label{subsetVC}
Follows from definition of $\C{k}{t}$, $\rightarrow^j$ ($\forall n.\ \langle
\sigma , v \rangle \rightarrow^n \langle \sigma , v \rangle$) for arbitrary
$j \leq k$ and \ref{frame}.

\subsubsection{$\forall \delta, \gamma, v.\ \delta(\gamma(v))\textrm{ is a value.}$}\label{valueSub}

By construction, $\delta$ and $\gamma$ only map variables to values, and values
are closed under substitution.

\subsubsection{$
    \forall k, \sigma, \sigma', e, e', t.  \ (\varsigma', e') \in \C{k}{t} \wedge
    \langle \sigma, e \rangle \rightarrow \langle \sigma', e' \rangle
    \Rightarrow (\varsigma, e) \in \C{k+1}{t}
$}\label{stepInC}

\begin{proof}

    \assume{arbitrary $j < k + 1$, and $\sigma_r$ such that $\varsigma
    \star \varsigma_r$ defined.\\}

    \step{}{\case{$j=0$. Clearly $\sigma_f = \sigma_s + \sigma_r$ and $e' = e$.}{
        Remains to show that if $e$ is a value then $(\varsigma_s \star
        \varsigma_r, e) \in \V{k}{t}$.\\
        This is true vacuously, because by assumption, $e$ is not a value.}}

    \step{}{\case{$j \geq 1$. We have $\langle \sigma, e \rangle
        \rightarrow^j\, = \langle \sigma', e' \rangle \rightarrow^{j-1}$.}{
        Instantiate $(\varsigma', e') \in \C{k}{t}$, with $j-1 < k$ and
        $\sigma_r$ to conclude the required conditions.}}

\end{proof}

\subsubsection{$j \leq k \Rightarrow \den{\_\ }{k}{\cdot} \subseteq \den{\_\ }{j}{\cdot}$}\label{subsetKJ}

Lemma~\ref{stepInC} is the inductive step for this lemma for the $\C{}{}$ case.\\
Need to prove for $\V{}{}$, by induction on $t$ and then index.

\begin{proof}

    \suffices{Consider only $t \multimap t'$ case, rest use $k$ directly on structure of type.}

    \assume{Arbitrary $j \leq k$ and $(\varsigma_{v'}, v') \in \V{k}{t
        \multimap t'}$.}

    \prove{$(\varsigma_{v'}, v') \in \V{j}{t \multimap t'}$.\\}

    \step{}{$v'$ is of the correct syntactic form (lambda or fixpoint) by
        assumption.}

    \step{}{\assume{arbitrary $j' \leq j$ and $(\varsigma_v, v) \in \V{j'}{t}$
        such that $\varsigma_{v'} \star \varsigma_v$ is defined.}}

    \step{}{\suffices{to show $(\varsigma_{v'} \star \varsigma_v, v' v) \in
        \C{j'}{t'}$.}}

    \step{}{This is true by instantiating $(\varsigma_{v'}, v') \in \V{k}{t
        \multimap t'}$ with $j' \leq k$ and $(\varsigma_v, v) \in \V{j'}{t}$.}

\end{proof}

\subsubsection{$\forall \Delta, \Gamma, t, k, \theta, \delta, \gamma.\ %
    \delta \in \den{I}{k}{\Delta}\theta \wedge \gamma \in \pi_2[\den{L}{k}{\Gamma}\theta]
    \Rightarrow \dom(\Delta) = \dom(\delta)$ and $\dom(\Gamma) = \dom(\gamma)$}\label{samedom}

\pf~By induction on $\Delta$ and $\Gamma$.

\subsubsection{$\forall k, \Gamma, \Gamma', \theta, \sigma_+, \gamma_+.\ %
    (\varsigma_+, \gamma) \in \den{L}{k}{ \Gamma, \Gamma' }\theta
    \wedge \Gamma, \Gamma' \textrm{ disjoint } \Rightarrow\\
    \exists \sigma, \gamma, \sigma' , \gamma' .\ \sigma_+ = \sigma + \sigma'
    \wedge \gamma, \gamma' \textrm{ disjoint }
    \wedge \gamma_+ = \gamma \cup \gamma'\\
    \wedge (\varsigma, \gamma) \in \den{L}{k}{\Gamma}
    \wedge (\varsigma', \gamma') \in \den{L}{k}{\Gamma'} $}\label{restriction}

\pf~By induction on $\Gamma'$.

\clearpage%
\subsection{Soundness}
\[
    \forall \Theta, \Delta, \Gamma, e, t.\ \Theta; \Delta ; \Gamma \vdash e : t \Rightarrow
    \forall k.\ \den{}{k}{ \Theta; \Delta ; \Gamma \vdash e : t }
\]

\begin{proof}

    \pfsketch{ Induction over the typing judgements.\\ }

    \assume{%
        \begin{pfenum}

            \item Arbitrary $\Theta, \Delta, \Gamma, e, t$ such that
                $\Theta; \Delta ; \Gamma \vdash e : t$.

            \item Arbitrary $k, \theta, \delta, \gamma, \sigma$ such that:
                \begin{pfenum}
                    \item $\textrm{dom}(\Theta) = \textrm{dom}(\theta)$
                    \item $(\varsigma, \gamma) \in \den{L}{k}{\Gamma}\theta$\label{inL}
                    \item $\delta \in \den{I}{k}{\Delta}\theta$.\label{inI}
                \end{pfenum}

            \item W.l.o.g., all variables are distinct,\\hence $\dom(\Delta)$ and
                $\dom(\Gamma)$ are disjoint\\so $\gamma \circ \delta = \delta
                \circ \gamma$ (as substitutions defined recursively over expressions).\label{disjoint}\\

        \end{pfenum}
    }

    \prove{$(\varsigma, \gamma(\delta(e))) \in \C{k}{\theta(t)}$.}

    \assume{Arbitrary $j < k$ and $\sigma_{r}$, such that $\varsigma \star \varsigma_r$ defined.}

    \suffices{$\langle \sigma + \sigma_r, e \rangle \rightarrow^j \ottkw{err}\ %
        \vee \exists \sigma_f, e_f.\ \langle \sigma + \sigma_r, e \rangle \rightarrow^j
        \langle \sigma_f + \sigma_r, e_f \rangle\\
        \qquad \wedge ( e_f \textrm { is a value } \Rightarrow
        ( \varsigma_f \star \varsigma_r, e_f ) \in \V{k-j}{t} )$.}

    \suffices{By \ref{frame}, to show $\langle \sigma, e \rangle \rightarrow^j \ottkw{err}\ %
        \vee \exists \sigma_f, e_f.\ \langle \sigma, e \rangle \rightarrow^j
        \langle \sigma_f, e_f \rangle\\
        \qquad \wedge ( e_f \textrm { is a value } \Rightarrow
        ( \varsigma_f, e_f ) \in \V{k-j}{t} )$\\}


    \step{}{%
        \case{%
            \textsc{Ty\_Let}.
        }{%
        }
    }

    \begin{proof}
        \step{}{By induction,
            \begin{pfenum}
                \item $\forall k.\ \den{}{k}{\Theta; \Delta; \Gamma \vdash e : t}$\label{IH1}
                \item $\forall k.\ \den{}{k}{\Theta; \Delta; \Gamma', x : t \vdash e' : t'}$\label{IH2}.
            \end{pfenum}}\pflabel{IH}

        \step{}{By \ref{inL}, \ref{disjoint} and \ref{restriction}, we know there exists the following:
            \begin{pfenum}
                \item $(\varsigma_e, \gamma_e) \in \den{L}{k}{\Gamma}$
                \item $\gamma = \gamma_e \cup \gamma_{e'}$
                \item $\sigma = \sigma_e + \sigma_{e'}$.
            \end{pfenum}}\pflabel{split}

        \step{}{So, using $k, \theta, \delta, \gamma_e, \sigma_e$, we have
            $(\varsigma_e, \gamma_e(\delta(e))) \in \C{k}{\theta(t)}$.}

        \step{}{By \pfref{split}, $(\varsigma_e, \gamma(\delta(e))) \in \C{k}{\theta(t)}$.}

        \step{}{By definition of $\C{k}{\cdot}$ and \pfref{split}, we
            instantiate with $j$ and $\sigma_r = \sigma_{e'}$ to
            conclude that $\langle \varsigma , \gamma(\delta(e)) \rangle$
            either takes $j$ step to $\ottkw{err}$ or another heap-and-expression
            $\langle \sigma_f, \gamma(\delta(e_f)) \rangle$.}

        \step{}{\case{$\ottkw{err}$}{By \textsc{Op\_Context\_Err},
            the whole expression reduces to $\ottkw{err}$ in $j < k$ steps.}}

        \step{}{\case{$j$ steps to another heap and expression.}By \textsc{Op\_Context} and,
            the whole expression does the same.}

        \step{}{If it is not a value, we are done. If it is $(\varsigma_f, v)
            \in \V{k-j}{\theta(t)}$ by \ref{valueSub}.
            \suffices{$(\varsigma_f \star \varsigma_{e'}, \ottkw{let}\, x\, =\, v\, \ottkw{in}\,
                \gamma(\delta(e'))) \in \C{k-j}{\theta(t')}$ by \ref{stepInC} $j$ times.}
            \suffices{$(\varsigma_f \star \varsigma_{e'}, \gamma(\delta(e')) [x/v]) \in
                \C{k-j-1}{\theta(t')}$ by \ref{stepInC}.}}

        \step{}{By \ref{subsetKJ}, $(\varsigma_{e'}, \gamma_{e'}[x \mapsto v]) \in
            \den{L}{k}{\Gamma', x : t}\theta \subseteq \den{L}{k-j-1}{\Gamma', x : t}\theta$.}

        \step{}{Instantiate \ref{IH2} of step \pfref{IH} with $\theta, k-j-1, \delta,
            \gamma_{e'}[x \mapsto v], \sigma_{e'}$ to conclude \\ $(\varsigma_{e'},
            \gamma_{e'}[x \mapsto v](\delta(e'))) \in \C{k-j-1}{\theta(t')}$.}

        \step{}{By \ref{disjoint}, we have $\gamma(\delta(e')) [x/v] =
            \gamma_{e'}[x \mapsto v](\delta(e'))$ and \\ by \ref{frame} we conclude
            $(\varsigma_f \star \varsigma_{e'}, \gamma(\delta(e')) [x/v]) \in \C{k-j-1}{\theta(t')}$}

    \end{proof}

    \step{}{%
        \case{%
            \textsc{Ty\_Pair\_Elim}.
        }{%
            {\pfsketch~Similar to \textsc{Ty\_Let}, but with the following key differences.}
        }
    }

    \begin{proof}

        \step{}{When $(\varsigma_f, v) \in \V{k-j}{\theta(t_1) \otimes \theta(t_2)}$, we have
            $v = (v_1, v_2)$.}

        \step{}{\suffices{$(\varsigma_{e'}, \gamma(\delta(e'))) \in
            \C{k-j-1}{\theta(t')}$ by \ref{stepInC} $j+1$ times.}}

        \step{}{By \ref{subsetKJ}, $(\varsigma_{e'}, \gamma_{e'}[a \mapsto v_1, b \mapsto v_2 ])
            \in \den{L}{k}{\Gamma', a : t_1, b : t_2 }\theta \subseteq \den{L}{k-j-1}{\Gamma',
            a : t_1, b : t_2}\theta$.}

        \step{}{Instantiate $\den{}{k-j-1}{\Theta; \Delta; \Gamma', a : t_1, b : t_2 \vdash e' : t'}$
            with $\theta, \delta, \gamma_{e'} [a \mapsto v_1, b \mapsto v_2 ] , \sigma_{e'}$.}

        \step{}{By \ref{disjoint} $(\varsigma_{e'}, \gamma(\delta(e'))) \in \C{k-j-1}{\theta(t')}$.}

    \end{proof}

    \step{}{%
        \case{%
            \textsc{Ty\_Bang\_Elim}.
        }{%
            {\pfsketch~Similar to \textsc{Ty\_Let}, but with the following key differences.}
        }
    }

    \begin{proof}

        \step{}{When $(\varsigma_f, v) \in \V{k-j}{\theta(!t)}$, since $\V{k-j}{\theta(!t)}
            = \V{k-j}{!\theta(t)}$,\\ we have $\varsigma_f = \empH$ and $v = \ottkw{Many}\, v'$
            for some $(\empH, v') \in \V{k-j}{\theta(t)}$.}

        \step{}{\suffices{$(\varsigma_{e'}, \ottkw{let}\, \ottkw{Many}\, x\, =\, \ottkw{Many}\,
            v'\, \ottkw{in}\, \gamma(\delta(e'))) \in \C{k-j}{\theta(t)}$.}}

        \step{}{\suffices{$(\varsigma_{e'}, \gamma(\delta(e')) [ x / v ]) \in \C{k-j-1}{\theta(t)}$
            by \ref{stepInC} $j+1$ times.}}

        \step{}{Instantiate $\den{}{k-j-1}{\Theta; \Delta, x : t, \Gamma' \vdash e' : t'}$ with
            $\theta, \delta_{e'} = \delta[x \mapsto v'], \gamma_{e'}, \sigma_{e'}$.}

        \step{}{By \ref{disjoint}, $(\varsigma_{e'}, \gamma(\delta(e')) [ x / v ]) \in
            \C{k-j-1}{\theta(t)}$.}

    \end{proof}

    \step{}{%
        \case{%
            \textsc{Ty\_Unit\_Elim}.
        }{%
            \prove{$(\sigma, \gamma(\delta(\ottkw{let}\, ()\, =\, e\, \ottkw{in}\, e'))) \in \C{k}{\theta(t)}$.}
            {\pf\ Similar to \textsc{Ty\_Let} but with \textsc{Op\_Let\_Unit}.}

        }
    }

    \begin{proof}

        \step{}{When $({\sigma_e}_f, v) \in \V{k-j}{\ottkw{unit}}$, we have ${\sigma_e}_f =
            \empH$ and $v = ()$.}

        \step{}{\suffices{$(\sigma_{e'}, \gamma(\delta(e'))) \in
            \C{k-j-1}{\theta(t')}$ by \ref{stepInC}.}}

        \step{}{\define{$\gamma_{e'}$ to be the restriction of $\gamma$ to $\dom(\Gamma')$.}
            Thus, by \ref{subsetKJ}, $(\sigma_{e'}, \gamma_{e'}) \in \den{L}{k}{\Gamma'}\theta
            \subseteq \den{L}{k-j-1}{\Gamma'}\theta$.}

        \step{}{Instantiate $\den{}{}{\Theta; \Delta; \Gamma' \vdash e' : t'}$ with $\theta,
            k-j-1, \delta, \gamma_{e'}, \sigma_{e'}$.}

        \step{}{By \ref{disjoint} $(\sigma_{e'}, \gamma(\delta(e'))) \in
            \C{k-j-1}{\theta(t')}$.}

    \end{proof}

    \step{}{%
        \case{%
            \textsc{Ty\_Bool\_Elim}.
        }{%
            \prove{$(\sigma, \gamma(\delta(\ottkw{if}\, e\, \ottkw{then}\,
                e_1\, \ottkw{else}\, e_2))) \in \C{k}{\theta(t)}$.}
                {\pf\ Similar to \textsc{Ty\_Unit\_Elim} but with \textsc{Op\_If\_\{True,False\}}\\
                and ${\sigma_e}_f = \empH$ and $v = \ottkw{Many}\, \ottkw{true}$ or
                $v = \ottkw{Many}\, \ottkw{false}$.}
        }
    }

    \step{}{%
        \case{%
            \textsc{Ty\_Bang\_Intro}.
        }{%
            \prove{$(\sigma, \gamma(\delta(\ottkw{Many}\, e))) \in \C{k}{\theta(!t)}$.}
            \suffices{$(\sigma, \ottkw{Many}\, \gamma(\delta(e))) \in \C{k}{!\theta(t)}$.}
        }

    }

    \begin{proof}

        \step{}{By assumption of \textsc{Ty\_Bang\_Intro}, $e = v$ for some
            value $v \neq l$, $\Gamma = \empH$ and so \\ $\den{}{}{\Theta;
            \Delta; \cdot \vdash v : t}$ by induction.}

        \step{}{\suffices{$(\empH, \ottkw{Many}\, \delta(v)) \in
            \C{k}{!\theta(t)}$ by \ref{disjoint} and \ref{inL}.}}

        \step{}{Instantiate $\den{}{}{\Theta; \Delta; \cdot \vdash v : t}$
            with $\theta, k, \delta, \gamma = [], \sigma = \empH$ to obtain
            $(\empH, \delta(v)) \in \C{k}{\theta(t)}$.}

        \step{}{Instantiate $(\empH, \delta(v)) \in \C{k}{\theta(t)}$ with $j=0$, and
            $\sigma_r = \empH$, to conclude $(\empH, v) \in \V{k}{\theta(t)}$.}

        \step{}{By definition of $\V{k}{!\theta(t)}$, \ref{valueSub} and \ref{subsetVC} we
            have $(\empH, \ottkw{Many}\, \delta(v)) \in \C{k}{!\theta(t)}$.}

    \end{proof}

    \step{}{%
        \case{%
            \textsc{Ty\_Pair\_Intro}.
        }{%
            \prove{$(\sigma, \gamma(\delta(\, ( e, e')\, ))) \in \C{k}{\theta(t \otimes t')}$.}
            \assume{Arbitrary $j \leq k$ and $\sigma_{r}$.}
            \suffices{Show whole expression either reduces to $\ottkw{err}$ or a heap and
                expression in $j$ steps.}
        }
    }

    % Intro dual of Unit_Elim
    \begin{proof}

        \step{}{\define{$(\sigma_1, \gamma_1) \in \den{L}{j}{\Gamma}$ similar to
            $(\sigma_e, \gamma_e)$ in \textsc{Ty\_Let}.}}

        \step{}{By induction,
            \begin{pfenum}
                \item $\den{}{}{\Theta; \Delta; \Gamma_1 \vdash e_1 : t_1 }$
                \item $\den{}{}{\Theta; \Delta; \Gamma_2 \vdash e_2 : t_2 }$.
            \end{pfenum}}

        \step{}{Instantiate the first with $\theta, k, \delta, \gamma_1, \sigma_1$.}

        \step{}{Therefore, $(\sigma_1, \gamma_1(\delta(e_1))) \in \den{C}{k}{\theta(t)}$.}

        \step{}{So, $(\sigma_1 \star \sigma_2, \gamma_1(\delta(e_1)))$ either reduces to
            $\ottkw{err}$ or a heap and expression in $j$ steps.}

        \step{}{\case{$\ottkw{err}$}{By \textsc{Op\_Context\_Err} and \ref{disjoint},
            so too does the whole expression. Since $j \leq k$ and $\sigma_{r}$ (for \ref{frame})
            are arbitrary, $(\sigma, \gamma(\delta(\, (e, e')\, ))) \in
            \C{k}{\theta(t \otimes t')}$.}}

        \step{}{\case{$j$ steps to another heap and expression.}By \textsc{Op\_Context} and
            \ref{disjoint}, the whole expression does the same.}

        \step{}{If it is not a value, we are done. If it is $({\sigma_1}_f, v_1)
            \in \V{k-j}{\theta(t_1)}$ by \ref{valueSub}. \suffices{By \ref{stepInC},
            $({\sigma_1}_f \star \sigma_{e_2}, \, (v_1, e_2)\, ) \in \C{k-j}{\theta(t_1 \otimes
            t_2)}$.}}

        \step{}{Instantiate the second IH with $\theta, j, \delta, \gamma_2, \sigma_2$ defined as
            per usual.}

        \step{}{So, $({\sigma_1}_f \star \sigma_2, \gamma_2(\delta(e_2)))$ either reduces to
            $\ottkw{err}$ or a heap and expression in $j$ steps.}

        \step{}{\case{$\ottkw{err}$}{By \textsc{Op\_Context\_Err}, \ref{disjoint}, so too
            does the whole expression. Since $j \leq k$ and $\sigma_r$ (for \ref{frame}) are
            arbitrary, $(\sigma_{e_2}, \, (v_1, e_2)\, ) \in \C{k-j}{\theta(t_1
            \otimes t_2)}$.}}

        \step{}{\case{$j$ steps to another heap and expression.}By \textsc{Op\_Context} and
            \ref{disjoint}, the whole expression does the same.}

        \step{}{If it is not a value, we are done. If it is $({\sigma_2}_f, v_2)
            \in \V{k-j}{\theta(t_2)}$ by \ref{valueSub}. \suffices{By \ref{stepInC},
            $({\sigma_1}_f \star {\sigma_2}_f, \, (v_1, v_2)\, ) \in \C{k-2j}{\theta(t_1
            \otimes t_2)}$.}}

        \step{}{By \ref{subsetKJ} and \ref{subsetVC}, $({\sigma_1}_f \star {\sigma_2}_f, \,
            (v_1, v_2)\, ) \in \V{k-j}{\cdot} \subseteq \V{k-2j}{\cdot} \subseteq \C{k-2j}{\cdot}$
            as needed.}

    \end{proof}

    \step{}{%
        \case{%
            \textsc{Ty\_Lambda}.
        }{%
            \prove{$(\sigma, \gamma(\delta(\ottkw{fun}\, x : t \rightarrow e))) \in
                \C{k}{\theta(t \multimap t')}$.}

            \suffices{By 6, to show $\ldots \in \V{k}{\theta(t \multimap t')}$.}

            \assume{Arbitrary $j<k$, $(\sigma_v, v) \in \V{j}{\theta(t)}$ such that $\sigma \star
                \sigma_v$ is defined.}

            \suffices{$(\sigma \star \sigma_v, \gamma(\delta(\ottkw{fun}\, x : t \rightarrow
                e))\, v) \in \C{j}{\theta(t')}$.}

            \suffices{$(\sigma \star \sigma_v, \gamma(\delta(e)) [ x / v]) \in \C{j}{\theta(t')}$.}

        }
    }

    \begin{proof}

        \step{}{By induction, $\den{}{}{\Theta; \Delta; \Gamma, x : t \vdash e}$.}

        \step{}{Instantiate it $\theta, j-1, \gamma[x \mapsto v], \sigma_v \star \sigma$.}

        \step{}{Hence, $(\sigma_v \star \sigma, \gamma[x \mapsto v](\delta(e))) \in \C{j-1}{\theta(t)}$.}

        \step{}{By \ref{disjoint}, we are done.}

    \end{proof}

    \step{}{%
        \case{%
            \textsc{Ty\_App}.
        }{%
            \prove{$(\sigma, \gamma(\delta(\,e \,e'\,))) \in \C{k}{\theta(t)}$.}
            \assume{Arbitrary $j$ and $\sigma_r$ such that $\sigma \star \sigma_r$ defined.}
            \suffices{Show whole expression either reduces to $\ottkw{err}$ or a heap and
            expression in $j$ steps.}
        }
    }

    % Intro dual of Let
    \begin{proof}

        \step{}{By induction,
            \begin{pfenum}
                \item $\den{}{}{\Theta; \Delta; \Gamma \vdash e : t' \multimap t }$
                \item $\den{}{}{\Theta; \Delta; \Gamma' \vdash e' : t' }$.
            \end{pfenum}}

        \step{}{Instantiate the first with $\theta, k, \delta, \gamma_e, \sigma_e$ as per usual
            definitions, \\ to conclude $(\sigma_e, \gamma_e(\delta(e))) \in
            \C{k}{\theta(t' \multimap t)}$.}

        \step{}{Instantiate \emph{this} with $j$ and $\sigma_{e'}$ to conclude
            $(\sigma = \sigma_e \star \sigma_{e'}, \gamma(\delta(\, e \, e'\,)))$ reduces to
            $\ottkw{err}$ or another heap and expression in $j$ steps (using \ref{disjoint}).}

        \step{}{\case{$\ottkw{err}$}{By \textsc{Op\_Context\_Err}, so too does the whole
            expression.\\ Since $j \leq k$ and $\sigma_{r}$ (for \ref{frame}) are arbitrary,
            $(\sigma, \gamma(\delta(\, e\, e'\, ))) \in \C{k}{\theta(t' \multimap t)}$.}}

        \step{}{\case{$j$ steps to another heap and expression.}By \textsc{Op\_Context},
            the whole expression does the same. \\ If it is not a value, we are done. \\
            If it is $({\sigma_e}_f, \ottkw{fun}\, x : t \rightarrow e_b) \in
            \V{k-j}{\theta(t' \multimap t)}$ by \ref{valueSub}.}

        \step{}{\suffices{By \ref{stepInC}, to show $({\sigma_e}_f \star \sigma_{e'},
            \gamma(\delta(\, (\ottkw{fun}\, x : t \rightarrow e_b)\, e'))) \in \C{k-j}{\theta(t)}$.}}

        \step{}{Instantiate the second IH with $\theta, j, \delta, \gamma_{e'}, \sigma_{e'}$
            defined as per usual.}

        \step{}{So, $({\sigma_e}_f \star \sigma_{e'}, \gamma_{e'}(\delta(e')))$
            either reduces to $\ottkw{err}$ or a heap and expression in $j$ steps.}

        \step{}{\case{$\ottkw{err}$}{By \textsc{Op\_Context\_Err} and \ref{disjoint}, so too
            does the whole expression. Since $j \leq k$ and $\sigma_r$ (for \ref{frame}) are
            arbitrary, $({\sigma_e}_f \star \sigma_{e'}, \gamma(\delta(\,
            (\ottkw{fun}\, x : t \rightarrow e_b)\, e') )) \in \C{k-j}{\theta(t)}$.}}

        \step{}{\case{$j$ steps to another heap and expression.} By \textsc{Op\_Context} and
            \ref{disjoint}, the whole expression does the same.}

        \step{}{If it is not a value, we are done. If it is, by definition of $({\sigma_e}_f,
            \ottkw{fun}\, x : t \rightarrow e_b) \in \V{k-j}{\theta(t' \multimap t)}$, we have
            $({\sigma_e}_f \star {\sigma_{e'}}_f, \gamma(\delta(\, (\ottkw{fun}\, x : t \rightarrow
            e_b)\, v'))) \in \C{k-2j}{\theta(t)}$.}

    \end{proof}

    \step{}{%
        \case{%
            \textsc{Ty\_Gen}.
        }{%
            \prove{$(\sigma, \gamma(\delta(\ottkw{fun}\, fc \rightarrow e))) \in \C{k}{\theta(\forall \, fc.\, t)}$.}
        }
    }

    \begin{proof}
    \end{proof}

    \step{}{%
        \case{%
            \textsc{Ty\_Spc}.
        }{%
            \prove{$(\sigma, \gamma(\delta(e\, [f]))) \in \C{k}{\theta(t\, [fc/f])}$.}
        }
    }

    \begin{proof}
    \end{proof}

    \step{}{%
        \case{%
            \textsc{Ty\_Fix}.
        }{%
            \prove{$(\sigma, \gamma(\delta(\ottkw{fix} (g, x : t, e : t')))) \in
                \C{k}{\theta(!(t \multimap t'))}$.}

            \suffices{to show $ \ldots \in \V{k}{!(\theta(t) \multimap
                \theta(t'))}$, by \ref{subsetVC}.}
        }
    }

    \begin{proof}

        \step{}{\assume{Arbitrary $j < k$ and $(\sigma, v) \in \V{j}{\theta(t)}$.}}

        \step{}{\suffices{$(\sigma, \emph{letManyG } g\, v) \in \C{j}{\theta(t')}$.}}

        \step{}{\pflet{$e_1 = e [ g / \ottkw{fun}\, x : t \rightarrow \emph{letManyG } g\, x ]$.}}

        \step{}{\suffices{by \ref{stepInC}, $(\sigma, (\ottkw{fun}\, x : t \rightarrow e_1)\, v)
            \in \C{j-1}{\theta(t')}$.}}

        \step{}{\suffices{by \ref{stepInC}, $(\sigma, e_1 [ x / v ]) \in \C{j-2}{\theta(t')}$.}}

        \step{}{By induction, we have $\den{}{}{\Theta; \Delta, g : t \multimap t';
            x : t \vdash e : t'}$.}

        \step{}{Instantiate this with $\theta, j-2, \delta[g \mapsto \ottkw{fun}\, x : t
            \rightarrow e_1 ], \gamma = [ x \mapsto v ], \sigma$ (???).
            \prove{$(\sigma, \ottkw{fun}\, x : t \rightarrow e_1) \in \V{j-2}{\theta(t)
            \multimap \theta(t')}$.}}

            \begin{proof}

                \step{}{\suffices{by \ref{stepInC}, $(\sigma', e_1 [ x / v']) \in
                    \C{j-2}{\theta(t')}$ for arbitrary $(\sigma', v') \in \V{j-2}{\theta(t)}$.}}

                \step{}{We can again use the induction hypothesis $\den{}{}{\Theta;
                        \Delta, g : t \multimap t'; x : t \vdash e : t'}$.}

                \step{}{But since it's true for $\C{0}{\cdot}$ (base case), it's true
                    by induction ???}

            \end{proof}

            \step{}{Lastly, we show $\delta(\gamma(e)) = e_1 [ x / v ]$, which follows
                by their definitions, \\ to conclude $(\sigma, e_1 [ x / v ]) \in
                \C{j-2}{\theta(t')}$.}

    \end{proof}

    \step{}{%
        \case{%
            \textsc{Ty\_Var\_Lin}.
        }{%
            \prove{$(\sigma, \gamma(\delta(x))) \in \C{k}{\theta(t)}$.}
        }
    }

    \begin{proof}
        \step{}{$\Gamma = \{ x : t\}$ by assumption of \textsc{Ty\_Var\_Lin}.}
        \step{}{\suffices{$(\sigma, \gamma(x)) \in \C{k}{\theta(t)}$ by \ref{disjoint}.}}
        \step{}{By \ref{inL}, there exist $(\sigma_x, v_x) \in \V{k}{\theta(t)}$, such
            that $\sigma=\sigma_x$ and $\gamma = [x \mapsto v_x]$.}
        \step{}{Hence, $(\sigma_x, v_x) \in \C{k}{\theta(t)}$, by \ref{subsetVC}.}
    \end{proof}

    \step{}{%
        \case{%
            \textsc{Ty\_Var}.
        }{%
            \prove{$(\sigma, \gamma(\delta(x))) \in \C{k}{\theta(t)}$.}
        }
    }

    \begin{proof}
        \step{}{$x : t \in \Delta$ and $\Gamma = \empH$ by assumption of \textsc{Ty\_Var}.}
        \step{}{\suffices{$(\empH, \delta(x)) \in \C{k}{\theta(t)}$ by \ref{disjoint} and \ref{inL}.}}
        \step{}{By \ref{inI}, there exists $v_x$ such that $(\empH, v_x) \in \V{k}{\theta(t)}$.}
        \step{}{Hence, $(\empH, v_x) \in \C{k}{\theta(t)}$, by \ref{subsetVC}.}
    \end{proof}

    \step{}{%
        \case{%
            \textsc{Ty\_Unit\_Intro}.
        }{%
            \prove{$(\sigma, \gamma(\delta(\, ()\,))) \in \C{k}{\theta(\Unit)}$.\\}
        }
    }

    \step{}{%
        \case{%
            \textsc{Ty\_Bool\_True}, \textsc{Ty\_Bool\_False}, \textsc{Ty\_Int\_Intro}, \textsc{Ty\_Elt\_Intro}.
        }{%
            Similar to \textsc{Ty\_Unit\_Intro}.
        }
    }

\end{proof}
