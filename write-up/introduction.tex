\chapter{TO DO: Introduction}

% No resetting page numbers: ISO 7144
% \pagenumbering{arabic}
% \setcounter{page}{0}

\begin{guidance}
    This is the introduction where you should introduce your work.  In
    general the thing to aim for here is to describe a little bit of the
    context for your work --- why did you do it (motivation), what was the
    hoped-for outcome (aims) --- as well as trying to give a brief
    overview of what you actually did.

    It's often useful to bring forward some ``highlights'' into
    this chapter (e.g.\ some particularly compelling results, or
    a particularly interesting finding).

    It's also traditional to give an outline of the rest of the
    document, although without care this can appear formulaic
    and tedious. Your call.
\end{guidance}

\prechapter{%
    In this thesis, I will argue that linear types are an appropriate,
    \emph{type-based formalism} for the problem of \emph{efficient}
    matrix-expression compilation. I will show that framing the problem using
    linear types can help \emph{reduce bugs} by making precise and explicit the
    informal, ad-hoc practices typically employed by human experts and linear
    algebra \emph{compilers} and automate checking them. As evidence for this
    argument, I will show programs written with this safety, precision and
    explicitness (1) can be just as pleasant and convenient for a programmer as
    less efficient, but higher-level linear algebra libraries and (2) perform
    just as \emph{efficiently and predictably} as lower-level, less readable
    and more error-prone linear algebra libraries.
}%

\section{Overview of Problem}

\section{Contributions}
