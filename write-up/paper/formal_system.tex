\section{Formal System}\label{sec:formal_system}

\subsection{Core Type Theory}

The full typing rules are in Appendix \ref{subsec:static_sem}, but the key
ideas are as follow.

A typing judgement consists of $ \Theta; \Delta; \Gamma
\vdash e : t$.

$\Theta$ is the environment that tracks which fractional permission variables
in scope. Fractional permissions (the \textsf{Perm} judgement) and types (the
\textsf{Type} judgement) are \emph{well-formed} if all of their free fractional
variables are in $\Theta$.

$\Delta$ is the environment storing non-linearly or \emph{inuitionistically}
typed variables.

$\Gamma$ is the environment storing linearly typed variables.  Note that rules
for typing $()$, booleans, integers and elements are typed with respect to an
\emph{empty} linear environment: this means no linear values are needed to
produce a value of those types.

\[
    \ottdruleTyXXUnitXXIntro{}
\]

Conversely, whenever two or more subexpressions need to be typed, they must
consume a disjoint set of linear values (pairs, let-expressions).  In the case
of if-expressions, both branches must consume the same set of linear values
(disjoint to the ones used to evaluate the condition).

\[
    \ottdruleTyXXBoolXXElim{}
\]

The \highl{Many} introduction and elimination rules are very important.
Producing !-type values may only be done if the expression inside is a
syntactic value which is not a location. This allows all safely duplicable
resources, including functions which capture non-linear resources from their
environments, but prevents producing aliases of (pointers to) arrays and
matrices. This is exactly the same as value-restriction from the world of
parametric polymorphism.

\[
    \ottdruleTyXXBangXXIntro{}
\]

Consuming a !-type value \emph{moves it} from the linear environment $\Gamma$
and \emph{into} the intuitionistic environment $\Delta$. This is exactly why
$\mathbf{let}\ !x = e_1\ \mathbf{in}\ e_2$ desugars to $\mathbf{let\ Many}\ x =
e_1\ \mathbf{in\ } \mathbf{let\ Many}\ x = \mathbf{Many}\ (\mathbf{Many}\ x)\
\mathbf{in}\ e_2$.

\[
    \ottdruleTyXXBangXXElim{}
\]

Rules \textsc{Ty\_Gen} and \textsc{Ty\_Spc} are for fractional permission
generalisation and specialisation respectively. They allow the definition and
use of functions that are polymorphic in the fractional permission index of
their results and one or more of their arguments.

\[
    \ottdruleTyXXGen{} \qquad\qquad \ottdruleTyXXSpc{}
\]

Rule \textsc{Ty\_Fix} shows how recursive functions are typed. Even though
recursive functions are fully annotated, type checking them is interesting for
two reasons: to type check the body of the fixpoint, the type of the recusive
function is in the \emph{intuitionistic} environment $\Delta$ (without this,
you would not be able to write a base case) whilst the argument and its type
are the \emph{only things in the linear environment} $\Gamma$. The latter means
that recursive functions can be type checked in an empty environment (thus be
wrapped in \highl{Many} and used zero or multiple times).

\[
    \ottdruleTyXXFix{}
\]

Lastly, types of almost all \lang\ primitives, as embedded in OCaml's type
system, are shown in Appendix \ref{subsec:primitives}, with some similar ones
(like those for binary arithmetic operators) omitted for brevity. The
main difference between the OCaml type of a primitive like \highl{gemm} and its
\lang\ counterpart is the inclusion of explicit `$\forall$'s.  So,
\highl{float bang -> ('a mat * bool bang) -> ('b mat * bool bang) -> float
bang -> z mat -> ('a mat * 'b mat) * z mat}
will correspond to \\
$!\ottkw{elt} \multimap \forall\, x.\ x \ \ottkw{mat} \ \otimes \ !\ottkw{bool}
\multimap \forall\, y.\  y \ \ottkw{mat} \ \otimes \ !\ottkw{bool} \multimap
\ !\ottkw{elt} \multimap z\ \ottkw{mat} \multimap ( x \ \ottkw{mat} \ \otimes y
\ \ottkw{mat} ) \ \otimes z\ \ottkw{mat}$

\subsection{Dynamic Semantics}\label{subsec:semantics}

The full, small-step transition relation is in Appendix \ref{subsec:dyn_sem},
but the key ideas are as follow.

Heaps ($\sigma$) are multisets containing triples of an abstract location $l$,
a fractional permission $f$ and sized matrices $m_{n,k}$. The notation $l
\mapsto_f m_{k_1, k_2}$ should be read as ``location $l$ represents $f$
ownership over matrix $m$ (of size $k_1 \times k_2$)''.  Each heap-and-expression
either steps to another heap-and-expression or a runtime error $\mathbf{err}$.
In the full grammar definition we see a definition of values and contexts in
the language.

We draw the reader's attention to the definitions relating to fractional
permissions. Specifically, unlike a lambda, the body of a $\ottkw{fun}\, f\!c
\rightarrow \_$ must be a syntactic value. The context $\ottkw{fun}\, f\!c
\rightarrow [-]$ means expressions can be reduced inside a fractional
permission generalisation. This is to emphasize that fractions are merely
\emph{compile-time constructs} and do not affect runtime behaviour. Correct
usage of fractions is enforced by the type system, so programs do not get
stuck. Fractional permissions are specialised using substitution over both the
heap and an expression (\textsc{Op\_Frac\_Perm}).
\[
    \ottdruleOpXXFracXXPerm{}
\]

Like with the static semantics, the interesting rules in the dynamic semantics
are those relating to primitives. Creating a matrix ($\ottkw{matrix}\ k_1\
k_2$) successfully (\textsc{Op\_Matrix}) requires non-negative dimensions and
returns a (fresh) location of a matrix of those dimensions, extending the heap
to reflect that $l$ represents a complete ownership over the new matrix.
\[
    \ottdruleOpXXMatrix{}
\]

Dually, \textsc{Op\_Free}, requires a location represent complete ownership
before removing it and the matrix it points to from the heap.
\[
    \ottdruleOpXXFree{}
\]

Choosing a multiset representation as opposed to a set allows for two
convenient invariants: multiplicity of a triple $l \mapsto_f m_{k_1, k_2}$ in
the heap corresponds to the number of aliases of $l$ in the expression with
type $f\ \ottkw{mat}$ and the sum of all the fractions for $l$ will always be
$1$ (for a closed, well-typed expression). With this in mind, the rules
\textsc{Op\_Share} and \textsc{Op\_Unshare\_Eq} are fairly natural.
\[
    \ottdruleOpXXShare{} \\
\]
\[
    \ottdruleOpXXUnshareXXEq{}
\]

Combining all of these features, we see that \textsc{Op\_Gemm\_Match} requires
that the location being updated ($l_3$) has complete ownership of over matrix
$m_3$ and can thus change what value it stores to $m_1 m_2 + m_3$. In
particular, this places no restriction on $l_2$ and $l_3$: they could be
$\ottkw{share}$d aliases of the same matrix. Transition rules for other
primitives (omitted) follow the same structure: $\mapsto_1$ for any locations
that are written to and $\mapsto_{f\!c}$ for anything else.
\[
    \ottdruleOpXXGemmXXMatch{}
\]

\subsection{Logical Relation}

First, we define an interpretation of heaps with fractional permissions in the
style of Bornat et. al~\cite{bornat} (interpreting the multiset as a partial
map from locations to the sum of all its associated fractions and a matrix) as
well as the n-fold iteration of $\rightarrow$.
\[
    \den{H}{}{\sigma} = \bigstar_{(l,f,m) \in \sigma} [ l \mapsto_f m ]
\]
where
\[
    (\varsigma_1 \star \varsigma_2)(l) \equiv
    \begin{cases}
        \varsigma_1(l) & \textrm{if } l \in \dom(\varsigma_1) \wedge l \notin \dom(\varsigma_2) \\
        \varsigma_2(l) & \textrm{if } l \in \dom(\varsigma_2) \wedge l \notin \dom(\varsigma_1) \\
        (f_1 + f_2, m) & \textrm{if } (f_1, m) = \varsigma_1(l) \wedge (f_2, m) = \varsigma_2(l) \wedge f_1 + f_2 \leq 1 \\
        \textrm{undefined} & \textrm{otherwise}
    \end{cases}
\]

We then define a step-indexed logical relation in the style of Morrisett et.
al~\cite{morrisett}. $(\varsigma, v) \in \V{k}{t}$ means it takes a heap with
exactly $\varsigma$ resources to produce a value $v$ of type $t$ in at most $k$
steps. So, something like a $\ottkw{unit}$ or a $!t$ need no resources, whereas
a $f\, \ottkw{mat}$ needs exactly $f$ ownership of a some matrix and a pair
needs a $\star$ combination of the heaps required for each component.
\begin{align*}
  \V{k}{ \Unit } &= \{ (\empH, \ast) \} \\
  \V{k}{ f \, \Mat } &= \{ (\{ l \mapsto_{2^{-f}} \_ \} , l) \} \\
  \V{k}{ \Bang t } &= \{ (\empH, \Many\, v) \mid (\empH, v) \in \V{k}{t} \} \\
  \V{k}{ t_1 \otimes t_2 } &= \{ (\varsigma_1 \star \varsigma_2, ( v_1, v_2 )) \mid (\varsigma_1, v_1) \in \V{k}{t_1} \wedge (\varsigma_2, v_2) \in \V{k}{t_2} \}
\end{align*}

The definition of $\V{k}{\forall f\!c.\ t}$ says a value and heap
must be the same regardless of what fraction is substituted into both; the
$k-1$ is to take into account fraction specialisation takes ones step
(\textsc{Op\_Spc}).
\[
    \V{k}{ \forall f\!c.\  t } = \{ (\varsigma, \ottkw{fun}\, f\!c \rightarrow \, v) \mid \forall f.\ (\varsigma [ f\!c / f ], v [ f\!c / f ]) \in \V{k-1}{ t [ f\!c / f ] } \}
\]

To understand the definition of $\V{k}{t' \multimap t}$, we must first look at
$\C{k}{t}$, the computational interpretation of types. Intuitively, it is a
combination of a frame rule on heaps (no interference), type-preservation and
termination (in $j < k$ steps) to either an error or a heap-and-expression,
with the further condition that if the expression is a syntactic value then it
is also one semantically.
\begin{align*}
    \C{k}{ t } &= \{ (\varsigma_s, e_s) \mid \forall \, j < k, \sigma_r.\ \varsigma_s \star \varsigma_r \textrm{ defined } \Rightarrow \langle \sigma_s + \sigma_r, e_s \rangle \rightarrow^j \ottkw{err}\ \vee \exists \sigma_f, e_f.\\
               & \qquad \qquad \langle \sigma_s + \sigma_r, e_s \rangle \rightarrow^j \langle \sigma_f + \sigma_r, e_f \rangle \wedge ( e_f \textrm { is a value } \Rightarrow ( \varsigma_f \star \varsigma_r, e_f ) \in \V{k-j}{t} ) \}
\end{align*}

In this light, $\V{k}{t' \multimap t}$ simply says
that $v$ is a function and that evaluating the application of it to any
argument (of the correct type, requiring its own set of resources, bounded by
$k$ steps) satisfies all the aforementioned properties.
\begin{align*}
    \V{k}{ t' \multimap t } &= \{ (\varsigma_v, v ) \mid ( v \equiv \ottkw{fun}\, x : t' \rightarrow e \vee v \equiv \ottkw{fix} (g, x : t' , e : t) ) \, \wedge\\
                            & \qquad \qquad \forall j \leq k, (\varsigma_{v'}, v') \in \V{j}{ t' }.\ \varsigma_v \star \varsigma_v' \textrm{ defined } \Rightarrow (\varsigma_v \star \varsigma_v', v\, v') \in \C{j}{t} \}
\end{align*}

The interpretation of typing environments $\Delta$ and $\Gamma$ are with
respect to an arbitrary substitution of fractional permissions $\theta$. Note
that only the interpretation of $\Gamma$ involves a (potentially) non-empty heap.
\begin{align*}
    \den{I}{k}{ \Delta, x : t } \theta &= \{ \delta[x \mapsto v_x ] \mid \delta \in \den{I}{k}{\Delta}\theta \wedge (\empH, v_x) \in \V{k}{\theta(t)} \} \\
    \den{L}{k}{ \Gamma, x : t } \theta &= \{ (\varsigma \star \varsigma_x, \gamma[x \mapsto v_x ]) \mid (\varsigma, \gamma) \in \den{L}{k}{\Gamma}\theta \wedge (\varsigma_x, v_x) \in \V{k}{\theta(t)} \}
\end{align*}

And so the final semantic interpretation of a typing judgement simply
quantifies over all possible fractional permission substitutions $\theta$,
linear value substitutions $\gamma$, intuitionistic value substitutions
$\delta$ and heaps $\sigma$.  Note that, $\varsigma \equiv \den{H}{}{\theta(\sigma)}$.
\begin{align*}
\den{}{k}{ \Theta; \Delta ; \Gamma \vdash e : t } &= \forall \theta, \delta, \gamma, \sigma.\ \Theta = \dom(\theta) \wedge (\varsigma, \gamma) \in \den{L}{k}{ \Gamma }\theta \wedge \delta \in \den{I}{k}{ \Delta }\theta \Rightarrow \\
                                                     & \qquad \qquad (\varsigma, \theta(\delta(\gamma(e)))) \in \C{k}{ \theta(t) }
\end{align*}

\subsection{Soundness Theorem}

\textbf{Theorem 1 (The Fundamental Lemma of Logical Relations)}
\[
    \forall \Theta, \Delta, \Gamma, e, t.\ \Theta; \Delta ; \Gamma \vdash e : t \Rightarrow
    \forall k.\ \den{}{k}{ \Theta; \Delta ; \Gamma \vdash e : t }
\]

To prove the above theorem, we need several lemmas; the interesting ones are:
the moral equivalent of the frame rule (\ref{frame}), monotonicity for the
step-index (\ref{subsetKJ}), splitting up environments corresponds to splitting
up heaps (\ref{restriction}) and heap-and-expressions take the same steps of
evaluation under any substitution of their free fractional permissions
(\ref{fracPermSub}).

The proof proceeds by induction on the typing judgement.  The case for
\textsc{Ty\_Fix} is the reason we quantify over the step-index $k$ in the
\emph{conclusion} of the soundness theorem. It allows us to then induct over
the step-index and assume exactly the thing we need to prove at a smaller index.

The case for \textsc{Ty\_Gen} follows a similar pattern, but has the extra
complication of reducing an expression with an arbitrary fractional permission
variable in it, and then instantiating it at the last momemnt to conclude,
which is where \ref{fracPermSub} (heap-and-expressions take the same steps of
evaluation under any substitution of their free fractional permissions) is
used.

The rest of the cases are either very simple base cases (variables, unit,
boolean, integer or element literals) or follow very similar patterns; for
these, only \textsc{Ty\_Let} is presented in full and other similar cases
simply highlight exactly what would be different.  The general idea is to split
up the linear substitution and heap along the same split of $\Gamma/\Gamma'$,
then (by induction) use $\C{k}{-}$ and one `half' of the  linear substitution
and heap to conclude the `first' sub-expression either takes $j< k$ steps to
$\ottkw{err}$ or another heap-and-expression.

In the first case, you use \textsc{Op\_Context\_Err} to conclude the whole
let-expression does the same. Similarly we use \textsc{Op\_Context} $j$ times
in the second case. However, a small book-keeping wrinkle needs to be taken
care of in the case that the heap-and-expression turns into a value in $i \leq
j$ steps: \textsc{Op\_Context} is not functorial for the n-fold iteration of
$\rightarrow$.  Basically, the following is not quite true:
\[
\ottdrule{%
    \ottpremise{\langle  \sigma  \ottsym{,}  \ottnt{e}  \rangle  \rightarrow^j  \langle  \sigma'  \ottsym{,}  \ottnt{e'}  \rangle}%
    }{
    \langle  \sigma  \ottsym{,}  \ottnt{C}  \ottsym{[}  \ottnt{e}  \ottsym{]}  \rangle  \rightarrow^j  \langle  \sigma'  \ottsym{,}  \ottnt{C}  \ottsym{[}  \ottnt{e'}  \ottsym{]}  \rangle}{%
    {\ottdrulename{Op\_Context}}{}%
}
\]
because after the $i$ steps, we need to invoke \textsc{Op\_Let\_Var} to proceed
evalution for any remaining $j-i$ steps. After that, it suffices to use the
induction hypothesis on the second sub-expression to finish the proof.  To do
so, we need to construct a valid linear substitution and heap (i.e., one in
$\den{L}{k}{\Gamma', x : t}\theta$). We take the other `half' of the linear
substitution and heap (from the inital split at the start) and extend it with
$[x \mapsto v]$, (where $x$ is the variable bound in the let-expression and $v$
is the value we assume the first sub-expression evaluated to in $i$ steps).

