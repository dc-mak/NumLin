\section{Introduction}

\lang\ is a functional programming language designed to express the APIs of
low-level linear algebra libraries (such as BLAS/LAPACK) safely and explicitly.
It does so by combining linear types, fractional permissions, runtime errors
and recursion into a small, easily understandable, yet expressive set of core
constructs.

\lang\ allows a novice to understand and work with complicated linear algebra
library APIs, as well as point out subtle aliasing bugs and reduce memory usage
in existing programs. In fact, we were able to use \lang\ to find linearity and
aliasing bugs in a linear algebra program that was \emph{generated} by another
program \emph{specifically designed to translate matrix expressions into an
efficient sequence of calls linear algebra routines}. We were also able to
reduce the number of temporaries used by the same program, using \lang's type
system to guide us.

\lang's implementation supports several syntactic conveniences as well as a
\emph{usable} integration with real OCaml libraries.

\subsection{Contributions}

In this paper
\begin{itemize}
    \item we describe \lang, a linearly typed language for linear algebra programs
    \item we illustrate that \lang's design and features are well-suited to its
        intended domain with progressively sophisticated examples
    \item we prove \lang's soundness, using a step-indexed logical relation
    \item we describe a very simple, unification based type-inference algorithm
        for polymorphic fractional permissions (similar to ones used for
        parametric polymorphism), demonstrating an alternative approach to
        dataflow analysis \cite{bierhoff}
    \item we describe an implementation that is both compatible with and usable
        from existing code
    \item we show an example of how using \lang\ helped highlight linearity
        and aliasing bugs, and reduce the memory usage of a \emph{generated}
        linear algebra program
    \item we show that using \lang, we can achieve parity with C for linear
        algebra routines, whilst having much better static guarantees about the
        linearity and aliasing behaviour of our programs.
\end{itemize}

