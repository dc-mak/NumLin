\documentclass[11pt]{article}

\usepackage[top=1.1in, bottom=1.25in, left=1.25in, right=1.25in]{geometry}
\usepackage{booktabs}
\usepackage{charter}
\usepackage{parskip}
\usepackage{setspace}
\usepackage[dvipsnames]{xcolor}
\usepackage[nottoc]{tocbibind}
\usepackage[utf8]{inputenc}
\usepackage[outputdir=../build]{minted}
\usepackage{hyperref}

% \linespread{1.5}
\newcommand{\guidance}[1]{\textsl{\textcolor{Gray}{#1}}}

\begin{document}

\centerline{\Large Applications of Linear Types}
\vspace{2em}
\centerline{\Large \emph{A Part III project proposal}}
\vspace{2em}
\centerline{\large D. C. Makwana (\emph{dcm41}), Trinity College}
\vspace{1em}
\centerline{\large Project Supervisor: Dr.~N.~K.~Krishnaswami, Dr~S.~Dolan}
\vspace{1em}

\begin{abstract}
\textsl{%
    Complex resource management is a challenge for programmers.  Linear types
    allow the compiler and the user to statically keep track of the resources
    that an implementation uses, and offer a promising solution to
    resource management. However, they have not made their way into mainstream
    programming languages. I aim to take a step towards this with an
    incremental approach, by allowing users to use linear types via an OCaml
    library that offers them the ability to manipulate primitives of the Owl
    numerical library in a safe and efficient manner, by inferring and removing
    redundant copies.
}%

\end{abstract}

\section{Introduction, approach and outcomes}

% \guidance{%
%     Provide an introduction to your project or essay. In particular, try to
%     motivate the work and explain the relevant context (general background, as
%     well as sufficient detail about any related work).
% }

% \guidance{%
%     What's the basic idea and approach? What are you thinking of doing, and how
%     is it going to solve the problem (or need) you've identified. What are you
%     going to ``produce''?  A project will typically produce one (or perhaps
%     more) of the following: a piece of software, an evaluation of a published
%     result, a proof, or the design (and perhaps the construction of) a new piece
%     of hardware. An essay will typically either review and critique a particular
%     area of the academic literature, or evaluate a published result or proof.
%     Try to name the specific things (and describe them) in this part of the
%     proposal -- this will allow you to refer to them in the next.
% }

Simply stated, a linear value is a value that must be used once and only once.
More abstractly, it is the type-theory corresponding to Girard's Linear Logic.
Examples of linear types in the real world are the recent efforts to extend GHC
with Linear Types~\cite{retrofitting} for safe, pure-functional streaming: a
pointer into a stream must be processed before moving onto the next element (one
use) but crucially, you cannot keep this pointer and refer to it again (only one
use).  Rust implements linear types behind the scenes but presents \emph{affine}
types to the programmer: if a destructor is not called explicitly, the compiler
inserts ones when it infers the value's lifetime has ended. Apart from memory
(which enables zero-copy buffers and state-management in a pure functional
language), linear types can also be used to control other resources such as
files, network communications, pointers with capabilities and concurrent
channels.

My project will focus on memory, but not manipulating raw memory block
primitives, but ensuring the primitive types (such as large matrices) that
Owl~\cite{owl} manipulates are used efficiently. Owl provides an immutable
interface and so performs several redundant copies during a computation. It does
provide more efficient implementations of common numeric algorithms as
primitives which are hand-optimised and abstract away any unsafe calls involved.
This approach trades performance for conciseness, safety and ease of reasoning.

The opposite approach -- to rely on the programmer to note and remove redundant
copies -- is used extensively by Fortran's BLAS~\cite{blas} and Boost's
uBLAS~\cite{ublas}. They offer many different conventions for each operation to
meet every situation. For example, a function implementing a binary operation on
two matrices may return a newly-allocated matrix containing the result or modify
one of two or three input arguments to hold the result (if there is no aliasing,
like in \texttt{noalias(C)~=~prod(A,~B)} where the \texttt{=} operator is
overloaded) and even specify temporary intermediate results with annotations (as
with \texttt{prod(A,~prod<temp\_type>(B,C))}).

I hypothesise that it is possible to have the best of both worlds: write your
code as you wish and then have the computer statically determine the most
space-efficient sequence of copies and in-place writes that produces the
desired result. A small example to illuminate the point follows. Suppose we
have $n\times n$ matrices A, B and C. Knowing all those matrices are used
once and only once, we would then infer that we can evaluate the expression
\mintinline{ocaml}{(A + B) + C} \emph{in-place}, say, using the memory that A
occupies.

\begin{minted}{ocaml}
    let y  = (A + B) + C in    let y  = ((A + B) + C) * A in
    (* is the same as *)       (* is the same as *)
    let i1 = A + B  in         let (A1, A2) = dup(A) in
    let i2 = i1 + C in         let i1 = A1 + B       in
    i2                         let i2 = i1 + C       in
    (* is compiled to *)       let i4 = i2 * A2      in
    A := A + B;                i4
    A := A + C;                (* is compiled to *)
    return A                   B := B + A; B := B + C;
                               B := B * A; return B
\end{minted}

However, when an expression is used more than once, we can be more explicit and
invoke a function \emph{dup} for duplicate, to save A for use later.

It is important to note that \emph{dup} is not necessarily a copy: it is merely
a signal to the compiler via the type system that A's memory should not be
written to until both the results of the \emph{dup} (A1 and A2) are used.
Indeed, knowing that B is not used again, we could store (intermediate) results
in its memory.

\section{Workplan}
% \guidance{%
%     Project students have approximately 28 weeks between the submission of
%     the proposal, and the submission of the dissertation. Essay students have
%     approximately 14 weeks.  This section should account for what you intend
%     to do during that time. One approach would be to divide the time into
%     two-week chunks, and describe the work to be done (and, as relevant,
%     milestones to be achieved) in each chunk. You should leave one chunk for
%     writing an essay or two chunks for writing a project dissertation. You
%     should leave 1 chunk for contingencies.
% }

The output of this project will be a library that will, given a linear algebra
expression, produce OCaml code to compute the expression using mutability and
unsafe functions that will, thanks the static guarantees provided by linear
types, be safe, correct and perform well (less memory consumption, fewer
dynamic checks). Evaluation will consist of comparing test programs against
Owl's default immutable interface, its lazy interface, Julia and Python/Numpy.

During implementation, I will also explore other questions such as: if a matrix
does need to be copied, then, when during execution is it best to do so?  What
is necessary to support the class of numerical algorithms like the small
example above, such as Kalman filters that although in their mathematical
expression appear to use a value twice, can be implemented to have values
modified in-place? Numerical libraries and lazy interfaces (that use mutability
behind the scenes by reference counting) tend to perform poorly for allocating
lots of small matrices (for example, in robust linear regression); will
statically determining memory usage improve their performance?

\vspace*{\fill}
\begin{table}[H]
\centering
\begin{tabular}{ll}
    \toprule

    Date & Work \\

    \midrule

    % December
    04-12-2017 & Set-up build environments (using Nix) Git repository          \\
               & continuous-integration and testing framework. Start           \\
               & designing explicitly-typed DSL (domain-specific language).    \\
                                                                               \\

    18-12-2017 & Finish designing DSL and write a tree-interpreter for it      \\
               & that outputs matrices using Owl.                              \\
                                                                               \\

    % January
    01-01-2018 & Write a lexer/parser and \emph{compiler} for this language.   \\
                                                                               \\

    15-01-2018 & Review with supervisor/break for assignments.                 \\
                                                                               \\

    29-01-2018 & Start with basic elaboration and inference.                   \\
                                                                               \\

    % February
    12-02-2018 & Buffer.                                                       \\
                                                                               \\

    26-02-2018 & Project progress review. Begin discussion of advanced         \\
               & extensions such as staging or further inference and analysis. \\
                                                                               \\

    % March
    12-03-2018 & `One-minute madness: project `elevator pitch''.               \\
               &  Start with chosen extension. Submit to ICFP.                 \\
                                                                               \\

    26-03-2018 & Plan evaluation and start and write-up.                       \\
                                                                               \\

    % April
    09-04-2018 & Buffer.                                                       \\
                                                                               \\

    23-04-2018 & Perform evaluation.                                           \\
                                                                               \\

    % May
    07-05-2018 & Write-up.                                                     \\
                                                                               \\

    21-05-2018 & Project title changes (25\textsuperscript{th}).               \\
               & Final changes to code and thesis. Prepare for POPL.           \\
                                                                               \\

    % June
    01-06-2018 & Submission Deadline.                                          \\
                                                                               \\

    12-06-2018 & MPhil and Part III Project presentations.                     \\
                                                                               \\

  \bottomrule

\end{tabular}
\end{table}
\vspace*{\fill}
\newpage
\bibliographystyle{plain}
\bibliography{proposal}

\end{document}
