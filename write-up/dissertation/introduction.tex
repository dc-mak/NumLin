\chapter{Introduction}

% No resetting page numbers: ISO 7144
% \pagenumbering{arabic}
% \setcounter{page}{0}

% \begin{guidance}
%     This is the introduction where you should introduce your work.  In
%     general the thing to aim for here is to describe a little bit of the
%     context for your work --- why did you do it (motivation), what was the
%     hoped-for outcome (aims) --- as well as trying to give a brief
%     overview of what you actually did.
% 
%     It's often useful to bring forward some ``highlights'' into
%     this chapter (e.g.\ some particularly compelling results, or
%     a particularly interesting finding).
% 
%     It's also traditional to give an outline of the rest of the
%     document, although without care this can appear formulaic
%     and tedious. Your call.
% \end{guidance}

Linear types allow the compiler and programmer to statically keep track of the
resources that a program uses, thus offering a promising solution to the
problems associated with complex resource management. However, they have not
made their way into many mainstream programming languages, in the same way
parametrically polymorphic types have. To illustrate their simplicity and
power, I implemented an OCaml library that allows users to learn about and
become familiar with linear types, specifically in the context of linear
algebra programs.

The main contributions of this thesis are:

\begin{itemize}

    \item An \textbf{original, usable implementation} of a type system that can express
        aliasing, read/write permissions, memory allocation, re-use and deallocation.

    \item An \textbf{original} demonstration of how that type system can be
        \textbf{applied to the APIs} of linear algebra libraries and the
        \textbf{benefits} of doing so.

    \item Many \textbf{new} examples of how that type system can \textbf{automatically
        check} for common aliasing, read/write permission, memory allocation,
        re-use and deallocation \textbf{errors} in the context of
        linear algebra programs.

    \item An \textbf{original} demonstration of how that type system can be
        \textbf{used} for \textbf{matrix expression compilation}.

    \item \textbf{New and readable implementations} of \textbf{non-trivial}
        linear algebra programs that \textbf{take advantage} of said type
        system.

    \item A \textbf{new solution} to the \textbf{dichotomy} of
        \textbf{readability}, \textbf{ease of reasoning and safety} of
        high-level linear algebra libraries versus the
        \textbf{memory-efficiency} of low-level linear algebra libraries.

    \item A \textbf{new library design} to provide said solution in a way that
        \textbf{integrates well} with existing OCaml code and linear algebra
        libraries.

\end{itemize}
