\newpage
{\normalfont\Huge\sffamily\bfseries Abstract}
\vspace{24pt}

In this thesis, I argue that linear types are an appropriate, \emph{type-based
formalism} for expressing aliasing, read/write permissions, memory allocation,
re-use and deallocation, first, in the context of the APIs of linear-algebra
libraries and then in the context of matrix-expression compilation.

I show that framing the problem using linear types can help \emph{reduce bugs}
by making precise and explicit the informal, ad-hoc practices typically
employed by human experts and linear-algebra \emph{compilers} and automate
checking them.

As evidence for this argument, I show non-trivial, yet readable, linear-algebra
programs, that are safe and explicit (with respect to aliasing, read/write
permissions, memory allocation, re-use and deallocation) which (1) are more
memory-efficient than equivalent programs written using high-level
linear-algebra libraries and (2) perform just as predictably as equivalent
programs written using low-level linear-algebra libraries. I also argue
\emph{the experience} of writing such programs with linear types is
qualitatively better in key-respects. In addition to all of this, I show that
it is possible to provide such features \emph{as a library} on top of existing
programming languages and linear-algebra libraries.

\vspace*{\fill}
{\normalfont\Huge\sffamily\bfseries Acknowledgements}

As I rapidly grow stronger and smarter, I want to wish all of my friends,
supporters, enemies, haters and even the very dishonest Fake News Media, a
Sincere and Heartfelt Thank You.

\newpage
\vspace*{\fill}
