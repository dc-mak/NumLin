\section{Lemmas}

\subsection{$
    \forall \sigma_s, \sigma_r, e.\ \varsigma_s \star \varsigma_r \textrm{ defined }
    \Rightarrow \forall n.\ \langle \sigma_s, e \rangle \rightarrow^n =
    \langle \sigma_s + \sigma_r , e \rangle \rightarrow^n
$}\label{frame}

\begin{proof}

    \suffices{By induction on $n$, consider only the cases $\langle \sigma_s, e
        \rangle \rightarrow \langle \sigma_f, e_f \rangle$ where $\sigma_s \neq
        \sigma_f$.\\}

    \pfsketch{~Only \textsc{Op\_\{Free, Matrix, Share, Unshare\_Eq,
        Gemm\_Match\}} change the heap: the rest are either parametric in the
        heap or step to an $\ottkw{err}$.\\}

    \prove{$\langle \sigma_s + \sigma_r, e \rangle \rightarrow
        \langle \sigma_f + \sigma_r, e_f \rangle$.\\}

    \step{}{\case{\textsc{Op\_Free}, $\sigma_s \equiv \sigma' + \{ l
        \mapsto_1 m \}$, $\sigma_f = \sigma'$.}{\pf~Instantiate
        \textsc{Op\_Free} with $(\sigma' + \sigma_r) + \{ l \mapsto_1 m
        \}$,\\valid because $l \notin \dom(\varsigma_r)$ by $\varsigma' \star
        [l \mapsto_1 m] \star \varsigma_r$ defined (assumption).}}

    \step{}{\case{\textsc{Op\_Matrix}}{
        \pf~Rule has no requirements on $\sigma_s$ so will also work with
        $\sigma_s + \sigma_r$.
    }}

    \step{}{\case{\textsc{Op\_Share}, $\sigma_s \equiv \sigma' + \{ l \mapsto_f
        m \}$, $\sigma_f = \sigma' + \{ l \mapsto_{\frac{1}{2} \cdot f} m \} +
        \{ l \mapsto_{\frac{1}{2} \cdot f} m \}$.}{
        \pf~Union-ing $\sigma_r$ does not remove $l \mapsto_f m$, so that can
        be split out of $ \sigma_s + \sigma_r$ as before.
    }}

    \step{}{\case{\textsc{Op\_Unshare\_Eq}, $\sigma_s \equiv \sigma' + \{ l
        \mapsto_{\frac{1}{2} \cdot f} m \} + \{ l \mapsto_{\frac{1}{2} \cdot f}
        m \}$, $\sigma_f = \sigma' + \{ l \mapsto_f m \}$.}{}}

    \begin{proof}

        \step{}{Union-ing $\sigma_r$ does not remove $l \mapsto_{\frac{1}{2}
            \cdot f} m$, so that can still be split out of $ \sigma_s + \sigma_r$.}

        \step{}{There may also be other valid splits introduced by $\sigma_r$.}

        \step{}{However, by assumption of $\varsigma_s \star \varsigma_r$
            defined, any splitting of $\sigma_s + \sigma_r$ will
            satisfy $f \leq 1$.}

    \end{proof}

    \step{}{\case{\textsc{Op\_Gemm\_Match}}{}}

    \begin{proof}

        \step{}{By assumption of $\varsigma_s \star \varsigma_r$ defined,
            either $l_1$ (or $l_2$, or both) are not in $\sigma_r$, or they are
            and the matrix values they point to are the same.}

        \step{}{The permissions (of $l_1$ and/or $l_2$) may differ, but
            \textsc{Op\_Gemm\_Match} universally quantifies over them and
            leaves them unchanged, so they are irrelevant.}

        \step{}{Only the pointed to matrix value at $l_3$ changes.}

        \step{}{\suffices{$l_3 \notin \pi_1 [ \sigma_r ]$.}}

        \step{}{By assumption of $\varsigma_s \star \varsigma_r$ defined, $l_3
            \notin \dom(\varsigma_r)$.}

        \step{}{Hence $l_3 \notin \pi_1 [ \sigma_r ]$.}

    \end{proof}

\end{proof}

\subsection{$\forall k, t.\ \V{k}{t} \subseteq \C{k}{t}$}\label{subsetVC}
Follows from definition of $\C{k}{t}$, $\rightarrow^j$ ($\forall n.\ \langle
\sigma , v \rangle \rightarrow^n \langle \sigma , v \rangle$) for arbitrary
$j \leq k$ and \ref{frame}.

\subsection{$\forall \theta, \delta, \gamma, v.\ \theta(\delta(\gamma(v)))\textrm{ is a value.}$}\label{valueSub}

$\theta$ is irrelevant because it only maps fractional permission variables to
fractional permissions. By construction, $\delta$ and $\gamma$ only map
variables to values, and values are closed under substitution.

\subsection{$
    \forall k, \sigma, \sigma', e, e', t.  \ (\varsigma', e') \in \C{k}{t} \wedge
    \langle \sigma, e \rangle \rightarrow \langle \sigma', e' \rangle
    \Rightarrow (\varsigma, e) \in \C{k+1}{t}
$}\label{stepInC}

In the lemma, and for the rest of its proof, $\varsigma = \den{H}{}{\sigma}$.

\begin{proof}

    \assume{arbitrary $j < k + 1$, and $\sigma_r$ such that $\varsigma
    \star \varsigma_r$ defined.\\}

    \step{}{\case{$j=0$. Clearly $\sigma_f = \sigma_s + \sigma_r$ and $e' = e$.}{
        Remains to show that if $e$ is a value then $(\varsigma_s \star
        \varsigma_r, e) \in \V{k}{t}$.\\
        This is true vacuously, because by assumption, $e$ is not a value.}}

    \step{}{\case{$j \geq 1$. We have $\langle \sigma, e \rangle
        \rightarrow^j\, = \langle \sigma', e' \rangle \rightarrow^{j-1}$.}{
        Instantiate $(\varsigma', e') \in \C{k}{t}$, with $j-1 < k$ and
        $\sigma_r$ to conclude the required conditions.}}

\end{proof}

\subsection{$j \leq k \Rightarrow \den{\_\ }{k}{\cdot} \subseteq \den{\_\ }{j}{\cdot}$}\label{subsetKJ}

For the rest of this proof, $\varsigma = \den{H}{}{\sigma}$.\\
Lemma~\ref{stepInC} is the inductive step for this lemma for the $\C{}{}$ case.\\
Need to prove for $\V{}{}$, by induction on $t$ and then index.

\begin{proof}

    \suffices{Consider only $t \multimap t'$ case, rest use $k$ directly on structure of type.}

    \assume{Arbitrary $j \leq k$ and $(\varsigma_{v'}, v') \in \V{k}{t
        \multimap t'}$.}

    \prove{$(\varsigma_{v'}, v') \in \V{j}{t \multimap t'}$.\\}

    \step{}{$v'$ is of the correct syntactic form (lambda or fixpoint) by
        assumption.}

    \step{}{\assume{arbitrary $j' \leq j$ and $(\varsigma_v, v) \in \V{j'}{t}$
        such that $\varsigma_{v'} \star \varsigma_v$ is defined.}}

    \step{}{\suffices{to show $(\varsigma_{v'} \star \varsigma_v, v' v) \in
        \C{j'}{t'}$.}}

    \step{}{This is true by instantiating $(\varsigma_{v'}, v') \in \V{k}{t
        \multimap t'}$ with $j' \leq k$ and $(\varsigma_v, v) \in \V{j'}{t}$.}

\end{proof}

\subsection{$\forall \Delta, \Gamma, t, k, \theta, \delta, \gamma.\ %
    \delta \in \den{I}{k}{\Delta}\theta \wedge \gamma \in \pi_2[\den{L}{k}{\Gamma}\theta]
    \Rightarrow \\ \dom(\Delta) = \dom(\delta)$ and $\dom(\Gamma) = \dom(\gamma)$}\label{samedom}

\pf~By induction on $\Delta$ and $\Gamma$.

\subsection{$\forall k, \Gamma, \Gamma', \theta, \sigma_+, \gamma_+.\ %
    (\varsigma_+, \gamma) \in \den{L}{k}{ \Gamma, \Gamma' }\theta
    \wedge \Gamma, \Gamma' \textrm{ disjoint } \Rightarrow\\
    \exists \sigma, \gamma, \sigma' , \gamma' .\ \sigma_+ = \sigma + \sigma'
    \wedge \gamma, \gamma' \textrm{ disjoint }
    \wedge \gamma_+ = \gamma \cup \gamma'\\
    \wedge (\varsigma, \gamma) \in \den{L}{k}{\Gamma}
    \wedge (\varsigma', \gamma') \in \den{L}{k}{\Gamma'} $}\label{restriction}

\pf~By induction on $\Gamma'$.

\subsection{$\forall e, \sigma, e', \sigma', \theta.\
    \langle \sigma, e \rangle \rightarrow \langle \sigma',  e' \rangle
    \Rightarrow \langle \theta(\sigma), \theta(e) \rangle \rightarrow
    \langle \theta(\sigma') , \theta(e') \rangle$}\label{fracPermSub}

\pf~By induction on $\rightarrow$.
\begin{proof}

    \step{}{\assume{Arbitrary $e, \sigma, e', \sigma', \theta$ such that 
        $\langle \sigma, e \rangle \rightarrow \langle \sigma', e' \rangle$.}}

    \step{}{\suffices{To consider only the following rules which mention
        fractional permission variables.}
        \textsc{Op\_Frac\_Perm}, \textsc{Op\_Share}, \textsc{Op\_Unshare\_(N)Eq}
        and \textsc{Op\_Gemm\_(Mis)Match}.}

    \step{}{\case{\textsc{Op\_Frac\_Perm.}}{Because substitution avoids capture,
        \\ $\langle \theta(\sigma), \theta((\ottkw{fun} fc \rightarrow v) \, [f])
         \rangle \rightarrow \langle \theta(\sigma'\, [fc/f]),
        \theta(v\, [fc/f]) \rangle$.}}

    \step{}{The rest of the cases are parametric in their use of fractional
        permission variables and so will take the same step ater any substitution.}

    \step{}{\textsc{Corollary:}
        If $\langle \sigma \, [fc/f_1], e\, [fc/f_1] \rangle \rightarrow^n
        \langle \sigma_2, e'_2 \rangle$ and
        $\langle \sigma\, [fc/f_2], e \, [fc/f_2] \rangle \rightarrow^n
        \langle \sigma_2, e'_2 \rangle$, then
        $\exists\,\sigma, e'.\ \sigma_1 = \sigma\,[fc/f_1] \wedge
        \sigma_2 = \sigma\,[fc/f_2] \wedge
        e'_1 = e'\,[fc/f_1] \wedge e'_2 = e'\,[fc/f_2]$.}

\end{proof}
