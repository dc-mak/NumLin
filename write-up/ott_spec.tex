\chapter{Ott Specification}\label{chap:ott_spec}

The following pages present a specification of the grammar and type-system used
by my project, produced using the Ott~\cite{sewell_ott} tool. It is important to note
that the type-system described here is not how it is implemented: it is easier
and clearer to describe the system as below for explaining. However, for implementing,
I found it much more and user- and debugging-friendly to:

\begin{itemize}

    \item Implement it so that the type-environment \emph{changes} as a result
        of type-checking an expression, similar to the rules shown in
        Figure~\ref{fig:example_rules}; with this, the below semantics describe
        the \emph{difference} between the environment after and before checking
        an expression. For example, in the pair-introduction rule, $\Gamma =
        \Gamma_2 - \Gamma_1$ and $\Gamma' = \Gamma_3 - \Gamma_2$, for an
        appropriate definition of $(-)$.

    \item \emph{Mark} variables as used instead of \emph{removing} them from the
        environment for better error messages.

    \item Have \emph{one} environment where variables were \emph{tagged} as
        linear and unused, linear and used, and intuitionistic. This was
        definitely an implementation convenience so that variable binding could
        be handled uniformly for linear and intuitionistic variables and
        scoping/variable look-up could be handled implicitly thanks to the
        associative-list structure of the environment. So, the variable
        rule would most accurately look like:
        \begin{prooftree}
            \AxiomC{}
            \RightLabel{\textsc{Ty\_Var}}
            \UnaryInfC{$\Theta; \Gamma, x \overset{n}{:} t \vdash x : t ; \Gamma, x \overset{n-1}{:} t$}
        \end{prooftree}
        for $n \in \{0\,\textrm{(used)}, 1\,\textrm{(unused)},
        \omega\,\textrm{(intuitionistic)}\}$, $\omega - 1 = \omega$ and  $1 - 1 = 0$.

\end{itemize}

\clearpage%
\ottall%
