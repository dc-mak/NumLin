\documentclass[a4paper,UKenglish]{lipics-v2019}
%This is a template for producing LIPIcs articles. 
%See lipics-manual.pdf for further information.
%for A4 paper format use option "a4paper", for US-letter use option "letterpaper"
%for british hyphenation rules use option "UKenglish", for american hyphenation rules use option "USenglish"
%for section-numbered lemmas etc., use "numberwithinsect"
%for enabling cleveref support, use "cleveref"
%for enabling cleveref support, use "autoref"

\bibliographystyle{plainurl}

\title{NumLin: Linear Types for Linear Algebra}

\author{Dhruv C.~Makwana}{Unaffiliated \url{dhruvmakwana.com} }{dcm41@cam.ac.uk}{[orcid]}{}

\author{Neelakantan R.~Krishnaswami}{Department of Computer Science and Technology, University of Cambridge, United Kingdom}{nk480@cl.cam.ac.uk}{[orcid]}{}

\authorrunning{D.\,C. Makwana and N.\,R. Krishnaswami}

\Copyright{Dhruv C. Makwana and Neelakantan R. Krishnaswami}% LIPIcs license is "CC-BY";  http://creativecommons.org/licenses/by/3.0/

\ccsdesc[300]{Theory of computation~Program specifications}

\keywords{numerical, linear, algebra, types, permissions, OCaml}

\supplement{\url{www.github.com/dc-mak/lt4la}}

%\nolinenumbers %uncomment to disable line numbering

%\hideLIPIcs  %uncomment to remove references to LIPIcs series (logo, DOI, ...), e.g. when preparing a pre-final version to be uploaded to arXiv or another public repository

%Editor-only macros:: begin (do not touch as author)%
\EventEditors{John Q. Open and Joan R. Access}
\EventNoEds{2}
\EventLongTitle{42nd Conference on Very Important Topics (CVIT 2016)}
\EventShortTitle{CVIT 2016}
\EventAcronym{CVIT}
\EventYear{2016}
\EventDate{December 24--27, 2016}
\EventLocation{Little Whinging, United Kingdom}
\EventLogo{}
\SeriesVolume{42}
\ArticleNo{23}

% Source code highlighting
\usepackage[outputdir=../build]{minted}
  % Magic incantation to stop minted from putting red boxes around shit
  \usepackage{etoolbox}
  \makeatletter
  \AtBeginEnvironment{minted}{\dontdofcolorbox}
  \def\dontdofcolorbox{\renewcommand\fcolorbox[4][]{##4}}
  \makeatother
\RecustomVerbatimEnvironment{Verbatim}{BVerbatim}{}
% Convenient inline syntax highlighting
\newmintinline[highl]{ocaml}{breaklines}

\usepackage{amsmath,amssymb}

% Ott Rules
\usepackage{supertabular}%
% \usepackage{geometry}%
\usepackage{ifthen}%
\usepackage{alltt}%hack%

% generated by Ott 0.25 from: semantics.ott
\newcommand{\ottdrule}[4][]{{\displaystyle\frac{\begin{array}{l}#2\end{array}}{#3}\quad\ottdrulename{#4}}}
\newcommand{\ottusedrule}[1]{\[#1\]}
\newcommand{\ottpremise}[1]{ #1 \\}
\newenvironment{ottdefnblock}[3][]{ \framebox{\mbox{#2}} \quad #3 \\[0pt]}{}
\newenvironment{ottfundefnblock}[3][]{ \framebox{\mbox{#2}} \quad #3 \\[0pt]\begin{displaymath}\begin{array}{l}}{\end{array}\end{displaymath}}
\newcommand{\ottfunclause}[2]{ #1 \equiv #2 \\}
\newcommand{\ottnt}[1]{\mathit{#1}}
\newcommand{\ottmv}[1]{\mathit{#1}}
\newcommand{\ottkw}[1]{\mathbf{#1}}
\newcommand{\ottsym}[1]{#1}
\newcommand{\ottcom}[1]{\text{#1}}
\newcommand{\ottdrulename}[1]{\textsc{#1}}
\newcommand{\ottcomplu}[5]{\overline{#1}^{\,#2\in #3 #4 #5}}
\newcommand{\ottcompu}[3]{\overline{#1}^{\,#2<#3}}
\newcommand{\ottcomp}[2]{\overline{#1}^{\,#2}}
\newcommand{\ottgrammartabular}[1]{\begin{supertabular}{llcllllll}#1\end{supertabular}}
\newcommand{\ottmetavartabular}[1]{\begin{supertabular}{ll}#1\end{supertabular}}
\newcommand{\ottrulehead}[3]{$#1$ & & $#2$ & & & \multicolumn{2}{l}{#3}}
\newcommand{\ottprodline}[6]{& & $#1$ & $#2$ & $#3 #4$ & $#5$ & $#6$}
\newcommand{\ottfirstprodline}[6]{\ottprodline{#1}{#2}{#3}{#4}{#5}{#6}}
\newcommand{\ottlongprodline}[2]{& & $#1$ & \multicolumn{4}{l}{$#2$}}
\newcommand{\ottfirstlongprodline}[2]{\ottlongprodline{#1}{#2}}
\newcommand{\ottbindspecprodline}[6]{\ottprodline{#1}{#2}{#3}{#4}{#5}{#6}}
\newcommand{\ottprodnewline}{\\}
\newcommand{\ottinterrule}{\\[5.0mm]}
\newcommand{\ottafterlastrule}{\\}
\newcommand{\ottmetavars}{
\ottmetavartabular{
 $ \mathit{fc} $ & \ottcom{fractional capability variable} \\
 $ \mathit{x} ,\, \mathit{g} ,\, \mathit{a} ,\, \mathit{b} $ & \ottcom{expression variable} \\
 $ \mathit{k} $ & \ottcom{integer variable} \\
 $ \mathit{el} $ & \ottcom{array-element variable} \\
}}

\newcommand{\ottterminals}{
\ottrulehead{symb}{::=}{}\ottprodnewline
\ottfirstprodline{|}{ \lambda }{}{}{}{}\ottprodnewline
\ottprodline{|}{ \otimes }{}{}{}{}\ottprodnewline
\ottprodline{|}{ \multimap }{}{}{}{}\ottprodnewline
\ottprodline{|}{ \vdash }{}{}{}{}\ottprodnewline
\ottprodline{|}{ \in }{}{}{}{}\ottprodnewline
\ottprodline{|}{ \forall }{}{}{}{}\ottprodnewline
\ottprodline{|}{ \textsf{Cap} }{}{}{}{}\ottprodnewline
\ottprodline{|}{ \textsf{Type} }{}{}{}{}\ottprodnewline
\ottprodline{|}{ \: ! }{}{}{}{}\ottprodnewline
\ottprodline{|}{ \rightarrow }{}{}{}{}\ottprodnewline
\ottprodline{|}{\ottkw{value}}{}{}{}{}}

\newcommand{\ottf}{
\ottrulehead{\ottnt{f}}{::=}{\ottcom{fractional capability}}\ottprodnewline
\ottfirstprodline{|}{\mathit{fc}}{}{}{}{\ottcom{variable}}\ottprodnewline
\ottprodline{|}{\ottkw{Z}}{}{}{}{\ottcom{zero}}\ottprodnewline
\ottprodline{|}{\ottkw{S} \, \ottnt{f}}{}{}{}{\ottcom{successor}}}

\newcommand{\ottt}{
\ottrulehead{\ottnt{t}}{::=}{\ottcom{linear type}}\ottprodnewline
\ottfirstprodline{|}{\ottkw{unit}}{}{}{}{\ottcom{unit}}\ottprodnewline
\ottprodline{|}{\ottkw{bool}}{}{}{}{\ottcom{boolean (true/false)}}\ottprodnewline
\ottprodline{|}{\ottkw{int}}{}{}{}{\ottcom{63-bit integers}}\ottprodnewline
\ottprodline{|}{\ottkw{elt}}{}{}{}{\ottcom{array element}}\ottprodnewline
\ottprodline{|}{\ottnt{f} \, \ottkw{arr}}{}{}{}{\ottcom{arrays}}\ottprodnewline
\ottprodline{|}{\ottnt{f} \, \ottkw{mat}}{}{}{}{\ottcom{matrices}}\ottprodnewline
\ottprodline{|}{\: !  \ottnt{t}}{}{}{}{\ottcom{multiple-use type}}\ottprodnewline
\ottprodline{|}{\forall \, \mathit{fc}  \ottsym{.}  \ottnt{t}}{}{\textsf{bind}\; \mathit{fc}\; \textsf{in}\; \ottnt{t}}{}{\ottcom{frac. cap. generalisation}}\ottprodnewline
\ottprodline{|}{\ottnt{t}  \otimes  \ottnt{t'}}{}{}{}{\ottcom{pair}}\ottprodnewline
\ottprodline{|}{\ottnt{t}  \multimap  \ottnt{t'}}{}{}{}{\ottcom{linear function}}\ottprodnewline
\ottprodline{|}{\ottnt{t}  \ottsym{\{}  \ottnt{f}  \ottsym{/}  \mathit{fc}  \ottsym{\}}} {\textsf{M}}{}{}{\ottcom{substitution}}\ottprodnewline
\ottprodline{|}{\ottsym{(}  \ottnt{t}  \ottsym{)}} {\textsf{S}}{}{}{\ottcom{parentheses}}}

\newcommand{\otte}{
\ottrulehead{\ottnt{e}}{::=}{\ottcom{expression}}\ottprodnewline
\ottfirstprodline{|}{\ottnt{p}}{}{}{}{\ottcom{primitives}}\ottprodnewline
\ottprodline{|}{\mathit{x}}{}{}{}{\ottcom{variable}}\ottprodnewline
\ottprodline{|}{\ottkw{let} \, \mathit{x}  \ottsym{=}  \ottnt{e} \, \ottkw{in} \, \ottnt{e'}}{}{\textsf{bind}\; \mathit{x}\; \textsf{in}\; \ottnt{e'}}{}{\ottcom{let binding}}\ottprodnewline
\ottprodline{|}{\ottsym{(}  \ottsym{)}}{}{}{}{\ottcom{unit introduction}}\ottprodnewline
\ottprodline{|}{\ottkw{let} \, \ottsym{(}  \ottsym{)}  \ottsym{=}  \ottnt{e} \, \ottkw{in} \, \ottnt{e'}}{}{}{}{\ottcom{unit elimination}}\ottprodnewline
\ottprodline{|}{\ottkw{true}}{}{}{}{\ottcom{true}}\ottprodnewline
\ottprodline{|}{\ottkw{false}}{}{}{}{\ottcom{false}}\ottprodnewline
\ottprodline{|}{\ottkw{if} \, \ottnt{e} \, \ottkw{then} \, \ottnt{e_{{\mathrm{1}}}} \, \ottkw{else} \, \ottnt{e_{{\mathrm{2}}}}}{}{}{}{\ottcom{if}}\ottprodnewline
\ottprodline{|}{\mathit{k}}{}{}{}{\ottcom{integer}}\ottprodnewline
\ottprodline{|}{\mathit{el}}{}{}{}{\ottcom{array element}}\ottprodnewline
\ottprodline{|}{\ottkw{Many} \, \ottnt{e}}{}{}{}{\ottcom{!-introduction}}\ottprodnewline
\ottprodline{|}{\ottkw{let} \, \ottkw{Many} \, \mathit{x}  \ottsym{=}  \ottnt{e} \, \ottkw{in} \, \ottnt{e'}}{}{}{}{\ottcom{!-elimination}}\ottprodnewline
\ottprodline{|}{\ottkw{fun} \, \mathit{fc}  \rightarrow  \ottnt{e}}{}{}{}{\ottcom{frac. cap. abstraction}}\ottprodnewline
\ottprodline{|}{\ottnt{e}  \ottsym{[}  \ottnt{f}  \ottsym{]}}{}{}{}{\ottcom{frac. cap. specialisation}}\ottprodnewline
\ottprodline{|}{\ottsym{(}  \ottnt{e}  \ottsym{,}  \ottnt{e'}  \ottsym{)}}{}{}{}{\ottcom{pair introduction}}\ottprodnewline
\ottprodline{|}{\ottkw{let} \, \ottsym{(}  \mathit{a}  \ottsym{,}  \mathit{b}  \ottsym{)}  \ottsym{=}  \ottnt{e} \, \ottkw{in} \, \ottnt{e'}}{}{\textsf{bind}\; \mathit{a} \cup  \mathit{b}\; \textsf{in}\; \ottnt{e'}}{}{\ottcom{pair elimination}}\ottprodnewline
\ottprodline{|}{\ottkw{fun} \, \mathit{x}  \ottsym{:}  \ottnt{t}  \rightarrow  \ottnt{e}}{}{\textsf{bind}\; \mathit{x}\; \textsf{in}\; \ottnt{e}}{}{\ottcom{abstraction}}\ottprodnewline
\ottprodline{|}{\ottnt{e} \, \ottnt{e'}}{}{}{}{\ottcom{application}}\ottprodnewline
\ottprodline{|}{\ottkw{fix} \, \ottsym{(}  \mathit{g}  \ottsym{,}  \mathit{x}  \ottsym{:}  \ottnt{t}  \ottsym{,}  \ottnt{e}  \ottsym{:}  \ottnt{t'}  \ottsym{)}}{}{\textsf{bind}\; \mathit{g} \cup  \mathit{x}\; \textsf{in}\; \ottnt{e}}{}{\ottcom{fixpoint}}}

\newcommand{\ottp}{
\ottrulehead{\ottnt{p}}{::=}{\ottcom{primitive}}\ottprodnewline
\ottfirstprodline{|}{\ottkw{set}}{}{}{}{\ottcom{array index assignment}}\ottprodnewline
\ottprodline{|}{\ottkw{get}}{}{}{}{\ottcom{array indexing}}\ottprodnewline
\ottprodline{|}{\ottsym{(}  \ottsym{+}  \ottsym{)}}{}{}{}{\ottcom{integer addition}}\ottprodnewline
\ottprodline{|}{\ottsym{(}  \ottsym{-}  \ottsym{)}}{}{}{}{\ottcom{integer subtraction}}\ottprodnewline
\ottprodline{|}{\ottsym{(}  \ottsym{*}  \ottsym{)}}{}{}{}{\ottcom{integer multiplication}}\ottprodnewline
\ottprodline{|}{\ottsym{(}  \ottsym{/}  \ottsym{)}}{}{}{}{\ottcom{integer division}}\ottprodnewline
\ottprodline{|}{\ottsym{(}  \ottsym{=}  \ottsym{)}}{}{}{}{\ottcom{integer equality}}\ottprodnewline
\ottprodline{|}{\ottsym{(}  \ottsym{<}  \ottsym{)}}{}{}{}{\ottcom{integer less-than}}\ottprodnewline
\ottprodline{|}{\ottsym{(}  \ottsym{+.}  \ottsym{)}}{}{}{}{\ottcom{element addition}}\ottprodnewline
\ottprodline{|}{\ottsym{(}  \ottsym{-.}  \ottsym{)}}{}{}{}{\ottcom{element subtraction}}\ottprodnewline
\ottprodline{|}{\ottsym{(}  \ottsym{*.}  \ottsym{)}}{}{}{}{\ottcom{element multiplication}}\ottprodnewline
\ottprodline{|}{\ottsym{(}  \ottsym{/.}  \ottsym{)}}{}{}{}{\ottcom{element division}}\ottprodnewline
\ottprodline{|}{\ottsym{(}  \ottsym{=.}  \ottsym{)}}{}{}{}{\ottcom{element equality}}\ottprodnewline
\ottprodline{|}{\ottsym{(}  \ottsym{<.}  \ottsym{)}}{}{}{}{\ottcom{element less-than}}\ottprodnewline
\ottprodline{|}{\ottkw{not}}{}{}{}{\ottcom{boolean negation}}\ottprodnewline
\ottprodline{|}{\ottkw{share}}{}{}{}{\ottcom{share array}}\ottprodnewline
\ottprodline{|}{\ottkw{unshare}}{}{}{}{\ottcom{unshare array}}\ottprodnewline
\ottprodline{|}{\ottkw{free}}{}{}{}{\ottcom{free arrary}}\ottprodnewline
\ottprodline{|}{\ottkw{array}}{}{}{}{\ottcom{Owl:  make array}}\ottprodnewline
\ottprodline{|}{\ottkw{copy}}{}{}{}{\ottcom{Owl:  copy array}}\ottprodnewline
\ottprodline{|}{\ottkw{sin}}{}{}{}{\ottcom{Owl:  map sine over array}}\ottprodnewline
\ottprodline{|}{\ottkw{hypot}}{}{}{}{\ottcom{Owl:  $x_i := \sqrt{x_i^2 + y_i^2}$}}\ottprodnewline
\ottprodline{|}{\ottkw{asum}}{}{}{}{\ottcom{BLAS: $\sum_i | x_i |$}}\ottprodnewline
\ottprodline{|}{\ottkw{axpy}}{}{}{}{\ottcom{BLAS: $x := \alpha x + y$}}\ottprodnewline
\ottprodline{|}{\ottkw{dot}}{}{}{}{\ottcom{BLAS: $x \cdot y$}}\ottprodnewline
\ottprodline{|}{\ottkw{rotmg}}{}{}{}{\ottcom{BLAS: see its docs}}\ottprodnewline
\ottprodline{|}{\ottkw{scal}}{}{}{}{\ottcom{BLAS: $x := \alpha x$}}\ottprodnewline
\ottprodline{|}{\ottkw{amax}}{}{}{}{\ottcom{BLAS: index of max. abs. value}}\ottprodnewline
\ottprodline{|}{\ottkw{setM}}{}{}{}{\ottcom{matrix index assignment}}\ottprodnewline
\ottprodline{|}{\ottkw{getM}}{}{}{}{\ottcom{matrix indexing}}\ottprodnewline
\ottprodline{|}{\ottkw{shareM}}{}{}{}{\ottcom{share matrix}}\ottprodnewline
\ottprodline{|}{\ottkw{unshareM}}{}{}{}{\ottcom{unshare matrix}}\ottprodnewline
\ottprodline{|}{\ottkw{freeM}}{}{}{}{\ottcom{free matrix}}\ottprodnewline
\ottprodline{|}{\ottkw{matrix}}{}{}{}{\ottcom{Owl:  make matrix}}\ottprodnewline
\ottprodline{|}{\ottkw{copyM}}{}{}{}{\ottcom{Owl:  copy matrix}}\ottprodnewline
\ottprodline{|}{ \textbf{copyM\_to} }{}{}{}{\ottcom{Owl:  copy matrix onto another}}\ottprodnewline
\ottprodline{|}{\ottkw{gemm}}{}{}{}{\ottcom{BLAS: $C := \alpha A^{T?} B^{T?} + \beta C$}}\ottprodnewline
\ottprodline{|}{\ottkw{symm}}{}{}{}{\ottcom{BLAS: $C := \alpha A B + \beta C$}}\ottprodnewline
\ottprodline{|}{\ottkw{posv}}{}{}{}{\ottcom{BLAS: Cholesky decomp. and solve}}\ottprodnewline
\ottprodline{|}{\ottkw{potrs}}{}{}{}{\ottcom{BLAS: solve with given Cholesky}}}

\newcommand{\ottT}{
\ottrulehead{\Theta}{::=}{\ottcom{fractional capability environment}}\ottprodnewline
\ottfirstprodline{|}{ \cdot }{}{}{}{}\ottprodnewline
\ottprodline{|}{\Theta  \ottsym{,}  \mathit{fc}}{}{}{}{}}

\newcommand{\ottG}{
\ottrulehead{\Gamma}{::=}{\ottcom{linear types environment}}\ottprodnewline
\ottfirstprodline{|}{ \cdot }{}{}{}{}\ottprodnewline
\ottprodline{|}{\Gamma  \ottsym{,}  \mathit{x}  \ottsym{:}  \ottnt{t}}{}{}{}{}\ottprodnewline
\ottprodline{|}{\Gamma  \ottsym{,}  \Gamma'}{}{}{}{}}

\newcommand{\ottD}{
\ottrulehead{\Delta}{::=}{\ottcom{linear types environment}}\ottprodnewline
\ottfirstprodline{|}{ \cdot }{}{}{}{}\ottprodnewline
\ottprodline{|}{\Delta  \ottsym{,}  \mathit{x}  \ottsym{:}  \ottnt{t}}{}{}{}{}}

\newcommand{\ottformula}{
\ottrulehead{\ottnt{formula}}{::=}{}\ottprodnewline
\ottfirstprodline{|}{\ottnt{judgement}}{}{}{}{}\ottprodnewline
\ottprodline{|}{\mathit{x}  \ottsym{:}  \ottnt{t} \, \in \, \Delta}{}{}{}{}\ottprodnewline
\ottprodline{|}{\mathit{x}  \ottsym{:}  \ottnt{t} \, \in \, \Gamma}{}{}{}{}\ottprodnewline
\ottprodline{|}{\mathit{fc} \, \in \, \Theta}{}{}{}{}\ottprodnewline
\ottprodline{|}{\ottkw{value} \, \ottsym{(}  \ottnt{e}  \ottsym{)}}{}{}{}{}}

\newcommand{\ottWellXXFormed}{
\ottrulehead{\ottnt{Well\_Formed}}{::=}{}\ottprodnewline
\ottfirstprodline{|}{\Theta  \vdash  \ottnt{f} \, \textsf{Cap}}{}{}{}{\ottcom{Valid fractional capabilities}}\ottprodnewline
\ottprodline{|}{\Theta  \vdash  \ottnt{t} \, \textsf{Type}}{}{}{}{\ottcom{Valid types}}}

\newcommand{\ottValues}{
\ottrulehead{\ottnt{Values}}{::=}{}\ottprodnewline
\ottfirstprodline{|}{\ottkw{value} \, \ottsym{(}  \ottnt{e}  \ottsym{)}}{}{}{}{\ottcom{Value restriction for !-introduction}}}

\newcommand{\ottTypes}{
\ottrulehead{\ottnt{Types}}{::=}{}\ottprodnewline
\ottfirstprodline{|}{\Theta  \ottsym{;}  \Delta  \ottsym{;}  \Gamma  \vdash  \ottnt{e}  \ottsym{:}  \ottnt{t}}{}{}{}{\ottcom{Typing rules for expressions}}}

\newcommand{\ottjudgement}{
\ottrulehead{\ottnt{judgement}}{::=}{}\ottprodnewline
\ottfirstprodline{|}{\ottnt{Well\_Formed}}{}{}{}{}\ottprodnewline
\ottprodline{|}{\ottnt{Values}}{}{}{}{}\ottprodnewline
\ottprodline{|}{\ottnt{Types}}{}{}{}{}}

\newcommand{\ottuserXXsyntax}{
\ottrulehead{\ottnt{user\_syntax}}{::=}{}\ottprodnewline
\ottfirstprodline{|}{\mathit{fc}}{}{}{}{}\ottprodnewline
\ottprodline{|}{\mathit{x}}{}{}{}{}\ottprodnewline
\ottprodline{|}{\mathit{k}}{}{}{}{}\ottprodnewline
\ottprodline{|}{\mathit{el}}{}{}{}{}\ottprodnewline
\ottprodline{|}{symb}{}{}{}{}\ottprodnewline
\ottprodline{|}{\ottnt{f}}{}{}{}{}\ottprodnewline
\ottprodline{|}{\ottnt{t}}{}{}{}{}\ottprodnewline
\ottprodline{|}{\ottnt{e}}{}{}{}{}\ottprodnewline
\ottprodline{|}{\ottnt{p}}{}{}{}{}\ottprodnewline
\ottprodline{|}{\Theta}{}{}{}{}\ottprodnewline
\ottprodline{|}{\Gamma}{}{}{}{}\ottprodnewline
\ottprodline{|}{\Delta}{}{}{}{}\ottprodnewline
\ottprodline{|}{\ottnt{formula}}{}{}{}{}}

\newcommand{\ottgrammar}{\ottgrammartabular{
\ottterminals\ottinterrule
\ottf\ottinterrule
\ottt\ottinterrule
\otte\ottinterrule
\ottp\ottinterrule
\ottT\ottinterrule
\ottG\ottinterrule
\ottD\ottinterrule
\ottformula\ottinterrule
\ottWellXXFormed\ottinterrule
\ottValues\ottinterrule
\ottTypes\ottinterrule
\ottjudgement\ottinterrule
\ottuserXXsyntax\ottafterlastrule
}}

% defnss
% defns Well_Formed
%% defn Cap_
\newcommand{\ottdruleWFXXCapXXVar}[1]{\ottdrule[#1]{%
\ottpremise{\mathit{fc} \, \in \, \Theta}%
}{
\Theta  \vdash  \mathit{fc} \, \textsf{Cap}}{%
{\ottdrulename{WF\_Cap\_Var}}{}%
}}


\newcommand{\ottdruleWFXXCapXXZero}[1]{\ottdrule[#1]{%
}{
\Theta  \vdash  \ottkw{Z} \, \textsf{Cap}}{%
{\ottdrulename{WF\_Cap\_Zero}}{}%
}}


\newcommand{\ottdruleWFXXCapXXSucc}[1]{\ottdrule[#1]{%
\ottpremise{\Theta  \vdash  \ottnt{f} \, \textsf{Cap}}%
}{
\Theta  \vdash  \ottkw{S} \, \ottnt{f} \, \textsf{Cap}}{%
{\ottdrulename{WF\_Cap\_Succ}}{}%
}}

\newcommand{\ottdefnWFXXCapXX}[1]{\begin{ottdefnblock}[#1]{$\Theta  \vdash  \ottnt{f} \, \textsf{Cap}$}{\ottcom{Valid fractional capabilities}}
\ottusedrule{\ottdruleWFXXCapXXVar{}}
\ottusedrule{\ottdruleWFXXCapXXZero{}}
\ottusedrule{\ottdruleWFXXCapXXSucc{}}
\end{ottdefnblock}}

%% defn Type_
\newcommand{\ottdruleWFXXTypeXXUnit}[1]{\ottdrule[#1]{%
}{
\Theta  \vdash  \ottkw{unit} \, \textsf{Type}}{%
{\ottdrulename{WF\_Type\_Unit}}{}%
}}


\newcommand{\ottdruleWFXXTypeXXBool}[1]{\ottdrule[#1]{%
}{
\Theta  \vdash  \ottkw{bool} \, \textsf{Type}}{%
{\ottdrulename{WF\_Type\_Bool}}{}%
}}


\newcommand{\ottdruleWFXXTypeXXInt}[1]{\ottdrule[#1]{%
}{
\Theta  \vdash  \ottkw{int} \, \textsf{Type}}{%
{\ottdrulename{WF\_Type\_Int}}{}%
}}


\newcommand{\ottdruleWFXXTypeXXElt}[1]{\ottdrule[#1]{%
}{
\Theta  \vdash  \ottkw{elt} \, \textsf{Type}}{%
{\ottdrulename{WF\_Type\_Elt}}{}%
}}


\newcommand{\ottdruleWFXXTypeXXArray}[1]{\ottdrule[#1]{%
\ottpremise{\Theta  \vdash  \ottnt{f} \, \textsf{Cap}}%
}{
\Theta  \vdash  \ottnt{f} \, \ottkw{arr} \, \textsf{Type}}{%
{\ottdrulename{WF\_Type\_Array}}{}%
}}


\newcommand{\ottdruleWFXXTypeXXBang}[1]{\ottdrule[#1]{%
\ottpremise{\Theta  \vdash  \ottnt{t} \, \textsf{Type}}%
}{
\Theta  \vdash  \: !  \ottnt{t} \, \textsf{Type}}{%
{\ottdrulename{WF\_Type\_Bang}}{}%
}}


\newcommand{\ottdruleWFXXTypeXXGen}[1]{\ottdrule[#1]{%
\ottpremise{\Theta  \ottsym{,}  \mathit{fc}  \vdash  \ottnt{t} \, \textsf{Type}}%
}{
\Theta  \vdash  \forall \, \mathit{fc}  \ottsym{.}  \ottnt{t} \, \textsf{Type}}{%
{\ottdrulename{WF\_Type\_Gen}}{}%
}}


\newcommand{\ottdruleWFXXTypeXXPair}[1]{\ottdrule[#1]{%
\ottpremise{\Theta  \vdash  \ottnt{t} \, \textsf{Type}}%
\ottpremise{\Theta  \vdash  \ottnt{t'} \, \textsf{Type}}%
}{
\Theta  \vdash  \ottnt{t}  \otimes  \ottnt{t'} \, \textsf{Type}}{%
{\ottdrulename{WF\_Type\_Pair}}{}%
}}


\newcommand{\ottdruleWFXXTypeXXLolly}[1]{\ottdrule[#1]{%
\ottpremise{\Theta  \vdash  \ottnt{t} \, \textsf{Type}}%
\ottpremise{\Theta  \vdash  \ottnt{t'} \, \textsf{Type}}%
}{
\Theta  \vdash  \ottnt{t}  \multimap  \ottnt{t'} \, \textsf{Type}}{%
{\ottdrulename{WF\_Type\_Lolly}}{}%
}}

\newcommand{\ottdefnWFXXTypeXX}[1]{\begin{ottdefnblock}[#1]{$\Theta  \vdash  \ottnt{t} \, \textsf{Type}$}{\ottcom{Valid types}}
\ottusedrule{\ottdruleWFXXTypeXXUnit{}}
\ottusedrule{\ottdruleWFXXTypeXXBool{}}
\ottusedrule{\ottdruleWFXXTypeXXInt{}}
\ottusedrule{\ottdruleWFXXTypeXXElt{}}
\ottusedrule{\ottdruleWFXXTypeXXArray{}}
\ottusedrule{\ottdruleWFXXTypeXXBang{}}
\ottusedrule{\ottdruleWFXXTypeXXGen{}}
\ottusedrule{\ottdruleWFXXTypeXXPair{}}
\ottusedrule{\ottdruleWFXXTypeXXLolly{}}
\end{ottdefnblock}}


\newcommand{\ottdefnsWellXXFormed}{
\ottdefnWFXXCapXX{}\ottdefnWFXXTypeXX{}}

% defns Values
%% defn Value_
\newcommand{\ottdruleValXXPrim}[1]{\ottdrule[#1]{%
}{
\ottkw{value} \, \ottsym{(}  \ottnt{p}  \ottsym{)}}{%
{\ottdrulename{Val\_Prim}}{}%
}}


\newcommand{\ottdruleValXXUnitXXIntro}[1]{\ottdrule[#1]{%
}{
\ottkw{value} \, \ottsym{(}  \ottsym{(}  \ottsym{)}  \ottsym{)}}{%
{\ottdrulename{Val\_Unit\_Intro}}{}%
}}


\newcommand{\ottdruleValXXBoolXXTrue}[1]{\ottdrule[#1]{%
}{
\ottkw{value} \, \ottsym{(}  \ottkw{true}  \ottsym{)}}{%
{\ottdrulename{Val\_Bool\_True}}{}%
}}


\newcommand{\ottdruleValXXBoolXXFalse}[1]{\ottdrule[#1]{%
}{
\ottkw{value} \, \ottsym{(}  \ottkw{false}  \ottsym{)}}{%
{\ottdrulename{Val\_Bool\_False}}{}%
}}


\newcommand{\ottdruleValXXIntXXIntro}[1]{\ottdrule[#1]{%
}{
\ottkw{value} \, \ottsym{(}  \mathit{k}  \ottsym{)}}{%
{\ottdrulename{Val\_Int\_Intro}}{}%
}}


\newcommand{\ottdruleValXXEltXXIntro}[1]{\ottdrule[#1]{%
}{
\ottkw{value} \, \ottsym{(}  \mathit{el}  \ottsym{)}}{%
{\ottdrulename{Val\_Elt\_Intro}}{}%
}}


\newcommand{\ottdruleValXXVar}[1]{\ottdrule[#1]{%
}{
\ottkw{value} \, \ottsym{(}  \mathit{x}  \ottsym{)}}{%
{\ottdrulename{Val\_Var}}{}%
}}


\newcommand{\ottdruleValXXFix}[1]{\ottdrule[#1]{%
}{
\ottkw{value} \, \ottsym{(}  \ottkw{fix} \, \ottsym{(}  \mathit{g}  \ottsym{,}  \mathit{x}  \ottsym{:}  \ottnt{t}  \ottsym{,}  \ottnt{e}  \ottsym{:}  \ottnt{t'}  \ottsym{)}  \ottsym{)}}{%
{\ottdrulename{Val\_Fix}}{}%
}}


\newcommand{\ottdruleValXXLambda}[1]{\ottdrule[#1]{%
}{
\ottkw{value} \, \ottsym{(}  \ottkw{fun} \, \mathit{x}  \ottsym{:}  \ottnt{t}  \rightarrow  \ottnt{e}  \ottsym{)}}{%
{\ottdrulename{Val\_Lambda}}{}%
}}


\newcommand{\ottdruleValXXGen}[1]{\ottdrule[#1]{%
\ottpremise{\ottkw{value} \, \ottsym{(}  \ottnt{e}  \ottsym{)}}%
}{
\ottkw{value} \, \ottsym{(}  \ottkw{fun} \, \mathit{fc}  \rightarrow  \ottnt{e}  \ottsym{)}}{%
{\ottdrulename{Val\_Gen}}{}%
}}


\newcommand{\ottdruleValXXSpc}[1]{\ottdrule[#1]{%
\ottpremise{\ottkw{value} \, \ottsym{(}  \ottnt{e}  \ottsym{)}}%
}{
\ottkw{value} \, \ottsym{(}  \ottnt{e}  \ottsym{[}  \mathit{fc}  \ottsym{]}  \ottsym{)}}{%
{\ottdrulename{Val\_Spc}}{}%
}}


\newcommand{\ottdruleValXXBangXXIntro}[1]{\ottdrule[#1]{%
\ottpremise{\ottkw{value} \, \ottsym{(}  \ottnt{e}  \ottsym{)}}%
}{
\ottkw{value} \, \ottsym{(}  \ottkw{Many} \, \ottnt{e}  \ottsym{)}}{%
{\ottdrulename{Val\_Bang\_Intro}}{}%
}}


\newcommand{\ottdruleValXXPairXXIntro}[1]{\ottdrule[#1]{%
\ottpremise{\ottkw{value} \, \ottsym{(}  \ottnt{e_{{\mathrm{1}}}}  \ottsym{)}}%
\ottpremise{\ottkw{value} \, \ottsym{(}  \ottnt{e_{{\mathrm{2}}}}  \ottsym{)}}%
}{
\ottkw{value} \, \ottsym{(}  \ottsym{(}  \ottnt{e_{{\mathrm{1}}}}  \ottsym{,}  \ottnt{e_{{\mathrm{2}}}}  \ottsym{)}  \ottsym{)}}{%
{\ottdrulename{Val\_Pair\_Intro}}{}%
}}

\newcommand{\ottdefnValXXValueXX}[1]{\begin{ottdefnblock}[#1]{$\ottkw{value} \, \ottsym{(}  \ottnt{e}  \ottsym{)}$}{\ottcom{Value restriction for !-introduction}}
\ottusedrule{\ottdruleValXXPrim{}}
\ottusedrule{\ottdruleValXXUnitXXIntro{}}
\ottusedrule{\ottdruleValXXBoolXXTrue{}}
\ottusedrule{\ottdruleValXXBoolXXFalse{}}
\ottusedrule{\ottdruleValXXIntXXIntro{}}
\ottusedrule{\ottdruleValXXEltXXIntro{}}
\ottusedrule{\ottdruleValXXVar{}}
\ottusedrule{\ottdruleValXXFix{}}
\ottusedrule{\ottdruleValXXLambda{}}
\ottusedrule{\ottdruleValXXGen{}}
\ottusedrule{\ottdruleValXXSpc{}}
\ottusedrule{\ottdruleValXXBangXXIntro{}}
\ottusedrule{\ottdruleValXXPairXXIntro{}}
\end{ottdefnblock}}


\newcommand{\ottdefnsValues}{
\ottdefnValXXValueXX{}}

% defns Types
%% defn Type_
\newcommand{\ottdruleTyXXVarXXLin}[1]{\ottdrule[#1]{%
}{
\Theta  \ottsym{;}  \Delta  \ottsym{;}   \cdot   \ottsym{,}  \mathit{x}  \ottsym{:}  \ottnt{t}  \vdash  \mathit{x}  \ottsym{:}  \ottnt{t}}{%
{\ottdrulename{Ty\_Var\_Lin}}{}%
}}


\newcommand{\ottdruleTyXXVar}[1]{\ottdrule[#1]{%
\ottpremise{\mathit{x}  \ottsym{:}  \ottnt{t} \, \in \, \Delta}%
}{
\Theta  \ottsym{;}  \Delta  \ottsym{;}   \cdot   \vdash  \mathit{x}  \ottsym{:}  \ottnt{t}}{%
{\ottdrulename{Ty\_Var}}{}%
}}


\newcommand{\ottdruleTyXXLet}[1]{\ottdrule[#1]{%
\ottpremise{\Theta  \ottsym{;}  \Delta  \ottsym{;}  \Gamma  \vdash  \ottnt{e}  \ottsym{:}  \ottnt{t}}%
\ottpremise{\Theta  \ottsym{;}  \Delta  \ottsym{;}  \Gamma'  \ottsym{,}  \mathit{x}  \ottsym{:}  \ottnt{t}  \vdash  \ottnt{e'}  \ottsym{:}  \ottnt{t'}}%
}{
\Theta  \ottsym{;}  \Delta  \ottsym{;}  \Gamma  \ottsym{,}  \Gamma'  \vdash  \ottkw{let} \, \mathit{x}  \ottsym{=}  \ottnt{e} \, \ottkw{in} \, \ottnt{e'}  \ottsym{:}  \ottnt{t'}}{%
{\ottdrulename{Ty\_Let}}{}%
}}


\newcommand{\ottdruleTyXXUnitXXIntro}[1]{\ottdrule[#1]{%
}{
\Theta  \ottsym{;}  \Delta  \ottsym{;}   \cdot   \vdash  \ottsym{(}  \ottsym{)}  \ottsym{:}  \ottkw{unit}}{%
{\ottdrulename{Ty\_Unit\_Intro}}{}%
}}


\newcommand{\ottdruleTyXXUnitXXElim}[1]{\ottdrule[#1]{%
\ottpremise{\Theta  \ottsym{;}  \Delta  \ottsym{;}   \cdot   \vdash  \ottnt{e}  \ottsym{:}  \ottkw{unit}}%
\ottpremise{\Theta  \ottsym{;}  \Delta  \ottsym{;}  \Gamma  \vdash  \ottnt{e'}  \ottsym{:}  \ottnt{t}}%
}{
\Theta  \ottsym{;}  \Delta  \ottsym{;}  \Gamma  \vdash  \ottkw{let} \, \ottsym{(}  \ottsym{)}  \ottsym{=}  \ottnt{e} \, \ottkw{in} \, \ottnt{e'}  \ottsym{:}  \ottnt{t}}{%
{\ottdrulename{Ty\_Unit\_Elim}}{}%
}}


\newcommand{\ottdruleTyXXBoolXXTrue}[1]{\ottdrule[#1]{%
}{
\Theta  \ottsym{;}  \Delta  \ottsym{;}   \cdot   \vdash  \ottkw{true}  \ottsym{:}  \: !  \ottkw{bool}}{%
{\ottdrulename{Ty\_Bool\_True}}{}%
}}


\newcommand{\ottdruleTyXXBoolXXFalse}[1]{\ottdrule[#1]{%
}{
\Theta  \ottsym{;}  \Delta  \ottsym{;}   \cdot   \vdash  \ottkw{false}  \ottsym{:}  \: !  \ottkw{bool}}{%
{\ottdrulename{Ty\_Bool\_False}}{}%
}}


\newcommand{\ottdruleTyXXBoolXXElim}[1]{\ottdrule[#1]{%
\ottpremise{\Theta  \ottsym{;}  \Delta  \ottsym{;}  \Gamma  \vdash  \ottnt{e}  \ottsym{:}  \ottkw{bool}}%
\ottpremise{\Theta  \ottsym{;}  \Delta  \ottsym{;}  \Gamma'  \vdash  \ottnt{e_{{\mathrm{1}}}}  \ottsym{:}  \ottnt{t'}}%
\ottpremise{\Theta  \ottsym{;}  \Delta  \ottsym{;}  \Gamma'  \vdash  \ottnt{e_{{\mathrm{2}}}}  \ottsym{:}  \ottnt{t'}}%
}{
\Theta  \ottsym{;}  \Delta  \ottsym{;}  \Gamma  \ottsym{,}  \Gamma'  \vdash  \ottkw{if} \, \ottnt{e} \, \ottkw{then} \, \ottnt{e_{{\mathrm{1}}}} \, \ottkw{else} \, \ottnt{e_{{\mathrm{2}}}}  \ottsym{:}  \ottnt{t}}{%
{\ottdrulename{Ty\_Bool\_Elim}}{}%
}}


\newcommand{\ottdruleTyXXIntXXIntro}[1]{\ottdrule[#1]{%
}{
\Theta  \ottsym{;}  \Delta  \ottsym{;}   \cdot   \vdash  \mathit{k}  \ottsym{:}  \: !  \ottkw{int}}{%
{\ottdrulename{Ty\_Int\_Intro}}{}%
}}


\newcommand{\ottdruleTyXXEltXXIntro}[1]{\ottdrule[#1]{%
}{
\Theta  \ottsym{;}  \Delta  \ottsym{;}   \cdot   \vdash  \mathit{el}  \ottsym{:}  \: !  \ottkw{elt}}{%
{\ottdrulename{Ty\_Elt\_Intro}}{}%
}}


\newcommand{\ottdruleTyXXBangXXIntro}[1]{\ottdrule[#1]{%
\ottpremise{\ottkw{value} \, \ottsym{(}  \ottnt{e}  \ottsym{)}}%
\ottpremise{\Theta  \ottsym{;}  \Delta  \ottsym{;}   \cdot   \vdash  \ottnt{e}  \ottsym{:}  \ottnt{t}}%
}{
\Theta  \ottsym{;}  \Delta  \ottsym{;}   \cdot   \vdash  \ottkw{Many} \, \ottnt{e}  \ottsym{:}  \: !  \ottnt{t}}{%
{\ottdrulename{Ty\_Bang\_Intro}}{}%
}}


\newcommand{\ottdruleTyXXBangXXElim}[1]{\ottdrule[#1]{%
\ottpremise{\Theta  \ottsym{;}  \Delta  \ottsym{;}  \Gamma  \vdash  \ottnt{e}  \ottsym{:}  \: !  \ottnt{t}}%
\ottpremise{\Theta  \ottsym{;}  \Delta  \ottsym{,}  \mathit{x}  \ottsym{:}  \ottnt{t}  \ottsym{;}  \Gamma'  \vdash  \ottnt{e'}  \ottsym{:}  \ottnt{t'}}%
}{
\Theta  \ottsym{;}  \Delta  \ottsym{;}  \Gamma  \ottsym{,}  \Gamma'  \vdash  \ottkw{let} \, \ottkw{Many} \, \mathit{x}  \ottsym{=}  \ottnt{e} \, \ottkw{in} \, \ottnt{e'}  \ottsym{:}  \ottnt{t'}}{%
{\ottdrulename{Ty\_Bang\_Elim}}{}%
}}


\newcommand{\ottdruleTyXXPairXXIntro}[1]{\ottdrule[#1]{%
\ottpremise{\Theta  \ottsym{;}  \Delta  \ottsym{;}  \Gamma  \vdash  \ottnt{e}  \ottsym{:}  \ottnt{t}}%
\ottpremise{\Theta  \ottsym{;}  \Delta  \ottsym{;}  \Gamma'  \vdash  \ottnt{e'}  \ottsym{:}  \ottnt{t'}}%
}{
\Theta  \ottsym{;}  \Delta  \ottsym{;}  \Gamma  \ottsym{,}  \Gamma'  \vdash  \ottsym{(}  \ottnt{e}  \ottsym{,}  \ottnt{e'}  \ottsym{)}  \ottsym{:}  \ottnt{t}  \otimes  \ottnt{t'}}{%
{\ottdrulename{Ty\_Pair\_Intro}}{}%
}}


\newcommand{\ottdruleTyXXPairXXElim}[1]{\ottdrule[#1]{%
\ottpremise{\Theta  \ottsym{;}  \Delta  \ottsym{;}  \Gamma  \vdash  \ottnt{e_{{\mathrm{12}}}}  \ottsym{:}  \ottnt{t_{{\mathrm{1}}}}  \otimes  \ottnt{t_{{\mathrm{2}}}}}%
\ottpremise{\Theta  \ottsym{;}  \Delta  \ottsym{;}  \Gamma'  \ottsym{,}  \mathit{a}  \ottsym{:}  \ottnt{t_{{\mathrm{1}}}}  \ottsym{,}  \mathit{b}  \ottsym{:}  \ottnt{t_{{\mathrm{2}}}}  \vdash  \ottnt{e}  \ottsym{:}  \ottnt{t}}%
}{
\Theta  \ottsym{;}  \Delta  \ottsym{;}  \Gamma  \ottsym{,}  \Gamma'  \vdash  \ottkw{let} \, \ottsym{(}  \mathit{a}  \ottsym{,}  \mathit{b}  \ottsym{)}  \ottsym{=}  \ottnt{e_{{\mathrm{12}}}} \, \ottkw{in} \, \ottnt{e}  \ottsym{:}  \ottnt{t}}{%
{\ottdrulename{Ty\_Pair\_Elim}}{}%
}}


\newcommand{\ottdruleTyXXLambda}[1]{\ottdrule[#1]{%
\ottpremise{\Theta  \vdash  \ottnt{t'} \, \textsf{Type}}%
\ottpremise{\Theta  \ottsym{;}  \Delta  \ottsym{;}  \Gamma  \ottsym{,}  \mathit{x}  \ottsym{:}  \ottnt{t'}  \vdash  \ottnt{e}  \ottsym{:}  \ottnt{t}}%
}{
\Theta  \ottsym{;}  \Delta  \ottsym{;}  \Gamma  \vdash  \ottkw{fun} \, \mathit{x}  \ottsym{:}  \ottnt{t'}  \rightarrow  \ottnt{e}  \ottsym{:}  \ottnt{t'}  \multimap  \ottnt{t}}{%
{\ottdrulename{Ty\_Lambda}}{}%
}}


\newcommand{\ottdruleTyXXApp}[1]{\ottdrule[#1]{%
\ottpremise{\Theta  \ottsym{;}  \Delta  \ottsym{;}  \Gamma  \vdash  \ottnt{e}  \ottsym{:}  \ottnt{t'}  \multimap  \ottnt{t}}%
\ottpremise{\Theta  \ottsym{;}  \Delta  \ottsym{;}  \Gamma'  \vdash  \ottnt{e'}  \ottsym{:}  \ottnt{t'}}%
}{
\Theta  \ottsym{;}  \Delta  \ottsym{;}  \Gamma  \ottsym{,}  \Gamma'  \vdash  \ottnt{e} \, \ottnt{e'}  \ottsym{:}  \ottnt{t}}{%
{\ottdrulename{Ty\_App}}{}%
}}


\newcommand{\ottdruleTyXXGen}[1]{\ottdrule[#1]{%
\ottpremise{\Theta  \ottsym{,}  \mathit{fc}  \ottsym{;}  \Delta  \ottsym{;}  \Gamma  \vdash  \ottnt{e}  \ottsym{:}  \ottnt{t}}%
}{
\Theta  \ottsym{;}  \Delta  \ottsym{;}  \Gamma  \vdash  \ottkw{fun} \, \mathit{fc}  \rightarrow  \ottnt{e}  \ottsym{:}  \forall \, \mathit{fc}  \ottsym{.}  \ottnt{t}}{%
{\ottdrulename{Ty\_Gen}}{}%
}}


\newcommand{\ottdruleTyXXSpc}[1]{\ottdrule[#1]{%
\ottpremise{\Theta  \vdash  \ottnt{f} \, \textsf{Cap}}%
\ottpremise{\Theta  \ottsym{;}  \Delta  \ottsym{;}  \Gamma  \vdash  \ottnt{e}  \ottsym{:}  \forall \, \mathit{fc}  \ottsym{.}  \ottnt{t}}%
}{
\Theta  \ottsym{;}  \Delta  \ottsym{;}  \Gamma  \vdash  \ottnt{e}  \ottsym{[}  \ottnt{f}  \ottsym{]}  \ottsym{:}  \ottnt{t}  \ottsym{\{}  \ottnt{f}  \ottsym{/}  \mathit{fc}  \ottsym{\}}}{%
{\ottdrulename{Ty\_Spc}}{}%
}}


\newcommand{\ottdruleTyXXFix}[1]{\ottdrule[#1]{%
\ottpremise{\Theta  \ottsym{;}  \Delta  \ottsym{,}  \mathit{g}  \ottsym{:}  \ottnt{t}  \multimap  \ottnt{t'}  \ottsym{;}   \cdot   \ottsym{,}  \mathit{x}  \ottsym{:}  \ottnt{t}  \vdash  \ottnt{e}  \ottsym{:}  \ottnt{t'}}%
}{
\Theta  \ottsym{;}  \Delta  \ottsym{;}   \cdot   \vdash  \ottkw{fix} \, \ottsym{(}  \mathit{g}  \ottsym{,}  \mathit{x}  \ottsym{:}  \ottnt{t}  \ottsym{,}  \ottnt{e}  \ottsym{:}  \ottnt{t'}  \ottsym{)}  \ottsym{:}  \: !  \ottsym{(}  \ottnt{t}  \multimap  \ottnt{t'}  \ottsym{)}}{%
{\ottdrulename{Ty\_Fix}}{}%
}}

\newcommand{\ottdefnTyXXTypeXX}[1]{\begin{ottdefnblock}[#1]{$\Theta  \ottsym{;}  \Delta  \ottsym{;}  \Gamma  \vdash  \ottnt{e}  \ottsym{:}  \ottnt{t}$}{\ottcom{Typing rules for expressions}}
\ottusedrule{\ottdruleTyXXVarXXLin{}}
\ottusedrule{\ottdruleTyXXVar{}}
\ottusedrule{\ottdruleTyXXLet{}}
\ottusedrule{\ottdruleTyXXUnitXXIntro{}}
\ottusedrule{\ottdruleTyXXUnitXXElim{}}
\ottusedrule{\ottdruleTyXXBoolXXTrue{}}
\ottusedrule{\ottdruleTyXXBoolXXFalse{}}
\ottusedrule{\ottdruleTyXXBoolXXElim{}}
\ottusedrule{\ottdruleTyXXIntXXIntro{}}
\ottusedrule{\ottdruleTyXXEltXXIntro{}}
\ottusedrule{\ottdruleTyXXBangXXIntro{}}
\ottusedrule{\ottdruleTyXXBangXXElim{}}
\ottusedrule{\ottdruleTyXXPairXXIntro{}}
\ottusedrule{\ottdruleTyXXPairXXElim{}}
\ottusedrule{\ottdruleTyXXLambda{}}
\ottusedrule{\ottdruleTyXXApp{}}
\ottusedrule{\ottdruleTyXXGen{}}
\ottusedrule{\ottdruleTyXXSpc{}}
\ottusedrule{\ottdruleTyXXFix{}}
\end{ottdefnblock}}


\newcommand{\ottdefnsTypes}{
\ottdefnTyXXTypeXX{}}

\newcommand{\ottdefnss}{
\ottdefnsWellXXFormed
\ottdefnsValues
\ottdefnsTypes
}

\newcommand{\ottall}{\ottmetavars\\[0pt]
\ottgrammar\\[5.0mm]
\ottdefnss}

%
\usepackage{ottlayout}%
\renewcommand{\ottpremise}[1]{\premiseSTY{#1}}%
\renewcommand{\ottusedrule}[1]{\usedruleSTY{#1}}%
\renewcommand{\ottdrule}[4][]{\druleSTY[#1]{#2}{#3}{#4}}%
\renewenvironment{ottdefnblock}[3][]{\defnblockSTY[#1]{#2}{#3}}{\enddefnblockSTY}%

% Proof macros
\newcommand{\den}[3]{ \mathcal{#1}_{#2} [\![ #3 ]\!] }%
\newcommand{\V}[2]{ \den{V}{#1}{#2} }%
\newcommand{\C}[2]{ \den{C}{#1}{#2} }%
 
\newcommand{\Unit}{\ottkw{unit}}%
\newcommand{\Bang}{\ottkw{!}}
\newcommand{\Bool}{\ottkw{bool}}%
\newcommand{\Int}{\ottkw{int}}%
\newcommand{\Elt}{\ottkw{elt}}%
\newcommand{\Mat}{\ottkw{mat}}%
\newcommand{\Zf}{\ottkw{Z}}%
\newcommand{\Sf}{\ottkw{S}}%
\newcommand{\Many}{\ottkw{Many}}%
\newcommand{\dom}{\mathrm{dom}}%
\newcommand{\empH}{\emptyset}%

\usepackage{pf2}
\beforePfSpace{15pt, 10pt, 10pt, 10pt, 5pt, 2pt}
\afterPfSpace{15pt, 10pt, 10pt, 10pt, 5pt, 2pt}
\interStepSpace{15pt, 10pt, 10pt, 10pt, 5pt, 2pt}
\pflongindent%

% Multi-line table cells
% tex.stackexchange.com/questions/2441/how-to-add-a-forced-line-break-inside-a-table-cell#19678
\newcommand{\specialcell}[3][c]{%
  \begin{array}[#1]{@{}#2@{}}#3\end{array}}

\newcommand{\alsocell}[3][c]{%
  \begin{tabular}[#1]{@{}#2@{}}#3\end{tabular}}

% Figures
\usepackage{graphicx}
\usepackage[dvipsnames]{xcolor}
\usepackage{lscape}
\usepackage{pgfplots}

% PL Stuff
\usepackage[nounderscore]{syntax}
\renewcommand{\syntleft}{\normalfont\itshape}
\renewcommand{\syntright}{}
\renewcommand{\ulitleft}{\normalfont\bf}%\syn@ttspace\frenchspacing}
\renewcommand{\ulitright}{}
\renewcommand{\litleft}{\bgroup\ulitleft}
\renewcommand{\litright}{\ulitright\egroup}

% NumLin
\newcommand{\lang}{\textsc{NumLin}}

\begin{document}


\begin{abstract}
    We present \lang, a functional programming language designed to express
    the APIs of low-level linear algebra libraries (such as BLAS/LAPACK) safely
    and explicitly, through a brief description of its key features and several
    illustrative examples. We show that \lang's type system is sound and that
    its implementation improves upon na{\"i}ve implementations of linear
    algebra programs, almost towards C-levels of performance. Lastly, we contrast
    it to other recent developments in linear types and show that using linear
    types and fractional permissions to express the APIs of low-level linear
    algebra libraries is a simple and effective idea.
\end{abstract}

\section{Introduction}

Programmers writing numerical software often find themselves caught on
the horns of a dilemma. The foundational, low-level linear algebra
libraries such as BLAS and LAPACK offer programmers very precise
control over the memory lifetime and usage of vector and matrix
values. However, this power comes paired with the responsibility to
manually manage the memory associated with each array object, and in
addition to bringing in the familiar difficulties of reasoning about
lifetimes, aliasing and sharing that plague low-level systems
programming; this also moves the APIs away from the linear-algebraic,
mathematical style of thinking that numerical programmers want to use.

As a result, programmers often turn to higher-level languages such as
Matlab, R and NumPy, which offer very high-level array abstractions
that can be viewed as ordinary mathematical values. This makes
programming safer, as well as making prototyping and verification much
easier, since it lets programmers write programs which bear a closer
resemblance to the formulas that the mathematicians and statisticians
designing these algorithms prefer to work with, and ensures that
program bugs will reflect incorrectly-computed values rather than heap
corruption.

The intention is that these languages can use libraries BLAS and
LAPACK, without having to expose programmers to explicit memory
management. However, this benefit comes at a price: because user
programs do not worry about aliasing, the language implementations
cannot in general exploit the underlying features of the low-level
libraries that let them explicitly manage and reuse memory. As a
result, programs written in high-level statistical languages can
be much less memory-efficient than programs that make full use
of the powers the low-level APIs offer. 

So in practice, programmers face a tradeoff: they can eschew safety
and exploit the full power of the underlying linear algebra libraries,
or they can obtain safety at the price of unneeded copies and worse
memory efficiency. In this work, we show that this tradeoff is not a
fundamental one.

\lang\ is a functional programming language whose type system is
designed to enforce the safe usage of the APIs of low-level linear algebra
libraries (such as BLAS/LAPACK).  It does so by combining linear types,
fractional permissions, runtime errors and recursion into a small, easily
understandable, yet expressive set of core constructs.

\lang\ allows a novice to understand and work with complicated linear
algebra library APIs, as well as point out subtle aliasing bugs and
reduce memory usage in existing programs. In fact, we were able to use
\lang\ to find linearity and aliasing bugs in a linear algebra
algorithm that was \emph{generated} by another program
\emph{specifically designed to translate matrix expressions into an
  efficient sequence of calls to linear algebra routines}. We were
also able to reduce the number of temporaries used by the same
algorithm, using \lang's type system to guide us.

\lang's implementation supports several syntactic conveniences as well as a
\emph{usable} integration with real OCaml libraries.

\subsection{Contributions}

In this paper
\begin{itemize}
    \item we describe \lang, a linearly typed language for linear algebra programs
    \item we illustrate that \lang's design and features are well-suited to its
        intended domain with progressively sophisticated examples
    \item we prove \lang's soundness, using a step-indexed logical relation
    \item we describe a very simple, unification based type-inference algorithm
        for polymorphic fractional permissions (similar to ones used for
        parametric polymorphism), demonstrating an alternative approach to
        dataflow analysis \cite{bierhoff}
    \item we describe an implementation that is both compatible with and usable
        from existing code
    \item we show an example of how using \lang\ helped highlight linearity
        and aliasing bugs, and reduce the memory usage of a \emph{generated}
        linear algebra program
    \item we show that using \lang, we can achieve parity with C for linear
        algebra routines, whilst having much better static guarantees about the
        linearity and aliasing behaviour of our programs.
\end{itemize}



\section{\lang\ Overview and Examples}\label{sec:lang_and_examples}

\subsection{Syntax}

\lang's concrete syntax is inspired by that of OCaml. It desugars
(Figure~\ref{fig:lang_desugar}) into a smaller, core type and expression
grammar (Figure~\ref{fig:core_grammar}).

\begin{figure}[p]
    \begin{center}
    \[\def\arraystretch{1.3}
    \begin{array}{rcl}
        f & ::= & {'\!f\!c}
                \mid \Zf
                \mid f\ \Sf \\
                \\
        t & ::= & \Unit
                \mid \Bool
                \mid \Int
                \mid \Elt
                \mid f\ \Arr
                \mid f\ \Mat
                \mid \Bang t
                \mid {'\!f\!c}.\ t
                \mid t \otimes t'
                \mid t \multimap t' \\
                \\
        e & ::= & p \textrm{ (primitives)}
                \mid x \textrm{ (variable)}
                \mid \Let\ x = e\ \In\ e'
                \mid ()
                \mid \Let\ () = e\ \In\ e'
                \mid \ottkw{true}
                \mid \ottkw{false}
                \\ && \ottkw{if}\ e\ \ottkw{then}\ e_1\ \ottkw{else}\ e_2
                \mid k \textrm{ (integer)}
                \mid l \textrm{ (heap location)}
                \mid el \textrm{ (array element)}
                \\ && \Many\ v
                \mid \Let\ \Many\ x = e\ \In\ e'
                \mid \ottkw{fun}\ '\!f\!c \rightarrow e
                    \textrm{ (frac. perm. abstraction)}
                \\ && e\ [f] \textrm{ (frac. perm. specialisiation)}
                \mid (e, e')
                \mid \Let\ (a,b) = e\ \In\ e'
                \\ && \ottkw{fun}\ x : t \rightarrow e
                \mid e\ e'
                \mid \ottkw{fix} \: (g, x : t, e : t')
    \end{array} \]
    \end{center}
    \caption{Core fraction $f$, type $t$ and expression $e$ grammar of
        \lang. Values $v$ are a subset of the expressions, their full
        definition and a list of all primitives $p$ is in
        Appendix~\ref{sec:grammar_def}.}\label{fig:core_grammar}
\end{figure}

\begin{figure}[p]
\begin{center}
\[\def\arraystretch{1.3}
    \begin{array}{rcl}
    x[e] &
    \Rightarrow &
    \ottkw{get}\ \_\ x\ (e) \:\qquad \textrm{(similarly for matrices)}
\\
    x[e_1] := e_2 &
    \Rightarrow &
    \ottkw{set}\ x\ (e_1)\ (e_2) \quad \textrm{(similarly for matrices)}
\\
\\
    pat & ::= & () \mid x \mid {!x} \mid \Many\ pat \mid (pat, pat)
\\
    \Let\ !x = e_1\ \In\ e_2 &
    \Rightarrow &
    \specialcell[t]{l}{%
        \Let\ \Many\ x = e_1\ \In \\
        \Let\ \Many\ x = \Many\ (\Many\ x)\ \In\ e_2}
\\
    \Let\ \Many\ \langle pat_x \rangle\ = e_1\ \In\ e_2 &
    \Rightarrow &
    \specialcell[t]{l}{%
        \Let\ \Many\ x = x\ \In \\
        \Let\ \langle pat_x \rangle\ = x\ \In\ e_2}
\\
    \Let\ (\langle pat_a \rangle, \langle pat_b \rangle)\ = e_1\ \In\ e_2 &
    \Rightarrow &
    \specialcell[t]{l}{%
        \Let\ (a,b)\ = a\_b\ \In\
        \Let\ \langle pat_a \rangle\ = a\ \In \\
        \Let\ \langle pat_b \rangle\ = b\ \In\ e_2}
\\
    \ottkw{fun}\ (\langle pat_x \rangle : t) \rightarrow e &
    \Rightarrow &
    \ottkw{fun}\ (x : t) \rightarrow \Let\ \langle pat_x \rangle = x\ \In\ e
\\
\\
    arg & ::= & \langle pat \rangle : t\ \mid\ {'x} \textrm{ (fractional permission variable)}
\\
    \ottkw{fun}\ \langle arg_{1 .. n} \rangle \rightarrow e &
    \Rightarrow &
    \ottkw{fun}\ \langle arg_1 \rangle \rightarrow ..
    \ \ottkw{fun}\ \langle arg_n \rangle \rightarrow e
\\
    \Let\ f\ {\langle arg_{1 .. n} \rangle} = e_1\ \In\ e_2 &
    \Rightarrow &
    \Let\ f = \ottkw{fun}\ {\langle arg_{1 .. n} \rangle} \rightarrow e_1\
    \In\ e_2
\\
    \Let\ !f\ {\langle arg_{1 .. n} \rangle} = e_1\ \In\ e_2 &
    \Rightarrow &
    \Let\ \Many\ f = \Many\ (\ottkw{fun}\ {\langle arg_{1 .. n} \rangle}
    \rightarrow e_1)\ \In\ e_2
\\
    \mathrm{fixpoint} & \equiv & \ottkw{fix}\ (f, x : t, \ottkw{fun}
    \ {\langle arg_{0 .. n} \rangle} \rightarrow e_1 : {t'} )
\\
    \Let\ \ottkw{rec}\ f\ (x : t)\ {\langle arg_{0 .. n} \rangle} : {t'} = e_1\ \In\ e_2 &
    \Rightarrow &
    \Let\ f = \mathrm{fixpoint}\ \In\ e_2
\\
    \Let\ \ottkw{rec}\ !f\ (x : t)\ {\langle arg_{0 .. n} \rangle} : {t'} = e_1\ \In\ e_2 &
    \Rightarrow &
    \Let\ \Many\ f = \Many\ \mathrm{fixpoint}\ \In\ e_2
    \end{array}
\]
\end{center}
\caption{Desugaring \lang.}\label{fig:lang_desugar}
\end{figure}

\subsection{Type System and Other Features}

The core type theory of \lang\ is a nearly off-the-shelf linear type
theory~\cite{barber1996}, supporting familiar features such as linear function
spaces $A \multimap B$ and tensor products $A \otimes B$. We adopt linearity --
the restriction that each program variable be used at most once -- since it
allows us to express purely functional APIs for numerical library routines that
mutate arrays and matrices~\cite{Wadler90}. Due to linearity, values cannot
alias and are only used once, which means that linearly-typed updates result in
no \emph{observable} mutation.

As a result, programmers can reason about \lang\ expressions as if they
were ordinary mathematical expressions -- as indeed they are! We are
merely adopting a stricter type discipline than usual to make managing
memory safe.

\subsubsection{Intuitionism: ! and \highl{Many}}

However, linearity by itself is not sufficient to produce an expressive enough
programming language. For values such as booleans, integers, floating-point
numbers as well as pure functions, we need to be able to use them
\emph{intuitionistically}, that is, more than once or not at all. For this
reason, we have the ! constructor at the type level and its corresponding
\highl{Many} constructor and \highl{let Many <id> = .. in ..} eliminator at
the term level. Because we want to restrict how a programmer can alias pointers
and prevent a programmer from ignoring them (a memory leak), \lang\ enforces
simple syntactic restrictions on which values can be wrapped up in a
\highl{Many} constructor (details in Section~\ref{sec:formal_system}).

\subsubsection{Fractional Permissions}

There are also valid cases in which we would want to alias pointers to a
matrix. The most common is exemplified by the BLAS routine \highl{gemm}, which
(rather tersely) stands for \emph{GEneric Matrix Multiplication}.  A
\emph{simplified} definition of \texttt{gemm($\alpha$, A, B, $\beta$, C)} is $C
:= \alpha AB + \beta C$. In this case, \texttt{A} and \texttt{B} may alias each
other but neither may alias \texttt{C}, because it is being written to.
Related to \emph{mutating} arrays and matrices is \emph{freeing} them. Here, we
would also wish to restrict aliasing so that we do not free one alias and then
attempt to use another. Although linearity on its own suffices to prevent
use-after-free errors when values are \emph{not} aliased (a freed value is
\emph{out of scope} for the rest of the expression), we still need another
simple, yet powerful concept to provide us with the extra expressivity of
aliasing \emph{without} losing any of the benefits of linearity.

Fractional permissions provide exactly this. Concretely, types of (pointers to)
arrays and matrices are \emph{parameterised} by a \emph{fraction}. A fraction
is either $1$ ($2^0$) or exactly \emph{half} of another fraction ($2^{-k}$, for
natural $k$). The former represents complete ownership of that value: the
programmer may mutate or free that value as they choose; the latter represents
read-only access or a \emph{borrow}: the programmer may read from the value but
not write to or free it. Creating an array/matrix gives you ownership of it, so
too does having one (with a fractional permission of $2^0$) passed in as an
argument.

In \lang, we can produce two aliases of a single array/matrix, by
\emph{sharing} it. If the original alias had a fractional permission of
$2^{-k}$ then the two new aliases of it will have a fractional permission of
$2^{-(k+1)}$ each. Thanks to linearity, the original array/matrix with a
fractional permission of $2^{-k}$ will be out of scope after the sharing.  When
an array/matrix is shared as such, we can prevent the programmer from freeing
or mutating it by making the types of \texttt{free} and \texttt{set} (for
mutation) require a \emph{whole} ($2^0$) permission.

If we have two aliases \emph{to the same matrix} with \emph{identical}
fractional permissions ($2^{-(k+1)}$), we can recombine or \emph{unshare} them
back into a single one, with a larger $2^{-k}$ permission. As before, thanks to
linearity, the original two aliases will be out of scope after unsharing.

\subsubsection{Runtime Errors}

Aside from out-of-bounds indexing, matrix unsharing is one of only \emph{two}
operations that can fail at runtime (the other being dimension checks, such as
for \texttt{gemm}). The check being performed is a simple sanity check that the
two aliasing pointers passed to \texttt{unshare} point to the same array/matrix.
Section~\ref{sec:discussion_related_work} contains an overview of how we could
remove the need for this by tracking pointer identities statically by
augmenting the type system further.

\subsubsection{Recursion}

The final feature of \lang\ which makes it sufficiently expressive is recursion
(and of course, conditional branches to ensure termination). Conditional
branches are implemented by ensuring that both branches use the same set of
linear values. A function can be recursive if it captures no linear values from
its environment. Like with \highl{Many}, this is enforced via simple syntactic
restrictions on the definition of recursive functions.

\subsection{Examples}

\subsubsection{Factorial}\label{subsubsec:factorial}

Although a factorial function (Figure~\ref{fig:lang_factorial}) may seem like
an aggresively pedestrian first example, in a linearly typed language such as
\lang\ it represents the culmination of many features.

To simplifiy the design and implementation of \lang's type system, recursive
functions must have full type annotations (non-recursive functions need only
their argument types annotated). Its body is a closed expression (with respect
to the function's arguments), so it type-checks (since it does not capture any
linear values from its environment).

The only argument is \highl{!x : !int}. The ! annotation on \texttt{x} is a
syntactic convenience for declaring the value to used intuitionistically, its
full and precise meaning is described in Section~\ref{subsec:core_tt}.

The condition for an \highl{if} may or may not use linear values (here, with
\highl{x < 0 || x = 0}, it does not). Any linear values used by the condition
would not be in scope in either branch of the \highl{if}-expression.  Both
branches use \texttt{x} differently: one ignores it completely and the other
uses it twice.

All numeric and boolean literals are implicitly wrapped in a \highl{Many} and
all primitives involving them return a \highl{!int}, \highl{!bool} or
\highl{!elt} (types of elements of arrays/matrices, typically 64-bit
floating-point numbers). The short-circuting \texttt{||} behaves in exactly the
same way as a boolean-valued \highl{if}-expression.

\begin{figure}[t]
    \centering
    \inputminted[fontsize=\small, linenos]{ocaml}{../../examples/factorial.lt}
    \caption{Factorial function in \lang.}\label{fig:lang_factorial}
\end{figure}

\subsubsection{Summing over an Array}

Now we can add fractional permissions to the mix:
Figure~\ref{fig:lang_sumarray} shows a simple, tail-recursive implementation of
summing all the elements in an array. There are many new features; first among
them is \highl{!x0 : !elt}, the type of array/matrix elements (64-bit floating
point).

\begin{figure}[t]
    \centering
    \inputminted[fontsize=\small]{ocaml}{../../examples/sum_array.lt}
    \caption{Summing over an array in \lang.}\label{fig:lang_sumarray}
\end{figure}

Second is \highl{('x) (row: 'x arr)} which is an array with a
universally-quantified fractional permission. In particular, this means the
body of the function cannot mutate or free the input array, only read from it.
If the programmer did try to mutate or free \texttt{row}, then they would get a
helpful error message (Figure~\ref{fig:lang_errormsg}).

\begin{figure}[t]
    \centering
    \begin{minted}[fontsize=\small]{ocaml}
let row = row[i] := x1 in (* or *) let () = free row in
(* Could not show equality:                                        *)
(*     z arr                                                       *)
(* with                                                            *)
(*     'x arr                                                      *)
(*                                                                 *)
(* Var 'x is universally quantified                                *)
(* Are you trying to write to/free/unshare an array you don't own? *)
(* In examples/sum_array.lt, at line: 7 and column: 19             *)
    \end{minted}
    \caption{Attempting to write to or free a read only array in
    \lang.}\label{fig:lang_errormsg}
\end{figure}

Alongside taking a \highl{row: 'x arr}, the function also returns an array
with exactly the same fractional permission as the \texttt{row} (which can only
be \texttt{row}).  This is necessary because of linearity: for the caller, the
original array passed in as an argument would be out of scope for the rest of
the expression, so it needs to be returned and then rebound to be used for the
rest of the function.

An example of this consuming and re-binding is in \highl{let (row, !x1) =
row[i]}. Indexing is implemented as a primitive $\ottkw{get}:\ {'x}.\ {{'x}\
\Arr} \multimap {!\Int} \multimap {'x\ \Arr} \otimes {!\Elt}$.
Although fractional permissions can be passed around explicitly  (as done in
the recursive call), they can also be \emph{automatically inferred at call
sites}: \highl{row[i] == get _ row i} takes advantage of this convenience.

\subsubsection{One-dimensional Convolution}

\begin{figure}[t]
    \centering
    \inputminted[fontsize=\small]{ocaml}{../../examples/weighted_avg_infer.lt}
    \caption{\emph{Simplified} one-dimensional convolution.}\label{fig:lang_oned_conv}
\end{figure}

Figure~\ref{fig:lang_oned_conv} extends the set of features demonstrated by the
previous examples by mutating one of the input arrays. A one-dimensional
convolution involves two arrays: a read-only kernel (array of weights) and an
input vector. It modifies the input vector \emph{in-place} by replacing each
\highl{write[i]} with a weighted (as per the values in the kernel) sum of it
and its neighbours; intuitively, sliding a dot-product with the kernel across
the vector.

What's implemented in Figure~\ref{fig:lang_oned_conv} is a \emph{simplified}
version of this idea, so as to not distract from the features of \lang. The
simplifications are:
\begin{itemize}
    \item the kernel has a length 3, so only the value of \highl{write[i-1]}
        (prior to modification in the previous iteration) needs to be carried
        forward using \highl{x0}
    \item \highl{write} is assumed to have length \highl{n+1}
    \item \texttt{i}'s initial value is assumed to be \texttt{1}
    \item \highl{x0}'s initial value is assumed to be \highl{write[0]}
    \item the first and last values of \texttt{write} are ignored.
\end{itemize}

Mutating an array is implemented similarly to indexing one -- a primitive
$\ottkw{set}:\ {\Zf\ \Arr} \multimap {!\Int} \multimap {!\Elt} \multimap
{\Zf\ \Arr}$. It consumes the original array and returns a new array with
the updated value.  \highl{let written = write[i] := <exp> } is just syntactic
sugar for \highl{let written = set write i <exp>}.

Since \highl{write: z arr} (where \Zf stands for $k=0$, representing a
fractional permission of $2^{-k} = 2^{-0} = 1$), we may mutate it, but since we
only need to read from \texttt{weights}, its fractional permission index can be
universally-quantified. In the recursive call, we see \texttt{\_} being used
explicitly to tell the compiler to \emph{infer} the correct fractional
permission based on the given arguments.

\subsubsection{Digression: Types of Primitives}

\emph{The most pertinent aspect of \lang\ is the types of its primitives}
(Figure~\ref{fig:primitive_types}).  While the types of operations such as
\textbf{get} and \textbf{set} might be borderline obvious, the types of
BLAS/LAPACK routines become an \emph{incredibly useful, automated check for
using the API correctly.}

\begin{figure}
    \begin{center}
    \[\def\arraystretch{1.3}
    \begin{array}{rcl}
%
        \ottkw{symm} &:& {!\Bool} \multimap {!\Elt} \multimap {'x}.\
            {{'x}\ \Mat} \multimap {'y}.\ {{'y}\ \Mat} \multimap {!\Elt}
            \multimap {\Zf\ \Mat} \multimap
            \\ && ( {{'x}\ \Mat} \otimes {{'y}\ \Mat} ) \otimes {\Zf\ \Mat} \\
%
        \ottkw{gemm} &:& {!\Elt} \multimap {'x}.\ {{'x}\ \Mat} \otimes
            {!\Bool} \multimap {'y}.\ {{'y}\ \Mat} \otimes {!\Bool}
            \multimap {!\Elt} \multimap
            \\ && {\Zf\ \Mat} \multimap ( {{'x}\ \Mat} \otimes {{'y}\ \Mat} )
            \otimes {\Zf\ \Mat} \\
%
        \ottkw{gesv} &:& {\Zf\ \Mat} \multimap {\Zf\ \Mat} \multimap
            {\Zf\ \Mat} \otimes {\Zf\ \Mat} \\
%
        \ottkw{posv} &:& {\Zf\ \Mat} \multimap {\Zf\ \Mat} \multimap
            {\Zf\ \Mat} \otimes {\Zf\ \Mat} \\
%
        \ottkw{potrs} &:& {'x}.\ {{'x}\ \Mat} \multimap {\Zf\ \Mat}
            \multimap {{'x}\ \Mat} \otimes {\Zf\ \Mat} \\
%
        \ottkw{syrk} &:& {!\Bool} \multimap {!\Elt} \multimap {'x}.\
            {{'x}\ \Mat} \multimap {!\Elt} \multimap {\Zf\ \Mat} \multimap
            {{'x}\ \Mat} \otimes {\Zf\ \Mat} \\
%
    \end{array} \]
    \end{center}
    \caption{Types of some \lang\ primitives.}\label{fig:primitive_types}
\end{figure}

We determine the types for these routines by consulting their documentation.
Each routine has a record of the expected aliasing behaviour and whether or not
it modifies or consumes its argument in any way. We use that to derive the
types in Figure~\ref{fig:primitive_types}. Since most of these low-level
routines are very careful not to do any allocation themselves, it is generally
very easy to give each a \lang\ type -- every argument that can modify/consume
its argument needs a full permission, and all others can be
fraction-polymorphic.  Taking Fortran as an example, it has a notion of
\texttt{in}, \texttt{out} and \texttt{inout} parameters. The latter two would
need full \Zf\ permissions; the first would be fraction-polymorphic.

\subsubsection{Squaring a Matrix}

\begin{figure}[t]
    \centering
    \inputminted[fontsize=\small]{ocaml}{../../examples/square.lt}
    \caption{Linear regression (OLS): $\hat\beta =
        \mathbf{(X^T X)^{-1} X^T y}$}\label{fig:lang_square}
\end{figure}

Figure~\ref{fig:lang_square} shows how a linearly-typed matrix squaring
function may be written in \lang. It is a \emph{non-recursive} function
declaration (the return type is inferred). Since we would like to be able to
use a function like \texttt{square} more than once, it is marked with a
\texttt{!} annotation (which also ensures it captures no linear values from the
surrounding environment).

To square a matrix, first, we extract the dimensions of the argument
\texttt{x}. Then, because we need to use \texttt{x} twice (so that we can
multiply it by itself) but linearity only allows one use, we use
$\ottkw{shareM}:\ {'x}.\ {{'x}\ \Mat} \multimap {{'x}\ \Sf\ \Mat} \otimes
{{'x}\ \Sf\ \Mat}$ to split the permission \texttt{'x} (which represents
$2^{-x}$) into two halves (\texttt{'x s}, which represents $2^{-(x+1)}$).

Even if \texttt{x} had type \texttt{z mat}, sharing it now enforces the
assumption of all BLAS/LAPACK routines that any matrix which is written to
(which, in \lang, is always of type \texttt{z mat}) does not alias any other
matrix in scope. So if we did try to use one of the aliases in mutating way,
the expression would not type check, and we would get an error similar to the
one in Figure~\ref{fig:lang_errormsg}.

The line \highl{let answer <- new (m,n) [| x1 * x2 |]} is syntactic sugar for
first creating a new $m \times n$ matrix (\highl{let answer = matrix m n}) and
then storing the result of the multiplication in it (\highl{let ((x1, x2),
answer) = gemm 1. _ (x1, false) _ (x2, false) 0. answer}). \highl{false} means
the matrix should not be accessed with indices transposed.

By using some simple pattern-matching and syntactic sugar
(Figure~\ref{fig:lang_matexp}), we can:
\begin{itemize}
    \item write normal-looking, apparently non-linear code
    \item use matrix expressions directly and have a call to an efficient
        call to a BLAS/LAPACK routine inserted with appropriate re-bindings
    \item retain the safety of linear types with fractional permissions by
        having the compiler statically enforce the aliasing and read/write rules
        implicitly assumed by BLAS/LAPACK routines.
\end{itemize}

\begin{figure}[t]
\begin{center}
\[\def\arraystretch{1.3}
    \begin{array}{rcl}
    \Let\ v \leftarrow x[e]\ \In\ e &
    \Rightarrow &
    \Let\ (x, !v)\ = x[e]\ \In\ e \qquad \textrm{(similarly for matrices)}
\\
    \Let\ x_2 \leftarrow \ottkw{new}\ [|\ x_1\ |]\ \In\ e &
    \Rightarrow &
    \Let\ (x_1, x_2)\ = \ottkw{copyM}\ \_\ x_1\ \In\ e
\\
    \Let\ x_2 \leftarrow [|\ x_1\ |]\ \In\ e &
    \Rightarrow &
    \Let\ (x_1, x_2)\ = \ottkw{copyM\_to}\ \_\ x_1\ x_2\ \In\ e
    \end{array}
\]
\[
    M ::= X\ \mid\ X^T\ \mid\ \textrm{sym}( X )
\]
\end{center}
\begin{align*}
% new
    \Let\ &Y \leftarrow \ottkw{new}\ (n, k)\ [|\ \alpha M_1 M_2 \ |]\ \In\ e
    \Rightarrow \\
    & \Let\ Y\ = \ottkw{matrix}\ n\ k\ \In
    \ \Let\ Y \leftarrow [|\ \alpha M_1 M_2 + 0Y\ |]\ \In\ e
\\[0.1\baselineskip]
%syrk
    \Let\ &Y \leftarrow [|\ \alpha X X^T + \beta Y\ |]\ \In\ e
    \Rightarrow \\
    & \Let\ (X,Y)\ = \ottkw{syrk}\ \ottkw{false}
    \ \alpha\ \_\ X\ \beta\ Y\ \In\ e
\\[0.1\baselineskip]
    \Let\ &Y \leftarrow [|\ \alpha X^T X + \beta Y\ |]\ \In\ e
    \Rightarrow \\
    & \Let\ (X,Y)\ = \ottkw{syrk}\ \ottkw{true}
    \ \alpha\ \_\ X\ \beta\ Y\ \In\ e
%symm
\\[0.1\baselineskip]
    \Let\ &Y \leftarrow\ [|\ \alpha\,\textrm{sym}(X_1)\,X_2 + \beta Y\ |]\ \In\ e
    \Rightarrow \\
    & \Let\ ((X_1,X_2),Y)\ = \ottkw{symm}\ \ottkw{false}
        \ \alpha\ \_ \ X_1 \_\ X_2\ \beta\ Y\ \In\ e
\\[0.1\baselineskip]
    \Let\ &Y \leftarrow\ [|\ \alpha\,X_2\,\textrm{sym}(X_1) + \beta Y\ |]\ \In\ e
    \Rightarrow \\
    & \Let\ ((X_1,X_2),Y)\ = \ottkw{symm}\ \ottkw{true}
        \ \alpha\ \_ \ X_1 \_\ X_2\ \beta\ Y\ \In\ e
\\[0.1\baselineskip]
%gemm
    \Let\ &Y \leftarrow [|\ \alpha X_1^{T?} X_2^{T?} + \beta Y\ |]\ \In\ e
    \Rightarrow \\
    & \Let\ ((X_1,X_2),Y)\ =\ \ottkw{gemm}\ \alpha
        \ \_\ \left(X_1, \substack{\ottkw{true}\\\ottkw{false}}\right)
        \ \_\ \left(X_2, \substack{\ottkw{true}\\\ottkw{false}}\right)
        \ \beta\ Y\ \In\ e
\end{align*}

\caption{Purely syntactic pattern-matching translations of
    matrix expressions.}\label{fig:lang_matexp}
\end{figure}

\subsubsection{Linear Regression}

\begin{figure}[t]
    \centering
    \inputminted[fontsize=\small]{ocaml}{../../examples/lin_reg.lt}
    \caption{Linear regression (OLS): $\hat\beta =
        \mathbf{(X^T X)^{-1} X^T y}$}\label{fig:lang_lin_reg}
\end{figure}

In Figure~\ref{fig:lang_lin_reg}, we wish to compute $\hat\beta = \mathbf{(X^T
X)^{-1} X^T y}$. To do that, first, we extract the dimensions of matrix
\texttt{x}. Then, we say we would like \texttt{xy} to be a new matrix, of
dimension $m \times 1$, which contains the result of $\mathbf{X^T y}$ (using
syntactic sugar for \texttt{matrix} and \texttt{gemm} calls similar to that
used in Figure~\ref{fig:lang_square}, with a \highl{^T} annotation on
\texttt{x} to set \texttt{x}'s `transpose indices'-flag to \highl{true}).

However, the line \highl{let x_T_x <- new (m,m) [| x^T * x |]}, works for a
slightly different reason: that pattern is matched to a BLAS call to
(\highl{syrk true 1. x 0. x_T_x}), which only uses \texttt{x} once. Hence
\texttt{x} can appear \emph{twice} in the \emph{pattern} without any calls to
\texttt{share}.

After computing \texttt{x\_T\_x}, we need to invert it and then multiply it by
\texttt{xy}. The BLAS routine $\ottkw{posv}:\ {\Zf\ \Mat} \multimap
{\Zf\ \Mat} \multimap {\Zf\ \Mat} \otimes {\Zf\ \Mat}$ does
exactly that: assuming the first argument is symmetric, \texttt{posv} mutates
its second argument to contain the desired value. Its first argument is also
mutated to contain the (upper triangular) Cholesky decomposition factor of the
original matrix. Since we do not need that matrix (or its memory) again, we
\texttt{free} it. If we forgot to, we would get a \texttt{Variable to\_del not
used} error. Lastly, we return the \texttt{answer} alongside the untouched
input matrices \texttt{(x,y)}.

\subsubsection{L1-norm Minimisation on Manifolds}

L1-norm minimisation is often used in optimisation problems, as a
\emph{regularisation} term for reducing the influence of outliers.  Although
the below formulation~\cite{bronstein} is intended to be used with \emph{sparse}
computations, \lang's current implementation only implements dense ones.
However, it still serves as a useful example of explaining \lang's features.

\begin{figure}[t]
    \centering
    \inputminted[fontsize=\small]{ocaml}{../../examples/l1_norm_min.lt}
    \caption{L1-norm minimisation on manifolds:
        $\mathbf{Q^{-1}U(I+U^TQ^{-1}U)^{-1}U^T}$}\label{fig:lang_l1_norm_min}
\end{figure}

Figure~\ref{fig:lang_l1_norm_min} shows even more pattern-matching. Patterns of
the form \highl{let <id> <- [| beta * c + alpha * a * b |]} are also desugared
to \texttt{gemm} calls. Primitives like $\ottkw{transpose}:\ {'x}.\ {{'x}\ \Mat}
\multimap {{'x}\ \Mat} \otimes {\Zf\ \Mat}$ and $\ottkw{eye}:\ {!\Int}
\multimap {\Zf\ \Mat}$ allocate new matrices; \texttt{transpose} returns
the transpose of a given matrix and \texttt{eye k} evaluates to a $k \times k$
identity matrix.

We also see our first example of re-using memory for different matrices: like
with \texttt{to\_del} and \texttt{posv} in the previous example, we do not need
the value stored in \texttt{tmp\_5\_5} after the call to \texttt{gesv} (a
primitive similar to \texttt{posv} but for a non-symmetric first argument).
However, we can re-use its memory much later to store \texttt{answer} with
\highl{let answer <- [| 0. * tmp_5_5 + q_inv_u * inv_u_T |]}. Again, thanks to
linearity, the identifiers \texttt{q} and \texttt{tmp\_5\_5} are out of scope
by the time \texttt{answer} is bound. Although during execution, all three
refer to the same piece of memory, logically they represent different values
throughout the computation.

\subsubsection{Kalman Filter}

A \emph{Kalman Filter}~\cite{kalman} is an algorithm for combining prior
knowledge of a state, a statistical model and measurements from (noisy) sensors
to produce an estimate a more reliable estimated of the current state.  It has
various applications (navigation, signal-processing, econometrics) and is
relevant here because it is usually presented as a series of complex matrix
equations.

\begin{figure}[t]
    \centering
    \inputminted[fontsize=\small]{ocaml}{../../examples/kalman.lt}
    \caption{Kalman filter: see Figure~\ref{fig:kalman_eqns} for the
        equations this code implements and Figure~\ref{fig:cblas_kalman}
        for an equivalent \textsc{Cblas/Lapacke} implementation.}\label{fig:lang_kalman}
\end{figure}

\begin{figure}[t]
    % align* uses too much vertical space
    {\centering
    $ \displaystyle
    \begin{aligned}
        \mu' &= \mu + \Sigma H^T (R + H \Sigma H^T)^{-1} (H \mu - \textrm{data})\\
        \Sigma' &= \Sigma ( I - H^T (R + H \Sigma H^T)^{-1} H \Sigma )
    \end{aligned}
    $ \par}
    \caption{Kalman filter equations (credit:
    \href{http://matthewrocklin.com/blog/work/2012/11/24/Kalman-Filter}{matthewrocklin.com}).}\label{fig:kalman_eqns}
\end{figure}

Figure~\ref{fig:lang_kalman} shows a \lang\ implementation of a Kalman filter
(equations in Figure~\ref{fig:kalman_eqns}). A few new features and techniques
are used in this implementation:
\begin{itemize}

    \item \texttt{sym} annotations in matrix expressions: when this is used, a
        call to \texttt{symm} (the equivalent of \texttt{gemm} but for
        symmetric matrices so that only half the operations are performed) is
        inserted

    \item \texttt{copyM\_to} is used to re-use memory by \emph{overwriting} the
        contents of its second argument to that of its first (erroring if
        dimensions do not match)

    \item \highl{let new_r <- new [| r_2 |]} creates a copy of \texttt{r\_2}

    \item \texttt{posvFlip} is like \texttt{posv} except for solving $XA = B$

    \item a lot of memory re-use; the following sets of identifiers alias each other:
        \begin{itemize}
            \item \texttt{r\_1}, \texttt{r\_2} and \texttt{k\_by\_k}
            \item \texttt{data\_1} and \texttt{data\_2}
            \item \texttt{mu} and \texttt{new\_mu}
            \item \texttt{sigma\_hT} and \texttt{x}
        \end{itemize}

\end{itemize}

The \lang\ implementation is much longer than the mathematical equations for
two reasons. First, the \lang\ implementation is a let-normalised form of the
Kalman equations: since there a large number of unary/binary (and occasionally
ternary) sub-expressions in the equations, naming each one line at a time makes
the implementation much longer.  Second, \lang\ has the additional task of
handling explicit allocations, aliasing and frees of matrices. However, it is
exactly this which makes it possible (and often, easy) to spot additional
opportunities for memory re-use. Furthermore, a programmer can explore those
opportunities easily because \lang's type system statically enforces correct
memory management and the aliasing assumptions of BLAS/LAPACK routines.



\section{Formal System}\label{sec:formal_system}

\subsection{Core Type Theory}

The full typing rules are in Appendix \ref{subsec:static_sem}, but the key
ideas are as follow.

A typing judgement consists of $ \Theta; \Delta; \Gamma
\vdash e : t$.

$\Theta$ is the environment that tracks which fractional permission variables
in scope. Fractional permissions (the \textsf{Perm} judgement) and types (the
\textsf{Type} judgement) are \emph{well-formed} if all of their free fractional
variables are in $\Theta$.

$\Delta$ is the environment storing non-linearly or \emph{inuitionistically}
typed variables.

$\Gamma$ is the environment storing linearly typed variables.  Note that rules
for typing $()$, booleans, integers and elements are typed with respect to an
\emph{empty} linear environment: this means no linear values are needed to
produce a value of those types.

\[
    \ottdruleTyXXUnitXXIntro{}
\]

Conversely, whenever two or more subexpressions need to be typed, they must
consume a disjoint set of linear values (pairs, let-expressions).  In the case
of if-expressions, both branches must consume the same set of linear values
(disjoint to the ones used to evaluate the condition).

\[
    \ottdruleTyXXBoolXXElim{}
\]

The \highl{Many} introduction and elimination rules are very important.
Producing !-type values may only be done if the expression inside is a
syntactic value which is not a location. This allows all safely duplicable
resources, including functions which capture non-linear resources from their
environments, but prevents producing aliases of (pointers to) arrays and
matrices. This is exactly the same as value-restriction from the world of
parametric polymorphism.

\[
    \ottdruleTyXXBangXXIntro{}
\]

Consuming a !-type value \emph{moves it} from the linear environment $\Gamma$
and \emph{into} the intuitionistic environment $\Delta$. This is exactly why
$\mathbf{let}\ !x = e_1\ \mathbf{in}\ e_2$ desugars to $\mathbf{let\ Many}\ x =
e_1\ \mathbf{in\ } \mathbf{let\ Many}\ x = \mathbf{Many}\ (\mathbf{Many}\ x)\
\mathbf{in}\ e_2$.

\[
    \ottdruleTyXXBangXXElim{}
\]

Rules \textsc{Ty\_Gen} and \textsc{Ty\_Spc} are for fractional permission
generalisation and specialisation respectively. They allow the definition and
use of functions that are polymorphic in the fractional permission index of
their results and one or more of their arguments.

\[
    \ottdruleTyXXGen{} \qquad\qquad \ottdruleTyXXSpc{}
\]

Rule \textsc{Ty\_Fix} shows how recursive functions are typed. Even though
recursive functions are fully annotated, type checking them is interesting for
two reasons: to type check the body of the fixpoint, the type of the recusive
function is in the \emph{intuitionistic} environment $\Delta$ (without this,
you would not be able to write a base case) whilst the argument and its type
are the \emph{only things in the linear environment} $\Gamma$. The latter means
that recursive functions can be type checked in an empty environment (thus be
wrapped in \highl{Many} and used zero or multiple times).

\[
    \ottdruleTyXXFix{}
\]

Lastly, types of almost all \lang\ primitives, as embedded in OCaml's type
system, are shown in Appendix \ref{subsec:primitives}, with some similar ones
(like those for binary arithmetic operators) omitted for brevity. The
main difference between the OCaml type of a primitive like \highl{gemm} and its
\lang\ counterpart is the inclusion of explicit `$\forall$'s.  So,
\highl{float bang -> ('a mat * bool bang) -> ('b mat * bool bang) -> float
bang -> z mat -> ('a mat * 'b mat) * z mat}
will correspond to \\
$!\ottkw{elt} \multimap \forall\, x.\ x \ \ottkw{mat} \ \otimes \ !\ottkw{bool}
\multimap \forall\, y.\  y \ \ottkw{mat} \ \otimes \ !\ottkw{bool} \multimap
\ !\ottkw{elt} \multimap z\ \ottkw{mat} \multimap ( x \ \ottkw{mat} \ \otimes y
\ \ottkw{mat} ) \ \otimes z\ \ottkw{mat}$

\subsection{Dynamic Semantics}\label{subsec:semantics}

The full, small-step transition relation is in Appendix \ref{subsec:dyn_sem},
but the key ideas are as follow.

Heaps ($\sigma$) are multisets containing triples of an abstract location $l$,
a fractional permission $f$ and sized matrices $m_{n,k}$. The notation $l
\mapsto_f m_{k_1, k_2}$ should be read as ``location $l$ represents $f$
ownership over matrix $m$ (of size $k_1 \times k_2$)''.  Each heap-and-expression
either steps to another heap-and-expression or a runtime error $\mathbf{err}$.
In the full grammar definition we see a definition of values and contexts in
the language.

We draw the reader's attention to the definitions relating to fractional
permissions. Specifically, unlike a lambda, the body of a $\ottkw{fun}\, f\!c
\rightarrow \_$ must be a syntactic value. The context $\ottkw{fun}\, f\!c
\rightarrow [-]$ means expressions can be reduced inside a fractional
permission generalisation. This is to emphasize that fractions are merely
\emph{compile-time constructs} and do not affect runtime behaviour. Correct
usage of fractions is enforced by the type system, so programs do not get
stuck. Fractional permissions are specialised using substitution over both the
heap and an expression (\textsc{Op\_Frac\_Perm}).
\[
    \ottdruleOpXXFracXXPerm{}
\]

Like with the static semantics, the interesting rules in the dynamic semantics
are those relating to primitives. Creating a matrix ($\ottkw{matrix}\ k_1\
k_2$) successfully (\textsc{Op\_Matrix}) requires non-negative dimensions and
returns a (fresh) location of a matrix of those dimensions, extending the heap
to reflect that $l$ represents a complete ownership over the new matrix.
\[
    \ottdruleOpXXMatrix{}
\]

Dually, \textsc{Op\_Free}, requires a location represent complete ownership
before removing it and the matrix it points to from the heap.
\[
    \ottdruleOpXXFree{}
\]

Choosing a multiset representation as opposed to a set allows for two
convenient invariants: multiplicity of a triple $l \mapsto_f m_{k_1, k_2}$ in
the heap corresponds to the number of aliases of $l$ in the expression with
type $f\ \ottkw{mat}$ and the sum of all the fractions for $l$ will always be
$1$ (for a closed, well-typed expression). With this in mind, the rules
\textsc{Op\_Share} and \textsc{Op\_Unshare\_Eq} are fairly natural.
\[
    \ottdruleOpXXShare{} \\
\]
\[
    \ottdruleOpXXUnshareXXEq{}
\]

Combining all of these features, we see that \textsc{Op\_Gemm\_Match} requires
that the location being updated ($l_3$) has complete ownership of over matrix
$m_3$ and can thus change what value it stores to $m_1 m_2 + m_3$. In
particular, this places no restriction on $l_2$ and $l_3$: they could be
$\ottkw{share}$d aliases of the same matrix. Transition rules for other
primitives (omitted) follow the same structure: $\mapsto_1$ for any locations
that are written to and $\mapsto_{f\!c}$ for anything else.
\[
    \ottdruleOpXXGemmXXMatch{}
\]

\subsection{Logical Relation}

First, we define an interpretation of heaps with fractional permissions in the
style of Bornat et. al~\cite{bornat} (interpreting the multiset as a partial
map from locations to the sum of all its associated fractions and a matrix) as
well as the n-fold iteration of $\rightarrow$.
\[
    \den{H}{}{\sigma} = \bigstar_{(l,f,m) \in \sigma} [ l \mapsto_f m ]
\]
where
\[
    (\varsigma_1 \star \varsigma_2)(l) \equiv
    \begin{cases}
        \varsigma_1(l) & \textrm{if } l \in \dom(\varsigma_1) \wedge l \notin \dom(\varsigma_2) \\
        \varsigma_2(l) & \textrm{if } l \in \dom(\varsigma_2) \wedge l \notin \dom(\varsigma_1) \\
        (f_1 + f_2, m) & \textrm{if } (f_1, m) = \varsigma_1(l) \wedge (f_2, m) = \varsigma_2(l) \wedge f_1 + f_2 \leq 1 \\
        \textrm{undefined} & \textrm{otherwise}
    \end{cases}
\]

We then define a step-indexed logical relation in the style of Morrisett et.
al~\cite{morrisett}. $(\varsigma, v) \in \V{k}{t}$ means it takes a heap with
exactly $\varsigma$ resources to produce a value $v$ of type $t$ in at most $k$
steps. So, something like a $\ottkw{unit}$ or a $!t$ need no resources, whereas
a $f\, \ottkw{mat}$ needs exactly $f$ ownership of a some matrix and a pair
needs a $\star$ combination of the heaps required for each component.
\begin{align*}
  \V{k}{ \Unit } &= \{ (\empH, \ast) \} \\
  \V{k}{ f \, \Mat } &= \{ (\{ l \mapsto_{2^{-f}} \_ \} , l) \} \\
  \V{k}{ \Bang t } &= \{ (\empH, \Many\, v) \mid (\empH, v) \in \V{k}{t} \} \\
  \V{k}{ t_1 \otimes t_2 } &= \{ (\varsigma_1 \star \varsigma_2, ( v_1, v_2 )) \mid (\varsigma_1, v_1) \in \V{k}{t_1} \wedge (\varsigma_2, v_2) \in \V{k}{t_2} \}
\end{align*}

The definition of $\V{k}{\forall f\!c.\ t}$ says a value and heap
must be the same regardless of what fraction is substituted into both; the
$k-1$ is to take into account fraction specialisation takes ones step
(\textsc{Op\_Spc}).
\[
    \V{k}{ \forall f\!c.\  t } = \{ (\varsigma, \ottkw{fun}\, f\!c \rightarrow \, v) \mid \forall f.\ (\varsigma [ f\!c / f ], v [ f\!c / f ]) \in \V{k-1}{ t [ f\!c / f ] } \}
\]

To understand the definition of $\V{k}{t' \multimap t}$, we must first look at
$\C{k}{t}$, the computational interpretation of types. Intuitively, it is a
combination of a frame rule on heaps (no interference), type-preservation and
termination (in $j < k$ steps) to either an error or a heap-and-expression,
with the further condition that if the expression is a syntactic value then it
is also one semantically.
\begin{align*}
    \C{k}{ t } &= \{ (\varsigma_s, e_s) \mid \forall \, j < k, \sigma_r.\ \varsigma_s \star \varsigma_r \textrm{ defined } \Rightarrow \langle \sigma_s + \sigma_r, e_s \rangle \rightarrow^j \ottkw{err}\ \vee \exists \sigma_f, e_f.\\
               & \qquad \qquad \langle \sigma_s + \sigma_r, e_s \rangle \rightarrow^j \langle \sigma_f + \sigma_r, e_f \rangle \wedge ( e_f \textrm { is a value } \Rightarrow ( \varsigma_f \star \varsigma_r, e_f ) \in \V{k-j}{t} ) \}
\end{align*}

In this light, $\V{k}{t' \multimap t}$ simply says
that $v$ is a function and that evaluating the application of it to any
argument (of the correct type, requiring its own set of resources, bounded by
$k$ steps) satisfies all the aforementioned properties.
\begin{align*}
    \V{k}{ t' \multimap t } &= \{ (\varsigma_v, v ) \mid ( v \equiv \ottkw{fun}\, x : t' \rightarrow e \vee v \equiv \ottkw{fix} (g, x : t' , e : t) ) \, \wedge\\
                            & \qquad \qquad \forall j \leq k, (\varsigma_{v'}, v') \in \V{j}{ t' }.\ \varsigma_v \star \varsigma_v' \textrm{ defined } \Rightarrow (\varsigma_v \star \varsigma_v', v\, v') \in \C{j}{t} \}
\end{align*}

The interpretation of typing environments $\Delta$ and $\Gamma$ are with
respect to an arbitrary substitution of fractional permissions $\theta$. Note
that only the interpretation of $\Gamma$ involves a (potentially) non-empty heap.
\begin{align*}
    \den{I}{k}{ \Delta, x : t } \theta &= \{ \delta[x \mapsto v_x ] \mid \delta \in \den{I}{k}{\Delta}\theta \wedge (\empH, v_x) \in \V{k}{\theta(t)} \} \\
    \den{L}{k}{ \Gamma, x : t } \theta &= \{ (\varsigma \star \varsigma_x, \gamma[x \mapsto v_x ]) \mid (\varsigma, \gamma) \in \den{L}{k}{\Gamma}\theta \wedge (\varsigma_x, v_x) \in \V{k}{\theta(t)} \}
\end{align*}

And so the final semantic interpretation of a typing judgement simply
quantifies over all possible fractional permission substitutions $\theta$,
linear value substitutions $\gamma$, intuitionistic value substitutions
$\delta$ and heaps $\sigma$.  Note that, $\varsigma \equiv \den{H}{}{\theta(\sigma)}$.
\begin{align*}
\den{}{k}{ \Theta; \Delta ; \Gamma \vdash e : t } &= \forall \theta, \delta, \gamma, \sigma.\ \Theta = \dom(\theta) \wedge (\varsigma, \gamma) \in \den{L}{k}{ \Gamma }\theta \wedge \delta \in \den{I}{k}{ \Delta }\theta \Rightarrow \\
                                                     & \qquad \qquad (\varsigma, \theta(\delta(\gamma(e)))) \in \C{k}{ \theta(t) }
\end{align*}

\subsection{Soundness Theorem}

\textbf{Theorem 1 (The Fundamental Lemma of Logical Relations)}
\[
    \forall \Theta, \Delta, \Gamma, e, t.\ \Theta; \Delta ; \Gamma \vdash e : t \Rightarrow
    \forall k.\ \den{}{k}{ \Theta; \Delta ; \Gamma \vdash e : t }
\]

To prove the above theorem, we need several lemmas; the interesting ones are:
the moral equivalent of the frame rule (\ref{frame}), monotonicity for the
step-index (\ref{subsetKJ}), splitting up environments corresponds to splitting
up heaps (\ref{restriction}) and heap-and-expressions take the same steps of
evaluation under any substitution of their free fractional permissions
(\ref{fracPermSub}).

The proof proceeds by induction on the typing judgement.  The case for
\textsc{Ty\_Fix} is the reason we quantify over the step-index $k$ in the
\emph{conclusion} of the soundness theorem. It allows us to then induct over
the step-index and assume exactly the thing we need to prove at a smaller index.

The case for \textsc{Ty\_Gen} follows a similar pattern, but has the extra
complication of reducing an expression with an arbitrary fractional permission
variable in it, and then instantiating it at the last momemnt to conclude,
which is where \ref{fracPermSub} (heap-and-expressions take the same steps of
evaluation under any substitution of their free fractional permissions) is
used.

The rest of the cases are either very simple base cases (variables, unit,
boolean, integer or element literals) or follow very similar patterns; for
these, only \textsc{Ty\_Let} is presented in full and other similar cases
simply highlight exactly what would be different.  The general idea is to split
up the linear substitution and heap along the same split of $\Gamma/\Gamma'$,
then (by induction) use $\C{k}{-}$ and one `half' of the  linear substitution
and heap to conclude the `first' sub-expression either takes $j< k$ steps to
$\ottkw{err}$ or another heap-and-expression.

In the first case, you use \textsc{Op\_Context\_Err} to conclude the whole
let-expression does the same. Similarly we use \textsc{Op\_Context} $j$ times
in the second case. However, a small book-keeping wrinkle needs to be taken
care of in the case that the heap-and-expression turns into a value in $i \leq
j$ steps: \textsc{Op\_Context} is not functorial for the n-fold iteration of
$\rightarrow$.  Basically, the following is not quite true:
\[
\ottdrule{%
    \ottpremise{\langle  \sigma  \ottsym{,}  \ottnt{e}  \rangle  \rightarrow^j  \langle  \sigma'  \ottsym{,}  \ottnt{e'}  \rangle}%
    }{
    \langle  \sigma  \ottsym{,}  \ottnt{C}  \ottsym{[}  \ottnt{e}  \ottsym{]}  \rangle  \rightarrow^j  \langle  \sigma'  \ottsym{,}  \ottnt{C}  \ottsym{[}  \ottnt{e'}  \ottsym{]}  \rangle}{%
    {\ottdrulename{Op\_Context}}{}%
}
\]
because after the $i$ steps, we need to invoke \textsc{Op\_Let\_Var} to proceed
evalution for any remaining $j-i$ steps. After that, it suffices to use the
induction hypothesis on the second sub-expression to finish the proof.  To do
so, we need to construct a valid linear substitution and heap (i.e., one in
$\den{L}{k}{\Gamma', x : t}\theta$). We take the other `half' of the linear
substitution and heap (from the inital split at the start) and extend it with
$[x \mapsto v]$, (where $x$ is the variable bound in the let-expression and $v$
is the value we assume the first sub-expression evaluated to in $i$ steps).



\section{Implementation}\label{sec:implementation}

Talk about how you implemented \lang\ and the general
architecture. Talk about how simple everything is, and also
about how implementing inference for fractions is. 

\subsection{Implementation Strategy}

\lang\ transpiles to OCaml and its implementation follows the structure of a
typical domain-specific language compiler. Although \lang's current
implementation is not as embedded DSL, its the general design is simple enough
to adapt to being so and also to target other languages.

Alongside the transpiler, a `Read-Check-Translate' loop, benchmarking program
and a test suite are included in executable artifacts accompanying this paper.

\begin{enumerate}

    \item \textbf{Parsing}. A generated, LR(1) parser parses a text file into a
        syntax tree. In general, this part will vary for different languages
        and can also be dealt with using combinators or syntax-extensions (the
        EDSL approach) if the host language offers such support.

    \item \textbf{Desugaring}. The syntax tree is then desugared into a
        smaller, more concise, abstract syntax tree. This allows for the type
        checker to be simpler to specify and easier to implement.

    \item \textbf{Matrix Expressions} are also desugared into the abstract
        syntax tree through some simple pattern-matching.

    \item \textbf{Type checking}. The abstract syntax tree is explicitly typed,
        with some inference to make writing typical programs more convenient.

    \item \textbf{Code Generation}. The abstract syntax tree is translated into
        OCaml, with a few `optimisations' to produce more readable code. This
        process is type-preserving: \lang's type system is embedded into
        OCaml's (Figure~\ref{fig:type_grammar}), and so the OCaml type checker
        acts as a sanity check on the generated code.

\end{enumerate}

A very pleasant way to use \lang\ is to have the build system generate code at
\emph{compile-time} and then have it be used by other modules like normal OCaml
functions. This makes it possible and even easy to use \lang\ alongside
existing OCaml libraries; in fact, this is exactly how the benchmarking program
and test-suite use code written in \lang.

\begin{figure}[tp]
    \centering
    \begin{minipage}{.3\textwidth}
        \centering
        \begin{grammar}
            <f> ::= `'
            \alt <fc>
            \alt `Z'
            \alt `S' <f>

            <t> ::= `'
            \alt `unit'
            \alt `bool'
            \alt `int'
            \alt `elt'
            \alt <f> `arr'
            \alt <f> `mat'
            \alt `!' <t>
            \alt \synt{$\forall$} <fc.> <t>
            \alt <t> \lit{$\otimes$} \synt{$t'$}
            \alt <t> \lit{$\multimap$} \synt{$t'$}
        \end{grammar}
    \end{minipage}
    \begin{minipage}{.3\textwidth}
        \centering
        \begin{minted}[fontsize=\small]{ocaml}
module Arr =
  Owl.Dense.Ndarray.D

type z = Z
type 'a s = Succ

type 'a arr =
  A of Arr.arr
  [@@unboxed]

type 'a mat =
  M of Arr.arr
  [@@unboxed]

type 'a bang =
  Many of 'a
  [@@unboxed]
        \end{minted}
    \end{minipage}
    \begin{minipage}{.3\textwidth}
        \begin{align*}
            [\![ f\!c ]\!] &= \texttt{'fc} \\
            [\![ \textbf{Z} ]\!] &= \texttt{z}\\
            [\![ \textbf{S} \, f ]\!] &= [\![ f ]\!]\, \texttt{s}\\
            [\![ \textbf{unit} ]\!] &= \texttt{unit}\\
            [\![ \textbf{bool} ]\!] &= \texttt{bool}\\
            [\![ \textbf{int} ]\!] &= \texttt{int}\\
            [\![ \textbf{elt} ]\!] &= \texttt{float}\\
            [\![ f\, \textbf{arr} ]\!] &= [\![ f ]\!]\, \texttt{arr}\\
            [\![ f\, \textbf{mat} ]\!] &= [\![ f ]\!]\, \texttt{mat}\\
            [\![ \textbf{!} \, t ]\!] &= [\![ t ]\!]\, \texttt{bang}\\
            [\![ \forall f\!c.\, t ]\!] &= [\![ t ]\!]\\
            [\![ t \otimes t' ]\!] &= [\![ t ]\!] \texttt{*} [\![ t' ]\!]\\
            [\![ t \multimap t' ]\!] &= [\![ t ]\!] \rightarrow [\![ t' ]\!]
        \end{align*}
    \end{minipage}
    \caption{\lang's type grammar (left) and its embedding into OCaml
        (right).}\label{fig:type_grammar}
\end{figure}

\subsubsection{Desugaring}\label{subsubsec:desugaring}

Straightforward enough.

\subsubsection{Matrix Expressions}

Pattern matching!

\subsubsection{Type Checking and Fractional Permission Inference}

Unification!

\subsubsection{Code Generation}

A few examples?

\subsection{Performance Metrics}

Here, evaluate the performance of the examples from the second
section.  Compare with your C implementations, and perhaps as well as
the straightforward math transcribed into (Matlab/R/Numpy?).


\section{Discussion and Related Work}\label{sec:discussion_related_work}

\subsection{Finding Bugs in SymPy's Output}\label{subsec:finding_bugs}

Prior to this project, we had little experience with linear algebra libraries
or the problem of matrix expression compilation. As such, we based our initial
\lang\ implementation of a Kalman filter using BLAS and LAPACK, on a popular
GitHub gist of a Fortran implementation, one that was \emph{automatically
generated} from SymPy's matrix expression compiler~\cite{rocklin_thesis}.

Once we translated the implementation from Fortran to \lang, we attempted to
compile it and found that (to our surprise) it did not type-check. This was
because the original implementation contained incorrect aliasing, unused
variables and unnecessary temporaries, and did not adhere to Fortran's
read/write permissions (with respect to \texttt{intent} annotations
\texttt{in}, \texttt{out} and \texttt{inout}) all of which were now highlighted
by \lang's type system.

The original implementation used 6 temporaries, one of which was immediately
spotted as never being used due to linearity. It also contained two variables
which were marked as \texttt{intent(in)} but would have been written over by
calls to `gemm', spotted by the fractional capabilities feature. Furthermore,
it used a matrix \emph{twice} in a call to `symm', once with a read permission
but once with a \emph{write} permission.  Fortran assumes that any parameter
being written to is not aliased and so this call was not only incorrect, but
illegal according to the standard, both aspects of which were captured by
linearity and fractional capabilities.

Lastly, it contained another unnecessary temporary, however one that was not
obvious without linear types. To spot it, we first performed live-range
splitting (checked by linearity) by hoisting calls to \highl{freeM} and then
annotated the freed matrices with their dimensions.  After doing so and
spotting two disjoint live-ranges of the same size, we replaced a call to
\highl{freeM} followed by allocating call to \highl{copy} with one, in-place
call to \highl{copyM_to}. We believe the ability to boldly refactor code which
manages memory is good evidence of the usefulness of linearity as a tool for
programming.

\subsection{Related Work}

Using linear types for BLAS routines is a particularly good domain fit (given
the implicit restrictions on aliasing arguments), and as a result the idea of
using substructural types to express array computations is not a particularly
new one~\cite{scholz,henriksen,bernardy2016}.  However, many of these designs
have been focused on building languages to \emph{implement} the kernel linear
algebra functions, and as a result, they tend to add additional limitations on
the language design. Both Futhark~\cite{henriksen} and Single Assignment
C~\cite{scholz} omit higher-order functions to facilitate compilation to GPUs.
The work of \cite{bernardy2016} forbids term-level recursion, in order to
ensure that all higher-order computations can be statically normalized away and
thereby maximize opportunities for array fusion.

In contrast, our approach is to begin with the assumption that we can take
existing efficient BLAS-like libraries, and then enforce their correct
\emph{usage} using a linear type discipline with fractional permissions. 

This approach is similar to the one taken in linear algebra libraries for Rust
-- these libraries typically take advantage of the distinction that Rust's type
system offers between mutable views/references to arrays.  The work of
\cite{weiss} and \cite{rustbelt} suggest that Rust's borrow-checker \emph{can
be expressed in simpler terms} using fractional permissions, though to our
knowledge the programmer-visible lifetime analysis in Rust has never been
formalized.

Working explicitly with fractional permissions has two main benefits. First,
our type system demonstrates that type systems for fractional permissions can
be dramatically simpler than existing state-of-the-art approaches, including
both industrial languages like Rust, as well as academic (such as those
developed by \cite{bierhoff}).  Bierhoff \emph{et al}'s type system, much like
Rust's, builds a complex dataflow analysis into the typing rules to infer when
variables can be shared or not. This allows for more natural-looking user
programs, but can create the impression that using fractional permissions
requires a heavy theoretical and engineering effort going well beyond that
needed for supporting basic linear types.

Instead, our approach, of requiring sharing to be made explicit, lets us
demonstrate that the existing unification machinery already in place for
ordinary ML-style type inference can be reused to support fractions. Basically,
we can view sharing a value as dividing a fraction by two, and after taking
logarithms all fractions are Peano numbers, whose equality can be established
with ordinary unification.

This fact is important because there are major upcoming implementations of
linear types such as Linear Haskell~\cite{bernardy2017linear}, which do not
have built-in support for fractional permissions. Instead, Linear Haskell takes
a slightly different definition of linearity, one based on \emph{arrows} as
opposed to \emph{kinds}: for $f : a \multimap b$, if $f u$ is used exactly once
\emph{then} $u$ is used exactly once. Whilst this has the advantage of being
backwards-compatible, it also means that the type system has no built-in
support for the concurrent reader, exclusive writer pattern that fractional
permissions enable.

However, since our type system shows demonstrates unification is ``all one
needs'' for fractions, it should be possible to \emph{encode} \lang's approach
to fractional permissions in Linear Haskell by adding a GADT-style natural
number index to array types tracking the fraction, which should enable
supporting high-performance BLAS bindings in Linear Haskell. Actually
implementing this is something we leave for future work, as there remains one
issue which we do not see a good encoding for. Namely, only having support for
linear functions makes it a bit inconvenient to manipulate linear values
directly -- programs end up taking on a CPS-like structure. This seems to
remain an advantage of a direct implementation of linear types over the Linear
Haskell style.


\subsection{Simplicity and Further Work}

We are pleasantly surprised at how simple the overall design and implementation
of \lang\ is, given its expressive power and usability.  So simple in fact,
that fractions, a convenient theoretical abstraction until this point, could be
implemented by restricting division and multiplication to be by 2 only
\cite{boyland2003}, thus turning any required arithmetic into unification.

Indeed, the focus on getting a working prototype early on (so that we could
test it with real BLAS/LAPACK routines as soon as possible) meant that we only
added features to the type system when it was clear that they were absolutely
necessary: these features were !-types and value-restriction for the
\highl{Many} constructor. 

Going forwards, one may wish to eliminate even more runtime errors from \lang,
by extending its type system. For example, we could have used existential types
to statically track pointer identities~\cite{ahmed20073}, or parametric
polymorphism.

We could also attempt to catch mismatched dimensions at compile time as well.
While this could be done with generative phantom types~\cite{abe2015simple},
using dependent types may offer more flexibility in \emph{partitioning}
regions~\cite{space_monads} or statically enforcing dimensions related
constraints of the arguments at compile-time.  ATS~\cite{cui2005ats} is an
example of a language which combines linear types with a sophisticated proof
layer. But although it provides BLAS bindings, it does not aim to provide
aliasing restrictions as demonstrated in this paper.

Taking this idea one step even further, since matrix dimensions are typically
fixed at runtime, we could \emph{stage} \lang\ programs and compile matrix
expressions using more sophisticated algorithms~\cite{barthels}. However, it is
worth noting that without care, such algorithms~\cite{rocklin_thesis}, usually
based on graph-based, ad-hoc dataflow analysis, can produce erroneous output
which would not get past a linear type system with fractions.

We also think that this concept (and the general design of its implementation)
need not be limited to linear algebra: we could conceivably `backport' this
idea to other contexts that need linearity (concurrency, single-use
continuations, zero-copy buffer, streaming I/O) or combine it with dependent
types to achieve even more expressive power to split up a single block of
memory into multiple regions in an arbitrary manner~\cite{space_monads}.


\clearpage
\bibliography{ourbib}

\clearpage
\appendix
\section{Appendix}

\begin{figure}[t]
\begin{center}
\[\def\arraystretch{1.3}
    \begin{array}{rcl}
    x[e] &
    \Rightarrow &
    \mathbf{get}\ \_\ x\ (e) \;\qquad \textrm{(similarly for matrices)}
\\
    x[e_1] := e_2 &
    \Rightarrow &
    \mathbf{set}\ x\ (e_1)\ (e_2) \quad \textrm{(similarly for matrices)}
\\
\\
    pat & ::= & ()\ \mid\ x\ \mid\ !x\ \mid\ \mathbf{Many\ } pat\ \mid\ (pat, pat)
\\
    \mathbf{let}\ !x = e_1\ \mathbf{in}\ e_2 &
    \Rightarrow &
    \specialcell[t]{l}{\mathbf{let\ Many}\ x = e_1\ \mathbf{in\ } \\
    \mathbf{let\ Many}\ x = \mathbf{Many}\ (\mathbf{Many}\ x)\ \mathbf{in}\ e_2}
\\
    \mathbf{let\ Many} \langle pat_x \rangle\ = e_1\ \mathbf{in}\ e_2 &
    \Rightarrow &
    \specialcell[t]{l}{%
        \mathbf{let\ Many}\ x = x\ \mathbf{in\ } \\
        \mathbf{let\ } \langle pat_x \rangle\ = x\ \mathbf{in\ } e_2}
\\
    \mathbf{let}\ (\langle pat_a \rangle, \langle pat_b \rangle)\  = e_1\ \mathbf{in}\ e_2 &
    \Rightarrow &
    \specialcell[t]{l}{%
        \mathbf{let\ } (a,b)\ = a\_b\ \mathbf{in\ }
        \ \mathbf{let\ } \langle pat_a \rangle\ = a\ \mathbf{in\ } \\
        \mathbf{let\ } \langle pat_b \rangle\ = b\ \mathbf{in\ } e_2}
\\
    \mathbf{fun}\ (\langle pat_x \rangle : t) \rightarrow e &
    \Rightarrow &
    \mathbf{fun}\ (x : t) \rightarrow \mathbf{let\ } \langle pat_x \rangle = x\ \mathbf{in\ } e
\\
\\
    arg & ::= & \langle pat \rangle : t\ \mid\ {'x} \textrm{ (fractional permission variable)}
\\
    \mathbf{fun}\ \langle arg_{1 .. n} \rangle \rightarrow e &
    \Rightarrow &
    \mathbf{fun}\ \langle arg_1 \rangle \rightarrow ..
    \ \mathbf{fun}\ \langle arg_n \rangle \rightarrow e
\\
    \mathbf{let}\ f\ {\langle arg_{1 .. n} \rangle} = e_1\ \mathbf{in}\ e_2 &
    \Rightarrow &
    \mathbf{let}\ f = \mathbf{fun}\ {\langle arg_{1 .. n} \rangle} \rightarrow e_1\
    \mathbf{in}\ e_2
\\
    \mathbf{let}\ !f\ {\langle arg_{1 .. n} \rangle} = e_1\ \mathbf{in}\ e_2 &
    \Rightarrow &
    \mathbf{let\ Many}\ f = \mathbf{Many}\ (\mathbf{fun}\ {\langle arg_{1 .. n} \rangle}
    \rightarrow e_1)\ \mathbf{in}\ e_2
\\
    \mathrm{fixpoint} & \equiv & \mathbf{fix}\ (f, x : t, \mathbf{fun}
    \ {\langle arg_{1 .. n} \rangle} \rightarrow e_1 : {t'} )
\\
    \mathbf{let\ rec}\ f\ (x : t)\ {\langle arg_{1 .. n} \rangle} : {t'} = e_1\ \mathbf{in}\ e_2 &
    \Rightarrow &
    \mathbf{let}\ f = \mathrm{fixpoint}\ \mathbf{in}\ e_2
\\
    \mathbf{let\ rec}\ !f\ (x : t)\ {\langle arg_{1 .. n} \rangle} : {t'} = e_1\ \mathbf{in}\ e_2 &
    \Rightarrow &
    \mathbf{let\ Many}\ f = \mathbf{Many}\ \mathrm{fixpoint}\ \mathbf{in}\ e_2
    \end{array}
\]
\end{center}
\caption{Desugaring from \lang\ concrete syntax to core constructs.}\label{fig:lang_desugar}
\end{figure}

\begin{figure}[p]
    \centering
    \begin{minted}[fontsize=\footnotesize]{ocaml}
let kalman sigma h mu r_1 data_1 =
  let h, _p_k_n_p_ = Prim.size_mat h in
  let k, n = _p_k_n_p_ in
  let sigma_h = Prim.matrix k n in
  let (sigma, h), sigma_h =
    Prim.symm (Many true) (Many 1.) sigma h (Many 0.) sigma_h
  in
  let (sigma_h, h), r_2 =
    Prim.gemm (Many 1.) (sigma_h, Many false) (h, Many true) (Many 1.) r_1
  in
  let (h, mu), data_2 =
    Prim.gemm (Many 1.) (h, Many false) (mu, Many false) (Many (-1.)) data_1
  in
  let h, new_h = Prim.copy_mat_to h sigma_h in
  let r_2, new_r = Prim.copy_mat r_2 in
  let chol_r, sol_h = Prim.posv new_r new_h in
  let chol_r, sol_data = Prim.potrs chol_r data_2 in
  let () = Prim.free_mat chol_r in
  let h_sol_h = Prim.matrix n n in
  let (h, sol_h), h_sol_h =
    Prim.gemm (Many 1.) (h, Many true) (sol_h, Many false) (Many 0.) h_sol_h
  in
  let () = Prim.free_mat sol_h in
  let h_sol_data = Prim.matrix n (Many 1) in
  let (h, sol_data), h_sol_data =
    Prim.gemm (Many 1.) (h, Many true) (sol_data, Many false) (Many 0.) h_sol_data
  in
  let mu, mu_copy = Prim.copy_mat mu in
  let (sigma, h_sol_data), new_mu =
    Prim.symm (Many false) (Many 1.) sigma h_sol_data (Many 1.) mu_copy
  in
  let () = Prim.free_mat h_sol_data in
  let h_sol_h_sigma = Prim.matrix n n in
  let (sigma, h_sol_h), h_sol_h_sigma =
    Prim.symm (Many true) (Many 1.) sigma h_sol_h (Many 0.) h_sol_h_sigma
  in
  let sigma, sigma_copy = Prim.copy_mat_to sigma h_sol_h in
  let (sigma, h_sol_h_sigma), new_sigma =
    Prim.symm (Many false) (Many (-1.)) sigma h_sol_h_sigma (Many 1.) sigma_copy
  in
  let () = Prim.free_mat h_sol_h_sigma in
  ((sigma, (h, (mu, (r_2, sol_data)))), (new_mu, new_sigma)) )
in
kalman
    \end{minted}
    \caption{OCaml code for a Kalman filter, generated (at \emph{compile time})
        from the code in Figure~\ref{fig:lang_kalman}, passed through
        \texttt{ocamlformat} for presentation.}\label{fig:ocaml_kalman}

\end{figure}

\begin{landscape}
\begin{figure}[p]
    \centering
    \begin{minted}[fontsize=\footnotesize]{c}
static void kalman( const int n,               const int k,                const double *sigma, /* n,n */
                    const double *h, /* k,n */ const double *mu, /* n,1 */ double *r,           /* k,k */
                    double *data,    /* k,1 */ double **ret_mu,  /* k,1 */ double **ret_sigma   /* n,n */ ) {
        double* k_by_n = (double *) malloc(k * n * sizeof(double));
/*16*/  cblas_dsymm(CblasRowMajor, CblasRight, CblasUpper, k, n, 1., sigma, n, h, n, 0., k_by_n, n);
/*17*/  cblas_dgemm(CblasRowMajor, CblasNoTrans, CblasTrans, k, k, n, 1., k_by_n, n, h, n, 1., r, k);
/*18*/  cblas_dgemm(CblasRowMajor, CblasNoTrans, CblasNoTrans, k, 1, n, 1., h, n, mu, 1, -1., data, 1);
/*19*/  cblas_dcopy(k * n, h, 1, k_by_n, 1);
        double* k_by_k = (double *) malloc(k * k * sizeof(double));
/*20*/  cblas_dcopy(k * k, r, 1, k_by_k, 1);
/*21*/  LAPACKE_dposv(LAPACK_ROW_MAJOR, 'U', k, n, k_by_k, k, k_by_n, n);
/*23*/  LAPACKE_dpotrs(LAPACK_ROW_MAJOR, 'U', k, 1, k_by_k, k, data, 1);
        free(k_by_k);
        double* n_by_n = (double *) malloc(n * n * sizeof(double));
/*24*/  cblas_dgemm(CblasRowMajor, CblasTrans, CblasNoTrans, n, n, k, 1., h, n, k_by_n, n, 0., n_by_n, n);
        free(k_by_n);
        double* n_by_1 = (double *) malloc(n * sizeof(double));
/*25*/  cblas_dgemm(CblasRowMajor, CblasTrans, CblasNoTrans, n, 1, k, 1., h, n, data, 1, 0., n_by_1, 1);
        double* new_mu = (double *) malloc(n * sizeof(double));
/*26*/  cblas_dcopy(n, mu, 1, new_mu, 1);
/*27*/  cblas_dsymm(CblasRowMajor, CblasLeft, CblasUpper, n, 1, 1., sigma, n, n_by_1, 1, 1., new_mu, 1);
        free(n_by_1);
        double* n_by_n2 = (double *) malloc(n * n * sizeof(double));
/*28*/  cblas_dsymm(CblasRowMajor, CblasRight, CblasUpper, n, n, 1., sigma, n, n_by_n, n, 0., n_by_n2, n);
/*29*/  cblas_dcopy(n*n, sigma, 1, n_by_n, 1);
/*30*/  cblas_dsymm(CblasRowMajor, CblasLeft, CblasUpper, n, n, -1., sigma, n, n_by_n2, n, 1., n_by_n, n);
        free(n_by_n2);
        *ret_sigma = n_by_n;
        *ret_mu = new_mu; }
    \end{minted}
    \caption{CBLAS/LAPACKE implementation of a Kalman filter. I used C instead
        of Fortran because it is what Owl uses under the hood and OCaml FFI
        support for C is better and easier to use than that for Fortran. A distinct
        `measure\_kalman' function that sandwiches a call to this function with
        \texttt{getrusage} is omitted for brevity.}\label{fig:cblas_kalman}

\end{figure}
\end{landscape}


\end{document}
