\section{\lang\ Overview and Examples}\label{sec:lang_and_examples}

\subsection{Overview}

Linearity is at the heart of \lang. Linearity allows us to express a
pure-functional API for numerical library routines that mutate arrays and
matrices. Linearity also restricts aliasing of (values which represent)
pointers.

\subsubsection{Intuitionism: ! and \highl{Many}}

However, linearity by itself is not sufficient to produce an expressive enough
programming language. For values such as booleans, integers, floating-point
numbers as well as pure functions, we need to be able to use them
\emph{intuitionistically}, that is, more than once or not at all. For this
reason, we have the !-constructor at the type level and its corresponding
\highl{Many}-constructor and \highl{let Many <id> = .. in ..}-eliminator at
the term level. Because we want to restrict how a programmer can alias pointers
and prevent a programmer from ignoring them (a memory leak), \lang\ enforces
simple syntactic restrictions on which values can be wrapped up in a
\highl{Many} constructor (details in Section~\ref{sec:formal_system}).

\subsubsection{Fractional Permissions}

There are also valid cases in which we would want to alias pointers to a
matrix. The most common is exemplified by the BLAS routine \texttt{gemm}, which
(rather tersely) stands for \emph{GEneric Matrix Multiplication}.  A
\emph{simplified} definition of \texttt{gemm($\alpha$, A, B, $\beta$, C)} is $C
:= \alpha AB + \beta C$. In this case, \texttt{A} and \texttt{B} may alias each
other but neither may alias \texttt{C}, because it is being written to.
Related to \emph{mutating} arrays and matrices is \emph{freeing} them. Here, we
would also wish to restrict aliasing so that we do not free one alias and then
attempt to use another. Although linearity on its own suffices to prevent
use-after-free errors when values are \emph{not} aliased (a freed value is
\emph{out of scope} for the rest of the expression), we still need another
simple, yet powerful concept to provide us with the extra expressivity of
aliasing \emph{without} losing any of the benefits of linearity.

Fractional permissions provide exactly this. Concretely, types of (pointers to)
arrays and matrices are \emph{parameterised} by a \emph{fraction}. A fraction
is either $1$ ($2^0$) or exactly \emph{half} of another fraction ($2^{-k}$, for
natural $k$). The former represents complete ownership of that value: the
programmer may mutate or free that value as they choose; the latter represents
read-only access or a \emph{borrow}: the programmer may read from the value but
not write to or free it. Creating an array/matrix gives you ownership of it, so
too does having one (with a fractional permission of $2^0$) passed in as an
argument.

In \lang, we can produce two aliases of a single array/matrix, by
\emph{sharing} it. If the original alias had a fractional permission of
$2^{-k}$ then the two new aliases of it will have a fractional permission of
$2^{-(k+1)}$ each. Thanks to linearity, the original array/matrix with a
fractional permission of $2^{-k}$ will be out of scope after the sharing.  When
an array/matrix is shared as such, we can prevent the programmer from freeing
or mutating it by making the types of \highl{free} and \highl{set} (for
mutation) require a \emph{whole} ($2^0$) permission.

If we have two aliases \emph{to the same matrix} with \emph{identical}
fractional permissions ($2^{-(k+1)}$), we can recombine or \emph{unshare} them
back into a single one, with a larger $2^{-k}$ permission. As before, thanks to
linearity, the original two aliases will be out of scope after unsharing.

\subsubsection{Runtime Errors}

Matrix unsharing is one of only \emph{two} operations that can fail at runtime
(the other being dimension checks, such as for \highl{gemm}). The check being
performed is a simple sanity check that the two aliasing pointers passed to
\highl{unshare} point to the same array/matrix.
Section~\ref{sec:discussion_related_work} contains an overview of how we could
remove the need for this by tracking pointer identities statically by
augmenting the type system further.

\subsubsection{Recursion}

The final feature of \lang\ which makes it sufficiently expressive is recursion
(and of course, conditional branches to ensure termination). Conditional
branches are implemented by ensuring that both branches use the same set of
linear values. A function can be recursive if it captures no linear values from
its environment. Like with \highl{Many}, this is enforced via simple syntactic
restrictions on the definition of recursive functions.

\subsection{Examples}

The more examples, the better. In fact, it's actually impossible to
have too many examples -- it's okay (indeed, desirable) to spend 5-6
pages on examples.

\begin{itemize}

    \item Simple: factorial, shows recursion, !-annotations etc.

    \item Less simple: summing over an array, indexing and safety

    \item Medium: one-dimensioal convolution, permissions

    \item Harder: linear regression, pattern-matching and apparent non-linearity

    \item Harder: L1 norm-minimisation, some frees, re-using memory

    \item Big finish: Kalman filter

\end{itemize}

\subsubsection{Introductory: Factorial}

Although a factorial function (Figure~\ref{fig:lang_factorial}) may seem like
an aggresively pedestrian first example, in a linearly typed language such as
\lang\ it represents the culmination of a rather number of features.

To simplifiy the design and implementation of \lang's type system, recursive
functions must have full type annotations (non-recursive functions need only
their argument types annotated). Its body is a closed expression (with respect
to the function's arguments), so it type-checks (since it does not capture any
linear values from its environment).

The only argument is \highl{!x : !int}. The !-annotation on \highl{x} is a
syntactic convenience for declaring the value to used intuitionistically, its
full and precise meaning is described in Section~\ref{sec:implementation}.

The condition for an \highl{if} may or may not use linear values (here, with
\highl{x < 0} \highl{ || x = 0}, it does not). Any linear values used by the conditional
would not be in scope in either branch of the \highl{if}-expression.  Both
branches use \highl{x} differently: one ignores it completely and the other
uses it twice.

All numeric and boolean literals are implicitly wrapped in a \highl{Many} and
all primitives involving them return a \highl{!int}, \highl{!bool} or
\highl{!elt} (types of elements of arrays/matrices, typically 64-bit
floating-point numbers). The short-circuting \highl{||} behaves in exactly the
same way as a boolean-valued \highl{if}-expression.

\begin{figure}[tp]
    \centering
    \inputminted[fontsize=\small, linenos]{ocaml}{../../test/examples/factorial.lt}
    \caption{Factorial function in \lang.}\label{fig:lang_factorial}
\end{figure}

\subsubsection{Introductory: Summing over an Array}

\begin{figure}[tp]
    \centering
    \inputminted[fontsize=\small]{ocaml}{../../test/examples/sum_array.lt}
    \caption{Summing over an array in \lang.}\label{fig:lang_sumarray}
\end{figure}

