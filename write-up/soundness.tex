\section{Soundness Proof}

\begin{proof}

    \pfsketch{\ %
        Use the contrapositive both ways. This turns the negated existential into
        witnesses we can work with.\\
    }

    \pflet{%
        $\phi(X) = $\\
        \emph{Note: $\forall X.\ \phi(X)\subseteq X$, $\not\subset\ \equiv\
        \not\subseteq \, \vee \, =$ and $\not\subseteq\ \Rightarrow\ \neq$}\\
        $a$, $b$, $c$ be elements of the Martelli's semiring \\
        $L^+ = a \cup \phi$ \\
        $L = \phi(L^+) = a \otimes (b \oplus c)$ \\
        $M^+ = $ \\
        $M = \phi(M^+) = (a \otimes b) \oplus (a \otimes c)$ \\
    }

    \prove{Distributivity holds, i.e.\ $L=M$.}
    \suffices{%
        Since $\oplus$ and $\otimes$ are commutative (definitions of $\oplus$
        and $\otimes$ are symmetric in their arguments because $\exists x.\
        \exists y.\ P(x,y) \Leftrightarrow \exists y.\ \exists x.\ P(x,y)$ and
        $\cup$ is commutative) it suffices to show only left-distributivity.\\
    }

    \pf{~We show $L \subseteq M$ and $M \subseteq L$.\\}

    \step{}{%
        \case{%
            $L \subseteq M$.
        }{%
            We show $m \notin M \Rightarrow m \notin L$ for arbitrary $m$.\\
            \pf{~We do this by cases on $m \in M^+$}.
            \suffices{\ %
                Because $L \subseteq L^+$, to show $m \notin L$ it suffices to
                show either $m \notin L^+$ or $\exists y \in L^+.\ y \subset m$.
            }
        }
    }

    \begin{proof}

        \step{}{%
            \case{%
                $m \in M^+$.
            }{%
                This means that $ \exists x \in a \otimes b,\ y \in a \otimes
                c.\ x \cup y = m $ and because $m \notin M$, we have $\exists
                x' \in a \otimes b,\ y \in a \otimes c.\ x' \cup y' = m'
                \subset m = x \cup y$. Assume, without loss of generality,
                they are the smallest such $x'$ and $y'$. Because $\phi(X)
                \subseteq X$ for any $X$, we proceed by cases: either $x' \in
                a$ or $y' \in a$ or both $x' \in b$ and $y' \in c$.
            }
        }

        \step{mplusRule}{%
            \case{%
                $m \notin M^+$.
            }{%
                This means $\forall\ x \in \phi(a \cup b),\ y \in \phi(a \cup
                c).\ m \neq x \cup y$.
            }
        }

        \step{}{%
            \textbf{Thus, if $m \notin M$, then $m \notin L$.}~Q.E.D.
        }

    \end{proof}

    \step{}{%
        \case{%
            $M \subseteq L$.
        }{%
            We show $l \notin L \Rightarrow l \notin M$ for arbitrary $l$.
            \suffices{\ %
                Because $M \subseteq M^+$, to show $l \notin M$ it suffices to
                show either $l \notin M^+$ or $\exists y \in M^+.\ y \subset l$.
            }
        }
    }

    \begin{proof}

        \step{lplusRule}{%
            \case{%
                $l \notin L^+$.
            }{%
                This means $l \notin a$ and $l \notin b \oplus c = \phi$.\\
                We conclude from the latter, that $\forall x \in b,\ y \in c.\ 
                x \cup y \neq l$.\\
                We reason by cases on \emph{why} $l \notin a$, to show that
                $\exists y \in M^+.\ y \subset l$ or $l \notin M^+$.
            }
        }

        \step{}{%
            \case{%
                $l \in L^+$.
            }{%
                Under the assumption $l \notin L$, we need only consider two
                cases: the rest produce the contradiction $l \in L$.
            }
        }

    \end{proof}


    \step{}{%
        \textbf{Thus, $L = M$}~Q.E.D.
    }

\end{proof}
