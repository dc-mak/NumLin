\section{Interpretation}

\subsection{Definitions}

% Changed to multiset because normal disjoint unions and subsets of cartesian
% products for the heap wouldn't capture _multiplicity_:different variables
% in the environment could have identical permissions/types.

Operationally, $\emph{Heap} \sqsubseteq \emph{Loc} \times \emph{Permission}
\times \emph{Matrix} $ (a multiset), denoted with a $\sigma$.\\
Define its \emph{interpretation} to be $\emph{Loc} \rightharpoonup
\emph{Permission} \times \emph{Matrix}$ with $\star:
\emph{Heap} \times \emph{Heap} \rightharpoonup \emph{Heap}$ as follows:
\[
    (\varsigma_1 \star \varsigma_2)(l) \equiv
    \begin{cases}
        \varsigma_1(l) & \textrm{if } l \in \dom(\varsigma_1) \wedge l \notin \dom(\varsigma_2) \\
        \varsigma_2(l) & \textrm{if } l \in \dom(\varsigma_2) \wedge l \notin \dom(\varsigma_1) \\
        (f_1 + f_2, m) & \textrm{if } (f_1, m) = \varsigma_1(l) \wedge (f_2, m) = \varsigma_2(l) \wedge f_1 + f_2 \leq 1 \\
        \textrm{undefined} & \textrm{otherwise}
    \end{cases}
\]
Commutativity and associativity of $\star$ follows from that of $+$.\\
$\varsigma_1 \star \varsigma_2$ is \emph{defined} if it is for all $l \in
\dom(\varsigma_1) \cup \dom(\varsigma_2)$.\\
%Note that $\forall \varsigma.\ \exists \textrm{ cardinally minimal}\, \sigma.\ \varsigma = \den{H}{}{\sigma}$.\\
%{\pf~Binary representation of $\varsigma(l)$ for each $l \in \emph{Loc}$.}\\
\textbf{Implicitly denote} $\varsigma \equiv \den{H}{}{\sigma} \equiv
\bigstar_{(l,f,m) \in \sigma} [ l \mapsto_f m ]$.\\
\\
The $n-$fold iteration for the $StepsTo$ (functional) relation, is also a (functional) relation:
\begin{align*}
    \forall n.\ \ottkw{err} &\rightarrow^n \ottkw{err} &
    \langle \sigma , v \rangle &\rightarrow^n \langle \sigma , v \rangle &
    \langle \sigma , e \rangle &\rightarrow^0 \langle \sigma , e \rangle &
    \langle \sigma , e \rangle &\rightarrow^{n+1} ((\langle \sigma , e \rangle \rightarrow) \rightarrow^n)
\end{align*}
Hence, all bounded iterations end in either an $\ottkw{err}$, a heap-and-expression or a
heap-and-value.

\subsection{Interpretation}

% TODO How to handle primitives?

\begin{align*}
  \V{k}{ \Unit } &= \{ (\empH, \ast) \} \\
\\
    \V{k}{ \Bool } &= \{ (\empH, true), (\empH, false) \} \\
\\
    \V{k}{ \Int } &= \{ (\empH, n) \mid 2^{-63} \leq n \leq 2^{63} -1 \} \\
\\
    \V{k}{ \Elt } &= \{ (\empH, f) \mid f \textrm{ a IEEE Float64 } \} \\
\\
    \V{k}{ f \, \Mat } &= \{ (\{ l \mapsto_{2^{-f}} \_ \} , l) \} \\
\\
    \V{k}{ \Bang t } &= \{ (\empH, \Many\, v) \mid (\empH, v) \in \V{k}{t} \} \\
\\
% Using substitution here directly helps avoid routing (fun fc -> v) [f] in C_k[[..]]
    \V{k}{ \forall fc.\  t } &= \{ (\varsigma, \ottkw{fun}\, fc \rightarrow \, v) \mid \forall f.\ (\varsigma, v [ fc / f ]) \in \V{k-1}{ t [ fc / f ] } \} \\
\\
    \V{k}{ t_1 \otimes t_2 } &= \{ (\varsigma_1 \star \varsigma_2, ( v_1, v_2 )) \mid (\varsigma_1, v_1) \in \V{k}{t_1} \wedge (\varsigma_2, v_2) \in \V{k}{t_2} \} \\
\\
% j <= k because beta-reduction is guaranteed to take one step AND our types aren't recursive.
    \V{k}{ t' \multimap t } &= \{ (\varsigma_v, v ) \mid ( v \equiv \ottkw{fun}\, x : t \rightarrow e \vee v \equiv \ottkw{fix} (g, x : t' , e : t) ) \, \wedge\\
                            & \qquad \qquad \forall j \leq k, (\varsigma_{v'}, v') \in \V{j}{ t' }.\ \varsigma_v \star \varsigma_v' \textrm{ defined } \Rightarrow (\varsigma_v \star \varsigma_v', v\, v') \in \C{j}{t} \} \\
\\
% j < k to match L3 Fluet/Ahmed paper
    \C{k}{ t } &= \{ (\varsigma_s, e_s) \mid \forall \, j < k, \sigma_r.\ \varsigma_s \star \varsigma_r \textrm{ defined } \Rightarrow \langle \sigma_s + \sigma_r, e_s \rangle \rightarrow^j \ottkw{err}\ \vee \exists \sigma_f, e_f.\\
               & \qquad \qquad \langle \sigma_s + \sigma_r, e_s \rangle \rightarrow^j \langle \sigma_f + \sigma_r, e_f \rangle \wedge ( e_f \textrm { is a value } \Rightarrow ( \varsigma_f \star \varsigma_r, e_f ) \in \V{k-j}{t} ) \} \\
\\
    \den{I}{k}{ \cdot } \theta &= \{ [] \} \\
\\
    \den{I}{k}{ \Delta, x : t } \theta &= \{ \delta[x \mapsto v_x ] \mid \delta \in \den{I}{k}{\Delta}\theta \wedge (\empH, v_x) \in \V{k}{\theta(t)} \} \\
\\
    \den{L}{k}{ \cdot } \theta &= \{ (\empH, []) \} \\
\\
    \den{L}{k}{ \Gamma, x : t } \theta &= \{ (\varsigma \star \varsigma_x, \gamma[x \mapsto v_x ]) \mid (\varsigma, \gamma) \in \den{L}{k}{\Gamma}\theta \wedge (\varsigma_x, v_x) \in \V{k}{\theta(t)} \} \\
\\
    \varsigma \equiv \den{H}{}{\sigma} &\equiv \bigstar_{(l,f,m) \in \sigma} [ l \mapsto_f m ]\\
\\
\den{}{k}{ \Theta; \Delta ; \Gamma \vdash e : t } &= \forall \theta, \delta, \gamma, \sigma.\ \Theta = \dom(\theta) \wedge (\varsigma, \gamma) \in \den{L}{k}{ \Gamma }\theta \wedge \delta \in \den{I}{k}{ \Delta }\theta \Rightarrow \\
                                                 & \qquad \qquad (\varsigma, \theta(\delta(\gamma(e)))) \in \C{k}{ \theta(t) }
\end{align*}
