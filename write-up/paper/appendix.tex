\section{Appendix}

\begin{figure}[t]
\begin{center}
\[\def\arraystretch{1.3}
    \begin{array}{rcl}
    x[e] &
    \Rightarrow &
    \mathbf{get}\ \_\ x\ (e) \;\qquad \textrm{(similarly for matrices)}
\\
    x[e_1] := e_2 &
    \Rightarrow &
    \mathbf{set}\ x\ (e_1)\ (e_2) \quad \textrm{(similarly for matrices)}
\\
\\
    pat & ::= & ()\ \mid\ x\ \mid\ !x\ \mid\ \mathbf{Many\ } pat\ \mid\ (pat, pat)
\\
    \mathbf{let}\ !x = e_1\ \mathbf{in}\ e_2 &
    \Rightarrow &
    \specialcell[t]{l}{\mathbf{let\ Many}\ x = e_1\ \mathbf{in\ } \\
    \mathbf{let\ Many}\ x = \mathbf{Many}\ (\mathbf{Many}\ x)\ \mathbf{in}\ e_2}
\\
    \mathbf{let\ Many} \langle pat_x \rangle\ = e_1\ \mathbf{in}\ e_2 &
    \Rightarrow &
    \specialcell[t]{l}{%
        \mathbf{let\ Many}\ x = x\ \mathbf{in\ } \\
        \mathbf{let\ } \langle pat_x \rangle\ = x\ \mathbf{in\ } e_2}
\\
    \mathbf{let}\ (\langle pat_a \rangle, \langle pat_b \rangle)\  = e_1\ \mathbf{in}\ e_2 &
    \Rightarrow &
    \specialcell[t]{l}{%
        \mathbf{let\ } (a,b)\ = a\_b\ \mathbf{in\ }
        \ \mathbf{let\ } \langle pat_a \rangle\ = a\ \mathbf{in\ } \\
        \mathbf{let\ } \langle pat_b \rangle\ = b\ \mathbf{in\ } e_2}
\\
    \mathbf{fun}\ (\langle pat_x \rangle : t) \rightarrow e &
    \Rightarrow &
    \mathbf{fun}\ (x : t) \rightarrow \mathbf{let\ } \langle pat_x \rangle = x\ \mathbf{in\ } e
\\
\\
    arg & ::= & \langle pat \rangle : t\ \mid\ {'x} \textrm{ (fractional permission variable)}
\\
    \mathbf{fun}\ \langle arg_{1 .. n} \rangle \rightarrow e &
    \Rightarrow &
    \mathbf{fun}\ \langle arg_1 \rangle \rightarrow ..
    \ \mathbf{fun}\ \langle arg_n \rangle \rightarrow e
\\
    \mathbf{let}\ f\ {\langle arg_{1 .. n} \rangle} = e_1\ \mathbf{in}\ e_2 &
    \Rightarrow &
    \mathbf{let}\ f = \mathbf{fun}\ {\langle arg_{1 .. n} \rangle} \rightarrow e_1\
    \mathbf{in}\ e_2
\\
    \mathbf{let}\ !f\ {\langle arg_{1 .. n} \rangle} = e_1\ \mathbf{in}\ e_2 &
    \Rightarrow &
    \mathbf{let\ Many}\ f = \mathbf{Many}\ (\mathbf{fun}\ {\langle arg_{1 .. n} \rangle}
    \rightarrow e_1)\ \mathbf{in}\ e_2
\\
    \mathrm{fixpoint} & \equiv & \mathbf{fix}\ (f, x : t, \mathbf{fun}
    \ {\langle arg_{1 .. n} \rangle} \rightarrow e_1 : {t'} )
\\
    \mathbf{let\ rec}\ f\ (x : t)\ {\langle arg_{1 .. n} \rangle} : {t'} = e_1\ \mathbf{in}\ e_2 &
    \Rightarrow &
    \mathbf{let}\ f = \mathrm{fixpoint}\ \mathbf{in}\ e_2
\\
    \mathbf{let\ rec}\ !f\ (x : t)\ {\langle arg_{1 .. n} \rangle} : {t'} = e_1\ \mathbf{in}\ e_2 &
    \Rightarrow &
    \mathbf{let\ Many}\ f = \mathbf{Many}\ \mathrm{fixpoint}\ \mathbf{in}\ e_2
    \end{array}
\]
\end{center}
\caption{Desugaring from \lang\ concrete syntax to core constructs.}\label{fig:lang_desugar}
\end{figure}

\begin{figure}[p]
    \centering
    \begin{minted}[fontsize=\footnotesize]{ocaml}
let kalman sigma h mu r_1 data_1 =
  let h, _p_k_n_p_ = Prim.size_mat h in
  let k, n = _p_k_n_p_ in
  let sigma_h = Prim.matrix k n in
  let (sigma, h), sigma_h =
    Prim.symm (Many true) (Many 1.) sigma h (Many 0.) sigma_h
  in
  let (sigma_h, h), r_2 =
    Prim.gemm (Many 1.) (sigma_h, Many false) (h, Many true) (Many 1.) r_1
  in
  let (h, mu), data_2 =
    Prim.gemm (Many 1.) (h, Many false) (mu, Many false) (Many (-1.)) data_1
  in
  let h, new_h = Prim.copy_mat_to h sigma_h in
  let r_2, new_r = Prim.copy_mat r_2 in
  let chol_r, sol_h = Prim.posv new_r new_h in
  let chol_r, sol_data = Prim.potrs chol_r data_2 in
  let () = Prim.free_mat chol_r in
  let h_sol_h = Prim.matrix n n in
  let (h, sol_h), h_sol_h =
    Prim.gemm (Many 1.) (h, Many true) (sol_h, Many false) (Many 0.) h_sol_h
  in
  let () = Prim.free_mat sol_h in
  let h_sol_data = Prim.matrix n (Many 1) in
  let (h, sol_data), h_sol_data =
    Prim.gemm (Many 1.) (h, Many true) (sol_data, Many false) (Many 0.) h_sol_data
  in
  let mu, mu_copy = Prim.copy_mat mu in
  let (sigma, h_sol_data), new_mu =
    Prim.symm (Many false) (Many 1.) sigma h_sol_data (Many 1.) mu_copy
  in
  let () = Prim.free_mat h_sol_data in
  let h_sol_h_sigma = Prim.matrix n n in
  let (sigma, h_sol_h), h_sol_h_sigma =
    Prim.symm (Many true) (Many 1.) sigma h_sol_h (Many 0.) h_sol_h_sigma
  in
  let sigma, sigma_copy = Prim.copy_mat_to sigma h_sol_h in
  let (sigma, h_sol_h_sigma), new_sigma =
    Prim.symm (Many false) (Many (-1.)) sigma h_sol_h_sigma (Many 1.) sigma_copy
  in
  let () = Prim.free_mat h_sol_h_sigma in
  ((sigma, (h, (mu, (r_2, sol_data)))), (new_mu, new_sigma)) )
in
kalman
    \end{minted}
    \caption{OCaml code for a Kalman filter, generated (at \emph{compile time})
        from the code in Figure~\ref{fig:lang_kalman}, passed through
        \texttt{ocamlformat} for presentation.}\label{fig:ocaml_kalman}

\end{figure}

\begin{landscape}
\begin{figure}[p]
    \centering
    \begin{minted}[fontsize=\footnotesize]{c}
static void kalman( const int n,               const int k,                const double *sigma, /* n,n */
                    const double *h, /* k,n */ const double *mu, /* n,1 */ double *r,           /* k,k */
                    double *data,    /* k,1 */ double **ret_mu,  /* k,1 */ double **ret_sigma   /* n,n */ ) {
        double* k_by_n = (double *) malloc(k * n * sizeof(double));
/*16*/  cblas_dsymm(CblasRowMajor, CblasRight, CblasUpper, k, n, 1., sigma, n, h, n, 0., k_by_n, n);
/*17*/  cblas_dgemm(CblasRowMajor, CblasNoTrans, CblasTrans, k, k, n, 1., k_by_n, n, h, n, 1., r, k);
/*18*/  cblas_dgemm(CblasRowMajor, CblasNoTrans, CblasNoTrans, k, 1, n, 1., h, n, mu, 1, -1., data, 1);
/*19*/  cblas_dcopy(k * n, h, 1, k_by_n, 1);
        double* k_by_k = (double *) malloc(k * k * sizeof(double));
/*20*/  cblas_dcopy(k * k, r, 1, k_by_k, 1);
/*21*/  LAPACKE_dposv(LAPACK_ROW_MAJOR, 'U', k, n, k_by_k, k, k_by_n, n);
/*23*/  LAPACKE_dpotrs(LAPACK_ROW_MAJOR, 'U', k, 1, k_by_k, k, data, 1);
        free(k_by_k);
        double* n_by_n = (double *) malloc(n * n * sizeof(double));
/*24*/  cblas_dgemm(CblasRowMajor, CblasTrans, CblasNoTrans, n, n, k, 1., h, n, k_by_n, n, 0., n_by_n, n);
        free(k_by_n);
        double* n_by_1 = (double *) malloc(n * sizeof(double));
/*25*/  cblas_dgemm(CblasRowMajor, CblasTrans, CblasNoTrans, n, 1, k, 1., h, n, data, 1, 0., n_by_1, 1);
        double* new_mu = (double *) malloc(n * sizeof(double));
/*26*/  cblas_dcopy(n, mu, 1, new_mu, 1);
/*27*/  cblas_dsymm(CblasRowMajor, CblasLeft, CblasUpper, n, 1, 1., sigma, n, n_by_1, 1, 1., new_mu, 1);
        free(n_by_1);
        double* n_by_n2 = (double *) malloc(n * n * sizeof(double));
/*28*/  cblas_dsymm(CblasRowMajor, CblasRight, CblasUpper, n, n, 1., sigma, n, n_by_n, n, 0., n_by_n2, n);
/*29*/  cblas_dcopy(n*n, sigma, 1, n_by_n, 1);
/*30*/  cblas_dsymm(CblasRowMajor, CblasLeft, CblasUpper, n, n, -1., sigma, n, n_by_n2, n, 1., n_by_n, n);
        free(n_by_n2);
        *ret_sigma = n_by_n;
        *ret_mu = new_mu; }
    \end{minted}
    \caption{CBLAS/LAPACKE implementation of a Kalman filter. I used C instead
        of Fortran because it is what Owl uses under the hood and OCaml FFI
        support for C is better and easier to use than that for Fortran. A distinct
        `measure\_kalman' function that sandwiches a call to this function with
        \texttt{getrusage} is omitted for brevity.}\label{fig:cblas_kalman}

\end{figure}
\end{landscape}
