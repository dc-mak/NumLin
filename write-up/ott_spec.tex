\chapter{Ott Specification}\label{chap:ott_spec}

The following pages present a specification of the grammar and type-system used
by my project, produced using the Ott\cite{sewell_ott} tool. It is important to note
that the type-system described here is not how it is implemented: it is easier
and clearer to describe the system as below for explaining. However, for implementing,
I found it much more and user- and debugging-friendly to:

\begin{itemize}

    \item Implement it so that the type-environment \emph{changes} as a result
        of type-checking an expression; with this, the below semantics describe
        the \emph{difference} between the environment after and before checking
        an expression.

    \item \emph{Mark} variables as used instead of \emph{removing} them from the
        environment for better error messages.

    \item Have \emph{one} environment where variables were \emph{tagged} as
        linear and un-used, linear and used, and intuitionistic. This was
        definitely an implementation convenience so that variable binding could
        be handled uniformly for linear and intuitionistic variables and
        scoping/variable look-up could be handled implicitly thanks to the
        associative-list structure of the environment.

\end{itemize}

\clearpage%
\ottall%
